%------------------------------------------------------------------------
% Chapter:  Example
%------------------------------------------------------------------------

\chapter{Example \label{example}}

This chapter takes you through all steps of a nanoparticle refinement.
The example will use data published in \cite{chneko2007}, which describes 
the refinement of ZnSe nanoparticles. We will quickly describe the main 
aspects of the model that is used to describe the experimental data. 
The main focus will be on the use of the \Suite and on the differences
to a refinement with \diffev.

The ZnSe nanoparticles can be described as a disordered Zincblende
structure. Stacking faults introduce a local Wurtzite structure. TEM 
images showed that the particles are of approximately ellipsoidal shape
with one long and two short axes.
Bulk ZnSe crystallizes in the Zincblende structure, and these nanoparticles
show a typical high defect concentration. 
For this example we choose to describe the structure as a stack of
layers based on a hexagonal metric. Thus the stacking faults are restricted
to one of the symmetrically equivalent [111] axes of the bulk cubic structure. 
A perfect ABCABC sequence of these layers would correspond to the  cubic
Zincblende structure, a perfect ABAB sequence would yield the hexagonal 
Wurtzite structure. For further details see \cite{chneko2007} and 
\cite{nedpro}.

\section{Parallel refinement \label{exa-par}}

A main advantage of \Suite compared to the stand alone programs is achieved 
in a parallel refinement. A stand alone refinement with \Diffev needs to
start \Discus and \Kuplot as slave programs within each refinement cycle.
Thus at each program start, the macros and the experimental data need to 
be read again. Furthermore, these separate programs need to communicate
the results via files on the hard disk, which slows down the performance. 
As the \Suite is a single program, internal communication is straightforward
and the makes most disk I/O obsolete. The \Suite performs its tasks 
within the individual sections and thus most parts of the actual refinement 
macros do not need to be changed.

To run in parallel mode, the \Suite must have been compiled with the MPI
option. The suite will then be started with a command like:

\begin{MacVerbatim}
   mpiexec -n 192 discus_suite -macro suite.mac  > /dev/null
\end{MacVerbatim}

If you use the Windows operating system, start a parallel refinement via:

\begin{MacVerbatim}
   parallel suite.mac
\end{MacVerbatim}

see chapter \ref{intro-par} for further info on the invocation.


The main \Suite macro can be very short:
\begin{MacVerbatim}
   1   diffev
   2      @diffev.mac
   3   exit
   4   exit
\end{MacVerbatim}

In line 1 we switch to the \Diffev section, perform the -slightly modified -
macro {\tt diffev.mac} and return to the main \Suite menu via the {\tt exit}
in line 3. In line 4 we finally finish the suite itself. 

This macro nesting is not a necessary feature. You could of 
course place all these lines into the {\tt diffev.mac} macro as well.

The {\tt diffev.mac} macro needs just few changes compared to a stand alone 
version:
\begin{MacVerbatim}
   1    set error,exit
   2    @cleanup.mac
   3    #
   4    @setup.mac
   5    @diffev_setup.mac
   6    #
   7    init silent
   8    #
   9    do i[0]=1,100
  10       echo "GENERATION %4d",REF_GENERATION
  11       run_mpi discus, nano.znse.mac, repeat:REF_NINDIV , compute:serial, logfile:/dev/null
  12       compare silent
  13    enddo
\end{MacVerbatim}

In line 1 we instruct \Suite to be picky with errors and to stop the execution 
is any error should occur. I do not want to run a refinement for several 
hours just to find out that somewhere there there has been an erroneous 
calculation.

The {\it cleanup.mac} macro may be used to remove all files related to an 
earlier refinement.

The {\it setup.mac} macro is used to define the number of individual 
repetitions that \Discus should perform for each member of the population.

Macro {\it diffev\_setup.mac} defines the size of the population, the 
number and range of the parameters to be refined etc.

In line 7 we initialize the parameters to their starting values. The 
qualifier {\it silent} instructs the \Suite not to write these 
trial parameters onto the disk. They will instead be passed silently 
between the \Diffev and \Discus sections.

The {\it run\_mpi} command in line 11 starts the simulation and the
assessment of the current agreement between experimental and calculated 
data. The \Diffev section will start the \Discus section and within 
\Discus operate the macro {\it nano.znse.mac}. The next parameter, here a
zero, tells \Diffev if the individual repetitions are to be carried out
in parallel or if they are to be done in serial operation by the 
\Discus section. If the parameters is zero, as in this example, any 
repetition is left to the \Discus section. If the parameter is larger than
one, \Diffev will calculate the individual repetitions in parallel on 
your system.

Finally in line 12 we tell \Diffev that the final agreement factor 
between experimental data and the calculated data shall not be read
from a file on your disk. Instead \Kuplot will pass this parameters
silently back to \diffev.
