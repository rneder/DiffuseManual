\chapter{SUITE commands}
\section{Summary}
\par
You can switch to the individual sections "discus", "diffev", 
and "kuplot" by typing the respective section name. To return 
to the suite type "exit" at the main menu of each section. 
\par
The variables i[*], r[*] and res[*] are global variables, 
a change in any section will be seen in any other section 
as well. 
The same holds for all user defined variables! 
\par
The section specific variables are local within each section. 
\par
If an output filename in "discus" starts with "kuplot", the 
data are written directly into the next available KUPLOT 
data set. This is available for Fourier output, powder, pdf. 
\section{News}
\subsection*{2018\_June}
\par
Revised the reaction to a CTRL-C 
\par
Added a ==$> $ 'set error, ... , "save" option 
\subsection*{2018\_Jan}
\par
The logical comparisons may now take the operators: 
$ <$, $ <$=, ==, /=, $> $=, $> $/ 
The classical fortran77 operators are still valid 
\par
New logical functions "isvar" and "isexp" can be used within an 
"if" construction. See help entry ==$> $'function' in the 
general "Command\_lang" section 
\subsection*{2017\_Sep}
\par
Throughout the program the internal calculation of random numbers 
was changed to the FORTRAN 90 intrinsic function. 
\subsection*{2016\_Dec}
\par
At a few select points colors are introduced into the output. 
Currently these are just the error messages. 
\subsection*{2016\_Oct}
\par
A new command 'parallel' has been added to the Windows version 
This allows to execute a macro in parallel. 
\par
\subsection*{2016\_June}
\par
The SUITE may now be interruted gracefully with a CTRL-c. 
This will cause the DISCUS part to write the current structure, 
and DIFFEV to shut down MPI if active. 
\subsection*{2015\_December}
\par
The branch command within the sections discus, diffev, kuplot may 
now take the form : 
branch discus -macro macro\_name par1, par2, ... 
\subsection*{2015\_June}
\par
Starting with Version 5.1, we have migrated to a X-Window 
environment for WINDOWS as well. As a small side effect, 
the technique to jump to the desired folder has changed slightly. 
See the help entry on "cd" in the general "Command\_lang" section 
for further information. The process is described in the 
package manual as well. 
\par
\section{diffev}
\par
Switches to the "diffev" section. 
\par
Within this section any standard DIFFEV command can be 
given. The behaviour of "diffev" is essentially the same 
as in the stand alone version. 
The 'diffev/run\_mpi' command will start a discus/kuplot 
section. The syntax of the command is unchanged. 
All trial parameters are placed into array $ <$r[]$> $ at 
entries 201, 202, ... 
The values of generation, member, children and number 
of parameters are placed into $ <$i[]$> $: 
i[201] = generation 
i[202] = member 
i[203] = children 
i[204] = parameter 
\par
Use an 'exit' to return to the suite. 
\section{discus}
\par
Switches to the "discus" section. 
\par
Within this section any standard DISCUS command can be 
given. The behaviour of "discus" is essentially the same 
as in the stand alone version. 
\par
Within the discus section you can use the command 
'branch kuplot' to switch to the kuplot branch. 
\par
In contrast to the stand alone DISCUS version, one can 
write an output file directly into the KUPLOT data sets. 
The number of data sets in KUPLOT is atumatically incremented. 
Currently this is implemented for the PDF and the powder 
output. Single crystal diffraction pattern to follow 
shortly. 
\section{kuplot}
\par
Switches to the "kuplot" section. 
\par
Within this section any standard KUPLOT command can be 
given. The behaviour of "kuplot" is essentially the same 
as in the stand alone version. 
\par
Within the kuplot section you can use the command 
'branch discus' to switch to the discus branch. 
\section{parallel}
{\bf parallel $ \{$$ <$numprocs$> $, $\} $,$ <$macro.mac$> $ $ \{$, $ <$para\_1...$\} $ \par }
\par
\vspace{3pt}
Starts an MPI driven parallel calculation. See the diffev help 
on a full explanation of parallel processing. 
\par
The parallel refinement will execute file $ <$macro.mac$> $, which 
must reside in the current directory. Make sure you have used 
cd $ <$path$> $ to change to the proper directory prior to the use 
of the 'parallel' command. The macro name must be given in 
full, including the ".mac" extension. If the macro requires 
parameters you must specify these following the macro name. 
\par
Optionally you can place the number or processes that MPI shall 
start prior tot he macro name. The numebr defaults to the value 
of the SHELL variable NUMBER\_OF\_PROCESSORS on your system. If this 
variable is not set, discus\_suite will start 4 processes. 
\par
\section{manual}
{\bf manual ["section:"$ \{$"suite"  $| $ "discus"  $| $ "diffev" $| $ \par }
{\bf                    "kuplot" $| $ "package" $| $ "mixscat"$\} $ \par }
{\bf        [,"viewer:"$ <$name$> $] \par }
\par
\vspace{3pt}
Opens a PDF viewer for one of the Manuals 
\par
The section defaults to the current program section that you are 
working with. 
On Linux systems, the viewer defaults to "qpdfview", on Windows 
system it defaults to "firefox". If DISCUS does not find the 
default or the user provided viewer, DISCUS will search 
a list of common PDF viewers. If none is found an error message 
points to the folder that contains the manuals. 
