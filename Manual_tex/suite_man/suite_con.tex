%------------------------------------------------------------------------
% Chapter:  Concepts
%------------------------------------------------------------------------

\chapter{Concepts \label{concepts}}
\section{Quick overview } 

The purpose of the \Suite is to run a combination of the sections: 
\href{./diffev\_man.pdf}\diffev, 
\href{./discus\_man.pdf}\discus,
\href{./refine\_man.pdf}\refine, 
and 
\href{./kuplot\_man.pdf}\Kuplot within a single program. The \Suite 
will be the initial section upon program start. This level serves
as a simple rooftop to the individual sections, which do the main
work. 

%It main application  will be a 
%run on a massive parallel system, but it will already pose an 
%improvement on a multi-core laptop. \Suite uses MPI to distribute 
%a part of its workload onto several processes. The original \Diffev
%starts \Discus as a separate slave program. This means that at each refinement
%cycle \Discus has to read all macros, and other relevant information over and 
%over again. This causes a lot of unnecessary I/O, which puts a considerable
%burden onto the central disk system.

%As \Suite includes \diffev, \discus, \Refine and \Kuplot within one single program,
%all relevant input can be read once at the beginning of the refinement cycle.
%It will from there on be kept in internal memory, thus eliminating the need of
%further input. This includes the trial parameters as well as the result values.
%Within \suite, the need to write these files onto the disk no longer exists and 
%\Suite instructs \Diffev to store these values only internally.  

As the main tasks are still performed within the sections \diffev, \discus,
\refine and \kuplot, the actual set of commands within \Suite is very limited. 
Besides the general command of the common command language the commands are
just the names of the three sections, Table \ref{cmd-tab}.

\begin{table}[!b]
\centering
\begin{tabularx}{\textwidth}{|p{30mm}|X|}
  \hline
  {\bf Command } & {\bf Description} \\
  \hline
  diffev & Switch to the {\tt diffev} section. in many cases this
           will be the only command used within the top menu of
           the suite.\\
  \hline
  discus & Switch to the {\tt discus} section.\\
  \hline
  refine & Switch to the {\tt refine} section.\\
  \hline
  kuplot & Switch to the {\tt kuplot} section.\\
  \hline
  parallel & Run a parallel refinement. \\
  \hline
\end{tabularx}
\caption[\Suite command verbs ]
        {\label{cmd-tab}\Suite command verbs.}
\end{table}

Within the four sections, a few details are slightly different compared to 
their stand alone versions. The main command syntax and typical command 
sequences are, however, identical. As \Suite is a single program, the 
global variables {\tt i[], r[], res[]} are identical throughout all 
sections. The same holds for user defined variables. Thus, for example,
if {\tt i[0]} is set to one value in any section, all other sections will
see this new value. Be careful when you combine old \Discus and \Kuplot 
macros. Their local variables my interfere with each other. 

Besides the three variables {\tt i[], r[], res[]} and the user defined
variables, each of the four section includes specific variables that are
valid and meaningful within this section only. These variables 
specific to any of the sections are visible within that section only.

\section{Modified concepts}

In order to fully utilize the common program architecture, a few details
should be used in a modified way within the different sections. These details
mostly concern disk I/O related parts. 


\subsection{\diffev}

The stand alone \Diffev used to communicate with \Discus and \Kuplot via (small) 
files that contain the trial parameters and the cost function results.
To avoid writing these to disk use:
\begin{MacVerbatim}
  init silent
\end{MacVerbatim}
The new {\tt silent} parameter tells the \Diffev section with the \Suite 
not to write an initial set of
trial parameters to disk. These values are automatically transferred to the 
slave sections.

Closely related is the modified command:
\begin{MacVerbatim}
  trialfile silent
\end{MacVerbatim}
which likewise tells \Diffev not to write trial files during the refinement
cycles.

In both cases the \Suite will use the internal variables 
to communicate the trial parameters to the slave sections. 
The trial parameters are set up within 
\href{./diffev\_man.pdf}\Diffev as variable names and can 
be referenced with these names within the other sections.
Likewise, the information on the current generation is transferred with 
special variable names. See the 
\href{./diffev\_man.pdf}{\Diffev}  
manual for full details.

As an example, to define the a and c lattice parameters for a 
Wurtzite type structure use the commands:
\begin{MacVerbatim}
newpara P_lata,  3.85, 3.95, 3.88, 3.92
newpara P_latc,  6.45, 6.55, 6.48, 6.52
\end{MacVerbatim}

See the chapter Example in the \Diffev manual for a fully documented
example. The remainder of this section is obsolete for versions 5.22 and 
later.

For backwards compatibility the variables {\tt ref\_para[entry\_number]}
are retained.
  Entries in this array
range from 1 to the number of parameters to be refined, as defined
by {tt pop\_dimx} in \diffev.
As of Versions 5.7 the data are still written to the array, {\tt r} as 
well. As a deliberate break compared
to the general flexibility in all \Suite sections, the trial values are 
stored in entries starting at {\tt r[201]}. Use of these entries is meant 
to preserve backwards compatibility but will be discontinued in 
future releases.

Another predefined section of the internal variables is the range:\\
\begin{tabularx}{\textwidth}{|p{30mm}|X|}
  \hline
  {\bf Entry } & {\bf Description} \\
  \hline
  {\tt i[201]} & Current generation number \\
  \hline
  {\tt i[202]} & Size of the population \\
  \hline
  {\tt i[203]} & Size of the child population \\
  \hline
  {\tt i[204]} & Number of parameters \\
  \hline
\end{tabularx}
{ } \\
The use of this predefined  section of the internal variables 
is discouraged and has been replaced by the variables:\\
\begin{tabularx}{\textwidth}{|p{35mm}|X|}
  \hline
  {\bf Entry } & {\bf Description} \\
  \hline
  {\tt REF\_GENERATION} & Current generation number \\
  \hline
  {\tt REF\_MEMBER} & Size of the population \\
  \hline
  {\tt REF\_CHILDREN} & Size of the child population \\
  \hline
  {\tt REF\_DIMENSION} & Number of parameters \\
  \hline
  {\tt REF\_KID} & Current child to work on \\
  \hline
  {\tt REF\_INDIV} & Current individual repetition of the current child \\
  \hline
\end{tabularx}
{ } \\
To retain the backwards compatibility, the actual child number and the number 
of the individual repetition are transferred as parameters on 
the {\tt run\_mpi} command line.

In the \Suite version of \kuplot, a new internal variable is used to store
and transfer the weighted R-value from \Kuplot to \diffev. Thus, the need
to write and read the result file ceases to exist. Do this via the commands:

\begin{MacVerbatim}
resultfile  silent
...                ! further initialization
run_mpi ...        ! Calculate with Discus and Kuplot
compare silent
\end{MacVerbatim}

The first and last lines instruct the \Diffev section not to read the 
resultfile but to use the value that is transferred internally.

MPI has been implemented in the current Windows installation as well
starting with Version 5.7.
Use the icon {\tt DISCUS\_Suite Parallel\_Version} for parallel 
refinements. 

Even in the single session version of  the \suite, the refinement
should use the {\tt run\_mpi} command in an identical fashion.

In order to use the suite efficiently, the 
{\tt branch} command is available within \diffev as well, and you
can branch to either \discus or \Kuplot. This command should replace the 
{\tt system discus < discus\_macro\_name} construction that is used in a
serial refinement set up:


\begin{MacVerbatim}
   1  diffev
   2     ... SETUP ...
   3     init silent
   4     branch discus
   5        @discus_macro.mac
   6        exit
   7     branch kuplot
   8        @kuplot_macro.mac
   9        exit
  10     compare silent
  11  exit
\end{MacVerbatim}

In this simplified example, we switch to the diffev section with 
the {\tt diffev} command in line 1. After initial setup done by 
\Diffev and the initialization of generation zero in line 3, 
the \Diffev section branches off to the \Discus section in line 4.
Now control is at \Discus and the loop over all children and all
calculations are performed by the macro {\tt diffev\_macro.mac} in 
line 5. The {\tt exit} command in line 5 returns to the \Diffev section. 
If you use a previous \Discus macro, the {\tt exit} line might 
be part of the \Discus macro! Lines 7 to 9 do the same for 
\kuplot. The {\tt compare} command (line 10) at the \Diffev level does 
the comparison between the generations, and finally we leave the 
\Diffev section in line 11. In a full refinement there would be a loop 
over all refinement cycles that includes lines 4 to 10 of this example.

\subsection{\discus}

The performance of \Discus is not directly affected by the \suite. To make 
full use of the \suite, a couple of concepts should be adhered to.

Use the trial parameter value through the variable names that are defined
within \Diffev. 
Use the variables
REF\_GENERATION, REF\_MEMBER, REF\_CHILDREN, REF\_KID, REF\_INDIV 
to obtain the current
generation number, the size of the population, the number of children 
and the number of trial parameters.  Do not open and read the file GENERATION.
This file is created within each refinement cycle, and has been augmented 
by a lot of information that \Diffev reads to continue an interrupted refinement.
Withinhte sections \Discus and \Kuplot there is, however, 
no longer any need to read it.


%{\it Read} the trial values from the real valued array {\tt ref\_para[]}, where 
%the trial values are stored in elements 1 and following. 
%Use the variables
%REF\_GENERATION, REF\_MEMBER, REF\_CHILDREN, REF\_KID, REF\_INDIV 
%to obtain the current
%generation number, the size of the population, the number of children 
%and the number of trial parameters.  Do not open and read the file GENERATION.
%This file is still created within each refinement cycle, there is, however, 
%no longer any need to read it.

If the initial asymmetric unit is not extremely long, it is best to build
the initial asymmetric unit from scratch. To build a Wurtzite type asymmetric
unit use the following command sequence:

\begin{MacVerbatim}
1   read
2      free P_lata, P_lata, P_latc, 90.00, 90.00, 120.00, P63mc
3   insert Zn, 1./3.,  2./3, P_zn_z, P-biso
4   insert S , 1./3.,  2./3, 0.00  , P-biso
5   save
6      outf internal.wurtzite.cell
7      omit all
8      write scat
9      write adp
10     run
11  exit
12  read
13     cell internal.wurtzite.cell, 5,5,5
\end{MacVerbatim}

In this example it is assumed that P\_lata      contains the a lattice parameter 
value, P\_latc      the c lattice parameter, P\_zn\_z     the z-position of Zn 
and P\_biso     
a common atomic displacement parameter B. The last parameter in line 2 tells
\Discus that the structure belongs to space group P63mc. The asymmetric unit
is stored internally into computer memory  in lines 5 to 11, and then expanded
into a block of 5x5x5 unit cells in lines 12 and 13. By providing the space
group symbol as last parameter to the {\tt free} command, the expansion in line
13 will ensure that the two atoms Zn and O are properly copied to create all
atoms in the unit cell. This macro section does not perform any disk I/O and
will be much more efficient on a large parallel system.

Do use the internal storage whenever you need to save a temporary crystal 
structure. Every time a file name is preceded with the string {\tt internal},
\Discus will save / read the structure to / from internal memory.

To perform any further manipulations with the calculated data like 
multiplication by a scale factor, the addition of a background etc., \Kuplot
needs to be executed once the \Discus section is finished. To allow
efficient use of massive parallel systems the new {\tt branch} command has 
been added to \Discus and \kuplot. It allows to switch directly from the
\Discus section to the \Kuplot section within \suite. 

\begin{MacVerbatim}
0   ...                        ! Discus initialization
1   do indiv=1,nindiv          ! Repeat nanoparticle generation nindiv times
2      @discus.zno.mac kid     ! do the actual simulation
3   enddo
4   branch   kuplot            ! Switch to KUPLOT
5      @kup.diffev.mac ., kid  ! Contains a final exit
6   exit                       ! Finish DISCUS
\end{MacVerbatim}

In this \Discus macro section the simulation of nanoparticles is repeated
several (nindiv) times. This might be necessary to average out the defect
distribution within the nanoparticles. Under these circumstances the individual
calculated diffraction pattern / PDFs need to be averaged after the \Discus 
section is finished. The branch to the \Kuplot section in line 4 allows
\Discus to do this while remaining on the identical slave process and thus 
on the identical compute node. On many massive parallel systems data transfer 
from a compute node to the central disk is time consuming and many of these 
systems have local fast (RAM) disks. The \Discus section can write the 
individual diffraction pattern / PDF to this local disk without much of a
penalty. But then, \Kuplot needs to operate on the same node in order to 
find the data. This is ensured by the {\tt branch} construct. 

In the alternative set up :
\begin{MacVerbatim}
0   ...                         ! Diffev initialization
1   run\_mpi discus, discus\_macro.mac  " Discus calculation
2   run\_mpi kuplot, kuplot\_macro.mac  ! Kuplot calculation
\end{MacVerbatim}

\Diffev will distribute the \Discus jobs onto all nodes, and then do 
the same with the \Kuplot jobs. As of version 5.18 the \Suite will ensure
taht 
the two jobs for a given child are performed on the identical node. 

\subsection{\kuplot}

In the same fashion as \Discus, the \Kuplot section should rely 
on the variables {\tt ref\_para[]}
REF\_GENERATION REF\_MEMBER, REF\_CHILDREN, REF\_KID, ref\_INDIV 
to determine the population size and if necessary the 
trial parameters. Do not read the GENERATION file.

Do not write a result file. The \Kuplot instruction:
\begin{MacVerbatim}
rvalue 1, 2
\end{MacVerbatim}

stores the weighted R-value internally and instructs the \Suite to
transfer the value from the slave process back to the master process.
The master process can then use this in the {\tt compare silent} command.

The {\tt branch} command is available within \Kuplot as well, and you
can branch to either \Diffev or \Discus. Most application will probably 
not need to do this.
