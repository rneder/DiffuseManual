%------------------------------------------------------------------------
% Chapter:  Introduction
%------------------------------------------------------------------------

\chapter{Introduction \label{intro}}
\section{What is DISCUS\_SUITE ?}

The \Suite is a combination of the components of the DISCUS program 
suite. It includes the sections alone programs 
\href{./discus\_man.pdf}\discus, 
\href{./diffev\_man.pdf}\diffev, 
\href{./refine\_man.pdf}\refine, and
\href{./kuplot\_man.pdf}\Kuplot under a common  command language. 
The command language is documented in the manual 
\href{./package\_man.pdf}{package\_man}.\\
In this combination, the 
\Suite allows you to seamlessly switch between the individual
sections. Its main advantage lies in the field of massive parallel 
computing. As the \Suite includes the full functionality of 
\diffev, the \Suite allows you to perform parallel refinements.
In contrast to a stand alone \diffev, the \Suite does have to
perform nearly no I/O during the refinement.

See the \Diffev and \Refine manuals for full details of the 
refinement process.

%------------------------------------------------------------------------

\section{Getting started \label{intro-get}}

\subsection{Windows}

Once the \Suite has been installed you should see an icon called
{\tt Discus\_SUITE\_WSL\ }
on your personal desktop. Double click the icon to start
the program of your choice.

At program start, each of the programs sets its starting folder to\\
\begin{MacVerbatim}
   C:\Users\your_name
\end{MacVerbatim}
where {\tt your\_name} will be your user name. The string {\tt Users}
may be slightly different depending on the language settings.
To work with data and macros that are stored in a separate folder,
you need to change to this folder.

Within the \Suite program window, type the command {\tt cd } including a
blank space after the {\tt cd}. At this point do not hit the enter key.
Open the desired folder with the Windows-Explorer. Left click on the
small folder symbol in the top line that indicates the path to your
folder. This should highlight the full path to the folder.

Select the highlighted path to the folder
with CTRL-c. Activate the \Suite window and press the
middle button on the mouse. This should place the full path into the
program window. As of version 5.25 you can also press CTRL-SHIFT-v to
insert the directory path. Incidentally this holds for any string that had
been copied into the clipboard with CTRL-c. 

Once you activate the window and hit the ENTER key the program will
work in this folder.

\subsection{Linux}


After the program {\it DISCUS\_SUITE} has been installed properly and the
environment variables are set, the program can be started by typing
'discus\_suite' at the operating systems prompt.

\subsection{MAC}

After the program {\it DISCUS\_SUITE} has been installed properly and the
environment variables are set, the program can be started by first 
starting a terminal window and typing
'discus\_suite' at the operating systems prompt.

The program uses a command language to interact with the user.  The
command {\tt exit} terminates the program and returns control to the
shell.  All commands of \Suite consist of a command verb,
optionally followed by one or more parameters.  All parameters must
be separated from one another by a comma ",".  There is no
predefined need for any specific sequence of commands.  \Suite     
is case sensitive, all commands and alphabetic parameters MUST be
typed in lower case letters.  If \Suite has been compiled
using the {\tt -DREADLINE} option (see installation files) basic
line editing and recall of commands is possible.  For more
information refer to the reference manual or check the online help
using ({\tt help command input}).  Names of input or output files
are to be typed as they will be expected by the shell.  If necessary
include a path to the file.  All commands may be abbreviated to the
shortest unique possibility. At least a single space is needed
between the command verb and the first parameter.  No comma is to
precede the first parameter. A line can be marked as comment by
inserting a "{\tt \#}" as first character in the line.\par

The symbols used throughout this manual to describe commands,
command parameters, or explicit text used by the program \Suite     
are listed in Table \ref{sym-tab}. There are several sources
of information, first \Suite  has a build in online help, which
can be accessed by entering the command {\tt help} or if help for a
particular command $<$cmd$>$ is wanted by {\tt help $<$cmd$>$}. This
manual describes background and principle functions of \Suite
and should give some insight in the ways to use this program. \par

\subsection{Overview}

\begin{table}[!tbh]
\centering
\begin{tabularx}{\textwidth}{|p{30mm}|X|}
  \hline
  {\bf Symbol} & {\bf Description} \\
  \hline\hline
  "text"     &  Text given in double quotes is to be understood as typed. \\
  \hline
  $<$text$>$ &  Text given in angled brackets is to be replaced by an
                appropriate value, if the corresponding line is used
                in \suite. It could, for example be the actual name
                of a file, or a numerical value. \\
  \hline
  {\tt text} &  Text in single quotes exclusively refers to \Suite
                commands. \\
  \hline
  $[$text$]$ &  Text in square brackets describes an optional parameter or
                command. If omitted, a default value is used, else
                the complete text given in the square brackets is to
                be typed. \\
  \hline
  \{text $|$ text\} &  Text given in curly brackets is a list of alternative
                parameters. A vertical line separates two alternative,
                mutually exclusive parameters. \\
  \hline
\end{tabularx}
\caption{\label{sym-tab}Used symbols}
\end{table}


\subsubsection{Command line options \label{intro-cmd}}
Several command line options are available for the Linux  and
MAC version.
The most important one is to start the execution of a macro:

\begin{MacVerbatim}
discus_suite -macro useful.mac
discus_suite -macro useful_with_params.mac  par1 par2 par3
\end{MacVerbatim}

The first line would start \Suite and begin the automatic execution
of macro {\tt useful.mac}. Likewise in the second line, the macro
{\tt useful\_with\_params.mac} is started, which takes three parameters. 
The actual strings that
you provide for par1, par2, par3 are handed down as parameters to
this macro. Note that there are no comma between the parameters.

For info on further command line options the the on-line help for
the Command language.

\section{Parallel execution \label{intro-par}}

The typical application of the \Suite is a run on a massive parallel 
system. You will have to check for details with your local administrator.
In general you should expect to start \Suite along the lines of:

\begin{MacVerbatim}
mpiexec -n 192 discus_suite -macro refine.mac > /dev/null
\end{MacVerbatim}

The command {\tt mpiexec} initiates the MPI system. Here in this example
we request 192 CPUs {\tt -n 192}. The MPI system then starts the \Suite
and read its input from the macro {\tt refine.mac}. All output is discarded
 to {\tt /dev/null} in order to minimize I/O. During development of your 
macro you might redirect the output to a logfile.

The example line above works fine on a single multi-core computer with 
Linux. The number of processes would be more reasonably placed around 
the number of CPUs on your computer.

Within the Windows version, parallel execution of such a macro is 
performed by \Suite command {\it parallel}:

\begin{MacVerbatim}
parallel refine.mac 
parallel 6, refine.mac 
\end{MacVerbatim}

The first command version instructs \Suite to run a parallel 
refinement via the macro {\it suite.mac}. The number of parallel 
threads is determined automatically. Alternatively you can tell 
\Suite how many parallel threads you would like to run by adding the
number of threads prior to the macro name. In the example above 
\Suite is instructed to use 6 threads.


%------------------------------------------------------------------------

\section{Command language}

The program includes a FORTRAN style interpreter that allows the
user to program complex modifications. A detailed discussion about the 
command language which is common to all \Discus package programs can be 
found in the separate \Discus package reference guide which is included with 
the package. At the moment, \Suite does not include any specific
variables.

