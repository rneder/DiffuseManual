%------------------------------------------------------------------------
% Chapter:  Least Squares refinement 
%------------------------------------------------------------------------

\chapter{Least Squares Refinement \label{lsq}}
\section{Refinement via least squares refinement \label{lsq-algo}}

Every time we measure some physical effect and wish to understand
how this effect works, we want to determine the parameters of a 
model function that will replicate the observations. The term 
refinement refers to the process by which the parameters of the
function are tuned such as to give the best agreement between
the observed and calculated values. The term {\it best agreement}
merits careful definition, for right now it is sufficient to say
that the sum over all squared differences between the observations and the 
calculations shall be minimized. Thus, refinement is but a special
case of general optimization. A very different example for an 
optimization could be the task to place as many integrated 
circuits into a chip and simultaneously achieve the fastest computations.
Quite well known is the traveling salesman problem. Here the 
optimization task requires to find the shortest route that 
visits a number of spots distributed in space.  

By far the fastest refinement technique is a least squares algorithm.
Such an algorithm can always be applied if we can describe the 
physical effect as a function of parameters:
\begin{equation}
  y ~=~ F(p_{0}, p_{1}, ..., p_{n}),
\end{equation}
and all the partial derivatives $\partial y/ \partial p_{i}$ can 
be calculated, either analytically or numerically. For each 
observed value y$_{obs}$, we calculate a value y$_{calc}$ and 
minimize the value of a weighted residual wR:
\begin{equation}
   wR = \sqrt{\frac{\sum_{i} w_{i} (y_{obs}(i) - y_{calc}(i))^2}
                   {\sum_{i} w_{i} y_{obs}(i)^2}}
  \label{diff-eq-rval}
\end{equation}
Here each difference is multiplied by a weight w$_{i}$ that reflects
the uncertainties of the experimental values. In case of crystal 
structure analysis, the observed values would be the observed 
intensities in a diffraction pattern and the calculated values 
those intensities that were calculated based on a structural model. 
Model parameters will be the lattice parameters, the positions
of the atoms in the unit cell, atomic displacement parameters etc.
as well as experimental parameters, such as the background.
Under the assumption that we have a periodic crystal, the 
partial derivatives of the intensity with respect to lattice
parameters, atom positions etc., can all be derived analytically.
This is the concept you will find within any single crystal structure
refinement program or a Rietveld program.

For disordered structures, the situation becomes more complicated.
Except for a few special cases like stacking faults or short-range
order problems, no general analytical function straightforwardly 
links the disorder parameter to the intensity. The intensity 
can still be calculated from structural models. The simulation,
however, usually involves the application of random choices 
to generate part or all of the atom positions, and the analytical
derivative of the intensity with respect to the order parameter
is no longer available. A numeric calculation of the derivatives
involves the repeated simulation of a new model for each parameter
and is fairly time consuming.

\section{Algorithm in the \Refine section \label{lsq-refine}}

The \Refine section uses a Least-Squares algorithm based on the software 
in the Numerical Recipes \cite{prflteve1989}. This software is based on the 
Levenberg-Marquart
algorithm. In any least-squares algorithm the derivatives are used to determine
a new estimate for each parameter. While the refinement is far from the final 
solution, each parameter can be modified by a large step in order to quickly 
approach the global minimum. Close to the minimum the steps need to be smaller
not to miss the minimum. Essentially the step size is roughly propertional to 
the derivative. The Levenberg-Marquardt algorithm, optimizes the steps to be
taken.

Within the \Refine section of the \Suite the derivatives are calculated
numerically. The program runs the simulation several times for each parameter:
at the current parameter value and at slightly larger and slightly smaller 
parameter values. The resulting R-values at each parameter value are analyzed
to calculate the derivative for this parameter. The \Refine section calculates
the R-value for a given parameter at:

\begin{tabular}{l|l}
  P & R-value(P) \\
  P+h & R-value(P+h) \\
  P-h & R-value(P-h) \\
  P+2h & R-value(P+2h) \\
  P-2h & R-value(P-2h) \\
\end{tabular}

The \Refine section calculates a polynomial of order two through these three 
points and uses this polynomial to derive the value of the derivative a the
current value of the Parameter {\tt P}. 

In any numerical determination of the derivative it is not straightforward 
to know what is the best value of the deviation {\tt h} from the current 
parameter value {\tt P}. If the function whose derivative we need to obtain
is a smooth function, a very small value of {\tt h} is best, as this is 
likely to yield a good approximation to the analytical derivative. The 
value {\tt h} must, however, be large enough to produce a R-value that 
differs significantly from the original R-value and whose calculation 
is not affected by rounding errors that are unavoidable for very small 
numerical values. Within the \Refine section the value of the absolute 
shift {\tt h} of the parameter {\tt P} is calculated as
{\tt P * shift}, where the value of {\tt shift} defaults to 0.002. 

As \Refine does not know the significance behind a user supplied parameter
the {\bf newparam} command comes with the optional parameter {\bf shift:}
that lets you set the relative parameter shift for each individual parameter.

A special difficulty is encountered for refined parameters that cause a
stepwise change in the simulated structure. The diameter of a (small) 
nanoparticle falls into this category. The number of atoms inside a
small nanoparticle is a discrete integer number. Increasing the 
formally real valued diameter results in no structural change until the 
diameter is big enough to add one or more atoms. If such a parameter
is modified by a small fraction or by a small absolute amount, no 
change in the structure might be encountered and the numerical derivative 
would appear to be zero. Only for a larger shift of the parameter will 
a significant structureal modification occur, which will in turn affect 
the R-value. A good value for diameters seems to be around 0.05, i.e.
the current diameter is modified by {\tt h= P*0.05}.

A second aspect to consider while using a Least-Squares algorithm that 
is based on the numerical calculation of the derivatives is the presence
of local minima.

A refinement within \Refine typically consists of two user supplied macro. In 
the main macro the refinement parameters, the input data and convergence 
criteria are defined. The actual calculation of the model function, respectively
the model structure and its diffraction pattern, are carried out by the second
macro.

The \Refine section uses four parameters to determine is the refinement has 
reached convergence. These are based on the absolute value of $\chi^2$, and
on the change of $\chi^2$, the relative parameter change and a confidence level.

Convergence is considered to have been reached if either the absolute value
of $\chi^2$ falls below a user defined leve, or if all three other criteria
are met. These three criteria are:
\begin{itemize}
  \item The value of $\chi^2$ changes less than a user defined value between 
        two cycles
  \item The largest relative parameter shift is less than a user defined value.
        The relative parameter shift is defined as the abolute value of the
        parameter change between two cycles divided by the parameter uncertainty.
  \item The confidence level reaches a user defined level.
\end{itemize}

The correct values of $\chi^2$, of the parameter uncertainties and the 
confidence value depend on the correct values of the data uncertainties. Often
the raw data do not provide an accurate estimate of the data uncertainties. In
this case the best strategy to start with is to assign unit wheights to all data 
points. As this implies that the optimum $\chi^2$ and the confidence level will
be unknown, \Refine also requires the user to set a maximum number of refinement 
cycles. 

The \Refine section defines the parameters that shall be refined as \Suite 
variable names. These variable names and their current values are passed on to
the user macro that is used to simulate the model structure. To tune the 
refinement, an initial value, a valid parameter range, and a status flag that 
indicates if this parameter is to be refined or to be kept fixed can be provided.
 
%------------------------------------------------------------------------
