%------------------------------------------------------------------------
% Chapter:  Least Squares refinement 
%------------------------------------------------------------------------

\chapter{Least Squares Refinement \label{lsq}}
\section{Refinement via least squares refinement \label{lsq-algo}}

Every time we measure some physical effect and wish to understand
how this effect works, we want to determine the parameters of a 
model function that will replicate the observations. The term 
refinement refers to the process by which the parameters of the
function are tuned such as to give the best agreement between
the observed and calculated values. The term {\it best agreement}
merits careful definition, for right now it is sufficient to say
that the sum over all squared differences between the observations and the 
calculations shall be minimized. Thus, refinement is but a special
case of general optimization. A very different example for an 
optimization could be the task to place as many integrated 
circuits into a chip and simultaneously achieve the fastest computations.
Quite well known is the traveling salesman problem. Here the 
optimization task requires to find the shortest route that 
visits a number of spots distributed in space.  

By far the fastest refinement technique is a least squares algorithm.
Such an algorithm can always be applied if we can describe the 
physical effect as a function of parameters:
\begin{equation}
  y ~=~ F(p_{0}, p_{1}, ..., p_{n}),
\end{equation}
and all the partial derivatives $\partial y/ \partial p_{i}$ can 
be calculated, either analytically or numerically. For each 
observed value y$_{obs}$, we calculate a value y$_{calc}$ and 
minimize the value of a weighted residual wR:
\begin{equation}
   wR = \sqrt{\frac{\sum_{i} w_{i} (y_{obs}(i) - y_{calc}(i))^2}
                   {\sum_{i} w_{i} y_{obs}(i)^2}}
  \label{diff-eq-rval}
\end{equation}
Here each difference is multiplied by a weight w$_{i}$ that reflects
the uncertainties of the experimental values. In case of crystal 
structure analysis, the observed values would be the observed 
intensities in a diffraction pattern and the calculated values 
those intensities that were calculated based on a structural model. 
Model parameters will be the lattice parameters, the positions
of the atoms in the unit cell, atomic displacement parameters etc.
as well as experimental parameters, such as the background.
Under the assumption that we have a periodic crystal, the 
partial derivatives of the intensity with respect to lattice
parameters, atom positions etc., can all be derived analytically.

For disordered structures, the situation becomes more complicated.
Except for a few special cases like stacking faults or short-range
order problems, no general analytical function straightforwardly 
links the disorder parameter to the intensity. The intensity 
can still be calculated from structural models. The simulation,
however, usually involves the application of random choices 
to generate part or all of the atom positions, and the analytical
derivative of the intensity with respect to the order parameter
is no longer available. A numeric calculation of the derivatives
involves the repeated simulation of a new model for each parameter
and is fairly time consuming.

%------------------------------------------------------------------------
