%------------------------------------------------------------------------
% Chapter:  Introduction
%------------------------------------------------------------------------

\chapter{Introduction \label{intro}}
\section{What is REFINE ?}

\Refine is the direct Least Squares Refinement section of the
\Suite program. Its internal engine is a Levenberg-Marquardt
type least squares fit that can be applied to 1D or 2D data.

See the \Diffev section on a more general 
generic evolutionary refinement program that implements the
differential evolutionary algorithm \cite{prstla2005}. Evolutionary or
genetic refinement algorithms allow the refinement of models, functions,
or more generally speaking the parameters of a cost function to obtain
a good solution. 

A least squares based refinement of a function 
$y = F(p_{0}, p_{1}, ..., p_{n})$ requires the calculation of all
partial derivatives $\partial y/ \partial p_{i}$, either from an 
analytical or a numeric solution. 
The refinement that is described in this section is intended to
work with disordered crystal structures as build by \Discus. 
To achieve this \Refine uses a macro that is provided by the user
to build a crystal and to calculate a diffraction pattern or a
PDF.  As the details of a calculation are hidden within the 
source code of the \Discus and \Kuplot section, \Refine
generally relies on a numerical calculation of the derivative. 

Since this macro could calculate any cost function, 
\Refine is not limited to the refinement of a particular physical
problem. 

%------------------------------------------------------------------------

\section{Getting started \label{intro-get}}

After the program {\it DISCUS\_SUITE} is installed properly and the
environment variables are set, the program can be started by typing
'discus\_suite' at the operating systems prompt.

\begin{table}[!tbh]
\centering
\begin{tabularx}{\textwidth}{|p{30mm}|X|}
  \hline
  {\bf Symbol} & {\bf Description} \\
  \hline\hline
  "text"     &  Text given in double quotes is to be understood as typed. \\
  \hline
  $<$text$>$ &  Text given in angled brackets is to be replaced by an
                appropriate value, if the corresponding line is used
                in \diffev. It could, for example be the actual name
                of a file, or a numerical value. \\
  \hline
  {\tt text} &  Text in single quotes exclusively refers to \Refine
                commands. \\
  \hline
  $[$text$]$ &  Text in square brackets describes an optional parameter or
                command. If omitted, a default value is used, else
                the complete text given in the square brackets is to
                be typed. \\
  \hline
  \{text $|$ text\} &  Text given in curly brackets is a list of alternative
                parameters. A vertical line separates two alternative,
                mutually exclusive parameters. \\
  \hline
\end{tabularx}
\caption{\label{sym-tab}Used symbols}
\end{table}

The section uses the identical command language to interact with the 
user as is used throughout the \Suite. The
command {\tt exit} terminates the section and returns control to the
top level of the \Suite.All commands of \Refine consist of a command verb,
optionally followed by one or more parameters.  All parameters must
be separated from one another by a comma ",".  There is no
predefined need for any specific sequence of commands.  \Refine     
is case sensitive, all commands and alphabetic parameters MUST be
typed in lower case letters.  If \Refine has been compiled
using the {\tt -DREADLINE} option (see installation files) basic
line editing and recall of commands is possible.  For more
information refer to the reference manual or check the online help
using ({\tt help command input}).  Names of input or output files
are to be typed as they will be expected by the shell.  If necessary
include a path to the file.  All commands may be abbreviated to the
shortest unique possibility. At least a single space is needed
between the command verb and the first parameter.  No comma is to
precede the first parameter. A line can be marked as comment by
inserting a "{\tt \#}" as first character in the line.\par

The symbols used throughout this manual to describe commands,
command parameters, or explicit text used by the program \Refine     
are listed in Table \ref{sym-tab}. There are several sources
of information, first \Refine  has a build in online help, which
can be accessed by entering the command {\tt help} or if help for a
particular command $<$cmd$>$ is wanted by {\tt help $<$cmd$>$}. This
manual describes background and principle functions of \Refine
and should give some insight in the ways to use this program. \par

\Refine is distributed as part of the diffuse scattering
simulation software \discus.  However, \Refine can be used
as general refinement program separate from the main purpose of 
the \Discus program package. 

%------------------------------------------------------------------------

\section{Command language}

%--\begin{table}[!tb]
%--\centering
%--\begin{tabularx}{\textwidth}{|p{30mm}|X|}
%--  \hline
%--  {\bf Variable} & {\bf Description} \\
%--  \hline \hline
%--  pop\_gen[1]    &  Current population number \\
%--  \hline 
%--  diff\_cr[1]     &  Cross over probability\\
%--  \hline \hline
%--  pop\_xmin[i]   &  Minimum allowed value for parameter no. i  \\
%--  \hline
%--\end{tabularx}
%--\caption[\Diffev structural variables]
%--        {\label{v1-tab}\Diffev variables. Variables marked
%--         with $^{*}$ are read-only and cannot be altered.}
%--\end{table}

The program includes a FORTRAN style interpreter that allows the
user to program complex modifications. A detailed discussion about the 
command language, which is common to all \Discus package programs can be 
found in the separate \Discus package reference guide which is included with 
the package. Currently there are no \Refine specific variables.

%--Table \ref{v1-tab} shows a summary of \Refine specific 
%--variables. Some of these variables cannot be modified, others like can be 
%--altered, thus allowing to modify the refinement strategy.
