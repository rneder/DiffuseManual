%------------------------------------------------------------------------
% Chapter:  Least Squares refinement  examples
%------------------------------------------------------------------------

\chapter{Example refinements \label{exa}}
\section{Simple noisy data function \label{exa-simple}}

In this first example the refinement to a simple function is
illustrated. The data were calculated as a simple polynomial,
see Fig. \ref{fexa-poly-exp}:

\begin{equation}
  y(x) = P_{const} + P_{lin}*x + P_{quad}*x^2 + P_{trip}*x^3
\end{equation}

A Gaussian distributed random error was added to each data point 
with a sigma equal to the absolute y-value of each data point.
If y was close to zero, the minimum sigma was set to 0.001.

\begin{figure}
   \includegraphics[angle=270,scale=0.45]{poly_data.epsi}
   \caption{Experimental polynomial function}
   \label{fexa-poly-exp}
\end{figure}

As y depends linearly on each of the parameters $P_{i}$, we can expect
that this refinement will run smoothly. Essentially arbitrary starting values
could be used to perform the refinement.

Two macro files are needed for the refinement, the main refinement macro and the
macro that caculates the theoretical function. The main refinement macro that
was used for this simple introductory example is:

\begin{MacVerbatim}
 1: refine
 2: data xy, DATA/triple.noise
 3: newparam, P_const, value:-8.01
 4: newparam, P_lin  , value:1.01
 5: newparam, P_quad , value:1.49
 6: newparam, P_trip , value:0.31
 7: set  cycle,   5
 8: set conv, dchi:0.50, pshift:0.005, conf:0.10, chisq:0.5
 9: set relax, start:0.02, success:0.5, fail:16.0
10: run triple.mac, plot:k_inter.mac
11: exit   ! back to SUITE
\end{MacVerbatim}

In line 1 we switch from the main \Suite level to the \Refine section. 
The data are loaded in line 2 as a simple "xy" file. All formats that
are supported by the \href{./kuplot\_man.pdf}\Kuplot {\tt load} command are 
available. \Refine stores the data set within its on memory. Thus no harm
is done if \Kuplot resets the data sets during its calculations. During 
a refinement in which you need to calculate and average multiple powder or 
PDF data sets such a reset migh be the norm rather than the exception.

Lines 3 to 6 define the four parameters we wish to refine. For 
each parameter a suitable name needs to be defined. This name has to be 
a variable name that is valid within the \suite. If the variable name does
not yet exist, it will be created. The command offers three optional 
parameters, only the "value" parameter is used in this example. it initiates
the parameter to the user supplied value, which can be a numerical expression.
As further optional parameter you can limit the range over which a parameter is
valid by the parameter "range:[<lower>,<upper>]. This will be helpful if you need
to exclude negative numbers or if the physical range of a parameter is limited. 
The last optional parameter allows to change the status of a parameter with
"status:fix" or "status:fixed" the parameter will be fixed. If a "value:" 
parameter is given, the parameter will be fixed to this value, otherwise it
takes its current value. If the optional parameter "status:" is omitted or if
its value is set to "refine" or "free", its value will be refined. 

Line 7 defines the maximum number of refinement cycles that the \Refine section
will perform. If the convergence criteria are met in an earlier cycle, the 
refinement will stop at that point. Several criteria are used to determine if 
the refinement has reached convergence. In this example they are all specified 
on line 8. The four criteria are:
\begin{itemize}
  \item chisq:<value>  Convergence has been reached if the value of chi square 
        drops below the user specified value.
  \item conf:<value>  If the confidence level is larger than this value
        and
  \item pshift:<value> the largest parameter shift defined as the absolute 
        value of the pareameter shift divided by the abolute value of the 
        parameter uncertainty
        and
  \item dchi:<value> the change in chi2 is less than this value then 
        convergence is assumed.
\end{itemize}
If either the {\tt chisq} or all three other criteria {\tt conf}, 
{\tt  pshift} and {\tt dchi} are met, onvergence is reached and the 
refinement stops. Keep in mind that the value of chisq and the parameter 
uncertainty wil  depend on the uncertainties of the input data values.

The Levenberg-Marquardt algorithm uses a parameter lamda to adjust the 
change of a parameter from one refinement cycle to the next. A small 
value results in a large parameter change and vice versa. \Refine will
dynamically adjust this parameter during the refinement. Starting with 
a value of {\tt 0.02} the value of lambda is decreased by multiplication with 
{\tt success:0.5} if the refinement succeeded i.e. chi2 decreased. The
underlying assumption is that as long as the algorithm decreases chi2, 
we can take larger parameter steps. Once the algorithm does not find a
better solution, its time to take smaller parameter steps in order to
find the minimum, and lambda is increased with {\tt fail:16.0}. The 
values in the xample are the default values. The main parameter you may have 
to play with is the start parameter. A larger value up to 2 seems to help
difficult refinements.
 
The "run" command on line 10 will start the refinement using the macro 
"triple.mac" to calculate the theory data set. The optional parameter
{\tt plot:} allows to specify a user supplied macro that will display
any data at each successful refinement step.

\begin{MacVerbatim}
 1: branch kuplot
 2: #
 3: if(F_DATA==0) then
 4:    reset
 5:    func (P_const)+(P_lin)*r[0]+(P_quad)*r[0]**2+(P_trip)*r[0]**3, F_XMIN, F_XMAX, F_XSTP
 6: else
 7:    func (P_const)+(P_lin)*r[0]+(P_quad)*r[0]**2+(P_trip)*r[0]**3, F_DATA
 8: endif
 9: exit
10: finished
\end{MacVerbatim}

The refinement begins within the \Refine section of the \suite. Since we want to 
calculate a simple x-y data set, the \Kuplot section is the best choice. With the
{\tt branch kuplot} command in line 1 \Refine switches to \kuplot.  In line 3 the 
macro tests if \Refine loaded the data into a \Kuplot data set with data set number
{\tt F\_DATA}. The \Refine build in variable {\tt F\_DATA} defines the data set number 
within \Kuplot into which \Refine was able to load the data set with the {\tt data} 
command, line 2 in the main fit macro. The actual limits and the step width 
that \Refine detects are
stored in the \Refine variables {\tt F\_XMIN}, {\tt F\_XMAX} and {\tt F\_XSTP}. 
Thes variables are all Read/Write but change their values with caution.

In \Kuplot the {\tt func} command will calculate a user supplied function over a
user defined range. The first parameter here the string 
"(P\_const)+(P\_lin)*r[0]+(P\_quad)*r[0]**2+(P\_trip)*r[0]**3" defines the 
function to be calculated. If three parameter follow, they define x-min, xmax
and the x-step for the function. The \Suite variable r[0] takes the role of the
functions x-coordinate. If only one parameter is supplied, it defines the data
set number whose range and step width are to be used for the function calculation. 
In the current example macro lines 5 and 7 will thus calculate the theory function 
over the identical range. 

A reason to deveate from the limits of the data set occures when you reifne 
against a powder diffraction data set. In order to take the effect of Bragg 
reflections slightly above the upper 2Theta limit into account, it is necessary
to calculate the powder pattern to a larger 2Theta value. This will ensure that 
the low 2Theta tail of these Bragg reflections will b e present within the 
calculated powder patttern, even if the peak position of the Bragg reflections
is not.

The user macro needs to ensure that the last data set loaded into \Kuplot is 
the final theory curve. 

At line 9 the user macro returns to \Refine. Finally in line 10, the 
\Refine command {\tt finished} tells \Refine that the user macro is finished.

For the current macro the output produced during the refinement will be the
following lines:

\begin{MacVerbatim}
Cyc Chi^2/(N-P)   MAX(dP/sig)   Conf          Lambda       wRvalue      Rexp
  0   1.0222      0.26239      0.34233      0.50000E-03  0.25018      0.24745    
  1   1.0219      0.14187E-02  0.34404      0.25000E-03  0.25015      0.24745    

Convergence reached 

 Information about the fit : 
   Chi^2      :  814.484          Chi^2/N :  1.01683    
   Conf. level: 0.344038      Chi^2/(N-P) :  1.02194    
   No.Data    :          801     No.Params:            4
   MRQ final  : 0.250000E-03
   wR value   : 0.250150          R exp   : 0.247451    

 Correlations larger than 0.8 :
   ** none **
  
   P_const          :   -3.92623      +-    0.107205    
   P_lin            :    2.16236      +-    0.772162E-01
   P_quad           :   0.512809      +-    0.314855E-01
   P_trip           :   -1.11685      +-    0.134130E-01
\end{MacVerbatim}

The optional plot macro needs to start within the \Kuplot section and return
with a final exit to \refine. For the current example you could use:

\begin{MacVerbatim}
 1: load xy, DATA/triple.noise
 2: kccal sub, 2,1
 3: r[0] = min(ymin[1],ymin[2]) - ymax[3]
 4: ccal add, wy, 3, r[0]
 5: skal
 6: mark
 7: rval 2,1, dat
 8: reset
 9: exit
\end{MacVerbatim}


\section{Nanoparticle PDF \label{exa-nano}}

In this section we will introduce the refinement of a nanoparticle PDF. This
initial example is fairly straightforward. The data set is the calculated PDF
of a spherical ZnO 45 \AA diameter nanoparticle without any defects, Fig.
\ref{fexa-nano-exp}. Thus we can, of course expect a rather perfect fit result.

\begin{figure}
   \includegraphics[angle=270,scale=0.45]{nano_data.epsi}
   \caption{Experimental nanoparticle PDF. The nanoparticle has the perfect
   ZnO structure with a diameter of 45 \AA. The insert shows enlarged the
   longer distances.}
   \label{fexa-nano-exp}
\end{figure}

The main parts of the main refinement macro are identical to the example
in section \ref{exa-simple}. 

\begin{MacVerbatim}
 1: refine
 2: #
 3: data xy, DATA/zno.grobs
 4: newparam  P_lata   , value:3.220, range:[2.90,3.50], shift:0.0020 , points:5
 5: newparam  P_latc   , value:5.050, range:[4.70,5.30], shift:0.0020 , points:5
 6: newparam  P_z_zn   , value:0.360, range:[0.30,0.42], shift:0.0020 , points:5
 7: newparam  P_biso   , value:0.400, range:[0.00,]    , shift:0.0001 , points:5
 8: newparam  P_ab_dia , value:45.00, range:[20.0,80.0], shift:0.0100 , points:5
 9: #
10: newparam  P_eta    , value:0.600, range:[0.00,1.00], shift:0.0020 , points:5
11: newparam  P_eta_l  , value:0.000, status:fixed
12: newparam  P_u      , value:0.000, status:fixed
13: newparam  P_v      , value:0.000, status:fixed
14: newparam  P_w      , value:0.001, range:[0.00,0.10], shift:0.0020 , points:5
15: newparam  P_qmax   , value:24.00, status:fixed
16: newparam  P_scale  , value:1.000, range:[0.10,10.0], shift:0.0020 , points:5
17: set cycle,  25
18: set relax, start,2.00
19: set conver, dchi:0.5, pshift:0.005, conf:0.90, chisq:1.1
20: #
21: run discus_main.mac, plot:k_inter.mac
22: #
23: exit
\end{MacVerbatim}

As before, the data set is loaded as a simple x-y ASCII data set in line 3.
The parameters are defined in lines 4 to 16. The main difference is the use 
of the additional optional parameters {\tt range}, {\tt shift} and {\tt points}.

With {\tt range} a range of allowed parameter limits can be defined. You can 
use this to restrict the refinement range to a sensible range, limited by 
intuition or by physical constraints. The optional parameter takes two 
values, the lower and upper boundary. One of these two values may be omitted,
as in the case of {\tt P\_biso}. The lower limit is defined as 0.00, as a 
atomic displacement parameter cannot assume negative values. The upper limit
is absent, allowing arbitrarily large values. 

The next optional parameter {\tt shift} allows to specify a multiplicative 
term that is used to calculate parameter points at {\tt p-2*delta}, 
{\tt p-1*delta}, {\tt p-2*delta}, {\tt p-2*delta}. These additional points are 
needed to calculate a numerical derivative of the function value with respect
to the parameter at hand. An accurate derivative will be calculated at the 
current parameter value if the shifts {\tt delta} are small. In a crystal 
structure simulation there are, however, a number of parameter types that need
a rather large shift {\tt delta}. The most prominent example of this class are
nanoparticle diameters. If these are increased by a small amount, no atom might 
be added to the structure at all, as the boundary surface might cut in between 
the positions of all atoms. In the example a value of {\tt shift:0.01} works 
well.

The last optional parameter {\tt points} specifies how many points are  used to 
calculated the derivative. Possible values are 3 or 5, which include the 
parameter value itself. The larger number of points will provide a more
accurate derivative, but will of course require more computing time.

Another issue to be considered is the determination and transfer of fixed 
parameters from the main fit macro to the slave macro. Two different styles 
might be adopted. You can create variables and set their values within a 
separate macro, which you use at all appropriate macros. The other option,
chosen in this example, is to use the {\tt newparam} command but to fix the 
parameter value. It is mostly a matter of personal choice which style you 
prefer. As an argument in favor of the {\tt newparam} style, consider the
profile parameters {\tt P\_u}, {\tt P\_v}. It will often be necessary to 
refine all three profile shape parameters. It makes sense though to start
with the simplest model, a profile function of constant width, determined
by parameter {\tt P\_w}. If, later on, you decide that a full profile 
function is needed, all you have to do is to free parameters {\tt P\_u} and
{\tt P\_v}. For the style of a separate macro, you would have to remove 
the parameters from the variable defining macro, remove all initilization
lines in any macro and add new parameters to the main fit routine. This is
likely more error prone.

The main \Discus macro {\discus\_main.mac} has to build a spherical 
nanoparticle based on the current refinement parameters, calculate the
necessary diffraction data and provide these as final data set within the 
\Kuplot section.

\begin{MacVerbatim}
 1: branch discus
 2: variable integer, ncellx
 3: variable integer, ncellz
 4: #
 5: read
 6:   stru CELL/zno.cell
 7: lat[1] = P_lata
 8: lat[2] = P_lata
 9: lat[3] = P_latc
10: z[1]   = P_z_zn
11: b[1]   = P_biso
12: b[2]   = P_biso
13: ncellx = 2.00*int(P_ab_dia/lat[1]) + 2
14: ncellz = 2.00*int(P_ab_dia/lat[3]) + 2
15: save
16:   outf internal.zno.cell
17:   omit all
18:   run
19: exit
20: read
21:   cell internal.zno.cell     , ncellx, ncellx, ncellz
22: surface
23:   boundary sphere, P_ab_dia/2.
24: exit
25: purge
26: @powder.mac
27: @output.mac TEMP/zno.grcalc
28: branch kuplot
29: rese
30: load xy, TEMP/zno.grcalc
31: skal
32: ccal mul, wy, 1 , P_scale
33: ksav 1
34:   form xy
35:   outf   TEMP/zno.grcalc
36:   run
37: exit
38: exit
39: finished
\end{MacVerbatim}

In lines 5 to 12 a template asymmetric unit is read, its lattice parameters
atom position and ADP values are adapted. The known lattice parameters
are used to calculate the number of unit cells required to fit a spherical 
object into a block of unit cells in lines 13 and 14. The modified 
asymmetric unit is saved to an internal file (lines 15 to 19) and a block of
unit cells is build using the current information, lines 20 and 21. Within the
{\tt surface} menu (lines 22 to 24) a sphere is cut with radius {\tt P\_dia/2} 
and all voids are removed from the structure, line 25. The PDF is determined
by calculating the powder diffraction pattern through the Debye Scattering 
Equation, macro {\tt powder.mac} line 26. This PDF is saved with the 
{\tt output} menu in line 27. In the remainder of the macro the PDF is scaled
and its final form saved to the hard disk. The {\tt exit} in line 37 returns to
the \Discus section and the {\tt exit} in line 38 returns to the \Refine 
section. The slave macro is terminated by the special \Refine command
{\tt finished}.

\begin{MacVerbatim}
 1: powder
 2: xray
 3: set axis,q
 4: set calc,debye
 5: set disp,off
 6: set delta,0.00
 7: set qmin, 0.5000
 8: set qmax, P_qmax
 9: set dq,   0.0001
10: set profile, pseudo
11: set profile, eta, P_eta, P_eta_l
12: set profile, uvw, P_u, P_v, P_w
13: set profile,  width, 11
14: set temp,use
15: set wvle,0.20000
16: set four,four
17: set lpcor,bragg,0.500
18: set scale, 1.000  
19: run
20: exit
\end{MacVerbatim}

The powder menu uses the Debye Scattering Equation (line 4) to calculate
the powder pattern for the finite sized spherical nanoparticle. While
the lower limit ${Q_{min}}$ is fixed in this example to 0.5, the upper limit
is determined by the (fixed) parameter {\tt P\_qmax}. A Pseudo Voigt profile
function is set up in lines 11 to 13.

\begin{MacVerbatim}
 1: output
 2: value PDF
 3: outf \$1
 4: form powder, r, 0.01,  50, 0.01
 5: run
 6: exit
\end{MacVerbatim}
 
In order to generate the PDF, the output menu is instructed to write the
data with value {\tt PDF}, line 2. Correspondingly, the format is set to
{\tt r} with user supplied limits in line 4. In this example, the limits
are fixed number, if necessary this can be changed to use variable names.

This refinement proceeds with the following initial output:

\begin{MacVerbatim}
Cyc Chi^2/(N-P)   MAX(dP/sig) Par   Conf          Lambda       wRvalue      Rexp
  0   2273.9      0.31748E-02   5   0.0000      0.10000E-01   1.0180      0.21348E-01
  1   1476.8      0.15617       5   0.0000      0.50000E-02  0.82039      0.21348E-01
  2   1194.3      0.56799E-01   5   0.0000      0.25000E-02  0.73777      0.21348E-01
  3   862.37      0.59080E-02   6   0.0000      0.12500E-02  0.62690      0.21348E-01
  4   557.78      0.13358E-01   5   0.0000      0.62500E-03  0.50418      0.21348E-01
  5   329.67      0.14493E-01   5   0.0000      0.31250E-03  0.38761      0.21348E-01
  6   171.28      0.10872E-01   5   0.0000      0.15625E-03  0.27939      0.21348E-01
  7   138.25      0.17644E-01   6   0.0000      0.78125E-04  0.25101      0.21348E-01
  8   51.188      0.99726E-02   5   0.0000      0.39062E-04  0.15274      0.21348E-01
  9   26.416      0.29501E-01   6   0.0000      0.19531E-04  0.10972      0.21348E-01
 10   8.3212      0.51764E-01   6   0.0000      0.97656E-05  0.61581E-01  0.21348E-01
 11   6.7917      0.62021E-02   5   0.0000      0.48828E-05  0.55635E-01  0.21348E-01
 12   6.7917       0.0000       1   0.0000      0.78125E-04  0.55635E-01  0.21348E-01
 13   6.7917       0.0000       1   0.0000      0.12500E-02  0.55635E-01  0.21348E-01
 14   6.7917       0.0000       1   0.0000      0.20000E-01  0.55635E-01  0.21348E-01
 15   6.7917       0.0000       1   0.0000      0.32000      0.55635E-01  0.21348E-01
 16   5.8721      0.16346E-02   5   0.0000      0.16000      0.51731E-01  0.21348E-01
 17   4.7877      0.26307E-02   5   0.0000      0.80000E-01  0.46711E-01  0.21348E-01
 18   3.7652      0.35205E-02   5   0.0000      0.40000E-01  0.41424E-01  0.21348E-01
 19   2.8465      0.19826E-02   5   0.0000      0.20000E-01  0.36017E-01  0.21348E-01
 20   1.6043      0.16427E-02   5   0.0000      0.10000E-01  0.27039E-01  0.21348E-01
 21   1.2085      0.17484E-03   5  0.12624E-21  0.50000E-02  0.23468E-01  0.21348E-01
Convergence reached
\end{MacVerbatim}

As you can see the inital agreement is not very well, nevertheless the algorithm 
manages to find and refine into the perfect solution in cycle 21. Note that in 
cycles 12 to 15 the algorithm does not find a better solution and increases 
{\tt lambda}. At cycle 21 the shift in chi squared is less than the user defined 
threshold and the refinement terminates.

The final output lists the convergence criteria, the final agreement values
and the refined parameters.

\begin{MacVerbatim}
 Data loaded as     : xy, DATA/zno.grobs
 Sigma not defined  
 Refinement macro   : discus_main.mac
   Convergence 1    : dP/sigma < AND conf >     AND dChi^2 <
                      0.500E-002     0.900          0.500     
   Convergence 2    : dChi^2 <   AND dP/sigma > 0.0          
                      0.500     
   Convergence 3    : Chi^2 <                                
                       1.10     

 Information about the fit : 
   Chi^2      :  6032.82          Chi^2/N :  1.20656    
   Conf. level: 0.126237E-21  Chi^2/(N-P) :  1.20850    
   No.Data    :         5000     No.Params:            8
   MRQ final  : 0.500000E-02
   wR value   : 0.234680E-01      R exp   : 0.213478E-01

 Correlations larger than 0.8 :
   P_w              - P_eta            : -0.953
  
 Refined parameters
   P_lata           :    3.24883      +-    0.126421E-04    [   2.90000     ,   3.50000     ]
   P_latc           :    5.20379      +-    0.296431E-04    [   4.70000     ,   5.30000     ]
   P_z_zn           :   0.367053      +-    0.108764E-04    [  0.300000     ,  0.420000     ]
   P_biso           :   0.462104      +-    0.327584E-03    [   0.00000     ,               ]
   P_ab_dia         :    39.7863      +-    0.129176E-01    [   20.0000     ,   80.0000     ]
   P_eta            :   0.867988      +-    0.174564E-01    [   0.00000     ,   1.00000     ]
   P_w              :   0.520064E-003 +-    0.197551E-04    [   0.00000     ,  0.100000     ]
   P_scale          :   0.986814      +-    0.837900E-03    [  0.100000     ,   10.0000     ]
 Fixed   parameters
   P_eta_l          :    0.00000     
   P_u              :    0.00000     
   P_v              :    0.00000     
   P_qmax           :    24.0000     
\end{MacVerbatim}

\begin{figure}
   \includegraphics[angle=270,scale=0.45]{nano_final.epsi}
   \caption{Experimental nanoparticle PDF overlayed with the final result.
   The difference curve in green is shifted down for clarity.
   }
   \label{fexa-nano-fin}
\end{figure}

Fig \ref{fexa-nano-fin} shows the excellent agreement between {\it observed} and 
calculated data, even at very large distances near the nanoparticle 
diameter. 

%------------------------------------------------------------------------
%------------------------------------------------------------------------
