\chapter{REFINE commands}
\section{Summary}
\par
Here is a list and brief description of valid REFINE commands. Further 
help can be obtained by typing the corresponding command name at the 
help prompt. 
\par
\begin{MacVerbatim}
News          : Information on program updates
\end{MacVerbatim}
\section{Example}
\par
A simple fit might refine parameters of a straight line to 
observed data. Lets assume that the data are in a simple x/y 
file "observed.data". A suitable macro to refine is: 
\par
\begin{MacVerbatim}
refine                       ! Switch from SUITE to REFINE section
data xy, observed.data       ! Load data set "observed.data"
                             ! The load command creates user variables
                             ! F_DATA = Number of Data set in Kuplot
                             ! F_XMIN, F_XMAX, F_XSTP
                             ! F_YMIN, F_YMAX, F_YSTP
                             ! That contain the data limits and step size.
newparam P_inter, value:1.0  ! Define y-axis intersept as first parameter
newparam P_slope, value:1.0  ! Define slope as second parameter
set cycle, 10                ! Define maximum number of cycles
run fit_work.mac             ! start the fit with user macro fit_work.mac
exit                         ! Back to SUITE
\end{MacVerbatim}
The user macro fit\_run should be something like: 
\par
\begin{MacVerbatim}
branch kuplot                ! Step into the KUPLOT section
                             ! or for structures into DISCUS
func P_inter + P_slope*r[0], F_DATA
                             ! Calculate the function,
                             ! limits are set automatically from
                             ! data set F_data that was loaded by refine.
exit                         ! Back to REFINE
finished                     ! Special keyword signals end of user macro
\end{MacVerbatim}
\section{data}
{\bf data "kuplot", $ <$number$> $ \par }
{\bf data $ <$type$> $, $ <$infile$> $ \par }
\par
\begin{MacVerbatim}
data "kuplot", <number>
\end{MacVerbatim}
\vspace{3pt}
The calculation will refine parameters versus the observed data 
that are stored in the KUPLOT data set number $ <$number$> $ 
\par
\begin{MacVerbatim}
data <type>,<infile>
\end{MacVerbatim}
The calculation will refine parameters versus the observed data 
that are loaded from file $ <$infile$> $. 
   See the ==$> $ 'kuplot/load' command for details on proper 
   ways to load a data set. 
\par
Refer to the KUPLOT section for instructions on loading data. 
\section{fix}
{\bf fix $ <$parname$> $ [,"value:"$ <$number$> $] \par }
\par
\vspace{3pt}
Fixes a parameter. Its value will remain at its current value 
or at $ <$number$> $, if the optional parameter "value:" is used. 
\par
A fixed parameter can be freed with a ==$> $ 'newparam' command. 
\section{newparam}
{\bf newparam $ <$parname$> $ [,"value:"$ <$start$> $] [,"status:"$ <$flag$> $] \par }
{\bf                    [,"range:["$ <$lower$> $,$ <$upper$> $"]"] \par }
\par
\vspace{3pt}
Defines a new parameter name $ <$parname$> $. This should be any valid 
user defined variable name, limited to a length of 16 characters. 
The user variable is allowed to have been defined previously, 
and its current value will be used if the "value:" option is omitted. 
\par
See the general command line section for details on the definition 
of variables. 
\par
Optional parameters are: 
value:$ <$start$> $ 
   The Parameter will be initialized to the $ <$start$> $ value. 
   If omitted, the parameter value will take on its current value. 
\par
status:refine 
status:free 
status:fix 
status:fixed 
   With the flags "refine" or "free" the parameter will be refined 
   during the refinement cycles. 
   With the flags "fixed" or "fix", the parameter will remain fixed 
   at its current value. Be carefull that the user macro does not 
   change the parameter value! 
   Default flag is "refine". 
\par
range:[$ <$lower$> $, upper$> $] 
   Defines a lower and upper boundary for the parameter. The 
   fit will ensure that the parameter does not move outside the 
   specified range. 
   If the "range:" parameter is omitted, the default behaviour 
   is to assume no boundaries. 
   In order to turn the boundaries off, simply state the 
   'newparam' command again for the refinement parameter 
   without the "range:" option. 
\section{set}
\begin{MacVerbatim}
set cycle,<maxc>
set conver [,"dchi:"<delta>] [,"pshift:"<max>] [,"conf:"<level>]
\end{MacVerbatim}
{\bf set cycle,$ <$maxc$> $ \par }
\vspace{3pt}
Sets the maximum number of refinement cycles 
\par
{\bf set conver [,"dchi:"$ <$delta$> $] [,"pshift:"$ <$max$> $] [,"conf:"$ <$level$> $] \par }
{\bf            [,"chi2:"$ <$level$> $] \par }
\par
\vspace{3pt}
Allows the user to define convergence criteria. 
\par
dchi:$ <$delta$> $ 
  If the value of Chi$**$2/(Ndata-Npara) decreases by less than $ <$delta$> $ 
  convergence is reached. 
  Defaults to 0.5 
pshift:$ <$max$> $ 
  If all refinement parameters change by less then $| $DeltaP/Sigma$| $ 
  convergence is reached. 
  Defaults to 0.005 
conf:$ <$level$> $ 
  If the confidence level is greater than $ <$level$> $ 
  convergence is reached. 
  Defaults to 0.010 
chi2:$ <$level$> $ 
  If the value of Chi$**$2/(Ndata-Npara) falls below $ <$level$> $ 
  convergence is reached. 
  Defaults to 0.500 
\par
Convergence is reached, if: 
Either 
   (dchi AND pshift AND conf) 
or 
   chi2 
are met. 
\section{sigma}
{\bf sigma "kuplot", $ <$number$> $ \par }
{\bf sigma $ <$type$> $,$ <$infile$> $ \par }
\par
\begin{MacVerbatim}
sigma "kuplot", <number>
\end{MacVerbatim}
\vspace{3pt}
The calculation will refine parameters versus the observed data 
and use sigmas 
that are stored in the KUPLOT data set number $ <$number$> $ 
\par
\begin{MacVerbatim}
sigma <type>,<infile>
\end{MacVerbatim}
The calculation will refine parameters versus the observed data 
and use sigmas 
that are loaded from file $ <$infile$> $. 
   See the ==$> $ 'kuplot/load' command for details on proper 
   ways to load a data set. 
\par
Refer to the KUPLOT section for instructions on loading data. 
