%------------------------------------------------------------------------
% Chapter:  FORTRAN interpreter
%------------------------------------------------------------------------

\chapter{FORTRAN style interpreter \label{fort}}

The programs include a FORTRAN style interpreter that allows the
user to program complex modifications.  The interpreter provides
variables, linked to data sets and free variables, loops, logical
construction, basic arithmetic and built in functions.  Commands
related to the FORTRAN interpreter are {\tt =, break, do, else,
elseif, enddo, endif, eval, if} and {\tt variable}. The command {\tt
eval} allows to examine the contents of a variable or to evaluate an
expression, e.g. {\tt eval r[1]*0.5}. \par

%------------------------------------------------------------------------

\section{Variables \label{var}}

The programs in the {\it DISCUS} suite use variables to store 
data. The programs distinguish three types of variables:
\begin{itemize}
  \item Variables with fixed name for general use
  \item Variables with user defined names for general use
  \item Variables with fixed name that carry program information
\end{itemize}

{\bf Variables with fixed name for general use}

There are two types of variables: variables of type integer or real, 
denoted by their fixed names "i" and "r", immediately followed by a left square bracket [, 
one or more indices (separated by a comma), and finally a right square bracket ]. 

Examples for build in variables are given next.

\begin{quote}
  {\it Example :\/} i[1], r[0], i[i[1]], x[2]
\end{quote}

The build-in variables {\tt i[n]} and {\tt r[n]} are general purpose 
variables. The index {\tt n} can take integer values from zero to a 
maximum value defined at program compilation, usually 500. In addition,
 each of the program has a number of build-in
variables related to its function. \Discus for example has a number of 
variables related to the structure that the user is using, like unit cell 
dimensions, atom positions etc. For a list of these variables,
please refer to the respective program users guides.
\par
\par

{\bf Variables with user defined names for general use}\\
The second set of variables are single valued variables with user
defined names. These allow the user to use more obvious variable 
names. The programs allow to define real valued, integer valued
and character variables.
With almost identical syntax, the user can provide 1-D and 2-D
integer and real valued arrays. Currently no Character arrays are provided.

An example of the use of defined variables is shown here:
%
\begin{MacVerbatim}
   1  variable real,alpha,90.0
   2  variable real,beta
   3  variable real,diff
   4  beta = 94.0
   5  diff = alpha - beta
   6  eval diff
\end{MacVerbatim}
%
In line one we define a real variable {\tt alpha} with an initial
value of 90. Next, two more variables are defined and the difference
between {\tt alpha} and {\tt beta} is calculated (line 5) and the
result displayed on the screen (line 6). 

Recently another type of variables was added to the command language:
character variables. These can be used in conjunction with format 
statements (see command reference under \textit{expressions}). An example 
setting the plot title in \textsc{Kuplot} to the current date is listed
below.
%
\begin{MacVerbatim}
   1  variable character,datum
   2  #
   3  datum="%c",fdate(0)
   4  tit2 "Plotted on %c",datum
\end{MacVerbatim}
%
Note the slightly different way character intrinsic functions are used.
A complete list of character functions can be found in Table
\ref{func-char}.

To define 1-D or 2-d arrays, simply add the optional parameter "dim"
to the variable definition.
\begin{MacVerbatim}
   1  variable real, mat_a, dim:[3:3]
   2  variable real, vec_a, dim:[3]
\end{MacVerbatim}

To work on user supplied matrices, the \Suite provides several commands:
\begin{itemize}
  \item matmul: Multiply matrices
  \item matadd: Add      matrices
  \item detmat: Calculate the determinant
  \item mattrans: Calculate the transposed matrix
\end{itemize}
%
\par

{\bf Variables with predefined name }\\

Starting with version 5.7.0 system wide variables with fixed name 
were introduced. Currently these are mostly related to the refinement,
see the \Diffev manual for details. 

The general predefined variables are explained in the following list:

\begin{itemize}
  \item UNDEF = -1  A status is undefined
  \item TRUE  = 1   A status is true
  \item FALSE = 0   A status is false
  \item IS\_TOP = 0 Possible value for STATE, if you are currently at the
                    top program level. This will be the case within the
                    suite. If the individual programs DISCUS, KUPLOT
                    or DIFFEV are used as stand alone programs you will 
                    be in a STATE=IS\_TOP in their main menu.
  \item IS\_SECTION = 1  Possible value for STATE, if you are currently at
                    one of the three sections DISCUS, DIFFEV, KUPLOT
                    within the suite.
  \item IS\_BRANCH = 2  Possible value for STATE, if you are currently in
                    a section after a "branch" command. 
  \item SUITE = 0   Possible value for PROGRAM, if you are within the suite.
  \item DISCUS = 1   Possible value for PROGRAM, if you are within discus.
  \item DIFFEV = 2   Possible value for PROGRAM, if you are within diffev.
  \item KUPLOT = 3   Possible value for PROGRAM, if you are within kuplot.
  \item MPI\_OFF = 0 Possible value for MPI, if the program was not started
                     with mpi.
  \item MPI\_ON  = 1 Possible value for MPI, if the program was started
                     with mpi.
  \item PROGRAM      Receives a value of (SUITE, DISCUS, DIFFEV, KUPLOT)
                     depending on which part you are currently within.
  \item STATE        Receives a value (IS\_TOP, IS\_SECTION, IS\_BRANCH),
                     depending on the way this section was entered.
  \item MPI\_FIRST   Relevant to \Diffev only. The value will be UNDEFined
                     in the other sections. Within \Diffev it tells a 
                     slave program if the slave was invoked as the first
                     'run\_mpi' command within the current refinement generation.
  \item MPI          Receives a value of MPI\_OFF or MPI\_ON. 
  \item PI           Pi = 3.1415... 
\end{itemize}  

One further system variable is {\tt PI}. The capitalization is mandatory.

{\bf Variables with fixed name that carry program information}

These variables, which are common to the whole program suite are 
used by the programs to return information to the user.

Currently just one such variable {\tt res[n]} exists. It is used
in a wide variety of commands and functions to return a successful
operation or to contain result values.

As an example see:
%
\begin{MacVerbatim}
   1  fexist dummy.file
   2  eval res[0]
   2  eval res[1]
\end{MacVerbatim}
%
Here the {\tt res} variable is used to flag the existence of the file
called "dummy.file". {\tt res[0]} usually carries the number of 
results returned, in this special case one. Here {\tt res[1]} would
be equal to one if the file exists and zero if it does not exist.


%------------------------------------------------------------------------

\section{Arithmetic expressions \label{arith-exp}}

The language allows the use of arithmetic expressions using the same
notation as in FORTRAN. Valid operators are {\tt +, -, *, /} and
for the exponentiation {\tt **}. Expressions can be grouped by 
round brackets ( and ). The
usual hierarchy for the operators holds. Values of expressions can
be assigned to any modifiable variable. If you know FORTRAN (or
another programming language) you will have no problems with these
examples.

\begin{quote}
  i[0] = 1 \\
  r[3] = 3.1415 \\
  r[i[1]] = 2.0*(i[5]-5.0/6.5)
\end{quote}

%------------------------------------------------------------------------

\section{Logical expressions \label{log-exp}}

Logical expressions are formed similar to FORTRAN. They may contain
numerical comparisons using the syntax: {\it $<$arithmetic
expression$><$operator$><$arithmetic expression$>$}. The allowed
operators are 

{\tt <, <=, >, >=, ==, /=}

or in the older Fortran77 style:

{\tt .lt., .le., .gt., .ge., .eq., .ne.} 

for operations less than, less equal, greater than,
greater equal, equal and not equal, respectively. Logical
expressions can be combined by the logical operators:

 {\tt .not., .and., .or., .xor., .eqv.}.

The following example shows an
expression that is true for values of {\tt i[1]} within the interval
of 3 and 11, false otherwise.

\begin{quote}
  i[1] >= 3 .and. i[1] <= 11 \\
  i[1].ge.3 .and. i[1].le.11
\end{quote}

Logical operations may be nested and grouped using round
brackets ( and ). For more examples see section \ref{if}.

%------------------------------------------------------------------------

\section{Character expressions\label{char-exp}}

Character expressions allow you to write annotations in \kuplot, nicely 
formatted comments in the output of any section or complex filenames.

The first use might be that of character variables within the sections 
of the \Discus suite. The simplest assignment of values to a character 
variable takes the form:
\begin{MacVerbatim}
  character variable, string
  string = 'This is a string'
\end{MacVerbatim}

In order to concatenate strings, or to place numerical values into a 
string three format specifiers are provided:

\begin{MacVerbatim}
  "%c"   or "%10c"
  "%d"   or "%10d"
  "%f"   or "%10.4f"
\end{MacVerbatim}

All specifiers require the leading {\tt \%} as control character. 
The {\tt \%c} specifier allows to place the value of a character value or 
a character string into the expression. A number between the {\tt \%} and 
the {\tt c} signifies the width of the string that will be placed. If the 
incoming string is too long it will be truncated as in the following 
examples:
\begin{MacVerbatim}
  character variable, string
  character variable, line
  string = 'abcdefghij'
  line   = "%c", string    ! place at variable length 
  echo "%c",line
  abcdefghij               ! This will be shown on screen
  line   = "%c5", string   ! place as 5 character long string
  echo "%c",line
  abcde                    ! This will be shown on screen
  line   = "%5c", string(4:8)  ! take a sub string 
  echo "%c",line
  defgh                    ! This will be shown on screen
\end{MacVerbatim}

The {\tt \%d} specifier allows to write the value of an integer expression
into a character string. The optional number between the {\tt \%} and the 
{\tt d} specifies the number of digits that will be printed. If a 
capital {\tt D} is used in combination with the the width defining 
number, any leading digits are printed as zeros.

\begin{MacVerbatim}
  echo "%d more %d", 1, 2
  1 more 2                ! This will be shown on screen
  echo "%4D more %4d", 1, 2
  0001 more    2          ! This will be shown on screen
\end{MacVerbatim}

The {\tt \%f} specifier allows to write the value of a real valued expression
into a character string. The optional number between the {\tt \%} and the 
{\tt f} specifies the width of the field printed and the number of 
digits that follow the decimal point. If a  
capital {\tt F} is used in combination with the the width defining
number, any leading digits are printed as zeros.

\begin{MacVerbatim}
  echo "%f more %d", 1.1, 20.2
  1.000000 more 20.0000000 ! This will be shown on screen
  echo "%4.1F more %6.2f", 1.1, 20.2
  01.0 more   20.20        ! This will be shown on screen
\end{MacVerbatim}
%------------------------------------------------------------------------
%------------------------------------------------------------------------

\section{Intrinsic functions \label{func}}

Several intrinsic functions are defined.  Each function is
referenced, as in FORTRAN, by its name followed by a pair of
parentheses ( and ) that include the list of arguments. The ( does
not have to immediately follow the function name.  Trigonometric and
arithmetic functions are listed in table \ref{func-trig}.  Table
\ref{func-ran} contains various random number generating functions.
Some programs such as \Discus have a set of specific intrinsic 
functions allowing one to for example calculate bond length. Again 
each individual users guide will contain a table of these specific
functions.

\begin{table}[!tbh]
\centering
\begin{tabularx}{\textwidth}{|p{12mm}|p{45mm}|X|}
  \hline
  {\bf Type} & {\bf Name} & {\bf Description} \\
  \hline\hline
  real & sin(r) cos(r) tan(r) &
         Sine, cosine and tangent of $<$r$>$ in radian \\
  real & sind(r) cosd(r) tand(r) &
         Sine, cosine and tangent of $<$r$>$ in degrees \\
  real & asin(r) acos(r) atan(r) &
         Arc sin, cosine, tangent of $<$r$>$, result in radian \\
  real & asind(r) acosd(r) atand(r) &
         Arc sin, cosine, tangent of $<$r$>$, result in degrees \\
  \hline
  real & sqrt(r) & Square root of $<$r$>$ \\
  real & exp(r) &  Exponential of $<$r$>$, base e \\
  real & ln(r) &   Logarithm of $<$r$>$ \\
  real & sinh(r) cosh(r) tanh(r) &
                   Hyperbolic sine, cosine and tangent of $<$r$>$ \\
  \hline
  real    & abs(r)      &  Absolute value of $<$r$>$ \\
  integer & mod(r1, r2) &  Modulo $<$r1$>$ of $<$r2$>$ \\
  integer & int(r)      &  Convert $<$r$>$ to integer \\
  integer & nint(r)     &  Convert $<$r$>$ to nearest integer \\
  real    & frac(r)     &  Returns fractional part of $<$r$>$ \\
  \hline
  real    & min(r1, r2) &  Smaller number of $<$r1$>$ and $<$r2$>$ \\
  real    & max(r1, r2) &  Larger number of $<$r1$>$ and $<$r2$>$ \\
  \hline
\end{tabularx}
\caption{\label{func-trig}Trigonometric and arithmetic functions}
\end{table}

\begin{table}[!tbh]
\centering
\begin{tabularx}{\textwidth}{|p{12mm}|p{30mm}|X|}
  \hline
  {\bf Type} & {\bf Name} & {\bf Description} \\
  \hline\hline
  real & ran(r) &         Uniformly distributed pseudo random number
                          between 0.0 inclusively and 1.0 exclusively.
                          Argument $<$r$>$ is a dummy\\
  real & gran(r1, typ) &   Gaussian distributed random number with mean
                          0 and a width given by $<$r1$>$. If $<$typ$>$
                          is "s" $<$r1$>$ is taken as sigma, if
                          $<$typ$>$ is "f" $<$r1$>$ is taken as
                          FWHM. \\
  real & gbox(r1, r2, r3) & Returns pseudo random number with
                          distribution given by a box centered
                          at 0 with a width of $<$r2$>$ and two half
                          Gaussian distributions with individual
                          sigmas of $<$r1$>$ and $<$r3$>$ to the left
                          and right, respectively. \\
  real & gskew(r1,typ) &  Skewed Gaussian distributed random number 
                          with mean
                          \\
  real & logn(r1,r2)      & Returns a lognormal distributed number
                          with mean $<$r1$>$ and sigma $<$r2$>$ of
                          the underlying
                          Gaussian distribution.\\
  integer & pois(r1)      & Returns a Poisson distributed random
                          number with mean $<$r1$>$.\\
  \hline
\end{tabularx}
\caption{\label{func-ran}Random number functions}
\end{table}

Finally the available functions returning a character string are
listed in Table \ref{func-char}.

\begin{table}[!tbh]
\centering
\begin{tabularx}{\textwidth}{|p{30mm}|X|}
  \hline
  {\bf Name} & {\bf Description} \\
  \hline\hline
  date(0)        & Returns current date as {\tt CCYYMMDDhhmmss.sss}.\\
  fdate(0)       & Returns current date as character string {\tt Day Mon dd hh:mm.ss yyyy}.\\
  fmodt(0)       & Returns modification date of the last file opened 
                   as character string {\tt Day Mon dd hh:mm.ss yyyy}.\\
  cdate(0)       & Returns current date as {\tt Day Mon DD hh:mm:ss CCYY}.\\
  getcwd(0)      & Returns the current working directory.\\
  getenv('VAR')  & Returns the value of the environment variable
                   {\tt VAR}.\\
  \hline
\end{tabularx}
\caption{\label{func-char}String functions}
\end{table}


%------------------------------------------------------------------------

\section{Intrinsic commands  \label{cmds}}

To operate on user variables that are 1-D or 2-D matrices, the \Suite 
provides a couple of standard operations. These are listed in Table \ref{cmds-cmds}:

\begin{table}[!tbh]
\centering
\begin{tabularx}{\textwidth}{|p{30mm}|X|}
  \hline
  {\bf Name} & {\bf Description} \\
  \hline\hline
  matmul         & multiplies two matrices or a matrix by a scalar  \\
  matadd         & Adds two matrices. The second matrix can be multiplied by a scalar. \\
  detmat         & Calculates the determinant of the input matrix and stores this in 
                   the output matrix. \\
  mattrans       & Calculates the transpose of the input matrix and stores this in 
                   the output matrix. \\
  \hline
\end{tabularx}
\caption{\label{cmds-cmds}Matrix commands}
\end{table}

%------------------------------------------------------------------------

\section{Loops \label{do}}

Loops can be programmed using the {\tt do} command.
Three different types of loops are implemented. The first type
executes a predefined number of times. The syntax for this type of
loop is

\begin{quote}
{\it do $<$variable$>$ = $<$start$>$,$<$end$>$ [,$<$increment$>$] \\
     \ldots commands to be executed \ldots \\
     enddo }
\end{quote}

Following modern Fortran, the word "enddo" may
also be spelled as "end do" with arbitrary blanks
between the words.

Loops may contain constants or arithmetic expressions for
$<$start$>$, $<$end$>$, and $<$increment$>$.  $<$increment$>$
defaults to 1.  The internal type of the variables is real.  The
loop counter is evaluated from {\it ($<$end$>$ -
$<$start$>$)/$<$increment$>$ + 1}. If this is negative, the loop is
not executed at all.  The parameters for the counter variable, start
end and increment variables are evaluated only at the beginning of
the do - loop and stored in internal variables.  It is possible to
change the values of $<$variable$>$, $<$start$>$ and/or $<$end$>$
within the loop without any effect on the performance of the loop.
This practice is not encouraged, could, however, be an unexpected
source of errors. \par

The second type of loop is executed while $<$logical expression$>$ is
true. Thus it might not be executed at all. The syntax for this type
of loop is

\begin{quote}
{\it do while $<$logical expression$>$ \\
     \ldots commands to be executed \ldots \\
     enddo }
\end{quote}

The last type of loop is executed until $<$logical expression$>$ is
true. This loop, however, is always executed once and has the
following syntax

\begin{quote}
{\it do \\
     \ldots commands to be executed \ldots \\
     enddo until $<$logical expression$>$ }
\end{quote}

In the body of commands any valid program commands can be used.
This includes calls to the sub levels, further do loops or macros,
even if these macros contain do loops themselves.  The maximum level
of nesting is limited by the parameter {\it MAXLEV} in the file {\it
doloop\_mod.f90}.  If necessary adjust this parameter to allow for deeper
nesting.  All commands from the first {\tt do} command to the
corresponding 'enddo' are read and stored in an internal array. This
array can take at most {\it MAXCOM} (defined in file {\it
doloop\_mod.f90} as well) commands at every level of nesting.  If lengthy
macro files are included in the do loop, this parameter might have
to be adjusted.  \par

If a do loop (or an if block) needs to be terminated, the {\tt
break} command will perform this function.  The parameter on the
{\tt break'} command line gives the number of nested levels of {\tt
do} and {\tt if} blocks to be terminated. The interpreter will
continue execution with the first command following the
corresponding {\tt enddo} or {\tt endif} command.  An example is
given below, note, that the line numbers are only given for better
orientation and are no actual part of the listed commands. \par

\begin{MacVerbatim}
     1  do i[2]=1,5
     2     do i[1]=1,5
     3        if ((i[1]+i[2]) .eq 6) then
     4           break 2
     5        endif
     6     enddo
     7  enddo
\end{MacVerbatim}

In this example, the execution of the inner do-loop will stop as
soon as the sum of the two increment variables {\tt i[1]} and {\tt
i[2]} is equal to 6.  The program continues with the {\tt enddo}
line of the outer do - loop.  Notice that two levels need to be
interrupted, the if block and the innermost do loop.  If the
parameter had been equal to one, only the if block would have been
interrupted, while the innermost do loop would have continued
without break.\par

%------------------------------------------------------------------------

\section{Conditional statements \label{if}}

Commands can be executed conditionally by using the {\tt if}
command. Analogous to FORTRAN, the if-control structure takes the
following form:

\begin{quote}
{\it if ( $<$logical expression$>$ ) then \\
     \ldots commands to be executed \ldots \\
     elseif ( $<$logical expression$>$ ) then \\
     \ldots commands to be executed \ldots \\
     else \\
     \ldots commands to be executed \ldots \\
     endif }
\end{quote}

Following modern Fortran, the words "elseif" and "endif" may
also be spelled as "else if" and "end if" with arbitrary blanks
between the words.

The logical expressions are explained in section \ref{log-exp}.
Enclosed within an if block any valid program command can be
used.  This includes calls to the sub levels further if blocks, do
loops or macros, even if these macros contain if blocks or do loops
themselves.  The {\tt elseif} and {\tt else} section is optional.
The maximum level of nesting is limited by the parameter {\it
MAXLEV} in the file {\it doloop\_mod.f90}.  If necessary adjust this
parameter to allow for deeper nesting.  All commands from the first
{\tt if} command to the corresponding {\tt endif} are read and
stored in an internal array.  This array can take at most {\it
MAXCOM} (defined in file {\it doloop\_mod.f90} as well) commands at every
level of nesting.  If lengthy macro files are included in the do
loop, this parameter might have to be adjusted.  \par

If an if block (or a do loop) needs to be terminated, the {\tt
break} command will perform this function.  The parameter on the
{\tt break} command line gives the number of nested levels of {\tt
do} and {\tt if} blocks to be terminated. The interpreter will
continue execution with the first command following the
corresponding {\tt enddo} or {\tt endif} command.  See the example
in section \ref{do} for further explanations. \par

\begin{MacVerbatim}
     1  #
     2  # Read crystal file
     3  #
     4  read
     5  cell cell.cll,10,10,10
     6  #
     7  # Remove atoms with probability 0.3
     8  #
     9  do i[0]=1,n[1]
    10     if(ran(0).lt.0.3) then
    11        remove i[0]
    12     endif
    13  enddo
\end{MacVerbatim}

The example listed above illustrates the use of loops and
conditional statements within {\it DISCUS}. Again, the line number
are given for easy reference and not part of the actual
input. The first three lines are just comments. In lines 4 and 5 an
asymmetric unit is read from the file {\it cell.cll} and expanded to
a crystal size of 10x10x10 unit cells. In line 9 starts a do-loop
over all atoms within the crystal. The variable n[1] contains this
information (see \Discus users guide). Since
the function {\tt ran} produces a uniformly distributed pseudo
random number in the range 0.0 to 1.0, the if statement in line 10
is true in about 30\% of its calls, at least for sufficiently large 
crystal sizes. Thus approximately 30\% of the atoms are removed
(line 11), and the corresponding amount of vacancies (VOID) created
within the crystal.

%------------------------------------------------------------------------

\section{Filenames \label{fnames}}

Usually, file names are understood as typed, including capital
letters. Unix operating systems distinguish between upper and lower
case typing ! However, sometimes it is required to be able to alter
a file name e.g. within a loop.  Thus, the command language allows the 
user to construct file names by writing additional (integer) numerical
input into the filename.  The syntax for this is:

\begin{quote}
  {\it "string\%dstring",$<$integer expression$>$}
\end{quote}

The file format MUST be enclosed in quotation marks.  The position
of each integer must be characterized by a {\tt \%d}.  The sequence
of strings and '\%d's can be mixed at will.  The corresponding
integer expressions must follow after the closing quotation mark. If
the command line requires further parameters (like {\tt addfile} for
example) they must be given after the format-parameters.  The
interpretation of the '\%d's follows the C syntax. Up to 10 numbers
can be written into a filename.  All of the following examples will
result in the file name {\it a1.1}:

\begin{MacVerbatim}
     i[5]=1
     outfile a1.1
     outfile "a%1d.%1d",1,1
     outfile "a%1d.%1d",4-3,i[5]
\end{MacVerbatim}

The second example shows how filenames are changed within a loop.
Here the output (e.g. Fourier transform) will be written to the
files {\it data1.calc} to {\it data11.calc}.

\begin{MacVerbatim}
     do i[1]=1,11
       ..
       outfile "data%d.calc",i[1]
       ..
     enddo
\end{MacVerbatim}

As personal style you might find it best to label the files 
{\it data01.calc} to {\it data11.calc} i.e. with leading zeros 
and a fixed number of digits that are used for the number.
This is readily achieved with the {\tt \%D} format specifier.

\begin{MacVerbatim}
     do i[1]=1,11
       ..
       outfile "data%2D.calc",i[1]
       ..
     enddo
\end{MacVerbatim}
%------------------------------------------------------------------------

\section{Macros \label{mac}}

Any list of valid program commands can be written to an ASCII
file and executed indirectly by the command {\tt @$<$filename$>$}.
The commands are executed as typed.  Macro files can be written by any
editor on your system or be generated by the 'learn' command.
'learn' starts to remember all the commands that follow and saves
them into the file given on the {\tt learn} command.  The learn
sequence is terminated by the {\tt lend} command.  The default
extension of the macro file is {\it .mac}.  Macro files can be
nested and even reference themselves directly or indirectly.  This
referencing of macro files is, however, just a nesting of the
corresponding text of each macro, not a call to a function.  All
variable retain their values.  If an error occurs while executing a
macro, the program immediately stops execution of all macros and
returns to the interactive prompt.  If the macro switched to a
sub level, and the error occurred inside of this sub level, the program
will remain within this sub level the interactive prompt
corresponding to this sub level is returned. The command {\tt stop}
allows the user to interrupt the execution of a macro, enter
commands and continue the macro using the command {\tt cont}. Note,
that the macro needs to continued in the same sub level it was
interrupted.
\par

On the command line of the macro command {\tt @}, optional
parameters can be supplied.  Within the macro these have to be
referenced as \$1, \$2 etc.  Upon execution of the macro the formal
parameters \$n are replaced by the character string of the actual
parameters as they are typed on the command line. Even if the actual
parameters on the macro command line consist of expressions or 
variables, initially the string as typed on the command line is
pasted into the macro. If necessary, these strings are evaluated
as the macro executes a specific line.

Parameter \$0 contains the number of
parameters specified on the command line. As any other command
parameters, these parameters must be separated by comma.  If a
formal parameter is referenced inside a macro without a
corresponding parameter on the command line, an error message is
given.  An example is given below:

\begin{MacVerbatim}
     # Adds two numbers supplied as command line parameters.
     # The value is stored in variable defined by parameter three
     #
     $3 = $1 + $2
     eval $3
\end{MacVerbatim}

If this macro is called with the following line, {\tt @add
1,2,i[4]}, the result is stored in variable i[4] which now has the
integer value 3. The commands are initially replaced by the 
macro parameters to yield:
\begin{MacVerbatim}
     # Adds two numbers supplied as command line parameters.
     # The value is stored in variable defined by parameter three
     #
     i[4] = 1 + 2
     eval i[4]
\end{MacVerbatim}

If macros are nested, the lines of an inner macro are executed
at the point where the outer macro encounters the inner macro calling 
line. The text of the inner macro is effectivley pasted into the main
macro. Keep in mind that this is just a pasting of macro lines into 
a single strem of lines. In contrast to programming languages, 
a macro does not provide its own reserved name space. All variables 
have the same value throughout the program. 
\par
Besides the plain string substitution the \Suite offers a second 
transfer mechanism, that will transfer the value of an expression or
variable to the macro instead of the string that is placed onto 
the macro line. To achieve this, enclose the macro parameter with the
string value(...).
\begin{MacVerbatim}
     variable integer, number
     number = 2
     @test.mac number, value(number), i[4]
\end{MacVerbatim}

The effect of this is that the macro listed previously will 
execute the line:
\begin{MacVerbatim}
     i[4] = number + 2
     eval i[4]
\end{MacVerbatim}
In this simple example, no difference results to the standard transfer
mode. The main application of the {\tt value(..)} transfer mode is the
possibility to place the result of an expression (character, numeric)
at a place, where \Suite does not offer the evaluation of such an
expression. With a macro call in the form:
\begin{MacVerbatim}
     variable integer, number
     number = 2
     @subsitute.mac value("base_%4D",number)
\end{MacVerbatim}

The corresponding parameter \$1 in macro {\tt substitute.mac} 
\begin{MacVerbatim}
   # macro substitute.mac
   echo $1
   variable integer, $1
\end{MacVerbatim}

would take the form:
\begin{MacVerbatim}
   echo base_0002
   variable integer, base_0002
\end{MacVerbatim}

\par

\subsection{Command line options}
If the program is started with command line parameters,
e.g. {\tt discus 1.mac 2.mac}, the program will execute the given
macros in the specified order, in our example first {\tt 1.mac} then
{\tt 2.mac}. You cannot, however, provide parameters to these macros.

Alternatively a single macro can be executed by starting the program
with the command line option 
{\tt -macro macro\_file\_name parameter(s)}

If a macro is not found in the current working
directory, a system macro director is searched. This system macro
directory is located at {\tt path\_to\_binary/sysmac/discus/}.
Commonly used macro files might be installed in this directory.
If a macro file is not found, an error message is displayed.

%------------------------------------------------------------------------

\section{Working with files \label{io}}

The command language offers the user several commands to write
variables to a file or read values from a file. First a file needs
to be opened using the command {\tt fopen}. An optional parameter
{\tt append} allows one to append data to an existing file. Once the
task is finished, the file must be closed via {\tt fclose}. In the
standard configuration, the program can open five files at the same
time. The first parameter of all file input/output related commands
is the unit number which can range from 1 to 5. The commands {\tt
fget} and {\tt fput} are used to read and write data, respectively.
The following example illustrates the usage of these commands:

\footnotesize
\begin{MacVerbatim}
      1  fopen 1,sin.dat
      2  fput 1,'Cool sinus function'
      3  #
      4  do i[1]=1,50
      5     r[1]=i[1]*0.1
      6     r[2]=sin(r[1])
      7     fput 1,r[1],r[2]
      8  enddo
      9  #
     10  fclose 1
\end{MacVerbatim}
\normalsize

In line 1 we open the file {\it sin.dat} and write a title (line 2).
If the file already exists it will be overwritten. Note that the
text must be given in {\it single} quotes. Text and variables may be
mixed in a single line. Next we have a loop calculating $y=\sin(x)$
and writing the resulting $x$ and $y$ values to the open file (line
7). Finally the file is closed (line 10). To read values from a file
use simply the command {\tt fget} and the read numbers will be
stored in the specified variables. In contrast to writing to a file,
mixing of text and number is not allowed when reading data. However,
complete lines will be skipped when the command {\tt fget} is
entered without any parameters.

%------------------------------------------------------------------------

\section{Remote control\label{remote}}

It is possible to remote control the programs in the \Discus package
using so called sockets. One program acts as server and receives and 
executes commands. In order to enable the server feature, the program
needs to be started using the {\tt -remote} command line switch as in
the example below:
%
\begin{MacVerbatim}
dhcp165057:prog> ./discus -remote

          ***********************************************************
          *               D I S C U S   Version 5.22.0              *
          *                                                         *
          *         Created : Thu Jun 28 10:00:00 JST 2018          *
          *---------------------------------------------------------*
          * (c) R.B. Neder  (reinhard.neder@fau.de)                 *
          *     Th. Proffen (tproffen@lanl.gov)                     *
          ***********************************************************
          *                                                         *
          * For information on current changes type: help News      *
          *                                                         *
          ***********************************************************

 Command line editing enabled ..

 User macros in   : /Users/thomasproffen/mac/discus/
 System macros in : /Users/thomasproffen/mac/discus/
 Start directory  : /Users/thomasproffen/Code/Diffuse/discus/prog

 ------ > Running in SERVER mode
 Running in local mode (127.0.0.1) ..
 Listening to port 3330 ..
 Allowing connections from 127.0.0.1 ..
\end{MacVerbatim}
%
Note that the host IP address and port number are given at the end
of the startup output. This information is needed to connect to this 
running version of in our case \discus. To allow connections from
computers other than 127.0.0.1 (localhost) or using a different port,
use the command line options {\tt -access} and {\tt -port}.

In order to connect to the \Discus server, we use the command 
{\tt socket} as in this example:
%
\begin{MacVerbatim}
 discus > socket open,127.0.0.1,3330
 Connecting to 127.0.0.1:3330 ..
 Server : ready
 Connected ..
 discus > socket send,echo This is cool
 Server :  This is cool
 Server : ready
 discus > 
\end{MacVerbatim}
%
Of course any program or script that can use sockets is able to connect
to the \Discus package programs in this way. For more details refer
to the {\tt socket} command reference later in this guide.

%------------------------------------------------------------------------
