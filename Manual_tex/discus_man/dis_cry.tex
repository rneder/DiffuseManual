%------------------------------------------------------------------------
% Chapter:  Crystallography
%------------------------------------------------------------------------

\chapter{Crystallography \label{cryst}}

In this chapter crystallographic calculations such as conversion
from direct to reciprocal space and {\it vice versa} are discussed
as well as symmetry operations and unit cell transformations.  The
mathematical background of the described calculations can be found
e.g. in \citet{sands}.

%------------------------------------------------------------------------

\section{Crystallographic calculations \label{cryst-calc}}

In order to determine the proper place of a new atom, or to check
existing atoms several functions perform calculations that are based
on the metric of the crystal.  Several intrinsic functions are
provided by \discus. They are listed in Table \ref{func-cryst}
in section \ref{get} of this manual.  Those function allow the user
to calculate e.g. bond length and bond angles within the current
crystal. \par
%
\begin{table}[!tbh]
\centering
\begin{tabularx}{\textwidth}{|p{30mm}|X|}
  \hline
  {\bf Command} & {\bf Description} \\
  \hline\hline
  {\tt d2r}     & Conversion from direct (real) space to reciprocal space \\
  {\tt r2d}     & Conversion from reciprocal space to direct space \\
  {\tt vprod}   & Vector product in real or reciprocal space \\
  {\tt proj}    & Calculates the projection from one vector on another in
                  real or reciprocal space \\
  \hline
\end{tabularx}
\caption{\label{cry-tab1}Commands for crystallographic calculations}
\end{table}
%
The crystallographic calculation commands provided by \Discus
are shown in Table \ref{cry-tab1}. All commands display the results
on the screen and store them in the variable res[i] (see section
Variables in the package manual), which allows further use of the 
results in macro files.
\par

The commands {\tt d2r} and {\tt r2d} calculate the components of a
given vector ${\bf u}$ in the complementary space. The vector
product ${\bf w} = {\bf u} \ast {\bf v}$ gives a vector ${\bf w}$
that is normal to the two vectors ${\bf u}$ and ${\bf v}$. The
length of the resulting vector ${\bf w}$ is given by $|{\bf u}|
\cdot |{\bf v}| \cdot \sin (\angle ({\bf u},{\bf v}))$ which is the
area of the parallelogram spanned by the vectors ${\bf u}$ and ${\bf
v}$. This vector product is calculated by the command {\tt vprod}
and any of the three vectors ${\bf u}$, ${\bf v}$ and ${\bf w}$ may
be in direct or reciprocal space. The command {\tt proj} computes
the projection of a vector ${\bf u}$ onto the vector ${\bf v}$ or on
the plane normal to vector ${\bf v}$. The length of the projected
vector is given by ${\bf u} \cdot {\bf v} / |{\bf v}|$. Again all
input or resulting vectors can be in direct or reciprocal space.
More details about these commands can be found in the online help of
\discus. This version of \Discus does not offer a special
command for the scalar product of two vectors. You can readily
calculate the scalar product by keeping in mind its definition:
$|{\bf u}| \cdot |{\bf v}| \cdot \cos (\angle ({\bf u},{\bf v}))$.
You can calculate this value by the expression: {\tt blen({\bf
u})*blen({\bf v})*cosd (bang({\bf u}, {\bf v}))}.

%------------------------------------------------------------------------

\section{Generalized symmetry operations \label{cryst-sym}}

Lets assume that you want to rotate the oxygen atoms of a $SiO_{4}$
tetrahedra around one of the Si-O bonds.  Or you want to create the
twinned structure of a triclinic crystal.  The operations require
the application of a symmetry element whose axis is not parallel to
any of the base vectors and/or whose rotation angle is different
from 60, 90, 120, or 180$^{\circ}$. Needless to say, the resulting
symmetry matrix will not contain just one's and zero's. A general
symmetry operation can be described by a 4x4 matrix of the following
form:
%
\begin{equation}
    S = \left (
        \begin{array}{cccc}
                w_{11} & w_{12} & w_{13} & t_{1}  \\
                w_{21} & w_{22} & w_{23} & t_{2}  \\
                w_{31} & w_{32} & w_{33} & t_{3}  \\
                0 & 0 & 0 & 1
        \end{array} \right )
        \label{cry-eq1}
\end{equation}
%
Here the components $w_{ij}$ define the rotation and $t_{i}$ the
translational part of the symmetry operation.  A detailed
description for the derivation of this general symmetry operation
is found in \citet{sands}.\par

Additionally \Discus allows one to select the atom or molecule
types to be included in the symmetry operation using the {\tt
sele/dese} and {\tt msel/mdes} commands.  The user can specify
whether the atom or molecule created by the symmetry operation
replaces its original or is inserted as a new atom or molecule in
the crystal. The direction of the symmetry axis can either be given
in direct or reciprocal space.  A pure mirror operation is performed
by an improper rotation by 180$^{\circ}$.\par

%------------------------------------------------------------------------

\section{Generalized shear operations \label{cryst-shear}}

A similar operation is required if you want to distort the structure
or part of the structure. Such a distortion may occur upon a 
orthorhombic to monoclinic transformation. If the whole crystal is
sheared under this transformation it is, of course, easier to simply
change the lattice constants. If, however, only part of the structure
is to be deformed to produce for example a fish bone like pattern
a local shear operation may be required.

The {\tt shear} menu in \Discus allows to perform a shear operation:
%
\begin{equation}
    S = \left (
        \begin{array}{cccc}
                a_{11} & a_{12} & a_{13} & t_{1}  \\
                a_{21} & a_{22} & a_{23} & t_{2}  \\
                a_{31} & a_{32} & a_{33} & t_{3}  \\
                0 & 0 & 0 & 1
        \end{array} \right )
        \label{cry-eq2}
\end{equation}
%
Here the matrix elements may take up any value in order to perform the
desired distortion. \Discus allows several ways to set up the shear
operation, see the on-line help for details. You can specify the full
matrix, define the Eigenvectors of a shear or define the shear by 
a plane parallel to which a shear operation is performed.

Additionally \Discus allows one to select the atom or molecule
types to be included in the shear operation using the {\tt
sele/dese} and {\tt msel/mdes} commands.

%------------------------------------------------------------------------

\section{Unit cell transformations \label{cryst-trans}}

A common task in crystallography is to transform the unit cell of a
crystal, i.e.  use an alternative setting for some specific reason.
Subsequently one needs to transform the coordinates of the atoms
within the crystal to the new set of basis vectors.  The {\tt trans}
segment of \Discus gives various options to perform this type
of task.  The transformation can either be defined as new unit cell
in terms of the old unit cell or in terms of the new atom
coordinates with respect to the old ones and {\it vice versa}.
Independent of this choice, the origin can be shifted by a user
defined amount.  \Discus allows the user to select individual
atoms or to transform the complete crystal to the new system. If all
atoms in the crystal are transformed to the new base vectors, then
the unit cell dimensions and the metric tensor are transformed as
well. \par

Let us assume we have a cubic crystal defined by $a = b = c = 5$\AA.
The base vectors for the old system are ${\bf a},{\bf b}, {\bf c}$.
The new set of basis vectors defining the new unit cell shall given
by ${\bf a}',{\bf b}',{\bf c}'$ which are given by the relations
${\bf a}' = {\bf a}+{\bf b}$, ${\bf b}' = {\bf a}-{\bf b}$ and ${\bf
c}'={\bf c}$.  After entering the {\tt trans} segment of the program
\Discus, these definitions could be entered as follows:
%
\begin{MacVerbatim}
     anew 1.0, 1.0, 0.0
     bnew 1.0,-1.0, 0.0
     cnew 0.0, 0.0, 1.0
\end{MacVerbatim}
%
Note that the relation between the old and the new system could also
be defined with respect to the atom coordinates.  All those
relationships are calculated by \Discus and can be displayed
using the {\tt show} command. Part of the output for our example can
be found below:
%

\begin{MacVerbatim}
     ( a(new) ) = (   1.00000,  1.00000,   .00000 )   ( a(old) )
     ( b(new) ) = (   1.00000, -1.00000,   .00000 ) * ( b(old) )
     ( c(new) ) = (    .00000,   .00000,  1.00000 )   ( c(old) )

     ( a(old) ) = (    .50000,   .50000,   .00000 )   ( a(new) )
     ( b(old) ) = (    .50000,  -.50000,   .00000 ) * ( b(new) )
     ( c(old) ) = (    .00000,   .00000,  1.00000 )   ( c(new) )

     ( x(new) ) = (    .50000,   .50000,   .00000 )   ( x(old) )   (   .00000)
     ( y(new) ) = (    .50000,  -.50000,   .00000 ) * ( y(old) ) + (   .00000)
     ( z(new) ) = (    .00000,   .00000,  1.00000 )   ( z(old) )   (   .00000)

     ( x(old) ) = (   1.00000,  1.00000,   .00000 )   ( x(new) )   (   .00000)
     ( y(old) ) = (   1.00000, -1.00000,   .00000 ) * ( y(new) ) + (   .00000)
     ( z(old) ) = (    .00000,   .00000,  1.00000 )   ( z(new) )   (   .00000)
\end{MacVerbatim}
%
This particular example might seem very simple but as soon as e.g.
a unit cell transformation in a triclinic system is needed, this
part of \Discus becomes the most popular feature.

%------------------------------------------------------------------------
