%------------------------------------------------------------------------
% Chapter:  Creating structure
%------------------------------------------------------------------------

\chapter{Creating structures \label{struc}}

The first steps towards any simulation are to create the desired
structure and possibly to modify it.  \Discus offers two
different ways to perform this task, the structure can either be
generated from the contents of a asymmetric unit of a unit cell or
be completely read from a file.  For each atom the program stores
its type, its fractional coordinates (x,y,z) an isotropic
thermal coefficient (B) and a set of property flags. 
 If the crystal is generated from an
asymmetric unit, the program uses the space group symbol read from
the data file to generate the unit cell from the asymmetric unit.
Prior to this generation, the space group symbol is checked for
consistency with the lattice constants.  An error message flags any
inconsistencies.  The unit cell is copied to generate a crystal of
desired (rectangular) dimensions.
\par

The following sections describe the structure file formats and how
to group atoms to (rigid) molecules.  Once a crystal (which could be
as small as one atom) is generated, several tools are provided to
modify the crystal.  These tools can be grouped into those that
affect individual atoms and those that affect the whole crystal.
The first group includes the following modifications of an
individual atom: moving, replacing, removing and inserting new
atoms.  These are discussed in chapter \ref{mod-simple}.  The second
group includes at present the following modifications of the crystal
at large, for more details see chapter \ref{mod}: thermal
displacement, waves, stacking faults and domains, generalized 
symmetry operations ( chapter \ref{cryst}), Monte-Carlo modelling
( chapter \ref{mc}), and Reverse-Monte-Carlo modelling ( chapter 
\ref{rmc}).

%------------------------------------------------------------------------

\section{Reading structure files \label{struc-read}}

\Discus uses its own file format for crystal structures. In part this 
is a legacy feature since the {\it CIF} file format was not around at the time 
\Discus was started. A second aspect concerns the fact that a {\it CIF} file 
is restricted to the content of a single unit cell. As \Discus 
often builds large blocks of unit cells with reduced periodicity, 
the \Discus file format still bears many merits.

For the use of initial input a {\it CIF} file is fine and can be 
read directly 
within the {\tt read} menu. Alternatively the {\it CIF} file can
be imported into the \Discus file format. Further file formats can 
be imported into the \Discus format.

As the standard choice the \Discus 
specific format is described in this section.

A unit cell (actually the asymmetric unit within the unit cell )
or a whole structure is read from file by the command
{\tt read}. The format of these two file types is identical.  If a
unit cell is read, the contents of the file is regarded as the
asymmetric unit of the unit cell.  The space group information is
used to generate the whole unit cell. If, on the other hand, a
structure is read, the contents of the file is taken as it is.  \par

A keyword controlled structure file format is used allowing more
flexibility, e.g. the use of molecules. However, \Discus is
capable to read both the very old and the new structure files
automatically recognizing the correct format. Furthermore the
command {\tt format nokey} in the {\tt save} segment allows the user
to save a structure using the old format (see section
\ref{struc-save}). As the non keyword file format lacks many of
the storage features its use is highly discouraged.


\subsection{Structure file format \label{struc-new}}

The standard \Discus structure file is a simple text file starting
with a section with keywords setting parameters like lattice
constants or space group followed by a section with the actual
atoms. The sequence of keywords is arbitrary with two exceptions,
the first line {\bf must} contain the keyword {\tt title} setting a
title and the last keyword {\bf must} be i{\tt atoms} followed by a list of
atoms and/or molecule definitions (see section \ref{mol}). Each atom
within the asymmetric unit or complete structure is defined 
at least by its
name (e.g. ZR), the fractional coordinates $(x,y,z)$ and an
isotropic temperature factor $B$. The property flag is optional
and defines if the atom is a normal atom, inside a molecule, inside
or outside the crystal, within a domain. 
Starting with version 5.18.0 one can in addition specify the 
molecule number to which the atom belongs and an occupancy for the atom.

A list of valid keywords within a
structure file is given in Table \ref{struc-new-tab1}.

\begin{table}[!tbh]
\centering
\begin{tabularx}{\textwidth}{|p{25mm}|X|}
  \hline
  {\bf Keyword} & {\bf Description} \\
  \hline\hline
  \#          & Allows comments in the keyword section of the structure
                file.. \\
  \hline
  adp         & A list of isotropic B values for all atom types. If 
                present, these values will override the parameter 
                specified for an atom upon reading a structure.\\
  \hline
  atoms       & Marks the start of the atom list and must be the {\bf
                last} keyword except for 'molecule'.\\
              & Parameters are the names of the columns that are 
                present for each atom. Valid columns are:\\
              & x, y, z, Biso:  These four columns must always be present.\\
              & Property:  The property flag\\
              & MoleNo, MoleAt: Molecule number and entry number for the 
                atom. These two columns are written by \Discus during
                a save. To build a structure with molecules it is 
                easier for you to use the {\tt molecule} keyword.\\
              & Occ: Occupancy for this atom type\\
              & The sequence of the columns is arbitrary, yet the recommended
                sequence is:\\
              & x,y,z, Biso, Property, MoleNo, MoleAt, Occ\\
  \hline
  cell        & Parameters $a,b,c,\alpha,\beta,\gamma$ define the
                lattice constants of the crystal. Note that {\bf all} six
                parameters must always be given.\\
  \hline
  generator   & Allows to define generators additional to the ones
                defined by the space group. The 12 parameters define the
                symmetry operation. \\
  \hline
  molecule    & Defines rigid molecules. The keyword must stand in
                the atoms section of the structure file (see section
                \ref{mol}).\\
  \hline
  ncells      & Values are {\tt ncellx, ncelly, ncellz, nc, natoms, ntypes
                nummole, nummoletype numnoleatom}. Parameters {\tt ncellx,
                ncelly, ncellz} are the number of unit cells of the crystal, 
                along the three dimensions, 
                {\tt nc} are the number of atoms within a
                unit cell. {\tt natoms} is the total number of atoms in the 
                crystal, {\tt nscat} the number of atom types. The last three 
                value describe the number of molecules, the number of 
                molecule types and the total number of atoms within all molecules.
                This information is needed if a complete
                structure is read in order to allow \Discus to
                determine the crystals size. Upon input of an asymmetric unit 
                this information is replaced by the contant of the 
                actual cell file.\\
  \hline
  occ         & A list of atom occupancy values\\
  \hline
  scat        & A list of atom types. This will ensure that the atom
                types have a fixed sequence.\\
  \hline
  spcgr       & Sets the space group for the crystal.\\
  \hline
  symmetry    & Defines an additional symmetry operation defined by
                the 12 parameters given.\\
  \hline
  title       & Sets a title for the structure file. This {\bf must}
                be the first keyword in the file.\\
  \hline
\end{tabularx}
\caption{\label{struc-new-tab1} List of keywords for structure files}
\end{table}

Keywords are processed the same way as normal commands and
parameters have to be separated by commas. An example input file for
cubic zirconia ($ZrO_{2}$) is shown below.

\begin{MacVerbatim}
    title Structure of cubic ZrO2
    spcgr Fm-3m
    cell  5.14, 5.14, 5.14, 90., 90., 90.
    atoms
    ZR    0.00000, 0.00000, 0.00000,  0.5
    O     0.25000, 0.25000, 1./4.  ,  1.0
\end{MacVerbatim}

A full information input file as of version 5.18.0 has the form:

\begin{MacVerbatim}
    title Structure of cubic ZrO2
    spcgr Fm-3m, setting:abc
    cell  5.14, 5.14, 5.14, 90., 90., 90.
    atoms x,    y,    z,     Biso, Property, MoleNo, MoleAt, Occ, St,  Sh,  Sk,  Sl
    ZR    0.00, 0.00, 0.00,  0.5,  1,        0,      0,      1.0,  _,   0,   0,   0
    O     0.25, 0.25, 1./4., 1.0,  1,        0,      0,      1.0,  _,   0,   0,   0
\end{MacVerbatim}


The first line is the required {\tt title} line that describes the
structure. The next line specifies the space group
$Fm\overline{3}m$.  The symbols used should be the Hermann-Mauguin
symbols used in International Tables for Crystallography Vol.  A
\cite{tables}.  A center of inversion as in this example should be
given as {\tt -} sign immediately preceding the axis. Lattice types need
to be given as capital characters, mirror planes as small
characters.  Monoclinic cell choices 2,3 or unique c-axis will be
assumed if the corresponding non standard Hermann-Mauguin symbol is
used. The optional parameter {\tt setting:} allows to use a 
non standard setting, especially for the orthorhombic space groups. 
You can specify either the standard space group name and an 
alternative setting or the non standard space group name.
Another optional parameter {\tt 1} or {\tt 2} is used to differentiate
origin settings one and two. 

\Discus checks the given space group symbol for
contradictions between the space group symbal and the lattice 
constants and in case of an error
the unit cell is not read.  A complete list of valid space group
symbols is part of the online help and can be accessed via 'help
space\_group' from the \Discus command line.  

The next line in the
example above gives the lattice constants of $a=b=c=5.14$\AA \ and
$\alpha=\beta=\gamma=90$ degrees.  Note that \Discus requires
all six values to be given.  The keyword 'atoms' in the example file
which must be the last keyword starts the section with the list of
atoms.  Here zirconium occupies site 4(a) on (0,0,0) and oxygen is
on 8(f) at ($\frac{1}{4}, \frac{1}{4}, \frac{1}{4}$).  The isotropic
temperature factors for Zr and O are set to 0.5 \AA$^{2}$ and 1.0
\AA$^{2}$ respectively. \par

If the lattice parameters or the atom coordinates are separated
by comma, you can use simple expressions like 1./2. or even 
variables instead of explicit numerical values. \par

Additional generators can be defined through the optional {\tt
generator} keyword.  These generators act identical to the
generators defined through the space group symbol.  All previously
generated copies of the atoms in the asymmetric unit are copied by
this generator, and will in turn be copied by any generators
following later.  Since these additional generators are applied
after the space group generators, you can use these generators to
create non-standard groups or to create a set of symmetries that
does not from a group. The syntax of the 'generator' keyword is as
follows:

\begin{MacVerbatim}
    generator g11,g12,g13,g14, g21,g22,g23,g24, g31,g32,g33,g34
\end{MacVerbatim}

Copies of an atom at (x,y,z) will be calculated using the following
equation:

\begin{equation}
        \left ( \begin{array}{c} x'\\ y'\\ z'\end{array} \right ) =
        \left (
        \begin{array}{ccc}
                g_{11} & g_{12} & g_{13}  \\
                g_{21} & g_{22} & g_{23}  \\
                g_{31} & g_{32} & g_{33}
        \end{array}
        \right ) \cdot
        \left ( \begin{array}{c} x\\ y\\ z\end{array} \right ) +
        \left ( \begin{array}{c} g_{14}\\ g_{24}\\ g_{34}\end{array} \right )
        \label{struc-new-eq1}
\end{equation}

Additional symmetry operations can be defined through the optional
{\tt symmetry} keyword.  These symmetry operations act different
than the generators described above which are defined through the
space group symbol or listed as additional generators.  The symmetry
operations copy only those atoms created by the generators.  In
contrast to generators they do not act on copies of the atoms
created by previous symmetry operations. Both keywords {\tt
generator} and {\tt symmetry} define the symmetry operation in a
similar way using 12 parameters as shown in equation
\ref{struc-new-eq1}
\par

The following example shall illustrate the difference between
generators and additional symmetry operations.  The following two
generators

\begin{MacVerbatim}
    generator 1,0,0,0.5, 0,1,0,0.5, 0,0,1,0.0
    generator 1,0,0,0.5, 0,1,0,0.0, 0,0,1,0.5
\end{MacVerbatim}

would create the following copies of an atom at (0,0,0):
$(\frac{1}{2}, \frac{1}{2}, 0), (\frac{1}{2}, 0, \frac{1}{2})$ and
$(0, \frac{1}{2}, \frac{1}{2})$.  In contrast similar symmetry
operations

\begin{MacVerbatim}
    symmetry 1,0,0,0.5, 0,1,0,0.5, 0,0,1,0.0
    symmetry 1,0,0,0.5, 0,1,0,0.0, 0,0,1,0.5
\end{MacVerbatim}

will only generate the following two copies of an atom at (0,0,0):
$(\frac{1}{2}, \frac{1}{2}, 0)$ and $(\frac{1}{2}, 0, \frac{1}{2})$
since the symmetry operations will not act on previously generated
copies of the atom at (0,0,0). The second symmetry operation copies
only the atom at (0,0,0), not the atom at $(\frac{1}{2}, 0,
\frac{1}{2})$, since this atom was created by the previous symmetry
operation.


\subsection{Using molecules \label{mol}}

The keyword controlled structure file format of \Discus allows
the definition of molecules using the {\tt molecule} keyword.  This
keyword is allowed anywhere between the atoms of the unit cell file.
It marks the beginning of a group of atoms that are grouped to form
a molecule.  The individual atoms are listed in the usual way (see
section \ref{struc-new}).  The keyword {\tt molecule end} signals
the end of a molecule.  All atoms still listed in the unit cell file
are treated as individual atoms.  The molecule related keywords are
listed in table \ref{struc-mole-tab1}.

\begin{table}[!tbh]
\centering
\begin{tabularx}{\textwidth}{|p{35mm}|X|}
  \hline
  {\bf Keyword} & {\bf Description} \\
  \hline\hline
  molecule           & Defines the start of a molecule.\\
  molecule atoms     & Lists atom numbers belonging to current molecule.\\
  molecule biso      & Defines an overall B-value for the current molecule.\\
  molecule character & Defines the character of a molecule.\\
  molecule content   & Defines the start of particular molecule type.\\
  molecule density   & Defines the density of molecules with "object"
                       character.\\
  molecule end       & Defines end of a molecule atom list.\\
  \hline 
                     & Starting with \Discus 3.6 the following 
                       keywords have become obsolete.\\
  molecule generator & Defines generators for the internal symmetry of
                       the molecule.\\
  molecule symmetry  & Defines internal symmetry operations for the molecule.\\
  \hline
\end{tabularx}
\caption{\label{struc-mole-tab1} List of keywords molecules}
\end{table}

Starting with \Discus 3.6, the internal symmetry of a molecule
is automatically determined from the Wyckoff position of the first
atom. It is mandatory that this first atom is at a Wyckoff position,
whose local site symmetry corresponds to the molecule symmetry. If 
the molecules does not posses an atom at this site, insert a 
"VOID" at the site.

The internal symmetry of the molecule can be specified using the
'generator' and 'symmetry' sub-keywords.  The generators are
internal symmetry operations of the molecule.  \Discus compares
the lists of atoms created by the space group and by the molecule
generators.  Identical sections are linked to one molecule.  Atoms
created by other symmetry operations, e.g.  lattice centering will
form a new molecule of the same type.  The generators of the
molecule symmetry should be the generators that create the site
symmetry.  See the section on site symmetry in the International
Tables A \cite{tables} for further details.  As in the previous
section, symmetry operations will only act on the {\it original}
atoms of the molecule whereas generators will operate on previously
generated copies of atoms as well. There is one {\bf important
restriction} how molecules are defined in {\discus}: The first
atom of any molecule defines the origin of the molecule used by
various subsequent commands.  In case the origin lies on a symmetry
element of the space group it must be located at the point of
highest symmetry of the molecule.  If the structure does not have an
atom at this site you must include a {\it void} on this site.  This
could be the case e.g.  if you have an empty triangle on a threefold
axis.
\par

Alternatively to defining a molecule as discussed above, the command
{\tt molecule content} and {\tt molecule atoms} might be used to
define molecule types and the corresponding list of atom indices
belonging to that molecule. This procedure is used to be able to
save structures containing molecules since the order of the atoms
required by various \Discus functions might prevent storing
atoms in groups belonging to a particular molecule. Check section
\ref {struc-save} for more details about saving structures.

\subsection{Using occupancies \label{occ}}

Occupancies are a concept that is not at the heart of \discus, as
the program is intended to simulate an actual crystal structure. In
a real crystal structure, a site in a given unit cell is occupied by 
an actual atom or a void. Averaging all unit cells will result in 
an occupancy that may deviate from 1.0. 

The usual style for \Discus to deal with partial occupancies or
with mixed occupancies by two or more atom types on a given site is
to start with a fully ordered structure. In a second step atoms 
can be replaced by another atom type with the build in command
{\tt replace}. A random replacement will be performed by this 
command. For a small crystal the actual occupancy after a {\tt replace} 
command may deviate considerably from the intended occupancy. 
At the extreme, it is impossible to have a real occupancy other than
1.0 or 0.0 within a single unit cell at least for a site with 
multiplicity of one. Even for a site with higher multiplicity the actual 
distribution of the two or more atom types onto the site will in most cases 
severely change the symmetry. 

To deal with structures that do not posses disorder other than randomly
distributed deviations from full occupancy, an occupancy parameter was
added to \Discus in version 5.18.0. 

The treatment of an occupancy parameter in the structure file
depends on the input mode. 
{\tt stru} If a structure is read with the {\tt stru} command, each
atom type retains its occupancy value. The effect on a Fourier, 
powder pattern and PDF is the same as if the actual structure would
have been modified with the {\tt replace} command. If you plot the 
structure no difference to a fully occupied structure will be 
recognized.

{\tt cell} The default behavior is to discard occupancies and to 
reset all values to 1.0. To change this behavior \Discus offers
an optional parameter to the {\tt cell} command. This parameter
allows to instruct \Discus to {\tt apply} the occupancies, i.e.
the atom type in question will be (partially) substituted by voids.
Alternatively the occupancy parameter can be {\tt kept} to give
the same result as if the structure were read with the {\tt stru}
command.
%------------------------------------------------------------------------

\section{Freely generated structure      \label{struc-free}}

As an alternative, \Discus can create a structure freely by the
{\tt free} command. This alternative is helpful to avoid 
disk I/O.

\begin{MacVerbatim}
    1  read
    2  free    5.14,  5.14,  5.14,  90.00, 90.00, 90.00, Fm-3m
    3  insert  Zr, 0.0000, 0.0000,  0.0000,  0.1
    4  insert  O , 0.2500, 0.2500,  0.2500,  0.1
\end{MacVerbatim}

This example illustrates the alternative style to create the 
asymmetric unit of cubic zirconia. The last parameter in line 2
instructs \Discus that the freely read structure should belong 
to space group $Fm\overline{3}m$. This parameter is optional
and defaults to P1. After these four commands \Discus has
created just the atoms within the asymmetric unit. To build 
a full crystal, save this asymmetric unit (optionally save it
as an internal file) and read it again with the {\tt cell} 
command. See section \ref{struc-save} for more information on 
the save procedure.

%------------------------------------------------------------------------

\section{Internal storage of a structure \label{struc-int}}

All atoms are stored sequentially in an array.  If a complete
structure is read using the {\tt stru} command, the atoms are stored
in the same sequence as found on the input file. If several unit
cells have been generated by the command {\tt cell} in the {\tt
read} sub level, the contents of all unit cells are stored
consecutively. The {\tt cell} command always generates a rectangular
block of $nx, ny, nz$ unit cells along the direct axes.  The
innermost, fastest loop is over x, the outermost, slowest loop is
over z.  

Within each unit cell, the atoms are stored in the sequence
that they have been read from file. Each atom is multiplied by the
symmetry generators of the respective space group.  The sequence of
generators has been chosen identical to that in the International
Tables for Crystallography \cite{tables}, Vol.  A.  The sequence of
symmetrically equivalent atoms is therefore identical to that of the
International Tables. 

The generators of a space group are applied in the same order as
given in Vol A of the International Tables. Thus the generated atoms
will be in the same sequence as printed for each Wyckoff position of 
each space group. 

If the space group is centered, the generators
that correspond to symmetry operations are applied first and then
each block is repeated by application of the centering generators.
This is illustrated by the following example:
 
\begin{quote}
  {\it Space group C2 No.5, Wycoff position 4(c) x,y,z \\
       The atoms on this site are } \\
       $(0,0,0; \frac{1}{2},\frac{1}{2},0)+$ \\
       $x,y,z; \overline{x},y,\overline{z}$
\end{quote}

\Discus stores these four atoms on position 4(c) in the
following sequence:

\begin{quote}
       $x,y,z$\\
       $\overline{x},y,\overline{z}$\\
       $x+\frac{1}{2},y+\frac{1}{2},z$\\
       $\overline{x}+\frac{1}{2},y+\frac{1}{2},\overline{z}$
\end{quote}

Prior to \Discus 3.6, in accordance with the header lines in
the International Tables,
the generators required for centered space groups were applied first.
The atoms generated by these generators immediately followed each
position generated by the ordinary symmetry operators as illustrated
for the following example:

\begin{quote}
       $x,y,z$\\
       $x+\frac{1}{2},y+\frac{1}{2},z$\\
       $\overline{x},y,\overline{z}$\\
       $\overline{x}+\frac{1}{2},y+\frac{1}{2},\overline{z}$
\end{quote}

The old sequence caused problems for molecules in rhombohedrally
centered space groups. 

In order to provide backward compatibility, the {\tt define}
command allows you to switch between the newer storage sequence
and the older one.

The origin within the model crystal is taken as the center of the
simulated structure, i.e.  the coordinates of an atom on site
(0,0,0) of a crystal 20x20x20 unit cells in size would range in x,y
and z from -10.0 to 9.0.
\par

If the chemistry or position of an atom is modified, its position
within the structure array remains the same.  If it is deleted, it
is replaced by atom type VOID.  The position of all other atoms
remains unchanged.  This feature enables the user to calculate the
position of an atom in the structure array, irrespective of any
changes that took place in the structure.  All new atoms are added
at the end of the structure array. Their sequence is entirely based
on the sequence of generation. \par

The command {\tt trans} in the {\tt chem} sub level of \Discus
allows to transform between the index of an atom and its unit cell
and site number. Alternatively the atom index $l'$ for an atom on
site $l$ in unit cell $i,j,k$ can be calculated using the following
equation:

\begin{equation}
        l' = \{ (k-1) \cdot nx \cdot ny +
                (j-1) \cdot nx +
                (i-1) \} \cdot n[3] + l
        \label{struc-store-eq}
\end{equation}

Here $nx,ny,nz$ are the crystals dimensions in unit cells and $n[3]$
(see table \ref{v1-tab} in section \ref{get}) contains the number 
of atoms per unit cell.
Some calculations of \Discus (e.g.  usage of lots in Fourier
sub level, quick mode for bond length calculations, ..) require this
particular order of storage of the atoms.  However, in order to save
CPU time for the Fourier transform and disk space when saving a
structure, the {\tt purge} command may be applied which will delete
all empty sites in the crystal and the given relation between atom
index and unit cell/site is invalid.  The authors generally do {\bf
not} recommend the usage of the 'purge' command.

%------------------------------------------------------------------------

\section{Saving structures \label{struc-save}}

A particular structure stored in \Discus can be saved to a file
using the {\tt save} command. In cases where {\tt save} is followed
by a filename, the structure is directly saved using the currently
set options. \par

If {\tt save} is entered alone, \Discus will enter a {\tt save}
sub level which allows the user to select various options and select
all or only particular atoms to be saved. Note that the settings are
valid for all subsequent calls of {\tt save filename}. Generally the
user can choose between the old structure file format (see section
\ref{struc-read}) and the new keyword controlled structure file
format, which is the recommended format to be used. Beside the
required keywords the user is able to select which additional
keyword shall be written to the file. Note that in cases where a
structure containing molecules is saved without selecting the {\tt
molecule} keyword, the molecule information of the crystal will be
lost. Check the online help for {\tt save} for a complete list of
options.

Starting with \Discus Version 5.0 you can save a structure into
an internal memory location within the CPU instead of onto the disk.
Simply prepend the filename with the characters "internal". \Discus
will allocate memory to store  the structure. To recover the structure
read it with {\tt read} {\tt cell} or {\tt stru}. The filename must
of course start again with "internal". This internal storage is highly
recommended for temporary or intermediate structures that you do not 
really need. This holds especially during a refinement on a high 
performance computer cluster.

%------------------------------------------------------------------------

\section{Exporting structures \label{struc-expo}}

\Discus can export the current crystal structure for two purposes.
The first purpose serves to use the structure with a different
program, the second serves to display the crystal structure.

\subsection{Exporting to other programs \label{struc-other}}

Currently \Discus can export the structure for the programs 
listed in Table  \ref{struc-exp-tab1}. 


\begin{table}[!tbh]
\centering
\begin{tabularx}{\textwidth}{|p{30mm}|X|}
  \hline
  {\bf Format } & {\bf Description} \\
  \hline\hline
  CIF & A minimum content CIF file\\
      & export cif, <filename>.cif \\
      & \Discus writes a minimal CIF file. All atoms are transformed
        into a single unit cell. Along each of the axes the new unit
        cell dimensions are an integer multiple of the original unit
        cell. The factors are chosen to the samllest possible integer
        value. The space group is set to P1\\
      & Information on molecules is lost.\\
  \hline
  RMCPROFILE  & Crystal structure storage format as used by RMCProfile\\
              & export rmcprofile, <filename>\{.rmc6f\}, version:6\\
              & export rmcprofile, <filename>\{.rmc6f\}, version:7\\
              & \Discus writes a RMC6F file. the extension ".rmc6f" is
                optional and will be appended to the filename automatically
                if absent. \\
              & The optional version: parameter allows to switch between 
                version 6 and 7. For version 7, a Deuterium atom is 
                writen as "2H". Furthermore, for version 7 the optional
                string [type\_number] is appended to the atom type. 
                This type number allows to distinguish atoms of identical
                chemical nature that are on different sites in the asymmetric unit.\\
              & The unit cell parameters are multiplied by the smallest
                integer number to include the whole crystal in one unit 
                cell.\\
              & The origin of the \Discus structure is shifted such that all 
                atom coordinates are positive.\\
              & Information on molecules is lost.\\
  \hline
  SHELX & Write an ins file as used by SHELX programs \\
        & export shelx, <filename>.ins\\
        & A regular SHELX instruction file is written. \\
        & Identical atom types on different sites are written as
          "O", "O\_\_A", , "O\_\_B", etc. \\
        & Information on molecules is retained.\\
  \hline
  VASP  & Writes a POSCAR file as used by VASP \\
        & export vasp, <filename>\\
        & export poscar, <filename>\\
        & The filename is optional and defaults to "POSCAR".\\
        & Atoms are written as fractional coordinates of the
          crystal structure. \\
        & The origin of the \Discus structure is shifted such that all 
          atom coordinates are positive.\\
  \hline
\end{tabularx}
\caption{\label{struc-exp-tab1} List of supported export file formats}
\end{table}

\subsection{Exporting to visualization programs \label{struc-visu}}

The structure, or parts thereof can be exported to file in formats
suitable for display by corresponding programs.  Currently supported
programs are \Kuplot which is part of the \Discus package
and a number of structure plotting packages such as {\it ATOMS} and
{\it DIAMOND}.  A simplified {\it CIF} file can be read by most
crystallographic packages. 
Check the online help for an up-to-date list of
supported programs.
\par

Different export option are available in the {\tt plot} sub level of
\discus.  Individual atom or molecule types to be included in
the output might be selected.  The extend of the crystal to be
exported as well as an optional limitation to a given slice within
the crystal allow the user to select specific regions within the
crystal.  Such a slice is defined by a point ${\bf v}$ in real space
its thickness and its normal. For each atom in the crystal the
vector from the point ${\bf v}$ to the atom is projected onto the
real space normal to the slice. If the length of this projection is
less than the thickness, the atom is plotted, else not. \Discus
simply writes the coordinates of selected atoms.  The sequence of
indices in the output file can be defined by the user as any
permutation of x,y,z; x,z,y;...  The output for \Kuplot will
result in a projection of the structure along one of the
crystallographic axes or a projection along the normal of the slice
onto the plane, while the {\it GNUPLOT} format can be used for a
three dimensional viewing.  The file format for \Kuplot
includes marker type, color and size for the individual atom or
molecule representation.  These properties can be altered using the
{\tt set} command within the {\tt plot} sub level.  Assuming the
structure has been written to file {\it atoms.xy}, the following
\Kuplot macro would be appropriate:

\begin{MacVerbatim}
    load cr,atoms.xy    ! loading the data file
    plot                ! plot it
\end{MacVerbatim}

For more information about the program \Kuplot refer to the
\Kuplot manual or online help.

To import into a CIF file viewer export with the {\tt program cif}
option:

\begin{MacVerbatim}
  plot
    program cif
    sel     all
    outfile your_name.cif
    run
  exit
\end{MacVerbatim}

Several of the CIF file viewer programs will automatically 
create a copy of an atom whose factional coordinate along 
any axis is zero. This atom will then be displayed with
coordinates "one" along the axis. To prevent this behavior,
\Discus shifts the structure such that the origin is a 
little "left" of all fractional coordinates. 

As a CIF file contains atoms within a single unit cell
only,  \Discus transforms the original unit cell to a
supercell to enclose the full crystal in a single cell.
To avoid issues with atoms whose fraction coordinates
are "zero" or "one", the origin is shifted. The new 
unit cell is set to the smallest integer multiple of
the original cell plus a margin of one extra unit cells.

If the program is set to "{\tt jmol}", \Discus will still 
write a CIF file to disk. In addition you can use the 
optional parameter {\tt plot:inter} on the {\tt run} 
command to start the Jmol program with an interactive
plot of the structure, see section \ref{struc-plot} for
fulöl details.

%------------------------------------------------------------------------

\section{Importing structures \label{struc-import}}

\Discus can import crystal structures from a few external file formats, 
as listed
in Table \ref{struc-imp-tab1}. The relevant commands are found in the 
{\tt import} menu. \Discus will read the import structure and automatically
save a \Discus type file, whose extension has been replaced by {\bf stru}.

\begin{table}[!tbh]
\centering
\begin{tabularx}{\textwidth}{|p{30mm}|X|}
  \hline
  {\bf Format } & {\bf Description} \\
  \hline\hline
  CIF         & The Crystallographic Information File format, as    
                described in the International Tables 
                \cite{tables} Vol. F.\\
  \hline
  RMCPROFILE  & Crystal structure storage format as used by RMCProfile\\
  \hline
  SHELX & Import a res or ins file as used by SHELX programs \\
  \hline
  CRYSTALMaker & Crystal structure storage format as used by CrystalMaker \\
  \hline
\end{tabularx}
\caption{\label{struc-imp-tab1} List of supported import file formats}
\end{table}

%------------------------------------------------------------------------

\section{Plotting / Visualization \label{struc-plot}}

The main task of the \Discus section certainly is to create a wide 
range of structure types, to modify these and to calculate the diffraction
pattern respectively the Pair Distribution Function. It is of course 
helpful to visualize a structure. The {\tt plot} menu in \Discus offers
these possibilities. 

Predominantly one will write an output file that can be interpreted 
by appropriate crystal structure visualization programs. While 
\Discus offers several output formats, probably the most widely accepted
file format is the {\tt CIF} file format. As the {\tt CIF} file 
format is restricted to a single unit cell, \Discus will transform the
structure into a large unit cell that is an integer multiple of the current  
unit cell dimensions. To deal with programs that like to wrap around
atom positions, the structure is shifted to avoid atom coordinates that
are equal to zero or one.

Starting with version {\tt 5.22} \Discus offers the possibility to
open JMOL, if JAVA and JMOL have been installed on your computer. 
On computers with Linux or MacOS, you need to install JAVA and JMOL. 
The Windows version includes a JMOL executable. You still need to 
install the Java Run time Environment JRE. 

To start a JMOL window, set the program to "jmol" and include 
on the {\tt run} command line
the optional parameter {\tt "plot:inter"} 
. \Discus will still write a {\tt CIF} file to 
the hard disk and pass a short script to JMOL to load this {\tt CIF} file.

\begin{MacVerbatim}
 1  discus
 2  read
 3    cell quartz.cif, 2, 2, 2
 4  plot
 5    program   jmol
 6    outf      quartz_plot.cif
 7    extend    all
 8    select    all
 9    bond      Si, O, 1.2, 1.8, 0.1, 0.0, 0.0, 0.0
10    polyhedra Si, O, nmin:4,nmax:4, dmin:1.2, dmax:1.8, face:flat, hue:solid, color:auto
11    background 230, 230, 230
12    run plot:inter, kill:yes
13  exit
14  exit  ! back to suite
\end{MacVerbatim}

In this example, \Discus reads a {\tt CIF} file of Quartz and expands the 
structure to 2x2x2 unit cells. An output {\tt CIF} file is written as
{\tt quartz\_plot.cif}. In order not to overwrite the input {\tt CIF} file,
the name is modified to {\tt quartz\_plot.cif}. The {\tt bond} command
defines bonds from a central Silicon atom to neighboring Oxygen atoms
in a distance range from 1.2 to 1.8 \AA{}. The bonds will have a thickness
of 0.a \AA. The RGB color triplet is currently ignored, as JMOL has a
sensible standard color scheme. 

The {\tt polyhedra} command tells JMOL to plot polyhedra with the 
following (optional) properties:

\begin{tabularx}{\textwidth}{p{40mm}X}
nmin/nmax & Minimum/Maximum number of neighbors an atom must have to 
      be inside a polyhedron.  \\
dmin/dmax & Minimum/Maximum distance to the neighbors. Only neighbors
      in this range are taken into account to build a polyhedron. \\
face {"flat"|"collapsed"} & JMOL can draw a polyhedron with flat surfaces 
      or with indented surfaces. \\
hue {"solid"|"trans"} & A transparent polyhedron allows to see the 
      central atom. \\
color & With "auto" JMOL will take the color of the central atom 
      as template for the surfaces. \\
\end{tabularx}

In this example the bond distances specified by the {\tt bond} and the
{\tt polyhedra} command coincide. As JMOL takes only a single 
polyhedron definition you can use the additional {\tt bond} command to 
restrict more clearly, which polyhedra will be drawn.

Finally the background can be set as an RGB triplet.

The optional parameter {\tt plot:inter} initiates the JMOL plot. Omit this 
or set it to {\tt plot:none} to skip the JMOL plot. 

Every time you use the {\tt plot:inter} parameter, a new JMOL window 
will be started. The optional parameter {\tt kill:yes} allows to
close all previous JMOL windows.

%------------------------------------------------------------------------

\section{Properties \label{struc-properties}}

Within a crystal you will typically have a lot of different atom types. 
In addition to the type (Fe, Mn, Cl ...) atoms may be distinguished by
further properties. An atom might be within a molecule or be a separate
atom. \Discus labels these different properties as logical flags 
and updates these 
automatically if any changes occur. Many of the modification and output
tool can utilize these properties to include or reject a specific atom.
The properties are listed in table \ref{struc-prp-tab}:

\begin{table}[!tbh]
\centering
\begin{tabularx}{\textwidth}{|p{25mm}|X|}
  \hline
  {\bf Keyword} & {\bf Description} \\
  \hline\hline
  Normal      & The atom is a normal atom or a void.\\
  \hline
  Molecule    & The atom is part of a molecule \\
  \hline
  Domain      & The atom is within a domain. \\
  \hline
  Outside     & The atom is outside the crystal, i.e. on the other
                side of a limiting surface. Usually these atoms are
                voids as well.\\
  \hline
  External    & The atom is within a user defined distance to an 
                external boundary, either on the inside or outside.\\
  \hline
  Internal    & The atom is within a user defined distance to an 
                internal boundary, either on the inside or outside.
                Internal boundaries arise through domains.\\
  \hline
  Ligand      & The Atom is part of a ligand molecule that has been 
                placed on top of the crystal.\\
  \hline
\end{tabularx}
\caption{\label{struc-prp-tab} List of property flags}
\end{table}

In addition you can define further property like behaviour. 
This behaviour is based on connectivities \ref{struc-connect} that have 
been defined for an atom. These property like featureas allow to 
restrict the selection of atoms to specific types that have 
coordination numbers within a user defined range or whose 
coordination numbers are outside an excluded range of coordination
numbers.

%------------------------------------------------------------------------

\section{Connectivity \label{struc-connect}}

Many of the tools in \Discus need to reference neighboring atoms around
a given atom. It is very time consuming to search throughout the entire
crystal if an atom fulfills the desired neighborhood conditions. To 
avoid these length loops \Discus provides two styles to reference a 
neighboring atom. 

The first style is the fastest but it depends on the initial storage 
order of the atoms, which was described in section \ref{struc-int}. As
long as this sequence is maintained \Discus can reference a neighbor by
a vector relationship:

\begin{MacVerbatim} 
  1   set vec, 1,  1, 1,   1, 0, 0
  2   set vec, 2,  3, 4,   1,-2, 4
\end{MacVerbatim} 

The first parameter after the {\tt vec} keyword numbers the vector 
relationships. The next two parameters describe which atom suites with
a set of unit cells is related. The last three parameters describe the 
relative shift of the unit cell in which the second site, the neighbor, 
is found.  Line (1) would thus create a relationship between any atom
on site one in a unit cell to the atom on site 1 in a unit cell 
shifted by [1,0,0]. Line (2) creates a relationship between an atom
on site 3 in a unit cell and an atom on site 4 in a unit cell shifted 
by [1,$\overline{2}$,4] unit cells. These relationships hold for all
pairs throughout the crystal. In order to quickly calculate which 
atom in our internal sequence is referenced, the atoms must still be in 
their original order.

If you add a lot of atoms, or if you shape the crystal with the 
boundary commands (see section \ref{mod-surface}) and remove 
unnecessary outside voids, the internal sequence of atoms my change
dramatically. Under these circumstances the vector relationships 
to not work efficiently any longer. For that reason you will find 
several warnings with respect to the {\tt purge} command. To allow
a fast reference to neighboring atoms \Discus offers the 
{\tt connectivity} menu. This menu allows you to define relationships
between any pair of atoms through distance intervals:

\begin{MacVerbatim} 
  1   connectivity
  2      reset
  3      add  Zn, O , 1.90, 2.10, zn_o_short
  4      add  Zn, Zn, 3.10, 3.30, zn_zn_second
  5      add  O , Zn, 1.90, 2.10, o_o_short
  6      add  O , O , 3.10, 3.30, o_o_second
  7      create
  8      show
  9   exit
 10   ...
 11   connectivity
 12      recreate Zn, zn_zn_second
 13   exit
\end{MacVerbatim} 

In this example we clear all previous connectivity lists in line (2) and
add two connectivities. The first (line 3) relates Zn and O atoms that are within 
the interval 1.90 to 2.10\AA\  and provides a name {\tt zn\_o\_short} 
for this relationship. This relationship is uni directional, Zn is 
always at the center of the connectivity polyhedron. Thus to group 
all pairs, lines 5 and 6 add the opposite relationships with the oxygen 
as central atom.

The {\tt add} command simply creates the definitions, the actual 
lists are generated with the {\tt create} command. As the program 
has to perform a double loop over all atom pairs this may take a bit 
of time, but you will easily regain this time during a Monte-Carlo 
simulation.

In line 12 we recreate the connectivity {\tt zn\_zn\_second} around
Zinc atoms. This might be necessary, if atoms have been moved around.

Starting with version 5.14 two new optional features modify the 
build up of a connectivity.

The {\tt first} optional parameter {\tt first:<no>} allows you to limit
the number of neighbors that will be included in a connectivity to
the value <no>. This will include only the first <no> neighbors
that are closest in distance to the central atom. 
The string "first" is used to indicate that this 
is an optional parameter. 

This option might be needed, if you want to constrain the neighbors
around a central atom to a specific shape. You would choose for 
example {\tt first:4} to limit the connectivity to the four closest 
neighbors.

The optional parameter {\tt molescope:} allows to restrict the 
connectivity for atoms that are located within a molecule. 
Possible options are: {\tt ignore}, {\tt within} and {\tt outside}.
With {\tt ignore} the molecular status of the central atom and its 
neighbors is ignored. Any pair of atoms may be connected based on the
distance shell only. With the {\tt within} option, the central atom 
has to be within a molecule. Only those atoms within the distance shell
will be included in the connectivity that are within the same molecule
as the central atom. All other atoms are ignored, even if they are 
within the distance specified by <rmin> and <rmax>. This is useful
to constrain a covalent distance or bond angle in {\tt mmc}
while ignoring other atoms that may be close by.
The final option {\tt outside} acts in an opposite sense. The central 
atom has to be within a molecule, while the neighbors have to be 
outside this molecule. A typical use might be to build a connectivity
between a Hydrogen atom within a molecule to Oxygen atoms outside 
the molecule in order to establish a Hydrogen bond.

%------------------------------------------------------------------------
