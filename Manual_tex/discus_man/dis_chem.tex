%------------------------------------------------------------------------
% Chapter:  Analyzing defect structures
%------------------------------------------------------------------------

\chapter{Analyzing defect structures \label{chem}}

After describing how to create and change structures this chapter
will give a summary of \Discus functions to analyze a given
disordered structure. All those commands are accessed via the {\tt
chem} sub level of the program. This segment is entered using the
command {\tt chem} and left again by the command {\tt exit}. The
command {\tt mode} determines whether the entered commands operate
on atoms ({\tt mode atom}) or molecules ({\tt mode mole}). If
working with molecules the molecule type number replaces the atom
name or number as parameter for the different commands. Note that
not all commands are available in molecule mode. Also note, that the
\Discus cookbook \citep{nedpro} includes a complete chapter on
the analysis of disordered structures with many examples and
exercises.

%------------------------------------------------------------------------

\section{Occupancies \label{chem-occ}}

The command {\tt elem} will display the chemistry of the current
model crystal. Depending on the selected mode the relative abundance
of the present atom or molecule types is written on the screen. The
results are saved in the variable {\tt res[i]} (see chapter
FORTRAN style interpreter in the package manual). Entry {\tt res[1]}
contains the relative amount of atom types {\tt void} and entries 
{\tt res[1]} the relative amount of atom type 1.

The relative abundances given by the command
{\tt elem} are for the complete crystal and contain no further
information about its homogeneity. \Discus allows the user to
obtain concentration as well as correlation (see section
\ref{chem-corr}) distributions within the crystal using the command
{\tt homo}. This is achieved by sampling the crystal using a
predefined volume.

%------------------------------------------------------------------------

\section{Distortions \label{chem-dist}}

After analyzing concentrations within a crystal, this section will
focus on the analysis of distortions. The simplest way is to compute
average distances between different atom types in a given range. An
example is given below. First the desired range of interatomic
distances is limited to values between 4.5\AA \ and 5.5\AA \ (line
1). Next the bond length distribution is calculated using the 
\Discus command {\tt blen}. Here distances between {\it all} atoms
will be considered and the resulting distribution is written to the
file {\it chem.2.blen} (line 2).
%

\begin{MacVerbatim}
     1  set blen,4.5,5.5
     2  blen all,all,chem.2.blen
\end{MacVerbatim}
%
Rather than using {\tt all} as parameters for the command {\tt
blen}, atom names or numbers could be used to calculate the average
bond length only for specify atom types.
\par

The command {\tt blen} averages {\bf all} distances within the given
range specified with {\tt set blen}. In cases where more specific
information about distances in given directions is needed, the
command {\tt disp} must be used. Now it is necessary to enter
neighbor definitions which can either be specified by a distance
criteria or by individual vectors. In the example below the command
{\tt set vec} in line 1 defines vector 1 (first 1) to be from site 1
in one unit cell to site 1 (2nd and 3rd parameter) in the next unit
cell in x-direction (1,0,0 - last 3 parameters). Next this vector is
used as neighbor definition 1 (line 2). Finally the lattice
averages are computed between all present atom types (line 3).
%
\begin{MacVerbatim}
     1  set vec,1,1,1,1,0,0
     2  set neig,vec,1
     3  disp all,all
\end{MacVerbatim}

%------------------------------------------------------------------------

\section{Correlations \label{chem-corr}}

In this chapter the concept of correlations as a measure for those
two-body interactions will be introduced. Although diffuse
scattering contains only information about two-body interactions the
concepts described here can easily be extended to include multi-site
correlations. It should be noted that although these multi-site
interactions do not show up in the diffraction pattern directly,
they can have a constraining influence on two-body interactions and
thus affect the diffraction pattern. However, \Discus is
currently limited to the calculation of two-body interaction
averages. \par

We will talk about atom types in the following section, however, all
correlation related commands are available for molecules as well. To
work with molecules use the command {\tt mode mole} and specify
molecule types rather than atom types or names as parameters for the
commands.

\subsection*{Occupational correlations \label{chem-corr-occ}}

One definition of the correlation coefficient $c_{ij}$ between a
pair of sites $i$ and $j$ based on a statistical definition of
correlation \citep{we85} is given in equation \ref{chem-eq1}.
%
\begin{equation}
    c_{ij} = \frac {P_{ij} - \theta^{2}} { \theta (1 - \theta)}
    \label{chem-eq1}
\end{equation}
%
$P_{ij}$ is the joint probability that both sites $i$ and $j$ are
occupied by the same atom type and $\theta$ is its overall
occupancy.  Negative values of $c_{ij}$ correspond to situations
where the two sites $i$ and $j$ tend to be occupied by {\it
different} atom types while positive values indicate that sites $i$
and $j$ tend to be occupied by the {\it same} atom type.  A
correlation value of zero describes a random distribution.  The
maximum negative value of $c_{ij}$ for a given concentration
$\theta$ is $-\theta/(1-\theta)$ ($P_{ij}=0$), the maximum positive
value is +1 ($P_{ij}=\theta$). Let us look at a simple example:
%
\begin{MacVerbatim}
     1  read
     2  stru chem.1.stru
     3  #
     4  chem
     5  #
     6    set mode,quick,periodic
     7  #
     8    set vec,1,1,1, 1, 0, 0
     9    set vec,2,1,1,-1, 0, 0
    10    set vec,3,1,1, 0, 1, 0
    11    set vec,4,1,1, 0,-1, 0
    12  #
    13    set vec,5,1,1, 1, 1, 0
    14    set vec,6,1,1,-1, 1, 0
    15    set vec,7,1,1, 1,-1, 0
    16    set vec,8,1,1,-1,-1, 0
    17  #
    18    set neig,vec,1,2,3,4
    19    set neig,add
    20    set neig,vec,5,6,7,8
    21  #
    22    corr occ,zr,void
    23  #
    24  exit
\end{MacVerbatim}
%
The macro starts with the reading of the disordered structure (lines
1-2). After the {\tt chem} sublevel is entered (line 4) periodic
crystal boundaries are selected (line 6).  The parameter {\tt quick}
selects a faster neighboring finding algorithm which only works for
crystals arranged in the \Discus storage order (see section
\ref{struc-int}).  Note that $c_{\langle 10 \rangle}$ stands for the
nearest neighbor correlations in all four symmetrically equivalent
$<$10$>$ directions of the two dimensional cubic test crystal, i.e.
$c_{10}$, $c_{\overline{1}0}$, $c_{01}$ and $c_{0 \overline{1}}$.
All eight neighboring directions for $c_{\langle 10 \rangle}$ and
$c_{\langle 11 \rangle}$ are defined as vectors 1 to 8 in lines 8 to
16 of the macro file.  Next vectors 1 to 4 are grouped as
neighboring definition for $c_{\langle 10 \rangle}$ (line 18) and
vectors 5 to 8 for $c_{\langle 11 \rangle}$ (line 20).  The command
'set neig,add' (line 19) stores the current neighboring definition
and allows the definition of a new one.  Finally the correlations
for the defined neighboring directions are calculated (line 22). The
screen output looks like this:
%
\begin{MacVerbatim}
    Calculating correlations
        Atom types : A = ZR   and B = VOID

        Neig.     AA         AB         BB         # pairs    correlation
        -----------------------------------------------------------------
           1    50.49 %    48.99 %      .51 %       40000       -.3061
           2    71.04 %     7.89 %    21.07 %       40000        .7897
\end{MacVerbatim}
%
The program lists the probabilities for AA, AB and BB pairs and the
corresponding correlations $c_{ij}$.  Here the value for $c_{\langle
10 \rangle}$ is negative, i.e.  vacancy neighbors in $<$10$>$
directions tend to be avoided.  Neighboring vacancies in $<$11$>$
direction on the other hand are much more likely compared to a
random vacancy distribution indicated by the large positive value of
$c_{\langle 11 \rangle}$.

\subsection*{Displacement correlations \label{chem-corr-disp}}

The correlation coefficient $c_{ij}$ for displacement correlations
between two sites $i$ and $j$ is defined as:
%
\begin{equation}
    c_{ij} = \frac { \langle x_{i} x_{j} \rangle }
                   { \sqrt { \langle x_{i}^{2} \rangle
                             \langle x_{j}^{2} \rangle } }
    \label{chem-eq2}
\end{equation}
%
Here $x_{i}$ is the displacement of the atom on site $i$ from the
average position in a given direction and $\langle \cdot \rangle$
stands for the average over the crystal.  Again a negative value
describes a situation where the pairing of corresponding
displacements are less likely than in a crystal with random
displacements whereas a positive value indicates a larger than
random probability. The definition of neighbors is identical to the
example in the previous section.  Additionally the command {\tt set
neig,dir} is used to determine the displacement direction to be
used. Note that the displacement direction for the two sites $i$ and
$j$ is not necessarily the same, e.g.  one could be interested in
the correlation between the x-displacement on one site and the
y-displacement on the neighboring site.

\subsection*{Correlation fields \label{chem-corr-field}}

In the previous sections a correlation $c_{ij}$ for a given pair of
neighboring atoms was computed. An interesting information, however,
is how these correlations extend within the crystal. The program
\Discus allows the calculation of correlation fields for
occupational and displacement correlations (command: {\tt field}).

%------------------------------------------------------------------------

\section{Bond valence sums \label{chem-bval}}

The concept of {\it bond valence} has found wide applicability in
solid state chemistry \citep{bral85,brok91}. One application is
the use of bond-valence sums at atoms as a check on the
reliability of a determined local structure. The valence of an
atom i is calculated by the following empirical expression:
%
\begin{equation}
  V_{i} = \sum_{ij} \exp \left\{ \frac{r^{0}_{ij} - d_{ij}}{b} \right\}
  \label{chem-eq-bval}
\end{equation}
%
Here $r^{0}_{ij}$ and $b$ are the so-called bond valence parameters,
$d_{ij}$ is the distance between the central atom i and the
neighboring atom j. The sum goes over all nearest neighbors. The
bond valence parameters used by \Discus were taken from a list
compiled by I.D. Brown, McMaster University, Hamilton, Canada from
various references. Those parameters can be displayed using the
command {\tt show}. An example is given below:
%
\begin{MacVerbatim}
    discus > show bval,zr4+,o2-
    Bond valence parameters ZR4+ - O2-  : r0 =  1.9280 b =  0.3700
\end{MacVerbatim}
%
The parameters are specific for a given atom pair, here Zr$^{4+}$
and O$^{2-}$. Note that it is required to use the atom names
indicating the oxidation state like in the example above. It should
also be noted, that bond valence parameters are not available for
all pairs of atoms.

%------------------------------------------------------------------------

\section{Other tools \label{chem-other}}

In this chapter a short summary of functions available in the 'chem'
sublevel not discussed previously will be given. \par

The average structure of a crystal can be calculated using the
command {\tt aver}. Occupancies, average positions and standard
deviations for these positions are calculated. This command is not
available when working with molecules. The neighborhood of a given
atom or molecule can be examined using the commands {\tt neig} and
{\tt env}. The command {\tt neig} will use the currently stored
neighbor definitions whereas {\tt env} will display {\bf all} atoms
found in a given distance from the chosen origin. Finally the
conversion between atom index and unit cell / atom site can be made
using the equation given in section \ref{struc-int} or via the
command {\tt trans} in the {\tt chem} level of \discus.

%------------------------------------------------------------------------
