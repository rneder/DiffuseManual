%------------------------------------------------------------------------
% Chapter:  Monte Carlo simulation
%------------------------------------------------------------------------

\chapter{Monte Carlo simulation \label{mc}}

In general Monte Carlo (MC) methods can be described as statistical
simulation methods involving sequences of random numbers to perform
the simulation. The MC algorithm goes back to \citet{merorotete53}.
It works as follows: The total energy of the crystal is expressed as
a function of random variables such as as site occupancies or
displacements from the average structure. The simulation process
proceeds as follows: A site within the model crystal is chosen at
random and the associated variables are altered by some random
amount.  The energy difference $\Delta E$ of the configuration
before and after the change is computed.  The new configuration is
accepted if the transition probability $P$ given by
%
\begin{equation}
    P = \frac { \exp ( - \frac {\Delta E } { kT } ) }
              { 1 + \exp ( - \frac {\Delta E } { kT } ) }
    \label{mc-eq1}
\end{equation}
%
is less than a random number $\eta$, chosen uniformly in the range
[0,1]. $T$ is the temperature and $k$ Boltzmann's constant. Thus the
energy of the model crystal in minimized by the MC simulation. Note
the a higher temperature $T$ means that more moves will be accepted
that lead to a higher total energy $E$. In order to analyze the
defect structure of a particular system, the diffraction pattern of
the generated defect structure can be calculated and compared to the
experimental data. By adjusting near-neighbor interactions defining
the energy $E$ the model can be changed until a match with the
experiment is obtained. On the other hand MC simulations can be used
to explore the relationship between certain correlations and the
corresponding diffraction pattern, e.g. for teaching purposes. \par

In order to be able to carry out MC simulations, the energy $E$ of
the crystal needs to be defined.  The following sections give
energies for occupational as well as displacement correlations based
on the energy for an Ising model. Again readers are encouraged to
have a look at the Discus cookbook \citep{nedpro} which
contains a detailed discussion, examples and exercises on MC
simulations using Discus. \par

\begin{quote}
  {\it Note that the current version of \Discus contains a new
  rewritten module {\tt mmc} which allows to define more complex
  energies. The old {\tt mc} module is no longer available.}
\end{quote}

%------------------------------------------------------------------------
\section{Occupational disorder \label{mc-corr-occ}}

To represent the distribution of atom or molecule types within a
crystal, it is convenient to use Ising spin variables $\sigma_{i}
= \pm 1$.  Here $\sigma_{i} = 1$ means site $i$ is occupied with
type A and $\sigma_{i} = -1$ stands for type B being present on
site $i$.  Using these variables, the Hamiltonian (or energy)
takes the following form:
%
\begin{equation}
    E_{occ} = \sum_{i} H \sigma_{i} +
              \sum_{i}\sum_{n} J_{n} \sigma_{i} \sigma_{i-n} + \ldots
    \label{mc-eq2}
\end{equation}
%
The sums are over all sites $i$ and neighbors $n$.  The value
$\sigma_{i-n}$ refers to the occupancy (spin) of the neighboring
site $i-n$ of site $i$.  The quantities $J_{n}$ are pair interaction
energies corresponding to the neighboring vector defined by $i$ and
$n$.  Although the Hamiltonian can easily be extended to include
multi-site interactions, we will neglect those in this manual.  The
quantity $H$ is a single site energy which has the effect of an
external field in magnetic Ising models. Here it controls the
overall concentration.\par

The interaction energies $H$ and $J_{n}$ are initially unknown and
Discus employs a feedback mechanism to determine their values.
This is done in the following way: After a MC cycle, defined as the
number of MC steps needed to visit every crystal site once on
average, has been carried out, the resulting correlations are
computed and compared to the target values defined by the user.  If
the computed lattice averages are too low, then the corresponding
$H$ and $J_{n}$ are decreased by an amount proportional to the
difference between calculated and required value and {\it vice
versa}. It should be noted that even the simplest 2D Ising model
possesses a phase transition which is important to avoid when using
the described feedback mechanism during a MC simulation. \par

Starting with \Discus version 5.1 a damping algorithm gradually
decreases the changes of the interaction energies within the feed
back loop. This reduces fluctuations and provides a better 
convergence towards the desired correlations.
%------------------------------------------------------------------------
\section{Displacement disorder \label{mc-corr-disp}}

The MC simulation technique described above to create structures
with certain occupational correlations can quite simply been applied
to the case of displacements. Displacement correlations were already
discussed in section \ref{chem-corr-disp}. The Ising spin variables
$\sigma_{i}$ are replaces by continuous variables $x_{i}$ describing
the displacement of the atom or molecule on site $i$. Furthermore we
assume that the variable $x_{i}$ is Gaussian distributed with mean
zero ($ \langle x_{i} \rangle = 0$). Thus the Hamiltonian becomes:
%
\begin{equation}
    E_{dis} = \sum_{i} \sum_{n} J_{n} x_{i} x_{i-n}
    \label{mc-eq3}
\end{equation}
%
Here, as before, the first sum is over all sites $i$ and the second
sum is over all neighbors $n$ of the site $i$. In this equation there is
no term that depends on $x_{i}$ alone, since this would introduce a
shift in the average value $\langle x_{i} \rangle$. The MC
simulation operates as described before. Note that there are two
different modes to model displacements {\tt shift} and {\tt swdisp}.
Further details can be found in section \ref{rmc-mode}.
\par

Note that the displacements $x_{i}$ are taken in the direction
defined by {\tt set neig,dir} and the energy defined in equation
\ref{mc-eq3} is {\it blind} to displacement components in other
directions. It is recommended to use the 'swdisp' mode to maintain
the overall displacements within the crystal. In cases, however,
where the {\tt shift} mode is used, the generated shifts should be
restricted to directions corresponding to the correlations desired
in the MC simulation. If e.g. the x-displacements of one atom are
correlated with the y-direction of neighboring atoms, the shifts
should be restricted via the command {\tt set move} to the xy-plane.

%------------------------------------------------------------------------

\section{Creating distortions \label{mc-disp}}

In the previous section displacement correlations were introduced,
but the average displacements remained constant.  The modeling of
distortions {\it via} MC simulations works in a similar way. 
\Discus offers two different potentials. The first one uses a
Hamiltonian (equation \ref{mc-eq4}, where the atoms or molecules
move in harmonic potentials (Hooke's law).
%
\begin{equation}
    E_{h} = \sum_{i} \sum_{n} k_{n} [ d_{in} - \tau_{in} d_{0} ] ^{2}
    \label{mc-eq4}
\end{equation}
%
The sums are over all sites $i$ within the crystal and all neighbors
$n$ around site $i$.  The distance between neighboring atoms or
molecules is given by $d_{in}$ and the average distance is $d_{0}$.
The desired distortions are defined by the factor $\tau_{in}$. The
value $k_{n}$ is a force constant for each individual neighbor type
$n$. The second potential is the Lennard-Jones potential :
%
\begin{equation}
   E_{lj} = \sum_{i} \sum_{n \ne i}
            \left [ \frac{A}{d_{in}^M} - \frac{B}{d_{in}^N} \right ]
   \label{sro-eq-lennard}
\end{equation}
%
with
%
\begin{equation}
  A = D \frac{N}{N-M} \tau_{in}^M ~ \mbox{and} ~
  B = D \frac{M}{N-M} \tau_{in}^N.
  \label{sro-eq-lennard2}
\end{equation}
%
The sums are over all sites $i$ within the crystal and all
neighbors $n$ around site $i$.  The values for $A$ and $B$ are
calculated from the target distance $\tau_{in}$, {\it i.e.} the
distance where Lennard-Jones has its potential minimum, and from the
potential depth $D$ which must be negative.
\par
A standard Lennard-Jones potential is calculated with the repulsive
term and M=12 and the attractive term with N=6. \Discus allows
you to use other exponents as well.

A purely repulsive potential is used to create an even distribution
of defects within a crystal:

\begin{equation}
  E = -E_{\infty} + \left ( \frac{r-r_{min}}{scale} \right )^M
\end{equation}
Here $E_{infty}$ is the energy between two atoms at infinite 
distance. As the MC algorithm looks at changes, its value is not
relevant. Atoms at distances shorter than $r_{min}$ are placed
at essentially infinite energy, for longer distances the energy
changes according to the exponent M. A large value of M will
cause a steep descent at short distances, but the descent at longer 
distances might will be less relevant. To extend the effect of
the repulsive potential over long distances, choose a large 
value of {\tt scale}.
\par
Another potential function is the Buckingham potential:
\par
Angular distortions are realized with a potential energy:

\begin{equation}
  E = K \left (\Theta - \Theta_{ideal} \right )^2
\end{equation}
Here $\Theta_{ideal}$ is the indented ideal bond angle.

%------------------------------------------------------------------------

\section{Working with molecules \label{mc-mol}}

Discus allows the use of molecules. The command 'set mole'
selects the molecule types to be used for the MC simulation and
automatically switches Discus to molecule mode. On the other
hand 'set atoms' will return to atoms mode and select the atom
types to be used for the MC simulation. \par

All neighboring definitions work exactly as for atoms. However, the
site label used to define neighboring vectors ({\tt set vec}) still
refers to the atom site within the crystal regardless of the current
working mode.  Thus the user has to check which site of the unit
cell is occupied by the origin of the selected molecules and define
the vectors accordingly.  The use of the 'distance' mode to define
neighbors works straight forward, however, this mode is much slower
compared to using the vector definitions. All operation modes work
on rigid molecules.  Note, there are currently no MC (or RMC) moves
defining rotations of the molecules.  Rotations and other symmetry
operations can be realized by creating the wanted different
orientations of the molecules as different types using the symmetry
segment of Discus (see chapter \ref{cryst-sym}) and
subsequently using the {\it swap chemistry} mode.

\section{Examples \label{mc-exa}}

Since the publication of the Discus book \cite{nedpro} the Monte-Carlo part
has undergone quite a few changes and updates. The examples in this section
illustrate the current possibilities.

\subsection{Simple Short range order}

With this example a simple short range order example will be used to 
illustrate the main Monte-Carlo concepts as implemented into \discus.

The example uses a binary two dimensional crystal. This serves to speed up 
the simulations compared to a full 3D-model but does not impose a restriction
with respect to the possible concepts.

A header of such a simulation might include the following lines:

\begin{MacVerbatim}
  1  discus                     ! Switch to Discus section
  2  #                          ! Define useful variables
  3  variable integer,size_cr   ! Crystal size in unit cells
  4  variable real,comp         ! composition copper/gold
  5  variable real,c100         ! intended correlation along [100]
  6  variable real,c010         ! intended correlation along [010]
  7  variable real,a100         ! Achieved correlation along [100]
  8  variable real,a010         ! Achieved correlation along [010]
  9  variable real,heat         ! Pseudo temperature
 10  #
 11  size_cr = 100              ! Set crystal size
 12  comp = 0.5                 ! Set composition
 13  #
 14  c100 =  0.75               ! Set desired correlations
 15  c010 = -0.75
 16  heat =  1.5                ! set pseudo temperature
 17  #
 18  read                       ! Read asymmetric unit and build crystal
 19    cell crystal.cell, size_cr, size_cr, 1
 20  replace cu,Au,all,comp     ! replace cooper by gold on all sites in the 
 21                             !   unit cell
 22  chem
 23    set mode, quick, period, xy   ! Set proper periodic boundary conditions
 24    element                       ! Check composition
 25  exit
 26  #
 27  set parallel, useomp:parallel   ! Increase speed by parallel calculation
\end{MacVerbatim}

This header section does not contain any material that is specific to the 
Monte-Carlo section. The definitions in lines 11 through 16 are set within this
macro. Alternatively you can make the macros more flexible by using parameters
{\tt \$1, \$2} etc that are passed down to the macro.

The initial crystal is build in lines 18 and 19, here a simple primitive unit 
cell with a single Copper atom at 0,0,0 is used. A fraction of the Copper atoms
are replaced by Gold atoms in line 20. The initial distribution of these Gold
atoms throughout the crystal is random without any correlations between Copper and Gold.

Lines 22 to 25 set periodic boundary conditions along the x- and y-axes, and exclude
these along the z-axis, as the crystal dimension along this axis is just a 
single unit cell. These settings are actually the default upon reading a unit cell
that is expanded to a crystal of dimension 1 along any axis. 

Line 27 uses an option introduced in Version 6 to use parallel processing in 
order to speed up the calculation. This is the default behaviour. At the 
moment several parallel threads will modify the crystal independently during the
Monte-Carlo run. 


\begin{MacVerbatim}
  1  mmc                            ! Step into the Monte-Carlo menu
  2  rese                           ! Reset Monte-Carlo to initial state
  3  set neig,rese                  ! Reset neighbor definitions
  4  set vec ,rese                  ! Reset vectors between atom pairs
  5  #
  6  set vec,1, 1,1,  1, 0, 0       !  The first two vector pairs are for the 
  7  set vec,2, 1,1, -1, 0, 0       !  targets
  8  set vec,3, 1,1,  0, 1, 0       !
  9  set vec,4, 1,1,  0,-1, 0       !
 10  #
 11  set neig,number:1,vec,1,2      ! Define a first neighborhood of vectors 1 and 2
 12  #
 13  set neig,number:next,vec,3,4   ! Define a next neighborhood
 14  #
 15  set mode, 1.0, swchem,all      ! Define the modification mode
 16  #
 17  set targ,1,corr,cu,Au,c100,0.0,CORR  !  target 1 is for [100]
 18  set targ,2,corr,cu,Au,c010,0.0,CORR  !  target 2 is for [010]
 19  #
 20  set cyc, 500*n[1]              ! Define number of modification cycles
 21  set feed, 50*n[1]              ! Define number of feedback intervals
 22  set temp,heat                  ! Set pseudo temperature
 23  set finish, stop:cycles        ! Define termination criteria
 24  show                           ! Show current settings
 25  run                            ! Start the actual Monte-Carlo run
 26  set vec,5, 1,1,  1, 1, 0       ! These next vectors are to check achieved
 27  set vec,6, 1,1, -1,-1, 0       ! correlations along additional directions
 28  set vec,7, 1,1,  1,-1, 0       !
 29  set vec,8, 1,1, -1, 1, 0       !
 30  exit                           ! Go back to Discus menu
\end{MacVerbatim}

This is the actual Monte-Carlo macro to perform a short range order run. After
entering the {\tt mmc} Monte-Carlo menu, the menu is reset to the intial state
that is encountered at program start.

Lines 6 to 9 define four vectors for the pair correlations. The first number on
the {\tt set vec} line is a numbering of all vector relationsships. The vectors
will be referenced under these numbers in lines 11 and 13. 

The next two parameters, here both set to "1" refer to the sites within the unit 
cells for which we want to establish a correlations. To obtain the list of sites
within a unit cell, read an asymmetric unit and expand it to a single unit cell
with commands like:
 
\begin{MacVerbatim}
  read
    cell crystal.cell
  show atom, all
\end{MacVerbatim}

The list of all atoms will give you all sites within the structure.

The last three parameters describe whether the two sites are within the same unit
cell or in neighboring unit cells. The command in line 6 establishes a relationship
between an atom on site 1 in a chosen unit cell and an atom on site 1 in the 
neighboring unit cell separated by vector [1, 0, 0]. If there are more than one site 
per unit cell, you could also specify a command like {\tt set vec, 5, 1, 3,  0, 0, 0} to 
establish as 5th vector a relationship between an atom on site 1 in a unit cell and an atom on 
site 3 within the same unit cell.

In this simple example, the four pairs will be immediate neighbors separated by
$\pm \vec{a}$ and $\pm \vec{b}$. Lines 11 and 13 group these four vectors into 
two neighborhoods. The first neighborhood consists of vectors 1 and 2, the 
second neighborhood of vectors 3 and 4. For all pairs of atoms related by vectors 
within the identical neighborhood \Discus will try to achieve identical pair
correlations. In the current example, \Discus will sort the crystal while trying 
to attempt that all atom pairs separated by either $\vec{a}$ or $-\vec{a}$ 
will be correlated to each other. The type of correlation will be defined by the 
targets in lines 17 and 18. Simultaneously with the first correlation \Discus
will sort the atoms to create the desired second correlation along 
$\vec{b}$ and $-\vec{b}$. 

The optional parameter {\tt number:1} or {\tt number:next} defines the neighborhood
number and was introduced in version 6.7. The {\tt set target} command in 
lines 17 and 18 refers to this numbering
scheme. In order to enhance flexibility, you can specify the number either as a 
value (or as an expression) or as the phrase {\tt numer:next}, which will simply
increment the number. This increment works for the first neighborhood as well.

Note that \Discus versions prior to 6.7 required a {\tt set neig, add} line
between all neighborhood definitions, but {\bf not} after the last 
neighborhood definitions. The corresponding macro section would have been: 

\begin{MacVerbatim}
 10  #
 11  set neig,vec,1,2               ! Define a neighborhood of vectors 1 and 2
 12  set neig,add                   ! Add space for a next neighborhood
 13  set neig,vec,3,4
 14  #
\end{MacVerbatim}


Trivially, to sort the crystal, we must modify its structure. \Discus allows 
several different modifications to be carried out while {\it optimizing} the
structure. In this example, we only need a single type of modification, the 
exchange of two atom types. With the 

{\tt set mode, 1.0, swchem, all} 

command \Discus is instructed to choose two different atoms at random locations 
throughout the 
crystal. The initial energy state is computed and then the two atom types are 
exchanged. If the new energy state is lower than the inital, the new state 
is accepted. Otherwise the exchange is rejected with a probability
defined by Eq. \ref{mc-eq1}. 

In this example only one type of modification is used, thus the probability for
this move, the second parameter, is set to 1.0. The keyword {\tt swchem} 
instructs \Discus to switch the chemistry of the two selected atoms. The last 
parameter {\tt all} defines that the two atoms my be at any location throughout the 
crystal. Other values of this last parameter allow to restrict the choice 
of the two atoms to be in close proximity or to be at the same site within their 
respective unit cells. See the on-line help for more details.

At the core of any Monte-Carlo simulation lies the choice of the energy that is 
to be minimized. \Discus uses the 

{\tt set target, 1, corr, Cu, Au, c100, 0.0, CORR}

command to achieve this definition. In the macro listed above lines 17 and 18 
perform the task. The first numerical parameter refers to the neighborhood 
defined in lines 11 and 13. The next parameter, in this example {\tt corr}, 
defines the energy, here a chemical correlation or Occupational disorder as 
defined in section \ref{mc-corr-occ}. With this correlation type the pair 
of atom types specified in parameters 3 and 4 {\tt Cu, Au} 
will be sorted by atom type. If you give a positive value to the correlation
parameter \Discus will sort equal atom types as neighbors within the 
neighborhood that the current target points to. In this example, the 
last parameter, the string {\tt CORR} specifies that \Discus shall sort the
structure in order to achieve a correlation value given as parameter 5. Here 
correlation is specified as variable {\tt c100} to reflect the correlation 
along [1 0 0]. Any other variable name or an explicit numerical value would
of course be valid as well. The 6th parameter {\tt 0.0} is the starting value
for the Ising interaction energy of Eq. \ref{mc-eq2}. Initially the value 
required to achieve the desired correlation of {\tt c100} is unknown. \Discus
uses a feedback algorithm to determine this energy term. A positive correlation
implies a negative value of J and a negative correlation a positive value of J.

As an alternative the command can be specified as 

{\tt set target, 1, corr, Cu, Au, 0.0, e100, ENER}

The last parameter as string {\tt ENER} tells \Discus to use a fixed value 
of {\tt e100} for the Ising energy term J. If the value of this term is positive,
a negative correlation will result. The exact value of the final correlation 
depends on the pseudo temperature used to accept {\it wrong} moves, see Eq. 
\ref{mc-eq1}.



