%------------------------------------------------------------------------
% Chapter:  Monte Carlo simulation
%------------------------------------------------------------------------

\chapter{Monte Carlo simulation \label{mc}}

In general Monte Carlo (MC) methods can be described as statistical
simulation methods involving sequences of random numbers to perform
the simulation. The MC algorithm goes back to \citet{merorotete53}.
It works as follows: The total energy of the crystal is expressed as
a function of random variables such as as site occupancies or
displacements from the average structure. The simulation process
proceeds as follows: A site within the model crystal is chosen at
random and the associated variables are altered by some random
amount.  The energy difference $\Delta E$ of the configuration
before and after the change is computed.  The new configuration is
accepted if the transition probability $P$ given by
%
\begin{equation}
    P = \frac { \exp ( - \frac {\Delta E } { kT } ) }
              { 1 + \exp ( - \frac {\Delta E } { kT } ) }
    \label{mc-eq1}
\end{equation}
%
is less than a random number $\eta$, chosen uniformly in the range
[0,1]. $T$ is the temperature and $k$ Boltzmann's constant. Thus the
energy of the model crystal in minimized by the MC simulation. Note
the a higher temperature $T$ means that more moves will be accepted
that lead to a higher total energy $E$. In order to analyze the
defect structure of a particular system, the diffraction pattern of
the generated defect structure can be calculated and compared to the
experimental data. By adjusting near-neighbor interactions defining
the energy $E$ the model can be changed until a match with the
experiment is obtained. On the other hand MC simulations can be used
to explore the relationship between certain correlations and the
corresponding diffraction pattern, e.g. for teaching purposes. \par

In order to be able to carry out MC simulations, the energy $E$ of
the crystal needs to be defined.  The following sections give
energies for occupational as well as displacement correlations based
on the energy for an Ising model. Again readers are encouraged to
have a look at the Discus cookbook \citep{nedpro} which
contains a detailed discussion, examples and exercises on MC
simulations using Discus. \par

\begin{quote}
  {\it Note that the current version of \Discus contains a new
  rewritten module {\tt mmc} which allows to define more complex
  energies. The old {\tt mc} module is no longer available.}
\end{quote}

%------------------------------------------------------------------------
\section{Occupational disorder \label{mc-corr-occ}}

To represent the distribution of atom or molecule types within a
crystal, it is convenient to use Ising spin variables $\sigma_{i}
= \pm 1$.  Here $\sigma_{i} = 1$ means site $i$ is occupied with
type A and $\sigma_{i} = -1$ stands for type B being present on
site $i$.  Using these variables, the Hamiltonian (or energy)
takes the following form:
%
\begin{equation}
    E_{occ} = \sum_{i} H \sigma_{i} +
              \sum_{i}\sum_{n} J_{n} \sigma_{i} \sigma_{i-n} + \ldots
    \label{mc-eq2}
\end{equation}
%
The sums are over all sites $i$ and neighbors $n$.  The value
$\sigma_{i-n}$ refers to the occupancy (spin) of the neighboring
site $i-n$ of site $i$.  The quantities $J_{n}$ are pair interaction
energies corresponding to the neighboring vector defined by $i$ and
$n$.  Although the Hamiltonian can easily be extended to include
multi-site interactions, we will neglect those in this manual.  The
quantity $H$ is a single site energy which has the effect of an
external field in magnetic Ising models. Here it controls the
overall concentration.\par

The interaction energies $H$ and $J_{n}$ are initially unknown and
Discus employs a feedback mechanism to determine their values.
This is done in the following way: After a MC cycle, defined as the
number of MC steps needed to visit every crystal site once on
average, has been carried out, the resulting correlations are
computed and compared to the target values defined by the user.  If
the computed lattice averages are too low, then the corresponding
$H$ and $J_{n}$ are decreased by an amount proportional to the
difference between calculated and required value and {\it vice
versa}. It should be noted that even the simplest 2D Ising model
possesses a phase transition which is important to avoid when using
the described feedback mechanism during a MC simulation. \par

Starting with \Discus version 5.1 a damping algorithm gradually
decreases the changes of the interaction energies within the feed
back loop. This reduces fluctuations and provides a better 
convergence towards the desired correlations.
%------------------------------------------------------------------------
\section{Displacement disorder \label{mc-corr-disp}}

The MC simulation technique described above to create structures
with certain occupational correlations can quite simply been applied
to the case of displacements. Displacement correlations were already
discussed in section \ref{chem-corr-disp}. The Ising spin variables
$\sigma_{i}$ are replaces by continuous variables $x_{i}$ describing
the displacement of the atom or molecule on site $i$. Furthermore we
assume that the variable $x_{i}$ is Gaussian distributed with mean
zero ($ \langle x_{i} \rangle = 0$). Thus the Hamiltonian becomes:
%
\begin{equation}
    E_{dis} = \sum_{i} \sum_{n} J_{n} x_{i} x_{i-n}
    \label{mc-eq3}
\end{equation}
%
Here, as before, the first sum is over all sites $i$ and the second
sum is over all neighbors $n$ of the site $i$. In this equation there is
no term that depends on $x_{i}$ alone, since this would introduce a
shift in the average value $\langle x_{i} \rangle$. The MC
simulation operates as described before. Note that there are two
different modes to model displacements {\tt shift} and {\tt swdisp}.
Further details can be found in section \ref{rmc-mode}.
\par

Note that the displacements $x_{i}$ are taken in the direction
defined by {\tt set neig,dir} and the energy defined in equation
\ref{mc-eq3} is {\it blind} to displacement components in other
directions. It is recommended to use the 'swdisp' mode to maintain
the overall displacements within the crystal. In cases, however,
where the {\tt shift} mode is used, the generated shifts should be
restricted to directions corresponding to the correlations desired
in the MC simulation. If e.g. the x-displacements of one atom are
correlated with the y-direction of neighboring atoms, the shifts
should be restricted via the command {\tt set move} to the xy-plane.

%------------------------------------------------------------------------

\section{Creating distortions \label{mc-disp}}

In the previous section displacement correlations were introduced,
but the average displacements remained constant.  The modeling of
distortions {\it via} MC simulations works in a similar way. 
\Discus offers two different potentials. The first one uses a
Hamiltonian (equation \ref{mc-eq4}, where the atoms or molecules
move in harmonic potentials (Hooke's law).
%
\begin{equation}
    E_{h} = \sum_{i} \sum_{n} k_{n} [ d_{in} - \tau_{in} d_{0} ] ^{2}
    \label{mc-eq4}
\end{equation}
%
The sums are over all sites $i$ within the crystal and all neighbors
$n$ around site $i$.  The distance between neighboring atoms or
molecules is given by $d_{in}$ and the average distance is $d_{0}$.
The desired distortions are defined by the factor $\tau_{in}$. The
value $k_{n}$ is a force constant for each individual neighbor type
$n$. The second potential is the Lennard-Jones potential :
%
\begin{equation}
   E_{lj} = \sum_{i} \sum_{n \ne i}
            \left [ \frac{A}{d_{in}^M} - \frac{B}{d_{in}^N} \right ]
   \label{sro-eq-lennard}
\end{equation}
%
with
%
\begin{equation}
  A = D \frac{N}{N-M} \tau_{in}^M ~ \mbox{and} ~
  B = D \frac{M}{N-M} \tau_{in}^N.
  \label{sro-eq-lennard2}
\end{equation}
%
The sums are over all sites $i$ within the crystal and all
neighbors $n$ around site $i$.  The values for $A$ and $B$ are
calculated from the target distance $\tau_{in}$, {\it i.e.} the
distance where Lennard-Jones has its potential minimum, and from the
potential depth $D$ which must be negative.
\par
A standard Lennard-Jones potential is calculated with the repulsive
term and M=12 and the attractive term with N=6. \Discus allows
you to use other exponents as well.

A purely repulsive potential is used to create an even distribution
of defects within a crystal:

\begin{equation}
  E = -E_{\infty} + \left ( \frac{r-r_{min}}{scale} \right )^M
\end{equation}
Here $E_{infty}$ is the energy between two atoms at infinite 
distance. As the MC algorithm looks at changes, its value is not
relevant. Atoms at distances shorter than $r_{min}$ are placed
at essentially infinite energy, for longer distances the energy
changes according to the exponent M. A large value of M will
cause a steep descent at short distances, but the descent at longer 
distances might will be less relevant. To extend the effect of
the repulsive potential over long distances, choose a large 
value of {\tt scale}.
\par
Another potential function is the Buckingham potential:
\par
Angular distortions are realized with a potential energy:

\begin{equation}
  E = K \left (\Theta - \Theta_{ideal} \right )^2
\end{equation}
Here $\Theta_{ideal}$ is the indented ideal bond angle.

%------------------------------------------------------------------------

\section{Working with molecules \label{mc-mol}}

Discus allows the use of molecules. The command 'set mole'
selects the molecule types to be used for the MC simulation and
automatically switches Discus to molecule mode. On the other
hand 'set atoms' will return to atoms mode and select the atom
types to be used for the MC simulation. \par

All neighboring definitions work exactly as for atoms. However, the
site label used to define neighboring vectors ({\tt set vec}) still
refers to the atom site within the crystal regardless of the current
working mode.  Thus the user has to check which site of the unit
cell is occupied by the origin of the selected molecules and define
the vectors accordingly.  The use of the 'distance' mode to define
neighbors works straight forward, however, this mode is much slower
compared to using the vector definitions. All operation modes work
on rigid molecules.  Note, there are currently no MC (or RMC) moves
defining rotations of the molecules.  Rotations and other symmetry
operations can be realized by creating the wanted different
orientations of the molecules as different types using the symmetry
segment of Discus (see chapter \ref{cryst-sym}) and
subsequently using the {\it swap chemistry} mode.
