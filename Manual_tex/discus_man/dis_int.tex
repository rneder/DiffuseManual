%------------------------------------------------------------------------
% Chapter:  Introduction
%------------------------------------------------------------------------

\chapter{Introduction}
\section{What is DISCUS ?}

\Discus \citep{prne97} is intended as a versatile tool to
simulate crystal structures and the corresponding intensity
distribution in reciprocal space.  The program offers several
features that enable the user to easily generate a structure and
to introduce various defects.  The main intend is to simulate
defect structures, the program, however, is not limited in that
respect.  Ideal structures can be simulated as well and usually
will form the basis from which the defect structure is developed.
\par

The program can read a whole crystal or the asymmetric unit of a
unit cell. The latter is expanded to the whole unit cell by use of
the space group symbol. The definition of rigid molecules is
supported by \discus. The structure can be stored as structure
file or certain layers or projections of the crystal can be saved
for graphical display. The program \Discus is part of the
diffuse program package. Several tools are available to
modify single atoms or molecules within the structure or to alter
the complete crystal. These tools include thermal displacements,
waves, domains, stacking faults and generalized symmetry
operations. One feature of \Discus is the possibility to
introduce correlated defects and distortions using Monte Carlo (MC)
simulations. Various tools allow the analysis of a given defect
structure including the calculation of the PDF of a given structure.
\par

The Fourier transform segment of the program allows the user to
calculate  the intensity distribution along a line, arbitrarily 
oriented plane or a three dimensional volume
through reciprocal space. The resolution in reciprocal space can be
chosen by the user. The resulting intensity maps can be written in
several output formats, to be displayed by standard visualization
programs, the program {\it KUPLOT} or be printed directly.
The list of output formats includes the NeXus format.  Neutron-
as well as X-ray and electron scattering can be calculated. 
\Discus supports
the subtraction of the average structure factor and the calculation
of the diffuse intensities as average over many small crystal
volumes to create smooth noise free diffraction pattern. A detailed
discussion is given in chapter \ref{four}. The direct analysis of
measured diffuse scattering is possible using the Reverse Monte
Carlo segment of the program. The program allows the refinement of
the scattering intensity directly as well as the refinement of the
PDF. \par

The program uses a command language to interact with the user. No
predefined requirement exists for a given sequence of commands. The
commands can be typed at the \Discus prompt, or read from a
macro file. The command language includes a FORTRAN style
interpreter that allows the user to program loops and logical
structures. Several real and integer variables as well as structural
variables can be used to design the intended defect structure. On
line help is provided which gives the user information on any of the
\Discus commands as well as examples for typical \Discus
sessions. More detailed information how to get further help is given
in section \ref{get}.

%------------------------------------------------------------------------

\section{Getting started \label{get}}

The \Discus section is reached from the main \Suite by typing the command
{\tt discus} at the the  prompt of the \suite.

The program uses a command language to interact with the user.  The
command {\tt exit} terminates the program and returns control to the
shell.  All commands of \Discus consist of a command verb,
optionally followed by one or more parameters.  All parameters must
be separated from one another by a comma ",".  There is no
predefined need for any specific sequence of commands.  \Discus
is case sensitive, all commands and alphabetic parameters MUST be
typed in lower case letters.   Basic 
line editing and recall of commands is possible.  For more
information refer to the reference manual or check the online help
using ('help command input').  Names of input or output files are to
be typed as they will be expected by the local operating system.  
If necessary
include a path to the file.  All commands may be abbreviated to the
shortest unique possibility. At least a single space is needed
between the command verb and the first parameter.  No comma is to
precede the first parameter. iAll text to the right of a 
hash tag "{\tt \#}" or exclamation mark "{\tt !}" is considered a
comment. iThis does not include these signs within a character string
that is enclosed by single or double quotation marks.  \par

The symbols used throughout this manual to describe commands,
command parameters, or explicit text used by the program \Discus 
are listed in Table \ref{sym-tab}. There are several sources
of information, first \Discus has a build in online help, which
can be accessed by entering the command {\tt help} or if help for a
particular command $<$cmd$>$ is wanted by {\tt help $<$cmd$>$}. This
manual describes background and principle functions of \Discus
and should give some insight in the ways to use this program. \par

\begin{table}[!tbh]
\centering
\begin{tabularx}{\textwidth}{|p{30mm}|X|}
  \hline
  {\bf Symbol} & {\bf Description} \\
  \hline\hline
  "text"     &  Text given in double quotes is to be understood as typed. \\
  \hline
  $<$text$>$ &  Text given in angled brackets is to be replaced by an
                appropriate value, if the corresponding line is used
                in \Discus. It could, for example be the actual name
                of a file, or a numerical value. \\
  \hline
  {\tt text} &  Text in single quotes exclusively refers to \Discus
                commands. \\
  \hline
  $[$text$]$ &  Text in square brackets describes an optional parameter or
                command. If omitted, a default value is used, else
                the complete text given in the square brackets is to
                be typed. \\
  \hline
  \{text $|$ text\} &  Text given in curly brackets is a list of alternative
                parameters. A vertical line separates two alternative,
                mutually exclusive parameters. \\
  \hline
\end{tabularx}
\caption{\label{sym-tab}Used symbols}
\end{table}

As we have mentioned before, \Discus is controlled by a powerful command
language which is common to all programs in the \Discus package. 
A detailed description of the command language can be found in the 
\Discus Reference Guide which is included in the distribution.
Variables specific to the program \Discus are shown in Tables 
\ref{v1-tab} and \ref{v2-tab}. \Discus also provides a number of structure
related functions which are listed in Tab. \ref{func-cryst} and
\ref{func-cryst2}.

\begin{table}[!tbh]
\centering
\begin{tabularx}{\textwidth}{|p{30mm}|X|}
  \hline
  {\bf Variable} & {\bf Description} \\
  \hline\hline
  n[1]      & Number of atoms within the crystal \\
  n[2]      & Number of different scattering types, i.e. atoms \\
  n[3]      & Number of atoms within the unit cell \\
  n[4]      & Number of molecules within the crystal \\
  n[5]      & Number of different molecule types \\
  n[6]      & Number of molecules within the unit cell \\
  n[7]      & Number of non-void atoms in the crystal \\
  \hline
  cdim[i,1] & Lowest coordinate of any atom (i=1,2,3 for x,y,z) \\
  cdim[i,2] & Highest coordinate of any atom (i=1,2,3 for x,y,z) \\
  \hline\hline
  env[i]    & Index of neighboring atoms after 'find' command \\
  menv[i]   & Index of neighboring molecules after 'find' command \\
  \hline\hline
  lat[i]    & Lattice parameters \
              (i=1..6 for $a,b,c,\alpha,\beta,\gamma$\\
  vol[1]    & Unit cell volume \\
  rvol[1]   & Reciprocal unit cell volume \\
  \hline
  sym\_n[1] & Space group number of the structure\\
  \hline
\end{tabularx}
\caption{\label{v1-tab}Crystal related variables}
\end{table}

\begin{table}[!tbh]
\centering
\begin{tabularx}{\textwidth}{|p{30mm}|X|}
  \hline
  {\bf Variable} & {\bf Description} \\
  \hline\hline
  m[i]   & Number of scattering type (i.e. atom type) for atom i \\
  b[j]   & isotropic thermal factor B for atom {\bf type} j \\
  occ[j]   & occupancy parameter for atom {\bf type} j \\
  at\_name[i] & Atom name for atom i\\
  at\_type[j] & Atom name for atom {\bf type} j\\
  in\_mole[i] & Molecule in which atom i is \\
  prop    [i] & Property of atom i \\
  surf[i,k]   & Surface vector for atom i, components k=1,2,3\\
  \hline
  at\_type[i] & Atom name for atom type i\\
  \hline
  x[i]   & fractional x position of atom i \\
  y[i]   & fractional y position of atom i \\ 
  z[i]   & fractional z position of atom i \\
  \hline\hline
  mol\_cont[i,j] & Index of atom j in molecule i \\
  mol\_len[i]    & Number of atoms in molecule i \\
  mol\_biso[i]   & Isotropic B-value for molecule i\\
  mol\_clin[i]   & Linear correlation term for molecule i \\
  mol\_cquad[i]  & Quadratic correlation term for molecule i \\
  mol\_dens[i]   & Density of object i \\
  mol\_type[i]   & Type of object / molecule / domain i \\ 
  \hline\hline
  pdf\_dens[i]   & Number density for pdf calculations \\
  pdf\_scal[i]   & Scale factor   for pdf calculations \\
  \hline
\end{tabularx}
\caption{\label{v2-tab}Variables related to individual atoms and
         molecules}
\end{table}

\begin{table}[!tbh]
\centering
\begin{tabularx}{\textwidth}{|p{10mm}|p{47mm}|X|}
  \hline
  {\bf Type} & {\bf Name} & {\bf Description} \\
  \hline\hline
  real & \raggedright bang(u1,u2,u3, v1,v2,v3 [,w1,w2,w3]) &
       Returns the bond angle in degrees between {\bf u} and {\bf v}
       at site {\bf w}. If {\bf w} is omitted, the angle between
       direct space vectors {\bf u} and {\bf v} is returned. \\
  real & \raggedright blen(u1,u2,u3 [,v1,v2,v3]) &
       Returns the length of the real space vector {\bf v}-{\bf u}.
       The vector {\bf v} defaults to zero. \\
  real & \raggedright dstar(h1,h2,h3 [,k1,k2,k3]) &
       Returns the length of reciprocal vector {\bf k} - {\bf h} in
       \AA$^{-1}$. Vector {\bf k} defaults to zero. \\
  real & \raggedright rang(h1,h2,h3, k1,k2,k3 [,l1,l2,l3]) &
       Returns the angle between reciprocal vectors {\bf k} - {\bf h} and
       {\bf k} - {\bf l} at site {\bf k}.  If {\bf l} is omitted, the angle
       between reciprocal vectors {\bf h} and {\bf k} is returned. \\
  real & \raggedright scalpro(u1,u2,u3,v1,v2,v3 [,{"dd"|"rr"|"dr"|"rd"}]) &
       Returns the scalar product between the two vectors {\bf u} 
       and {\bf v}.  Both vectors may be given in direct
       or real space coordinates, flagged in parameter no 7. "d" 
       means real space, "r" reciprocal space. \\
  real & \raggedright vprod <u1>,<v1>,<w1>, <u2>,<v2>,<w2> [ ,<flag> ] &
       This is not a function but rather a command that calculates the 
       vector product {\bf 1} X {\bf 2}. The <flag> is a string of three
       characters that indicates whether the input vectors or the output
       vector are given in direct or reciprocal space coordinates. The
       three characters correspond to:

       first input vector,
       second input vector
       resulting vector

       "drd" means: vector one is given in direct space coordinates, vector 2
                is in reciprocal space coordinates (hkl) and the
                resulting vector product is to be given in direct space
                coordinates

       Any combination of "d" and "r" is allowed. \\
  \hline
\end{tabularx}
\caption{\label{func-cryst}Crystallographic functions}
\end{table}


\begin{table}[!tbh]
\begin{tabularx}{\textwidth}{|p{10mm}|p{47mm}|X|}
  \hline
  {\bf Type} & {\bf Name} & {\bf Description} \\
  \hline\hline
   logical & \raggedright isprop(atom [,[and:]property] [,or:property]) &
       Tests if an atom has the property / properties specified by
       property. The test can check a single property, if the 
       property is given by its name as in :

       'normal', 'molecule', 'domain', 'outside', 'external', 'internal',
       'ligand'. 

       Alternatively several properties can be checked 
       simultaneously if the property is a list of capital and/or 
       small letters as in 'NMDOEIL', 'nmdoeil'. A capital letter 
       signifies that the property must be present, a small letter 
       that the property must be absent. 

       The properties in the "and:" parameter must ALL be present, 
       i.e. they are tested with a logical "and". The properties in
       the "or:" parameter are tested with a logical "or", i.e. the
       function is true if any of the listed properties fulfills
       the condition (capital letter==present, small letter==absent).
       At least one of the two parameters must be present. If only 
       one parameter is given and the "and:" is omitted, the default
       is to test with the "and:" condition. \\
  \hline
\end{tabularx}
\caption{\label{func-cryst2}Crystallographic functions continued}
\end{table}

\section{Technical notes \label{tec}}

Most of the time you don not have to worry about the interior 
technical design of \discus. Most of the time the program will 
allocate its required memory and will not need user intervention.
As the command language is designe for utmost flexibility, the 
program will not know if you still need a piece of the allocated
memory or not. Again, most of the time this is not an issue, as 
few section of the program may require huge amount of memory. 
These sections are the Fourier menus (fourier and powder) and
particularly the pdf menu. THe latter can be very memory hungry 
if you deal with many different atom types and many different 
molecule types. It has been design to 
release temporary memory requirements. As such it should not really
need help with the memory issues. 

The Fourier menus might take up quite a bit of memory, if large
(3-dimensional) sections of reciprocal space are calculated. 
\Discus cannot know if you are done with the final Fourier 
map, After all, you might want to save it in the output menu 
in different formats. This prevents \Discus from a fully 
automatic release of the allocated memory. 

To free all memory, you can use the {\tt reset} command within
any menu or at the \Discus top level. This command will 
free all allocated arrays and furthermore set all menu 
settign back to the default values at program start. 

%------------------------------------------------------------------------
