%------------------------------------------------------------------------
% Chapter:  Fourier transform
%------------------------------------------------------------------------

\chapter{Fourier transform \label{four}}

\section{Calculating scattering intensities \label{four-int}}

Basically {\Discus} calculates the Fourier transform (neutron, 
X-ray, or electron)
according to the standard formula for kinematic scattering given in
equation \ref{four-eq1}.

\begin{equation}
        F({\bf h}) = \sum_{i=1}^{N} f_{i}({\bf h}) e ^{2 \pi i {\bf hr}_{i}}
                     \cdot e ^{- \frac{B |{\bf h}|^{2}}{4}}
        \label{four-eq1}
\end{equation}

The sum is over all N atoms in the crystal, where $f_{i}$ is the
atomic form factor (or scattering length in case of neutrons), ${\bf
r}_{i}$ the fractional coordinate of the atom.  The sum is
calculated at all points {\bf h} in reciprocal space.  The form
factors are tabulated and calculated once at each {\bf h} for all
species present in the crystal.  Optionally the Debye-Waller factor
is calculated.  Isotropic B are used for the representation of
thermal disorder. The realization of the actual code to calculate
the Fourier transform is based on the program {\it DIFFUSE}
\citep{buwe92}.  By limiting the calculation to an equidistant grid
and splitting the sum into sums over equal atom types, the computing
time required dropped by a factor of 4 to 6 (depending on compiler
and hardware) compared to calculating the sum given in equation
\ref{four-eq1} in a straight forward way.  More details about the
algorithm used can be found \citet{buwe92}. There are two
considerations when calculating the scattering intensities: Unwanted
scattering contributions due to the finite size of the model crystal
and the fact that the complete model scatters coherently, which
leads to unwanted high frequency oscillations in the calculated
pattern. Finite size effects are avoided by using a grid of $n/N$
where $N$ is the number of unit cells. Coherence might be limited by
using the {\tt lots} command in the {\tt four} module (see online
help). A detailed discussion of these topics can be found in
\cite{nedpro}.
\par

Let us consider a simple example. Assuming a structure has been read
or created, the scattering pattern can be calculated using these
commands:

\begin{MacVerbatim}
     1  four
     2    xray
     3    wvle moa1
     4    ll 0.0,0.0,0.0
     5    lr 2.0,0.0,0.0
     6    ul 0.0,2.0,0.0
     7    na 101
     8    no 101
     9    set aver,0.0
    10    lots off
    11    run
    12  exit
\end{MacVerbatim}

After entering the Fourier segment of {\Discus} (line 1),
$MoK\alpha$ radiation is selected in lines 2 and 3. Next the lower
left, lower right and upper left corner of the desired plane in
reciprocal space are specified (lines 4-6). The number of grid
points in both directions are set in lines 7 and 8. Finally the
subtraction of the average scattering intensity (Bragg peaks)
$\langle F \rangle$ is disabled (line 9) and the lot option is
switched off (line 10). Then the Fourier transform is calculated
(line 11). Note that one needs to save the result in the {\tt
output} segment of {\Discus} as we will discuss later in this
chapter.
\par
Starting with \Discus 5.0 you can also calculate a 3-dimensional 
volume in reciprocal space. A third axis is added to the 
values defined in the previous macro:

\begin{MacVerbatim}
     1  four
     2    xray
     3    wvle moa1
     4    ll 0.0, 0.0, 0.0
     5    lr 2.0, 0.0, 0.0
     6    ul 0.0, 2.0, 0.0
     7    tl 0.0, 0.0, 2.0
     8    na 101
     9    no 101
    10    nt 101
    11    set aver,0.0
    12    lots off
    13    run
    14  exit
\end{MacVerbatim}

This third axis defines the {\tt top left} corner of the reciprocal
space volume. The corresponding data can be saves as a series of slices
parallel to the layer defined by the corners {\tt ll, lr, ul} or as a
3D file in the NeXus file format or as a series of 2D slices.

Starting with \Discus 6.06 three new features were added to the Fourier
calculation.

The first new feature allows to calculate and merge several diffraction 
pattern. This allows you to simulate smaller crystals whose individual 
diffraction might be rather noisy but can be calculated rapidly. The
average of all the respective diffraction pattern will give you a smoother
final diffraction pattern. Consider the following macro:

\begin{MacVerbatim}
     1  fourier
     2    ...               ! Define all relevant Fourier settings
     3    set mode:init     ! Initialize the sum of Fourier calculations
     4  exit
     5  do LOOP=1, 5        ! A loop over five simulations
     6    @build_a_nice_crystal.mac   ! Details omitted, just some nice
     7                                ! simulation with local disorder
     8    fourier
     9      set mode:accumulate       ! Accumulate all individual calculations
    10      run
    11    exit
    12  enddo
    13  fourier
    14     set mode:finish            ! We are done with the accumulation
    15  exit
\end{MacVerbatim}

The command {\tt set mode:init} initializes the accumulation of several 
Fourier calculations. All prior calculations will be lost. To go back to
the single mode use the Fourier command {\tt set mode:single}. Within the 
loop the individual Fourier transforms are accumulated into temporary storage
as indicated by the {\tt set mode:accumulate} command. Once the loop is
finished, all pattern are merged with the final Fourrier command 
{\tt set mode:finish}. The merged pattern is now ready to be written 
using the {\tt output} menu.

The second new feature is the option to convolute the Fourier transform
with a Gaussian resolution function.

\begin{MacVerbatim}
     1  fourier
     2    ...               ! Define all relevant Fourier settings
     3    run  [ sigabs:[0.01, 1, 0, 0], sigord:[0.01, 0,1,0], sigtop:[0.01, 0,0,1]
     4  exit
\end{MacVerbatim}

\Discus allows to define a general triaxial ellipsoid. In the first example
above, the sigma is 0.01\AA{}$^{-1}$ along all three axes. With:

\begin{MacVerbatim}
     1  fourier
     2    ...               ! Define all relevant Fourier settings
     3    run  [ sigabs:[0.005, 1.0, 1.0, 0], sigord:[0.02, -1.0,1.0,0], sigtop:[0.01, 0,0,1]
     4  exit
\end{MacVerbatim}

three different half axes have been specified. Along reciprocal space direction [110] 
a sigma of 0.005\AA{}$^{-1}$, along [$\overline{1}10$] a sigma of 0.02\AA{}$^{-1}$
and finally along [001] a sigma of 0.01\AA{}$^{-1}$ is used. As pointed out in this
second example, the three half axes of the ellipsoid may point in arbitrary 
reciprocal space directions. They do not even have to be at mutual 90$^\circ$ to each 
other.

The third enhancement involves symmetry averaging. If requested, the crystals 
point group symmetry is used to average the calculated diffraction pattern. 
Currently the abscissa, ordinate and top axes must be parallel to a$^{*}$, 
b$^{*}$ and c$^{*}$ respectively. The center need not be at the origin of 
reciprocal space.  The average will work most evenly, however, if the center of the
calculated data is at the origin.

\par
%------------------------------------------------------------------------

\section{Powder diffraction \label{four-powder}}

In addition to the calculation of single crystal scattering
intensities, {\Discus} can also computer a powder diffraction
pattern of a given structure. There are two ways {\Discus} can
calculate powder intensities: One is a complete integration of
reciprocal space which is then mapped onto the desired range between
$2\Theta_{min}$ and $2\Theta_{max}$. The second algorithm uses the
Debye scattering equation to calculate the powder pattern. More details as well
as examples are given in our book \citep{nedpro}. Information about
the related commands are found in the online help using the command
{\tt help powder}.
\par


%------------------------------------------------------------------------

\section{Fourier methods \label {four-methods}}

Additional to the Fourier transform of a real space structure,
three other Fourier transforms are available in the current
version of {\Discus}: {\it difference Fourier, inverse Fourier}
and calculation of the {\it Patterson} (Fourier transform of
scattering intensities). All three of these Fourier transforms use
the following equation:
%
\begin{equation}
  \rho ({\bf r}) = \frac {1}{V} \sum_{i=1}^{N} F_{i} ({\bf h})
                   e ^{-2 \pi i {\bf hr}_{i}}
  \label{four-eq-inv}
\end{equation}
%
The resulting density function $\rho ({\bf r})$ and the Fourier
coefficient $F ({\bf h})$ take different meanings depending on the
intended inverse Fourier transformation listed in table
\ref{inv-tab}.
%
\begin{table}[!tbh]
\centering
\begin{tabularx}{\textwidth}{|p{22mm}|p{35mm}|X|}
  \hline
  {\bf Command} & {\bf F(h)} & {\bf $\rho$(r)} \\
  \hline\hline
  {\tt diff} & $F_{obs} - F_{calc}$ & difference scattering density \\
  {\tt inv}  & $F_{obs}$ & scattering density \\
  {\tt patt} & $I_{obs}$ & Patterson density \\
  \hline
\end{tabularx}
\caption{\label{inv-tab}List of available Fourier methods}
\end{table}
%
The Fourier coefficient are in general complex numbers.
Therefore, {\Discus} requires in most cases two input files for
the inverse Fourier transforms.  The allowed combinations are
listed in the help file and the command reference.

%------------------------------------------------------------------------

\section{Output file formats \label{four-out}}

All results of the Fourier transforms are written at the {\tt
output} segment of {\Discus}.  The following values can be saved
after you have calculated a single crystal diffraction pattern:
%
\begin{quote}
        {\it intensity, amplitude, real part, imaginary part, 
             phase angle (in degrees) and the 3D PDF }
\end{quote}
%
Following a powder diffraction pattern calculation, you can write:
%
\begin{quote}
        {\it intensity, S(Q), F(Q), $<f^2>$, $<f>^2$ 
             and the powder PDF}
\end{quote}
%
The single crystal Fourier transform calculates the real and 
imaginary part, all
other values are calculated at the time of output.  If the average
structure factor $\langle F \rangle$ was subtracted during the
calculation of the Fourier transform, the corresponding values of
$\langle F \rangle$ can be saved as well.  Note that if {\tt lots}
were used to calculate the Fourier transform, only the intensity
values can be saved to a file.  There is no need to calculate the
Fourier again, if several of the values listed are to be written to
file. Just define a new output value and output file name and run
the output again. The desired output format is selected using the
{\tt format} command. The available output formats are listed in the
table \ref{out-tab}.
%
\begin{table}[!tbh]
\centering
\begin{tabularx}{\textwidth}{|p{42mm}|X|}
  \hline
  {\bf Output format} & {\bf Description} \\
  \hline\hline
  {\tt stan} & Output is saved as real numbers in a format suitable for
               {\Kuplot} which is part of the {\Discus} program
               package. \\
  \hline
  {\tt hdf5} &  Create output in an HDF5 file format. \\
  \hline
  {\tt vesta} &  Create output in an ASCII file format suitable
                for input into the VESTA programm. \\
  \hline
  {\tt mrc} &  Create output in the MRC file format. \\
  \hline
  {\tt nexus} & Output is saved in the NeXus file format. See
               nexus.org for information on the NeXus file format
               and file viewers. \\
  \hline
  {\tt gnu}  & Real numbers suitable for the program {\it GNUPLOT} which
               can be obtained on most software archive sites. \\
  \hline
  {\tt pgm, ppm} & Integer bitmaps in PNM format as defined by Jef
                 Poskanzer (gray scale and colour). Various programs
                 are capable of reading PNM files and the
                 {\it pnmplus-package} is a freely available
                 collection of tools and conversion programs from PNM
                 to virtually any other graphics format. \\
  \hline
  {\tt post} &   Creates a Postscript bitmap suitable for direct printing or
                 to be imported by other programs. \\
  \hline
\end{tabularx}
\caption{\label{out-tab}Output formats for single crystal scattering intensities}
\end{table}
%
The standard and gnu output formats write the actual numbers calculated by the
Fourier transform.  The {\Kuplot} or standard file format is defined as:
%
\begin{MacVerbatim}
    1      na,no
    2      xmin,xmax,ymin,ymax
    3ff    z,z,z,z,z
\end{MacVerbatim}
%
The values 'na' and 'no' given in the first line define the size of
the section and are identical to the values given for the 'na' and
'no' commands at the Fourier segment.  {\Kuplot} uses the
coordinates along the abscissa and ordinate to scale the resulting
picture.  Since only two pairs of coordinates are read, the user has
to define the necessary indices. 'xmin' and 'ymin' are the is the
'x' and 'y' coordinates of the lower left corner in reciprocal
space, 'xmax' the 'x' coordinate of the lower right corner and
'ymax' the 'y' coordinate of the upper left corner.  Which of the
indices, h,k or l is interpreted as 'x' and 'y' coordinate depends
on the values given for the 'abs' and 'ord' commands at Fourier
sub level.  If for example the $(hkl)*[1 \overline{1} 0] = 0.0$ layer
is calculated, suitable values would be 'abs h' and 'ord l'. Now the
h index of the lower left corner is written as 'xmin', and the l
index as 'ymin'.  The values are written row by row, each row
consisting of the values along the abscissa.  An empty line
separates the rows in the output file.\par

The {it HDF5} output is a binary file written with the help of
the HDF5 library. Currently the format is compatible to the 
data read by {\bf Yell}, the 3D-PDF program by Arkadiy Simonov.
The \Kuplot section can read these files and step through the 3D-data.

Two file formats are intended to visualize 3D data with external 
programs such as Vesta and Chimera. For the Vesta program, 
\Discus writes two files, the {\tt base.vesta} and the {\tt base.grd}
files. The first contains display setting read by Vesta, while the
second file is an ASCII 3D bitap of the scattering data. Likewise,
the {\tt base.mrc} data are 3D data which can be read and displayed
for example with Chimera. 

As bitmaps for 3D-PDF calculations can be large, \Discus offers 
the option to write only a reduced section of the output into these
bitmaps. The {\tt range} command has several optional parameters to 
place and restrict the output range.

\begin{MacVerbatim}
    1  output
    2    ...
    3    range full    ! write full range of the bitmap
    4    range center:middle, pixel:[41, 51, 61]  
    5    range center:[101,131,51], pixel:[41, 51, 61]  
    6    range quad:llr
\end{MacVerbatim}

With line 3, the full range of the bitmap would be written into the output
file. Line 4 defines the center of the section at the middle of the complete
data set and writes a bitmap of full width with [41, 51, 61] pixels along
the three axes. With a command like in line 6 the center of the data is 
placed at the specified coordinate. Finally the option presented in line 6
will write a quadrant/octant of the data. These will always include the center 
of the complete data set and one of the four/eight possible quadrants/octants. 
In the example along the first two axes the left half is chosen and along the 
third axis the right half. 

The {\it GNUPLOT} output is written row by row, each data point
within each row in its own line of output.  The format of one such
row is: "$ h_{1}, h_{2}, z-value, h_{3}$".  Again the sequence of
indices h,k and l depends on the values given for the commands {\tt
abs} and {\tt ord} at Fourier sub menu. For the bitmap output the
calculated value is scaled linearly to values between 0 and 255. All
values less than a definable threshold using the 'thresh' command
are set to zero, all values above a maximum threshold are set to
255. Again, the bitmap is written row by row.  A color map included
in file {\it color.map} is used to attribute colors to the output
values.

Table \ref{out-tab-pow} shows the output options for powder diffraction data.
Possible values were listed at the beginning of this section.

%
\begin{table}[!tbh]
\centering
\begin{tabularx}{\textwidth}{|p{42mm}|X|}
  \hline
  {\bf Output format} & {\bf Description} \\
  \hline\hline
  {\tt powder} & Output is saved as powder diffraction data in a
               simple two column ASCII file. Data can be written on a
               2Theta of Q scale as defined by the first parameter.
               Optionally the range of data along the chosen scale
               can be limited. These settings are independent of the 
               scale on which the data were calculated.\\
  \hline
  {\tt pdf} &  The powder diffraction data are converted into the 1D-powder 
               PDF. \Discus takes care of the scaling and normalization.\\
  \hline
\end{tabularx}
\caption{\label{out-tab-pow}Output formats for powder diffraction data.}
\end{table}

%------------------------------------------------------------------------
