%------------------------------------------------------------------------
% Chapter:  FORTRAN interpreter
%------------------------------------------------------------------------

\chapter{FORTRAN style interpreter \label{fort}}

The program includes a FORTRAN style interpreter that allows the
user to program complex modifications.  The interpreter provides
variables, linked to data sets and free variables, loops, logical
construction, basic arithmetic and built in functions.  Commands
related to the FORTRAN interpreter are {\tt =, break, do, else,
elseif, enddo, endif, eval, if} and {\tt variable}. The command {\tt
eval} allows to examine the contents of a variable of evaluate an
expression, e.g. {\tt eval r[1]*0.5}. \par

%------------------------------------------------------------------------

\section{Variables \label{var}}

There are two types of variables: user defined single named
variables of type integer or real and build in variables, denoted by
a name, immediately followed by a left square bracket [, one or more
indices and a right square bracket ]. An example of the use of
defined variables is shown here:
%
\begin{MacVerbatim}
   1  variable  real,alpha,90.0
   2  variable  real,beta
   3  variable  real,diff
   4  beta = 94.0
   5  diff = alpha - beta
   6  eval diff
\end{MacVerbatim}
%
In line one we define a real variable {\tt alpha} with an initial
value of 90. Next two more variables are defined and the difference
between {\tt alpha} and {\tt beta} is calculated (line 5) and the
result displayed on the screen (line 6). Examples for build in
variables are given next.

\begin{quote}
  {\it Example :\/} i[1], r[0], i[i[1]], x[1,2]
\end{quote}

The left square brackets {\bf must} immediately follow the name of
the variable.  Blanks within the square brackets are not
significant.  The index can be any integer expression, especially
any of the integer variables again.  Two free variables are
provided, an integer, {\tt i[n]}, and a real, {\tt r[n]} variable.
Each of these variables is actually an array of dimension n given in
squared brackets. The allowed range for n is 0 to MAXPARAM, which is
defined at compilation time by the parameter MAXPARAM in the file
{\it param.inc}. More details are given in appendix
\ref{app-install}. \par

Beside these variables {\tt i[n]} and {\tt r[n]} for general use a
large variety of variables is linked to specific information of {\it
PDFFIT}, e.g. of the currently loaded structural data or PDF data.
Some of these variables c an not be modified, others like atomic
positions or thermal factors can be altered, thus allowing to modify
the loaded structure. Table \ref{v1-tab} shows a summary of {\it
PDFFIT} specific variables related to a loaded structure. Note that
all structure related variables are linked to the currently active
phase. To alter structural information of another phase use the
command {\tt phase} first. Variables related to PDF data are listed
in Table \ref{v2-tab} and refinement variables can be found in Table
\ref{v3-tab}.

\begin{table}[!htb]
\centering
\begin{tabularx}{\textwidth}{|p{30mm}|X|}
  \hline
  {\bf Variable} & {\bf Description} \\
  \hline \hline
  {\tt n[1]}$^{*}$ & Total number of atoms of current phase \\
  {\tt n[2]}$^{*}$ & Number of different atoms of current phase \\
  {\tt n[3]}$^{*}$ & Number of atoms per unit cell (current phase) \\
  {\tt n[4]}$^{*}$ & Number of phases \\
  {\tt n[5]}$^{*}$ & Number of currently active phase \\
  \hline
  {\tt x[n]}      & x-position of atom n of current phase \\
  {\tt y[n]}      & y-position of atom n of current phase \\
  {\tt z[n]}      & z-position of atom n of current phase \\
  {\tt m[n]}      & Number of atom type on site n (current phase) \\
  {\tt u[i,n]}    & Anisotropic thermal factor $U_{lm}$ for site n with
                    ($i=1..6$ for $U_{11}, U_{22}, U_{33}, U_{12}, U_{13},
                     U_{23}$)\\
  {\tt o[n]}      & Occupancy of site n (current phase) \\

  {\tt lat[i]}    & Lattice parameters ($i=1..6$ for $a,b,c,\alpha,\beta,
                    \gamma$) \\
  {\tt csca[p]}   & Scaling factor $f_{p}$ for phase $p$ \\
  {\tt rhoz[p]}   & Number density $\rho_{0}$ for phase $p$ \\
  \hline
\end{tabularx}
\caption[{\it PDFFIT} structural variables]
        {\label{v1-tab}{\it PDFFIT} structural variables. Variables marked
         with $^{*}$ are read-only and can not be altered.}
\end{table}

\begin{table}[!htb]
\centering
\begin{tabularx}{\textwidth}{|p{30mm}|X|}
  \hline
  {\bf Variable} & {\bf Description} \\
  \hline \hline
  {\tt n[6]}$^{*}$    & Number of loaded experimental PDFs \\
  {\tt np[s]}$^{*}$   & Number of data points of data set $s$ \\
  \hline
  {\tt pc[n,s]}       & Value of calculated PDF point $n$ of data set
                        $s$ for current phase \\
  {\tt po[n,s]}       & Value of observed PDF point $n$ of data set $s$ \\
  {\tt pw[n,s]}       & Value of weight ($1/\sigma^{2}$) of point $n$
                        of data set $s$ \\
  \hline
  {\tt rmin[s]}       & Minimum $r$-value for dataset $s$ \\
  {\tt rmax[s]}       & Maximum $r$-value for dataset $s$ \\
  {\tt delr[s]}       & Step size $\Delta r$ for dataset $s$ \\
  {\tt rang[i,s]}     & Range in $r$ to be calculated or refined ($i=1,2$
                        for lower and upper limit) \\
  \hline
  {\tt delt[p]}       & Value of $\delta$ for quadratic PDF peak sharpening for
                        phase $p$\\
  {\tt gamm[p]}       & Value of $\gamma$ for linear PDF peak sharpening for
                        phase $p$\\
  {\tt rcut[p]}       & Value of $r_{cut}$ below which the PDF peak width
                        is multiplied with $\phi_{0}$ \\
  {\tt srat[p]}       & Value of $\phi_{0}$ (or $\sigma$-ratio) \\
  {\tt qmax[s]}       & Maximum $Q$-value for data set $s$ \\
  {\tt qalp[s]}       & Value of $\alpha(s)$ (resolution broadening) for
                        data set $s$ \\
  {\tt qsig[s]}       & Value of $\sigma_{Q}$ (resolution dampening) for
                        data set $s$ \\
  {\tt dsca[s]}       & Scale factor $f_{s}$ for data set $s$ \\
  {\tt bave[s]}       & Average scattering length \\
  \hline
\end{tabularx}
\caption[{\it PDFFIT} PDF related variables]
        {\label{v2-tab}{\it PDFFIT} PDF related variables. Variables marked
         with $^{*}$ are read-only and can not be altered.}
\end{table}

\begin{table}[!htb]
\centering
\begin{tabularx}{\textwidth}{|p{30mm}|X|}
  \hline
  {\bf Variable} & {\bf Description} \\
  \hline \hline
  {\tt n[7]}$^{*}$    & Maximum number of parameters \\
  {\tt n[8]}$^{*}$    & Number of used refinement parameters \\
  \hline
  {\tt p[n]}          & Value of refinement parameter $p_{n}$ \\
  {\tt dp[n]}$^{*}$   & Standard deviation of {\tt p[n]} \\
  {\tt pf[n]}         & Refinement flag for $p_{n}$ (1=refine, 0=fixed) \\
  {\tt cl[n,m]}$^{*}$ & Value of correlation between parameters $p_{n}$ and
                        $p_{m}$ \\
  {\tt rw[1]}$^{*}$   & Expected R-value of refinement \\
  {\tt rw[2]}$^{*}$   & Resulting R-value of refinement \\
  {\tt rw[3]}$^{*}$   & Resulting weighted R-value of refinement \\
  \hline
  {\tt dx[n]}          & Standard deviation of {\tt x[n]} \\
  {\tt dy[n]}          & Standard deviation of {\tt y[n]} \\
  {\tt dz[n]}          & Standard deviation of {\tt z[n]} \\
  {\tt du[i,n]}$^{*}$  & Standard deviation of {\tt u[i,n]} \\
  {\tt do[n]}$^{*}$    & Standard deviation of {\tt o[n]} \\
  {\tt drhoz[p]}$^{*}$ & Standard deviation of {\tt rhoz[p]} \\
  {\tt dlat[i]}        & Standard deviation of {\tt lat[i]} \\
  {\tt ddelt[n]}$^{*}$ & Standard deviation of {\tt delt[n]} \\
  {\tt dgamm[n]}$^{*}$ & Standard deviation of {\tt gamm[n]} \\
  {\tt dsrat[n]}$^{*}$ & Standard deviation of {\tt srat[n]} \\
  {\tt dcsca[n]}$^{*}$ & Standard deviation of {\tt csca[n]} \\
  {\tt ddsca[s]}$^{*}$ & Standard deviation of {\tt dsca[s]} \\
  {\tt dqsig[s]}$^{*}$ & Standard deviation of {\tt qsig[s]} \\
  {\tt dqalp[s]}$^{*}$ & Standard deviation of {\tt qalp[s]} \\
  \hline
\end{tabularx}
\caption[{\it PDFFIT} refinement variables]
        {\label{v3-tab}{\it PDFFIT} refinement variables. Variables marked
         with $^{*}$ are read-only and can not be altered.}
\end{table}

Another variable is {\tt res[i]} which contains the results of a
particular {\it PDFFIT} command. The variable {\tt res[0]} contains
the number of elements returned in {\tt res[i]}. Check the online
help for the different commands to see what results might be
returned in this variable. \par

%------------------------------------------------------------------------

\section{Arithmetic expressions \label{arith-exp}}

{\it PDFFIT} allows the use of arithmetic expressions using the same
notation as in FORTRAN. Valid operators are `+', `-', `*', `/' and `**'.
Expressions can be grouped by round brackets ( and ). The usual
hierarchy for the operators holds. Values of expressions can be
assigned to any modifiable variable. If you know FORTRAN (or another
programming language) you will have no problems with these examples.

\begin{MacVerbatim}
  i[0] = 1
  r[3] = 3.1415
  r[i[1]] = 2.0*(i[5]-5.0/6.5)
\end{MacVerbatim}

%------------------------------------------------------------------------

\section{Logical expressions \label{log-exp}}

Logical expressions are formed similar to FORTRAN.  They may contain
numerical comparisons using the syntax: {\it $<$arithmetic
expression$> <$operator$> <$arithmetic expression$>$}.  The allowed
operators within {\it PDFFIT} are {\it .lt., .le., .gt., .ge., .eq.}
and {\it .ne.} for operations less than, less equal, greater than,
greater equal, equal and not equal, respectively.\par

Logical expressions can be combined by the logical operators {\it
.not., .and., .or.} and {\it .xor.}.  The following example shows an
expression that is true for values of {\tt i[1]} within the interval
of 3 and 11, false otherwise.

\begin{MacVerbatim}
  i[1].ge.3 .and. i[1].le.11
\end{MacVerbatim}

Logical operations may be nested and grouped using round brackets (
and ). For more examples see section \ref{if}.

%------------------------------------------------------------------------

\section{Intrinsic functions \label{func}}

Several intrinsic functions are defined.  Each function is
referenced, as in FORTRAN, by its name {\bf immediately} followed by
a pair of parentheses ( and ) that include the list of arguments.
Trigonometric and arithmetic functions are listed in table
\ref{func-trig}.  Table \ref{func-ran} contains various random
number generating functions.

\begin{table}[!tb]
\centering
\begin{tabularx}{\textwidth}{|p{15mm}|p{45mm}|X|}
  \hline
  {\bf Type} & {\bf Name} & {\bf Description} \\
  \hline \hline
  real & sin(r) cos(r) tan(r) &
         Sine, cosine and tangent of $<$r$>$ in radian \\
  real & sind(r) cosd(r) tand(r) &
         Sine, cosine and tangent of $<$r$>$ in degrees \\
  real & asin(r) acos(r) atan(r) &
         Arc sin, cosine, tangent of $<$r$>$, result in radian \\
  real & asind(r) acosd(r) atand(r) &
         Arc sin, cosine, tangent of $<$r$>$, result in degrees \\
  \hline
  real & sqrt(r) & Square root of $<$r$>$ \\
  real & exp(r) &  Exponential of $<$r$>$, base e \\
  real & ln(r) &   Logarithm of $<$r$>$ \\
  real & sinh(r) cosh(r) tanh(r) &
                   Hyperbolic sine, cosine and tangent of $<$r$>$ \\
  \hline
  real    & abs(r)      &  Absolute value of $<$r$>$ \\
  integer & mod(r1, r2) &  Modulo $<$r1$>$ of $<$r2$>$ \\
  integer & int(r)      &  Convert $<$r$>$ to integer \\
  integer & nint(r)     &  Convert $<$r$>$ to nearest integer \\
  real    & frac(r)     &  Returns fractional part of $<$r$>$ \\
  \hline
  real    & min(r1, r2) &  Smaller number of $<$r1$>$ and $<$r2$>$ \\
  real    & max(r1, r2) &  Larger number of $<$r1$>$ and $<$r2$>$ \\
  \hline
\end{tabularx}
\caption{\label{func-trig}Trigonometric and arithmetic functions}
\end{table}

\begin{table}[!tb]
\centering
\begin{tabularx}{\textwidth}{|p{15mm}|p{30mm}|X|}
  \hline
  {\bf Type} & {\bf Name} & {\bf Description} \\
  \hline\hline
  real & ran(r) &         Uniformly distributed pseudo random number
                          between 0.0 and 1.0. Argument $<$r$>$ is a
                          dummy\\
  \hline
  real & gran(r1, typ) &  Gaussian distributed random number with mean
                          0 and a width given by $<$r1$>$. If $<$typ$>$
                          is "s" $<$r1$>$ is taken as sigma, if
                          $<$typ$>$ is "f" $<$r1$>$ is taken as
                          FWHM. \\
  \hline
  real & gbox(r1, r2, r3) & Returns pseudo random number with
                          distribution given by a box centered
                          at 0 with a width of $<$r2$>$ and two half
                          Gaussian distributions with individual
                          sigmas of $<$r1$>$ and $<$r3$>$ to the left
                          and right, respectively. \\
  real & logn(r1,r2)      & Returns a lognormal distributed number
                          with mean $<$r1$>$ and sigma $<$r2$>$ of
                          the underlying
                          Gaussian distribution.\\
  integer & pois(r1)      & Returns a poisson distributed random
                          number with mean $<$r1$>$.\\
  \hline
\end{tabularx}
\caption{\label{func-ran}Random number functions}
\end{table}

{\it PDFFIT} offers some crystallographic functions to calculate
bond length, bond angles and related values. These functions are
summarised in Table \ref{func-cryst}.

\begin{table}[!tb]
\centering
\begin{tabularx}{\textwidth}{|p{15mm}|p{30mm}|X|}
  \hline
  {\bf Type} & {\bf Name} & {\bf Description} \\
  \hline
  real & \raggedright bang(u1,u2,u3, v1,v2,v3 [,w1,w2,w3]) &
       Returns the bond angle in degrees between {\bf u} and {\bf v}
       at site {\bf w}. If {\bf w} is omitted, the angle between
       direct space vectors {\bf u} and {\bf v} is returned. \\
  \hline
  real & \raggedright blen(u1,u2,u3 [,v1,v2,v3]) &
       Returns the length of the real space vector {\bf v}-{\bf u}.
       The vector {\bf v} defaults to zero. \\
  \hline
  real & \raggedright dstar(h1,h2,h3 [,k1,k2,k3]) &
       Returns the length of reciprocal vector {\bf k} - {\bf h} in
       \AA$^{-1}$. Vector {\bf k} defaults to zero. \\
  \hline
  real & \raggedright rang(h1,h2,h3, k1,k2,k3 [,l1,l2,l3]) &
       Returns the angle between reciprocal vectors {\bf k} - {\bf h} and
       {\bf k} - {\bf l} at site {\bf k}.  If {\bf l} is omitted, the angle
       between reciprocal vectors {\bf h} and {\bf k} is returned. \\
  \hline
\end{tabularx}
\caption{\label{func-cryst}Crystallographic functions}
\end{table}

Finally we have (currently only one) functions related to the
refinement itself listed in Table \ref{func-fit}.

\begin{table}[!tb]
\centering
\begin{tabularx}{\textwidth}{|p{15mm}|p{30mm}|X|}
  \hline
  {\bf Type} & {\bf Name} & {\bf Description} \\
  \hline
  real & rval(s [,rmi,rma]) &
       Returns the weighted R-value for the given data set $<$s$>$.
       Optionally a the region in $r$ taken into account can be
       restricted by giving the values $<$rmi$>$ and $<$rma$>$. \\
  \hline
\end{tabularx}
\caption{\label{func-fit}Refinement functions}
\end{table}

%------------------------------------------------------------------------

\section{Loops \label{do}}

Loops can be programmed in {\it PDFFIT} using the {\tt do} command.
Three different types of loops are implemented. The first type
executes a predefined number of times. The syntax for this type of
loop is

\begin{quote}
{\it do $<$variable$>$ = $<$start$>$,$<$end$>$ [,$<$increment$>$] \\
     \ldots commands to be executed \ldots \\
     enddo }
\end{quote}

Loops may contain constants or arithmetic expressions for $<$start$>$,
$<$end$>$, and $<$increment$>$.  $<$increment$>$ defaults to 1.  The
internal type of the variables is real.  The loop counter is evaluated from
{\it ($<$end$>$ - $<$start$>$)/$<$increment$>$ + 1}. If this is negative,
the loop is not executed at all.  The parameters for the counter variable,
start end and increment variables are evaluated only at the beginning of
the do - loop and stored in internal variables.  It is possible to change
the values of $<$variable$>$, $<$start$>$ and/or $<$end$>$ within the loop
without any effect on the performance of the loop.  This practice is not
encouraged, could, however, be an unexpected source of errors. \par

The second type of loop is executed while $<$logical expression$>$ is
true. Thus it might not be executed at all. The syntax for this type
of loop is

\begin{quote}
{\it do while $<$logical expression$>$ \\
     \ldots commands to be executed \ldots \\
     enddo }
\end{quote}

The last type of loop is executed until $<$logical expression$>$ is
true. This loop, however, is always executed once and has the
following syntax

\begin{quote}
{\it do \\
     \ldots commands to be executed \ldots \\
     enddo until $<$logical expression$>$ }
\end{quote}

In the body of commands any valid {\it PDFFIT} command can be used.
This includes calls to the sublevels, further do loops or macros,
even if these macros contain do loops themselves.  The maximum level
of nesting is limited by the parameter {\it MAXLEV} in the file {\it
doloop.inc}.  If necessary adjust this parameter to allow for deeper
nesting.  All commands from the first {\tt do} command to the
corresponding {\tt enddo} are read and stored in an internal array.
This array can take at most {\it MAXCOM} (defined in file {\it
doloop.inc} as well) commands at every level of nesting.  If lengthy
macro files are included in the do loop, this parameter might have
to be adjusted.  \par

If a do loop (or an if block) needs to be terminated, the {\tt
break} command will perform this function.  The parameter on the
{\tt break} command line gives the number of nested levels of {\tt
do} and {\tt if} blocks to be terminated. The interpreter will
continue execution with the first command following the
corresponding {\tt enddo} or {\tt endif} command.  An example is
given below, note, that the line numbers are only given for better
orientation and are no actual part of the listed commands. \par

\begin{MacVerbatim}
     1  do i[2]=1,5
     2    do i[1]=1,5
     3      if ((i[1]+i[2]) .eq 6) then
     4        break 2
     5      endif
     6    enddo
     7  enddo
\end{MacVerbatim}

In this example, the execution of the inner do-loop will stop as
soon as the sum of the two increment variables {\tt i[1]} and {\tt
i[2]} is equal to 6.  The program continues with the {\tt enddo}
line of the outer do - loop.  Notice that two levels need to be
interrupted, the if block and the innermost do loop.  If the
parameter had been equal to one, only the if block would have been
interrupted, while the innermost do loop would have continued
without break. \par

%------------------------------------------------------------------------

\section{Conditional statements \label{if}}

Commands can be executed conditionally by using the {\tt if}
command. Analogous to FORTRAN, the if-control structure takes the
following form:

\begin{quote}
{\it if ( $<$logical expression$>$ ) then \\
     \ldots commands to be executed \ldots \\
     elseif ( $<$logical expression$>$ ) then \\
     \ldots commands to be executed \ldots \\
     else \\
     \ldots commands to be executed \ldots \\
     endif }
\end{quote}

The logical expressions are explained in section \ref{log-exp}.
Enclosed within an if block any valid {\it PDFFIT} command can be
used.  This includes calls to the sublevels further if blocks, do
loops or macros, even if these macros contain if blocks or do loops
themselves.  The {\tt elseif} and {\tt else} section is optional.
The maximum level of nesting is limited by the parameter {\it
MAXLEV} in the file {\it doloop.inc}. If necessary adjust this
parameter to allow for deeper nesting.  All commands from the first
{\tt if} command to the corresponding {\tt endif} are read and
stored in an internal array. This array can take at most {\it
MAXCOM} (defined in file {\it doloop.inc} as well) commands at every
level of nesting.  If lengthy macro files are included in the do
loop, this parameter might have to be adjusted. \par

If an if block (or a do loop) needs to be terminated, the {\tt
break} command will perform this function.  The parameter on the
{\tt break} command line gives the number of nested levels of {\tt
do} and {\tt if} blocks to be terminated. The interpreter will
continue execution with the first command following the
corresponding {\tt enddo} or {\tt endif} command.  See the example
in section \ref{do} for further explanations. \par

\begin{MacVerbatim}
     1  do i[1]=1,n[7]
     2    if(dp[i[1]].gt.0.0) then
     3      p[i[1]]=(1.0+0.1*(0.5-ran(0)))*p[i[1]]
     4    endif
     5  enddo
\end{MacVerbatim}

The example listed above illustrates the use of loops and
conditional statements within {\it PDFFIT}.  Again, the line numbers
are given for easy reference and not part of the actual {\it PDFFIT}
input. This little sequence of commands will add a $\pm 5\%$ random
noise to all parameters that have a standard deviation greater than
zero. In line 1 starts a do-loop over all possible parameters.  The
variable n[7] contains this information (see table \ref{v1-tab} in
section \ref{var}). In line 2 we check whether the standard
deviation is greater than zero and if this is true, the value of
{\tt p[n]} is altered in line 3. Since the function 'ran' produces a
uniformly distributed pseudo random number in the range 0.0 to 1.0,
the factor the parameter is multiplied with (line 3) ranges from
0.95 to 1.05.

%------------------------------------------------------------------------

\section{Filenames \label{fnames}}

Usually, file names are understood as typed, including capital
letters. Unix operating systems distinguish between upper and lower
case typing ! However, sometimes it is required to be able to alter
a file name e.g. within a loop.  Thus, {\it PDFFIT} allows the user
to construct file names by writing additional (integer) numerical
input into the filename.  The syntax for this is:

\begin{quote}
  {\it "string\%dstring",$<$integer expression$>$}
\end{quote}

The file format MUST be enclosed in quotation marks.  The position
of each integer must be characterised by a {\tt \%d}.  The sequence
of strings and \%d's can be mixed at will.  The corresponding
integer expressions must follow after the closing quotation mark. If
the command line requires further parameter they must be given after
the format parameters. The interpretation of the \%d's follows the C
syntax.  Up to 10 numbers can be written into a filename.  All of
the following examples will result in the file name {\it a1.1.pdf}:

\begin{MacVerbatim}
     i[5]=1
     save pdf,1,a1.1
     save pdf,1,"a%d.%d",1,1
     save pdf,1,"a%d.%d",4-3,i[5]
\end{MacVerbatim}

The second example shows how filenames are changes within a loop.
Here the structure files {\it phase1.stru} to {\it phase3.stru} will
be loaded.

\begin{MacVerbatim}
     do i[1]=1,3
       ..
       read stru,"data%d.calc",i[1]
       ..
     enddo
\end{MacVerbatim}

%------------------------------------------------------------------------

\section{Macros \label{mac}}

Any list of valid {\it PDFFIT} commands can be written to an ASCII
file and executed indirectly by the command {\tt @$<$filename$>$}.
Macro files can be written by any editor on your system or be
generated by the {\tt learn} command. The command {\tt learn} starts
to remember all the commands that follow and saves them into the
given file. The learn sequence is terminated by the {\tt lend}
command.  The default extension of the macro file is {\it .mac}.
Macro files can be nested and even reference themselves directly or
indirectly.  This referencing of macro files is, however, just a
nesting of the corresponding text of each macro, not a call to a
function.  All variable retain their values. \par

On the command line of the macro command {\tt @}, optional
parameters can be supplied.  Within the macro these have to be
referenced as {\tt \$1}, {\tt \$2} etc.  Upon execution of the macro
the formal parameters '\$n' are replaced by the character string of
the actual values from the command line. The parameter {\tt \$0}
contains the number of parameter passed to the macro. As any other
command parameters, these parameters must be separated by comma.  If
a formal parameter is referenced inside a macro without a
corresponding parameter on the command line, an error message is
given.  An example is given below:

\begin{MacVerbatim}
     # Adds two numbers supplied as command line parameters.
     # The value is stored in variable defined by parameter three
     #
     $3 = $1 + $2
\end{MacVerbatim}

If this macro is called with the following line, {\tt @add
1,2,i[4]}, the result is stored in variable {\tt i[4]} which now
has the integer value 3. \par

If the program {\it PDFFIT} is started with command line parameters,
e.g. {\tt pdffit 1.mac 2.mac}, the program will execute the given
macros in the specified order, in our example first {\tt 1.mac} then
{\tt 2.mac}. If a macro is not found in the current working
directory, a system macro director is searched. This system macro
directory is located at {\tt path\_to\_binary/sysmac/pdffit/}.
Commonly used macro files might be installed in this directory.

%------------------------------------------------------------------------

\section{Working with files \label{io}}

The command language offers the user several commands to write
variables to a file or read values from a file. First a file needs
to be opened using the command {\tt fopen}. An optional parameter
'append' allows one to append data to an existing file. Once the
task is finished, the file must be closed via {tt fclose}. In the
standard configuration, the program can open five files at the same
time. The first parameter of all file input/output related commands
is the unit number which can range from 1 to 5. The commands {\tt
fget} and {\tt fput} are used to read and write data, respectively.
The following example illustrates the usage of these commands:

\begin{MacVerbatim}
      1  fopen 1,sin.dat
      2  fput 1,'Cool sinus function'
      3  #
      4  do i[1]=1,50
      5    r[1]=i[1]*0.1
      6    r[2]=sin(r[1])
      7    fput 1,r[1],r[2]
      8  enddo
      9  #
     10  fclose 1
\end{MacVerbatim}

In line 1 we open the file {\it sin.dat} and write a title (line 2).
If the file already exists it will be overwritten. Note that the
text must be given in {\it single} quotes. Text and variables may be
mixed in a single line. Next we have a loop calculating $y=\sin(x)$
and writing the resulting $x$ and $y$ values to the open file (line
7). Finally the file is closed (line 10). To read values from a file
use simply the command {\tt fget} and the read numbers will be
stored in the specified variables. In contrast to writing to a file,
mixing of text and number is not allowed when reading data. However,
complete lines will be skipped when the command {\tt fget} is
entered without any parameters.

%------------------------------------------------------------------------
