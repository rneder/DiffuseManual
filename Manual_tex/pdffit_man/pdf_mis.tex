%------------------------------------------------------------------------
% Chapter:  Misc. functions
%------------------------------------------------------------------------

\chapter{Calculating structural properties \label{misc}}

{\it PDFFIT} offers like {\it DISCUS} the intrinsic functions
{\tt blen} and {\tt bang} to calculate the bond length and bond
angles between atoms or general positions within the model
crystal. However, {\it PDFFIT} allows the user to calculated
bond length and bond angles with standard deviations propagated
from the uncertainties of lattice parameters and positions obtained
during the refinement

%------------------------------------------------------------------------

\section{Calculating single bond lengths and angles \label{misc_simp}}

The bond length between any two atoms within the structure can simply
be calculated using the command {\tt blen} followed by the two atom
numbers. The following example calculates the distance between the
atoms {\tt 5} and {\tt 9}:

\footnotesize
\begin{MacVerbatim}
   pdffit > blen 5,9
       MN (#    5) -    O (#    9)   =   1.965(5) A
\end{MacVerbatim}
\normalsize

\noindent The names of the two selected atoms and distance is
given. If the standard deviations of the lattice parameters or the
positions of the atoms are not zero, the standard deviation of the
bond length is given as well. As usual the standard deviation is
given in round brackets after the last significant digit. In our
example the bond length is $1.965 \pm 0.005$\AA. The bond angle
defined by three atoms can be calculated in a similar way using
the command {\tt bang} as demonstrated below:

\footnotesize
\begin{MacVerbatim}
   pdffit > bang 9,5,13
       O (#    9) -   MN (#    5) -    O (#   13)   =   74.1(9) degrees
\end{MacVerbatim}
\normalsize

\noindent It should be noted, that these functions calculates the
direct distance between the given atoms, although the same atom
might be closer to a given atom in a neighbouring unit cell. No
periodic boundary conditions are applied. In order to obtain a
specific bond length or angle it might be necessary to modify the
atom coordinates by $\pm 1$. This can simply be done using the
corresponding variables, e.g. {\tt x[1]=x[1]+1.0}. Note that this
is actually modifying the current structure. Alternatively one can
compute all bond length in a given interval as discussed in the
next section.

%------------------------------------------------------------------------

\section{Calculating multiple bond lengths \label{misc_mult}}

The command {\tt blen} allows one to calculate all bond length
between selected atom types in a given interval. This can be
useful to monitor a specific bond length. In the following example
all $Mn-O$ bonds within a range of $1.8$ and $2.4$\AA\ are
calculated.

\footnotesize
\begin{MacVerbatim}
  pdffit > blen mn,o,1.8,2.4
   Bond lengths found in current phase :
     MN (#    5) -    O (#    9)   =          1.965(5) A
     MN (#    5) -    O (#   13)   =           1.93(1) A
     MN (#    5) -    O (#   18)   =           2.15(1) A
     MN (#    5) -    O (#   14)   =           2.15(1) A
     MN (#    5) -    O (#   17)   =           1.93(1) A
     MN (#    5) -    O (#   11)   =          1.965(5) A

     MN (#    6) -    O (#   13)   =           2.15(1) A
     MN (#    6) -    O (#   12)   =          1.965(5) A
     MN (#    6) -    O (#   18)   =           1.93(1) A
     MN (#    6) -    O (#   10)   =          1.965(5) A
     MN (#    6) -    O (#   14)   =           1.93(1) A
     MN (#    6) -    O (#   17)   =           2.15(1) A

     MN (#    7) -    O (#    9)   =          1.965(5) A
     MN (#    7) -    O (#   15)   =           1.93(1) A
     MN (#    7) -    O (#   20)   =           2.15(1) A
     MN (#    7) -    O (#   11)   =          1.965(5) A
     MN (#    7) -    O (#   16)   =           2.15(1) A
     MN (#    7) -    O (#   19)   =           1.93(1) A

     MN (#    8) -    O (#   10)   =          1.965(5) A
     MN (#    8) -    O (#   19)   =           2.15(1) A
     MN (#    8) -    O (#   20)   =           1.93(1) A
     MN (#    8) -    O (#   12)   =          1.965(5) A
     MN (#    8) -    O (#   15)   =           2.15(1) A
     MN (#    8) -    O (#   16)   =           1.93(1) A
\end{MacVerbatim}
\normalsize

\noindent As one can see around each of the four $Mn$ in the unit
cell we find six oxygens forming an octahedra. In this mode of the
command {\tt blen} periodic boundary conditions are applied in the
same way as for the calculation of the PDF and as a result
distances to atoms outside the structural box (or unit cell) are
also found.
\par

The command {\tt blen} might also be used to identify which atom
pairs contribute to a given PDF peak. Assume the peak in question
is between 3.5 and 3.6\AA. Simply enter the command {\tt blen
all,all,3.5,3.6} and {\it PDFFIT} will list all atom pairs that
are separated by a distance between 3.5 and 3.6\AA.

%------------------------------------------------------------------------
