%------------------------------------------------------------------------
% Chapter:  Refining using PDFFIT
%------------------------------------------------------------------------

\chapter{Running a refinement \label{fit}}
\section{Refinement variable coding \label{fit_code}}

When using {\it PDFFIT} we have to distinguish between refinement
parameters, called $p_{i}$ (or {\tt p[i]}) and experimental and structural
parameters listed in Table \ref{tab_par}. The parameters that can be
assigned to refinement parameters are a subset of all available variables
discussed in section \ref{var}.

\begin{table}[tbh]
\centering
\begin{tabularx}{\textwidth}{|p{30mm}|X|}
  \hline
  {\bf Structural}  & (Set phase using {\tt phase} command !) \\
  \hline
  {\tt x[n]}        & x-position of atom $n$ in fractional coordinates \\
  {\tt y[n]}        & y-position of atom $n$ in fractional coordinates \\
  {\tt z[n]}        & z-position of atom $n$ in fractional coordinates \\
  {\tt o[n]}        & Occupancy for site $n$ \\
  {\tt u[i,n]}      & Anisotropic thermal parameter $U_{ij}$ for atom
                      $n$ ($i=1..6$ for $U_{11},U_{22},U_{33},U_{12},
                      U_{13}$ and $U_{23}$) \\
  {\tt lat[i]}      & Lattice parameters ($i=1..6$ for $a, b, c, \alpha,
                      \beta$ and $\gamma$) \\
  {\tt delt[p]}     & Quadratic correlation factor $\delta$ for phase $p$ \\
  {\tt gamm[p]}     & Linear correlation factor $gamma$ for phase $p$ \\
  {\tt srat[p]}     & Peak width ratio for $r<r_{cut}$ for phase $p$ \\
  {\tt csca[p]}     & Scale factor $f_{p}$ for phase $p$ \\
  \hline
  {\bf Experimental}& \\
  \hline
  {\tt dsca[s]}     & Overall scale factor $f_{s}$ for data set $s$ \\
  {\tt qalp[s]}     & Broadening factor $\alpha$ for data set $s$ \\
  {\tt qsig[s]}     & Resolution factor $\sigma_{Q}$ for data set $s$ \\
  \hline
\end{tabularx}
\caption{\label{tab_par}{\it PDFFIT} experimental and structural
         refinement variables.}
\end{table}

Each experimental or structural parameter that needs to be refined must
be assigned to one or more refinement parameters via the command
{\tt par}. The maximum number of parameters $p_{i}$ allowed in a single
{\tt par} command is defined by the variable MAXDPP in the file
'{\it config.inc}' (see appendix \ref{app-install}). For each parameter
$p_{i}$ that appears in the definition we have to also specify the
derivative of the corresponding structural parameter with respect
to all refinement parameters $p_{i}$. The value of $i$ of the
refinement parameters can be freely chosen by the user, however
there is of course an upper limit. This limit together with other program
related limits can be determined via the command {\tt show config}. \par

Let us consider a simple case like in the example given in section
\ref{quick}.  The lattice parameters $a,b,c$ of $Ni$ shall be refined
using a single refinement parameter since $Ni$ is cubic. The definitions
we need to enter are listed below. In lines 1--3 refinement parameter
$p_{1}$ is associated with the lattice constants $a,b,c$. It is important
to remember, that one needs to assign a suitable starting value to
$p_{1}$ as we have done in line 5 of our example.

\footnotesize
\begin{MacVerbatim}
    1 par lat[1]=p[1],1.0
    2 par lat[2]=p[1],1.0
    3 par lat[3]=p[1],1.0
    4 #
    5 p[1]=lat[1]
\end{MacVerbatim}
\normalsize

\noindent Next we consider a slightly more complicated
construction. It is again taken from the example in section
\ref{quick}. Assuming we want to refine a single isotropic
temperature factor for all atoms within the unit cell, we can
either type in all definitions explicitly of use a {\tt do} loop
as in the example below. The loop index is {\tt i[1]} and {\tt
n[1]} contains the total number of atoms for the currently active
phase. Note that {\it PDFFIT} replaces all variables in the
arguments, i.e. enclosed in {\tt [.]}, with the actual value at
the time the command is processed. Try the example below and check
the stored definitions using the command {\tt show const}. Again
we need to set the used refinement parameters to a suitable
stating value (line 7).

\footnotesize
\begin{MacVerbatim}
    1 do i[1]=1,n[1]
    2   par u[1,i[1]]=p[2],1.0
    3   par u[2,i[1]]=p[2],1.0
    4   par u[3,i[1]]=p[2],1.0
    5 enddo
    6 #
    7 p[2]=u[1,1]
\end{MacVerbatim}
\normalsize

\noindent As a final example we consider a (hypothetical) system
containing a rigid molecule consisting of two atoms, one at
$(\frac{1}{4}, 0,0)$ and one at $(-\frac{1}{4},0,0)$. Note that
{\it DISCUS} supports the definition of molecules in its structure
file. However, this extension is {\it not} supported by {\it
PDFFIT}. The rotation of the molecule around the z-axis through
the center of the molecule (conveniently located at $(0,0,0)$)
shall be modelled by a single refinement parameter, the rotation
angle $\psi$. This rotation can be described by the following
matrix

\begin{equation}
R = \left ( \begin{array}{ccc}
            \cos \psi & \sin \psi & 0 \\
           -\sin \psi & \cos \psi & 0 \\
            0         & 0         & 1 \\
            \end{array} \right ).
    \label{eq_rot}
\end{equation}

\noindent All we need to do now is to describe the new positions
of the two atoms as a function of the rotation angle $\psi$ and
specify the corresponding derivatives. One should keep in mind
that the atom positions are in {\it fractional coordinates}.
Furthermore one should {\it not} use definition of the type {\tt
par x[1]=x[1]*..} since {\tt x[1]} is changing during the
refinement. So we need to store the relevant information in some
general variable {\tt r[n]}. The parameter coding for our example
is shown below:

\footnotesize
\begin{MacVerbatim}
    1 r[1]=x[1]
    2 r[2]=y[1]
    3 r[3]=x[2]
    4 r[4]=y[2]
    5 #
    6 par x[1]=    r[1]*cosd(p[5])+r[2]*sind(p[5]),-1.*r[1]*sind(p[5])+r[2]*cosd(p[5])
    7 par y[1]=-1.*r[1]*sind(p[5])+r[2]*cosd(p[5]),-1.*r[1]*cosd(p[5])-r[2]*sind(p[5])
    8 par x[2]=    r[3]*cosd(p[5])+r[4]*sind(p[5]),-1.*r[3]*sind(p[5])+r[4]*cosd(p[5])
    9 par y[2]=-1.*r[3]*sind(p[5])+r[4]*cosd(p[5]),-1.*r[3]*cosd(p[5])-r[4]*sind(p[5])
   10 #
   11 p[5]=4.5
\end{MacVerbatim}
\normalsize

\noindent In lines 1--4 the $x$ and $y$ position of both atoms is
stored in the variables {\tt r[1]} to {\tt r[4]}. Note that $z$ is
not changing for our specific rotation. In lines 6--9 we give the
equations according to (\ref{eq_rot}) and the corresponding
derivative with respect to parameter $p_{5}$. The intrinsic
functions {\tt sind} and {\tt cosd} will expect their argument to
be in degrees whereas {\tt sin} and {\tt cos} need the argument to
be specified in radian. We chose to use degrees in our example.
More information about intrinsic functions can be found in section
\ref{func}. Currently the command language interpreter will not
accept expressions that start with '-' like {\tt -r[1]}, so one
needs to specify more completely {\tt -1.*r[1]}. The command {\tt
show const} will list all parameter definitions on the screen. The
output for our example is listed below:

\footnotesize
\begin{MacVerbatim}
     --------------------------------------------------------------------------
     PARAMETER DEFINITIONS :
     --------------------------------------------------------------------------

     Definition  : x[1]=r[1]*cosd(p[5])+r[2]*sind(p[5])
     Derivatives : d/d(p[   5]) = -1.*r[1]*sind(p[5])+r[2]*cosd(p[5])

     Definition  : y[1]=-1.*r[1]*sind(p[5])+r[2]*cosd(p[5])
     Derivatives : d/d(p[   5]) = -1.*r[1]*cosd(p[5])-r[2]*sind(p[5])

     Definition  : x[2]=r[3]*cosd(p[5])+r[4]*sind(p[5])
     Derivatives : d/d(p[   5]) = -1.*r[3]*sind(p[5])+r[4]*cosd(p[5])

     Definition  : y[2]=-1.*r[3]*sind(p[5])+r[4]*cosd(p[5])
     Derivatives : d/d(p[   5]) = -1.*r[3]*cosd(p[5])-r[4]*sind(p[5])
\end{MacVerbatim}
\normalsize

\noindent Basically this type of assigning refinement parameters
to structural parameters allows the user to realize nearly all
type of constraint models. In the next sections specific details
of structural parameters as well as refinement setting will be
discussed.

%------------------------------------------------------------------------
\subsection{Structural parameters \label{fit_stru}}

The structural parameters that can be refined using {\it PDFFIT}
were already listed in Table \ref{tab_par} at the beginning of
this chapter. Parameters {\tt x[n], y[n]} and {\tt z[n]} are the
{\it fractional} coordinates of atom $n$ of the currently active
structural phase (see section \ref{fit_mult}). The site occupancy
is given by {\tt o[n]}. The anisotropic temperature factor
$U_{lm}$ is stored in the variable {\tt u[i,n]} with $i=1..6$ for
$U_{11}, U_{22}, U_{33}, U_{12}, U_{13}$ and $U_{23}$. Note that
$U_{lm} = \langle u_{l}u_{m} \rangle$ and in particular $U_{ll} =
\langle u_{ll}^{2} \rangle$.\footnote{The old {\it RESPAR} program
was using $\langle u \rangle$ in its input file !} Two other forms
of the temperature factor commonly used in the literature are:
$B_{lm} = 8 \pi^{2}U_{lm} = 8 \pi^{2} \langle u_{l}u_{m} \rangle$
and $\beta_{lm} = \frac{1}{4} {\bf a}^{*}_{l} {\bf a}^{*}_{m}
B_{lm}$ where ${\bf a}^{*}$ are the reciprocal lattice parameters.
\par The remaining parameters are the scale factor {\tt csca[p]}
and the parameters {\tt delt[p]} and {\tt srat[p]} modelling the
$r$ dependence of the PDF peak width as discussed in the next
section. Note that these parameters take the number of the
corresponding structural phase $p$ as argument.

%------------------------------------------------------------------------
\subsection{R-dependence of the PDF peak width \label{fit_pwid}}

\begin{figure}[!t]
   \centering
   \includegraphics[scale=0.5, angle=270.0]{alp.1.eps}
   \caption[Influence of PDF peak broadening $\alpha$]
           {Result of PDF refinement of $Ni$ showing the influence
            of the PDF peak broadening parameter, $\alpha$. The
            filled circles are the data collected on the instrument
            NPDF at the Lujan Center at Los Alamos National Laboratory.
            The dotted line is a refinement without $\alpha$ and
            the dashed line is a refinement using the PDF broadening
            with $\alpha=0.002$. Note that the range displayed extends
            from 80\AA\ to 90\AA.}
   \label{fit-fig1}
\end{figure}

The PDF peak width contains contributions from thermal and zero
point displacements as well as static disorder. For large
distances $r$ the motion of the two contributing atoms is
uncorrelated, for small distances, however, the motion can be
strongly correlated leading to a sharpening of the first peak(s)
in the observed PDF (see \cite{jeprmjbi98}). The program PDFFIT
provides three different correction terms for the PDF peak width.
The final width is given by

\begin{equation}
  \sigma_{ij}= \sqrt {\sigma_{ij}^{'2} -
                      \frac{\delta}{r_{ij}^{2}} -
                      \frac{\gamma}{r_{ij}} +
                      \alpha^{2} r_{ij}^{2}}.
  \label{eq_fitw}
\end{equation}

\noindent Here $\sigma'$ is the peak width without correlation
given by the structural model. The first two terms correct for the
effects of correlated motion. Details about these terms can be
found in \cite{jehe02}. Within the scope of the users guide, we
just mention that the term $\delta / r^{2}$ describes the low
temperature behavior and the term $\gamma / r$ describes the high
temperature case. Since the two parameters are highly correlated,
one will in practice choose which one to refine. The last term in
equation \ref{eq_fitw} models the PDF peak broadening as a result
of the $Q$ resolution of the diffractometer. Details about this
corrections can be found in \cite{thlele02}. In many cases this
term will only be significant for refinements of wider ranges in
$r$. Note that the $Q$ resolution also results in an exponential
dampening of the PDF peaks which is modelled using the parameter
$\sigma_{Q}$. Complete details of the equations used in PDFFIT,
refer to appendix \ref{app-deriv}.
\par

The last mechanism to sharpen PDF peaks at low $r$ becomes
necessary e.g. when a system has a static displacement component
which is reflected by the structural model only up to a certain
distance. In this case {\it PDFFIT} allows the user to sharpen the
PDF peaks below a value of $r=r_{cut}$ by a factor $\phi < 1$.
Thus below $r_{cut}$ the temperature factor reflect only thermal
and zero-point motion whereas at higher $r$ a combination of
static and dynamic displacements are modelled by the thermal
parameters. The cutoff value is set using the variable {\tt
rcut[p]} and the ratio $\phi$ is set by {\tt srat[p]} (read sigma
ratio).

%------------------------------------------------------------------------
\subsection{Refining multiple phases or data sets \label{fit_mult}}

{\it PDFFIT} is capable of refining multiple structural phases and
multiple PDF data sets. Let us consider multiple data sets first.
Simply read each data set using the command {\tt read data} after
starting the program or using the command {\tt reset}. Assign a
refinement parameter to the experimental variables of each data set.
Setup the structural model as before and run the refinement. The
calculated PDF for each data set needs to be saved using the
command {\tt save pdf,n,file} where {\tt n} is the number of the
data set. When refining a model with more than one structural phase
read the corresponding structure using the command {\tt read stru}. Next
the definition of a structural model needs to be done for each phase.
Use the command {\tt phase} to make the phase active before entering
the corresponding parameter definitions. Since the variables
{\tt delt[p], rcut[p], srat[p]} and {\tt csca[p]} use the phase
number as argument, the corresponding definitions can be entered
without using the command {\tt phase}. Again after running the
refinement, the resulting structure of each phase must be saved
separately via {\tt save stru,p,file} where {\tt p} is the number of
the phase. \par

The new command {\tt psel} allows the user to associate certain phases
$p_{i}, i=1..n$ with a given data set $s$. One extreme would be to have
e.g. two data sets and one phase each completely separate which is
not really different from running the refinements separately.
However, by using identical refinement parameters one can constrain
any structural parameters between the phases. The default is that all
structural phases are associated with all loaded data sets.
\par

When using multiple phases and data sets one needs to be careful
who to use the scale factors {\tt csca[p]} and {\tt dsca[s]}. The
program makes no internal normalization and the scale factors might
be highly correlated.

%------------------------------------------------------------------------
\subsection{Difference modeling \label{fit_diff}}

Sometimes it might be beneficial to refine the {\it change} of
a PDF e.g. when crossing a phase transition rather than the PDF
itself. One advantage is that systematic errors might cancel when
calculating the difference. {\it PDFFIT} has the option to refine
a difference PDF $D(r) = G(r) - G_{r}(r)$. The normal PDF $G(r)$
is derived from the model structure and subsequently a reference
PDF $G_{r}(r)$ is subtracted. \par

Difference modeling is enabled by using the command {\tt read
diff} rather than {\tt read data} when reading the PDF data. The
command {\tt read diff} requires an additional parameter containing
the name of a file containing the reference PDF $G_{r}(r)$. The data
file must now contain the difference between the observed and
reference PDF. It is important that both PDFs cover identical points
in $r$. The command {\tt save pdf} will save the resulting difference
PDF and {\tt save dif} will save the difference between the calculated
difference PDF and the experimental difference PDF, I hope this is
not confusing.

%------------------------------------------------------------------------
\section{Setting refinement options \label{fit_opt}}

There are only two settings the user can change before a refinement
is started using the command {\tt run}: the maximum number of iterations
({\tt cycle}) and the magic number URF (some German: Unterer Relaxations
Faktor, command {\tt urf}). This value determines who 'fast' the fit
will move to its minimum or how much the parameter values are changed
in each cycle depending on the deviations. A small value (e.g. 0.1) might
lead to a fast convergence but might also miss the minimum. A larger
value (e.g. 100.0) will give a slow convergence which more certain
finds the minimum, but might be caught in local minima rather than in
the global one. Understood ? Well just try different values until your
fit converges nicely to the global minimum. The convergence of a
refinement can be judged by the final difference in the R-value displayed
on the screen. Obviously a small difference is the goal. {\it PDFFIT}
leaves the convergence criterion to the user. The command {\tt show fit}
will list the current refinement settings and results. If the number of
actually computed iteration is equal to the maximum number of iterations,
the refinement needs to be continued by simply issuing the command
{\tt run} again. One might also need to increase the maximum number
of iterations using the command {\tt cycle}. \par

When refining a model one can rarely refine all desired parameters
from the start. So a good start is for example to refine the scale
factor first followed by lattice parameters and so on. So one could
add the parameter definitions gradually. However, {\it PDFFIT} offers
refinement flags {\tt pf[n]}. To refine a certain parameter $n$ simply
use {\tt pf[n]=1}, to keep it fixed use {\tt pf[n]=0}. The default is
that all parameters that are used in any definition will be refined.

%------------------------------------------------------------------------
\section{Saving results \label{fit_save}}

After a refinement has been carried out, the results must be
saved before {\it PDFFIT} is terminated, otherwise {\bf all
results will be lost}. The following sequence of command will
save all results of a refinement:

\footnotesize
\begin{MacVerbatim}
      1 save pdf,1,result.pdf
      2 save dif,1,result.dif
      3 save str,1,result.stru
      4 save res,result.res
\end{MacVerbatim}
\normalsize

\noindent In lines 1--2 the calculated PDF and the difference
between observed and calculated PDF are written to the files {\it
result.pdf} and {\it result.dif}. The file formats were already
discussed in section \ref{file_pdf}. The parameter {\tt 1} stands
for the PDF and difference corresponding to the experimental
dataset one. In cases where multiple datasets are refined, the
results for each set must be saved separately. In line 3 the
resulting structure is saved using the {\it PDFFIT} structure file
format. Here the {\tt 1} stands for structural phase one. Again
for refinements using multiple phases, each resulting structural
phase needs to be save separately. Finally the a summary of the
refinement settings and results is saved to the file {\it
result.res} (line 4). The resulting structure can alternatively be
saved in the {\it DISCUS} structure file format using the command
{\tt save disc,1,file.stru}. This allows one to used {\it DISCUS}
for further analysis of the resulting structure. However, some
information like the standard deviations and the anisotropic
temperature factors $U_{ij}$ will be lost. Thus it is recommended
to save the resulting structure always in {\it PDFFIT} format as
well.

%------------------------------------------------------------------------
