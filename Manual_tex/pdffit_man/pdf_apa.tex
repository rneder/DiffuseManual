%------------------------------------------------------------------------
% Chapter:  Appendix A
%------------------------------------------------------------------------

\chapter{Computational Details \label{app-deriv}}

Here we give the details of the computation of the model PDF $G(r)$. In
the introduction we defined $G(r)$ as follows:

\begin{equation}
  G(r) = \frac{1}{r} \sum_{i}\sum_{j} \left [
         \frac{b_{i}b_{j}}{\langle b \rangle ^{2}}
         \delta (r - r_{ij}) \right ]   - 4 \pi r \rho_{0}
  \label{eq_gr}
\end{equation}

As we discussed already in this manual, the sums go over all atoms
within the model crystal and $r_{ij}$ is the separation distance
between atoms $i$ and $j$. The value $b_{i}$ is the scattering length
for atom $i$ and $\langle b \rangle$ is the average scattering length
of the model crystal. Finally $\rho_{0}$ is the number density. \par

As we have seen in the introduction to this users guide, the
experimental PDF is obtained by Fourier transform of the reduced
structure factor. However, the accessible range in $Q$ is limited by
$Q_{max}$. This can be described by a multiplication of the
structure factor up to infinity with a step function cutting off at
$Q=Q_{max}$ resulting in the convolution of the PDF with the Fourier
transform $C(r)$ of the step function. {\it PDFFIT} models the
finite $Q$-range by convoluting the model PDF $G(r)$ with

\begin{equation}
  C(r) = \frac{\sin(Q_{max} \cdot r)}{r}
  \label{eq_sinc}
\end{equation}

In the following section we will omit this convolution and discuss how
$G(r)$ and the partial derivatives with respect to the refinement
parameters, $\partial G(r)/ \partial p_{n}$, are actually calculated.

%------------------------------------------------------------------------

\section{Calculating the PDF}

The program {\it PDFFIT} refines the function $G(r)$ at discreet
values of $r=r_{k}$ for an experimental dataset $s$. Thus our
equation for the model PDF becomes:

\begin{equation}
  G(r_{k},s) = f_{s} B_{k}(s)
               \sum_{p=1}^{P} f_{p} S_{p}(r_{k},s_{i}) G_{p}(r_{k},s)
  \label{eq_grk}
\end{equation}

\noindent
with

\begin{eqnarray}
  G_{p}(r_{k},s)  & = & \frac{1}{N_{p}r_{k}}
                        \sum_{i}\sum_{j} \left [ A_{ij}(p) \cdot
                        T_{ij}(r_{k},p) \right ] - 4\pi r_{k}\rho_{0}(p) \\
  \label{eq_grkp}
  B_{k}(s)        & = & \exp \left [ - \frac{(r_{k}\sigma_{Q}(s))^{2}}
                        {2}\right ] \\
  \label{eq_bk}
  A_{ij}(p)       & = & \frac{c_{i}(p)c_{j}(p)b_{i}b_{j}}
                        {\langle b \rangle ^{2}}
  \label{eq_aij}           \\
  T_{ij}(r_{k},p) & = & \frac{1}{\sqrt{2\pi}\sigma_{ij}(p)}
                        \exp \left [ - \frac{(r_{k}-r_{ij}(p))^{2}}
                        {2 \sigma_{ij}^{2}(p)} \right ]
                        \left [ 1 + \left ( \frac{r_{k} - r_{ij}}{r_{ij}}
                        \right ) \right ]
  \label{eq_tij}
\end{eqnarray}

\noindent where $S_{p}(r_{k},s_{i})$ is a dampening function for
finite nano-particles discussed in the next section. $f_{s}$ stands
for the overall scale factor and $B_{k}(s)$ is the experimental
resolution factor for data set $s$. The first sum in (\ref{eq_grk})
is over the different structural phases $p$ in a multi phase
refinement. The relative abundance of each phase $p$ is given by
$f_{p}$. $G_{p}(r_{k},s)$ is the model PDF for a single phase $p$
given in (\ref{eq_grkp}). These values are actually stored in an
array after the PDF is calculated. The indices $i$ and $j$ sum over
all atoms within the structural phase $p$ or in other words we sum
over all atom {\it pairs} in that phase. The contribution of each
pair of atoms $i$ and $j$ is weighted with a factor $A_{ij}(p)$
which depends on the scattering length $b_{i}$ and concentration
$c_{i}$ of both atoms. The average scattering length for the sample
({\bf summed over all phases}) is $\langle b \rangle =
\sum_{i}c_{i}b_{i}/N_{p}$. The average number density is given by
$\rho_{0}(p) = N_{p}/V_{p}$ where $N_{p} = \sum_{i}c_{i}$ is the
number of atoms within phase $p$ and $V_{p}$ is the volume of the
unit cell in \AA$^{3}$. The calculation of $V_{p}$ is presented
later in this section. \par

The final term in (\ref{eq_grk}) we need to discuss is
$T_{ij}(r_{k},p)$. Generally there are two different ways to
account for displacements (either thermal or static) from the
average position. First one can use a large enough model
containing the desired displacements and perform an ensemble
average. This is the method used by the program {\it DISCUS} where
thermal displacements can be introduced according to a given
(isotropic) Debye-Waller factor. Secondly one can convolute each
contribution given by $\delta (r - r_{ij})$ in (\ref{eq_gr}) with
a modified Gaussian $T_{ij}(r_{k},p)$ accounting for the
displacements. The deviation from a Gaussian shape is caused by
anisotropic averaging and discussed in detail in \cite{thlele02}.
The width of the function $T_{ij}$ is given by the anisotropic
thermal factors $U_{lm} = \langle u_{l}u_{m} \rangle$ of atoms $i$
and $j$. Furthermore the $\sigma_{ij}(p)$ shows an $r$-dependence
given in (\ref{eq_sharp}). Note that this definition has been
extended since version 1.2 of PDFFIT. In some cases an additional
sharpening of the PDF peaks below a user defined cutoff value,
$r_{cut}$ needs to be introduced (see \ref{fit_pwid} for details).
This is done in (\ref{eq_cut}).

\begin{equation}
  \sigma_{ij}(p) = \phi(r_{ij},p) \cdot
                   \sqrt {\sigma_{ij}^{'2}(p) -
                          \frac{\delta(p)}{r_{ij}^{2}(p)} -
                          \frac{\gamma(p)}{r_{ij}(p)} +
                          \alpha^{2}(s) r_{ij}^{2}(p)}
  \label{eq_sharp}
\end{equation}

\noindent
with

\begin{equation}
  \phi(r_{ij},p) = \left \{ \begin{array}{ll}
                    \phi_{0}(p) & \mbox{for $r_{ij} < r_{cut}$} \\
                    1.0         & \mbox{otherwise} \\
                    \end{array} \right .
  \label{eq_cut}
\end{equation}

\noindent
and

\begin{equation}
  \sigma'_{ij}(p) = \frac{1}{|r_{ij}|} \sqrt{\sum_{l,m=1}^{3} \left[
                    r^{l}_{ij}(p)r^{m}_{ij}(p) (U_{lm}(i) + U_{lm}(j))
                    \right ]}
  \label{eq_therm}
\end{equation}

\noindent
with $r^{l}_{ij}$ standing for the $l$-th component of the difference
vector between atoms $i$ and $j$. Note that so far $r_{ij}$ referred to
a scalar, the distance between the atoms without any directional
information. \par

The last piece of the puzzle is the relation between $r_{ij}(p)$ and
$V_{p}$ and the lattice parameters $a,b,c,\alpha,\beta$ and $\gamma$.
{\it PDFFIT} works with fractional coordinates, i.e. in units
of the unit cell. However, to calculate the distance between two atoms,
the difference vector must be converted to \AA. A helpful concept when
dealing with non-orthogonal coordinate systems is the so-called
{\it metric tensor}, $g_{ij}$, which is defined as

\begin{equation}
   g_{ij} = {\bf{a}}_{i} \cdot {\bf{a}}_{j} =
   \left ( \begin{array}{ccc}
           a^{2}         & ab \cos\gamma & ac \cos\beta  \\
           ab \cos\gamma & b^{2}         & bc \cos\alpha \\
           ac \cos\beta  & bc \cos\alpha & c^{2}
           \end{array} \right )
   \label{eq_metric}
\end{equation}

\noindent
with $\bf{a}_{i}$ describing the basis vectors of the crystal lattice
in an orthonormal system. Using this metric tensor, the scalar
product of two vectors $\bf{u}$ and $\bf{v}$ is simply ${\bf{uv}}
= \sum_{ij} u_{i}v_{j}g_{ij}$ or \begin{math} V_{p} = abc
[1 - \cos^{2}\alpha - \cos^{2}\beta - \cos^{2}\gamma + 2 \cos\alpha
\cos\beta \cos\gamma]^{\frac{1}{2}} \end{math} which is simply the
determinant of the metric tensor $g_{ij}$. \par

Let us assume we want to calculate the distance between the atoms $i$
and $j$ on positions ${\bf u}_{i}$ and ${\bf u}_{j}$ in the crystal
system (We will omit the reference to phase $p$ is this part). Now the
difference vector still in fractional coordinates is simply
${\bf d}_{ij} = {\bf u}_{j} - {\bf u}_{i}$. In order to get the
components $r^{l}_{ij}$ needed to calculate $\sigma'_{ij}$ in (\ref
{eq_therm}), we simply multiply with the corresponding lattice parameter:

\begin{equation}
  r^{1}_{ij} = a d^{1}_{ij}, \;
  r^{2}_{ij} = b d^{2}_{ij}, \;
  r^{3}_{ij} = c d^{3}_{ij}
  \label{eq_met1}
\end{equation}

\noindent
The length of the difference vector in \AA\ is given by
$r_{ij} = [{\bf d}_{ij} \cdot {\bf d}_{ij}] ^{\frac{1}{2}}$ which can
be written as follows using (\ref{eq_metric}):

\begin{eqnarray}
  r_{ij} & = & [ a^{2} d^{2}_{ij_{x}} +
                 b^{2} d^{2}_{ij_{y}} +
                 c^{2} d^{2}_{ij_{z}} + \nonumber \\
         &   &   2 ab \cos\gamma d_{ij_{x}}d_{ij_{y}} +
                 2 ac \cos\beta  d_{ij_{x}}d_{ij_{z}} +
                 2 bc \cos\alpha d_{ij_{y}}d_{ij_{z}} ] ^{\frac{1}{2}}
  \label{eq_met2}
\end{eqnarray}

Now we have all necessary relationships to calculate $G(r_{k},s)$ from
the experimental and structural parameters we might want to refine.
The needed analytical derivatives are given in the next section and the
actual parameter names used by {\it PDFFIT} are listed in the last
section of this appendix.

%------------------------------------------------------------------------

\section{Particle size envelopes}

With the possibility to measure and model PDFs over a wide range in
distances $r$, the finite size of nano-particles requires an extra
envelope function $S_{p}(r_{k},s_{i})$ that will depend on the
shape, size and size distribution of the measured particles. For
long range ordered crystals, this factor simply becomes unity. The
parameters $s_{i}$ are the refinable quantities defining the shape,
size and distribution.

%------------------------------------------------------------------------

\subsection*{Sphere}

At the moment only spherical particles can be simulated. In this
case the only refinable parameter, $s_{1}=R$, is the radius of the
particle and the envelope function becomes\cite{hoprco06} :

\begin{equation}
  S_{p}(r_{k},R) = \left ( 1 - \frac{3} {4} \left ( \frac{r_{k}}{R} \right )
                             + \frac{1}{16} \left ( \frac{r_{k}}{R} \right )^{3}
                          \right ) \Theta (2R - r_{k})
  \label{eq_shape_sphere}
\end{equation}

\noindent with  $\Theta(x)=0$ for $x<0$ and $\Theta(x)=1$ for $x
\geq 0$. More shapes and distributions will be added as their
analytical expression become available.

%------------------------------------------------------------------------

\section{Partial derivatives}

It should be noted that the derivatives given in this section
actually need to be convoluted with the function given in
(\ref{eq_sinc}) as well since $\partial (G(r,p) * C(r)) / \partial p
= (\partial G(r,p) / \partial p) * C(r)$.

\subsection*{Scale factors}

The derivatives with respect to the two scale factors $f_{s}$ and
$f_{p}$ are straight forward and listed below:

\begin{eqnarray}
  \frac{\partial G(r_{k},s)}{\partial f_{s}} & = &
     B_{k}(s) \sum_{p=1}^{P} f_{p} S_{p}(r_{k},s_{i}) G_{p}(r_{k},s) \\
  \frac{\partial G(r_{k},s)}{\partial f_{p}} & = &
     f_{s}B_{k}(s) S_{p}(r_{k},s_{i}) G_{p}(r_{k},s)
  \label{eq_d/ds}
\end{eqnarray}

%------------------------------------------------------------------------

\subsection*{Experimental resolution factor}

The experimental resolution $\sigma_{Q}$ only appears in $B_{k}(s)$,
thus the derivative is simply:

\begin{equation}
  \frac{\partial G(r_{k},s)}{\partial \sigma_{Q}(s)} =
     - r_{k}^{2} \sigma_{Q}(s) \; f_{s}B_{k}(s)
                 \sum_{p=1}^{p} f_{p} S_{p}(r_{k},s_{i}) G_{p}(r_{k},s)
  \label{eq_d/dsq}
\end{equation}

%------------------------------------------------------------------------

\subsection*{Quadratic dynamic correlation factor}

The dynamic correlation factor $\delta(p)$ appears only in $T_{ij}
(r_{k},p)$ where $\sigma_{ij}(p)$ appears. Thus we can rewrite the
derivative for a given phase $p$ (we omit the reference to $p$ in
the following equations) as follows:

\begin{equation}
  \frac{\partial G(r_{k},s)}{\partial \delta} =
    \frac{f_{s}f_{p} S_{p}(r_{k},s_{i}) B_{k}(s)}{N_{p}r_{k}} \sum_{i} \sum_{j}
    A_{ij} \frac{\partial T_{ij}(r_{k})}{\partial \delta}
  \label{eq_d/delta1}
\end{equation}

\noindent
with

\begin{eqnarray}
  \frac{\partial T_{ij}(r_{k})}{\partial \delta} & = &
    \frac{\partial T_{ij}(r_{k})}{\partial \sigma_{ij}}
    \frac{\partial \sigma_{ij}} {\partial \delta} \nonumber \\
  & = & \frac{T_{ij}(r_{k})}{\sigma_{ij}}
    \left [\frac{(r_{k} - r_{ij})^{2}}{\sigma_{ij}^{2}} - 1 \right ]\cdot
    \frac{- \phi^{2}(r_{ij})}{2 r_{ij}^{2} \sigma_{ij}}
  \label{eq_d/delta2}
\end{eqnarray}

\noindent
which gives the final result for the derivative of

\begin{equation}
  \frac{\partial G(r_{k},s)}{\partial \delta} =
    - \frac{f_{s}f_{p} S_{p}(r_{k},s_{i}) B_{k}(s)}{N_{p}r_{k}} \sum_{i} \sum_{j}
      \frac{\phi^{2}(r_{ij})A_{ij}(p)T_{ij}(r_{k})}{2 r_{ij}^{2} \sigma^{2}_{ij}}
      \left [ \frac{(r_{k} - r_{ij})^{2}}{\sigma_{ij}^{2}} - 1
      \right ]
  \label{eq_d/delta3}
\end{equation}

%------------------------------------------------------------------------

\subsection*{Linear dynamic correlation factor}

The linear correlation factor $\gamma(p)$ appears only where
$\sigma_{ij}$ appears. Similar to (\ref{eq_d/delta1}) we find
using

\begin{equation}
  \frac{\partial \sigma_{ij}}{\partial \gamma}=
  \frac{- \phi^{2}(r_{ij})}{2  r_{ij} \sigma_{ij}}
  \label{eq_d/gamma1}
\end{equation}

\noindent which gives the resulting expression of

\begin{equation}
  \frac{\partial G(r_{k},s)}{\partial \gamma} =
    - \frac{f_{s}f_{p} S_{p}(r_{k},s_{i}) B_{k}(s)}{N_{p}r_{k}} \sum_{i} \sum_{j}
      \frac{\phi^{2}(r_{ij})A_{ij}(p)T_{ij}(r_{k})}{2 r_{ij} \sigma^{2}_{ij}}
      \left [ \frac{(r_{k} - r_{ij})^{2}}{\sigma_{ij}^{2}} - 1
      \right ]
  \label{eq_d/gamma2}
\end{equation}

%------------------------------------------------------------------------

\subsection*{Resolution broadening factor}

The resolution broadening $\alpha(s)$ appears only where
$\sigma_{ij}$ appears. We omit the reference to $s$ in the
following equations for clarity. Similar to (\ref{eq_d/delta1}) we
find using

\begin{equation}
  \frac{\partial \sigma_{ij}}{\partial \alpha}=
  \frac{\alpha \phi^{2}(r_{ij}) r_{ij}^{2}}{\sigma_{ij}}
  \label{eq_d/alpha1}
\end{equation}

\noindent which gives the resulting expression of

\begin{equation}
  \frac{\partial G(r_{k},s)}{\partial \gamma} =
      \frac{f_{s}f_{p} S_{p}(r_{k},s_{i}) B_{k}(s)}{N_{p}r_{k}} \sum_{i} \sum_{j}
      \frac{\alpha \phi^{2}(r_{ij}) r_{ij}^{2}
            A_{ij}(p)T_{ij}(r_{k})}{\sigma^{2}_{ij}}
      \left [ \frac{(r_{k} - r_{ij})^{2}}{\sigma_{ij}^{2}} - 1
      \right ]
  \label{eq_d/alpha2}
\end{equation}

%------------------------------------------------------------------------

\subsection*{Peak with ratio}

The parameter $\phi_{0}$ describing the additional sharpening of the
PDF peaks below $r_{cut}$ appears only where $\sigma_{ij}$ appears.
For values of $r$ above $r_{cut}$, the derivative is zero since no
term depends on $\phi_{0}$. For values of $r$ below $r_{cut}$, we
can write similar to (\ref{eq_d/delta1})

\begin{equation}
  \frac{\partial G(r_{k},s)}{\partial \phi_{0}} =
    \frac{f_{s}f_{p} S_{p}(r_{k},s_{i}) B_{k}(s)}{N_{p}r_{k}} \sum_{i} \sum_{j}
    A_{ij} \frac{\partial T_{ij}(r_{k})}{\partial \phi_{0}}
  \label{eq_d/rat1}
\end{equation}

\noindent
with

\begin{eqnarray}
  \frac{\partial T_{ij}(r_{k})}{\partial \phi_{0}} & = &
    \frac{\partial T_{ij}(r_{k})}{\partial \sigma_{ij}}
    \frac{\partial \sigma_{ij}} {\partial \phi_{0}} \nonumber \\
  & = & \frac{T_{ij}(r_{k})}{\sigma_{ij}}
    \left [ \frac{(r_{k} - r_{ij})^{2}}{\sigma_{ij}^{2}} - 1 \right ] \cdot
    \frac{\sigma_{ij}}{\phi_{0}}.
  \label{eq_d/rat2}
\end{eqnarray}

%------------------------------------------------------------------------

\subsection*{Lattice parameters}

The lattice parameters $a_{n}$ (for now symbolic for $a, b, c, \alpha,
\beta$ and $\gamma$) appear in $r_{ij}(p), \sigma_{ij}(p)$ and
the unit cell volume $V_{p}$. Thus only $T_{ij}(r_{k},p)$ and
$\rho_{0}(p)$ depend on these parameters. Again we omit the reference
to the phase $p$ in the equations below. Let us consider the
derivative of $T_{ij}$ first.

\begin{eqnarray}
  \frac{\partial T_{ij}(r_{k})}{\partial a_{n}} & = &
     \frac{\partial T_{ij}(r_{k})}{\partial \sigma_{ij}}
     \frac{\partial \sigma_{ij}} {\partial a_{n}} +
     \frac{\partial T_{ij}(r_{k})}{\partial r_{ij}}
     \frac{\partial r_{ij}}      {\partial a_{n}} \nonumber \\
  & = & \frac{T_{ij}(r_{k})}{\sigma_{ij}} \left \{ \left [
     \frac{(r_{k} - r_{ij})^{2}}{\sigma_{ij}^{2}} - 1 \right ]
     \frac{\partial \sigma_{ij}} {\partial a_{n}} +
     \left [ \frac{r_{k} - r_{ij}}{\sigma_{ij}}
     - \frac{\sigma_{ij} r_{k}}{r^{2}_{ij}}    \right ]
     \frac{\partial r_{ij}}{\partial a_{n}} \right \}
  \label{eq_d/da1}
\end{eqnarray}

\noindent
Using Equation (\ref{eq_sharp}) we find for the derivative of $\sigma_{ij}$
with respect to the lattice parameters:

\begin{eqnarray}
    \frac{\partial \sigma_{ij}} {\partial a_{n}} & = &
    \frac{\partial \sigma_{ij}} {\partial \sigma'_{ij}}
    \frac{\partial \sigma'_{ij}}{\partial a_{n}} +
    \frac{\partial \sigma_{ij}} {\partial r_{ij}}
    \frac{\partial r_{ij}}      {\partial a_{n}}  \nonumber  \\
    & = &
    \frac{\phi^{2}(r_{ij})}{2 \sigma_{ij}} \left [
    \frac{\partial \sigma'_{ij}}{\partial a_{n}} \cdot 2 \sigma'_{ij}
    +
    \frac{\partial r_{ij}}{\partial a_{n}} \left [
    2 \alpha^{2} r_{ij} + \frac{2 \delta}{r^{3}_{ij}} +
    \frac{\gamma}{r^{2}_{ij}} \right ] \right ]
  \label{eq_d/da2}
\end{eqnarray}

\noindent
The definition of $\sigma_{ij}$ (\ref{eq_therm}) shows that the lattice
parameters appear in the distance $r_{ij}$ as well as the components
of the difference vector $r_{ij}^{l}$ (\ref{eq_met1}). Using these
relations we get

\begin{eqnarray}
  \frac{\partial \sigma_{ij}'} {\partial a_{n}} & = &
    \frac{\partial \sigma_{ij}'} {\partial r^{l}_{ij}}
    \frac{\partial r^{l}_{ij}}   {\partial a_{n}} +
    \frac{\partial \sigma_{ij}'} {\partial r_{ij}}
    \frac{\partial r_{ij}} {\partial a_{n}} \nonumber \\
  & = & \frac{1}{r_{ij}^{2} \sigma_{ij}'} \left [
    \sum_{m=1}^{3} a_{m}d_{ij}^{n}d_{ij}^{m} \left (
    U_{nm}(i) + U_{nm}(j) \right ) \right ] -
    \frac{\sigma_{ij}'}{r_{ij}}
    \frac{\partial r_{ij}} {\partial a_{n}}
  \label{eq_d/da3}
\end{eqnarray}

\noindent
The first term in (\ref{eq_d/da3}) is zero for $n=4,5,6$, in other
words for the derivatives with respect to the cell angles. Next we
show the derivative of the distance $r_{ij}$ with respect to the
lattice parameters. Note that we now use $a,b,c,\alpha,\beta$ and
$\gamma$ explicitly rather than $a_{n}$ as before.

\begin{eqnarray}
  \frac{\partial r_{ij}} {\partial a} & = &
    \frac{1}{r_{ij}} (a d_{ij_{x}}^{2} + b\cos\gamma d_{ij_{x}}d_{ij_{y}} +
    c \cos\beta d_{ij_{x}} d_{ij_{z}} )  \\
  \frac{\partial r_{ij}} {\partial b} & = &
    \frac{1}{r_{ij}} (b d_{ij_{y}}^{2} + a\cos\gamma d_{ij_{x}}d_{ij_{y}} +
    c \cos\alpha d_{ij_{y}} d_{ij_{z}} )  \\
  \frac{\partial r_{ij}} {\partial c} & = &
    \frac{1}{r_{ij}} (c d_{ij_{z}}^{2} + a\cos\beta d_{ij_{x}}d_{ij_{z}} +
    b \cos\alpha d_{ij_{y}} d_{ij_{z}} )  \\
  \frac{\partial r_{ij}} {\partial \alpha} & = &
    - \frac{bc}{r_{ij}} (\sin\alpha d_{ij_{y}} d_{ij_{z}} ) \\
  \frac{\partial r_{ij}} {\partial \beta}  & = &
    - \frac{ac}{r_{ij}} (\sin\beta  d_{ij_{x}} d_{ij_{z}} ) \\
  \frac{\partial r_{ij}} {\partial \gamma} & = &
    - \frac{ab}{r_{ij}} (\sin\gamma d_{ij_{x}} d_{ij_{y}} )
  \label{eq_d/da4}
\end{eqnarray}

Finally we need see who the number density $\rho_{0} = N_{p}/V_{p}$ depends
on the lattice parameters. Considering the definition of the unit cell
volume, we get

\begin{equation}
  \frac{\partial \rho_{0}} {\partial a_{n}} =
    - \frac{N_{p}}{V_{p}^{2}} \frac {\partial V_{p}} {\partial a_{n}} =
    \left \{ \begin{array}{ll}
             - \frac{\rho_{0}}{a_{n}} & \mbox{for $a_{n}=a,b,c$} \\ & \\
             - \frac{\rho_{0}}{V_{p}^{2}}a^{2}b^{2}c^{2}
               (\sin\alpha\cos\alpha - \sin\alpha\cos\beta\cos\gamma)
                                          & \mbox{for $a_{n}=\alpha$} \\
             - \frac{\rho_{0}}{V_{p}^{2}}a^{2}b^{2}c^{2}
               (\sin\beta\cos\beta   - \sin\beta\cos\alpha\cos\gamma)
                                          & \mbox{for $a_{n}=\beta$} \\
             - \frac{\rho_{0}}{V_{p}^{2}}a^{2}b^{2}c^{2}
               (\sin\gamma\cos\gamma - \sin\gamma\cos\alpha\cos\beta)
                                          & \mbox{for $a_{n}=\gamma$} \\
             \end{array}\right.
   \label{eq_d/da5}
\end{equation}

\noindent
Using all equations given in this section, the desired derivative of the
calculated PDF $G_{p}(r_{k},s)$ with respect to the lattice parameters
can be calculated using

\begin{equation}
  \frac{\partial G_{p}(r_{k},s)} {\partial a_{n}} =
    f_{p}f_{s} S_{p}(r_{k},s_{i}) B_{k}(s) \left \{ \frac{1}{N_{p}r_{k}} \sum_{i} \sum_{j}
    \left [ A_{ij} \frac{\partial T_{ij}(r_{k})}{\partial a_{n}} \right ]
    - 4 \pi r_{k} \frac{\partial \rho_{0}} {\partial a_{n}} \right \}
  \label{eq_d/da6}
\end{equation}

%------------------------------------------------------------------------

\subsection*{Thermal parameters}

Only the PDF peak width given by  $\sigma_{ij}$ is a function of the
thermal parameters $U_{lm}(i)$ and $\sigma_{ij}$ appears only in $T_{ij}
(r_{k})$. Thus with some of the work already done in the previous
section, the desired derivative can be written as

\begin{equation}
  \frac{\partial G(r_{k},s)} {\partial U_{lm}(i)} =
    \frac{f_{p}f_{s} S_{p}(r_{k},s_{i}) B_{k}(s)}{N_{p}r_{k}} \sum_{i} \sum_{j}
    \left [ A_{ij}(p) \frac{\partial T_{ij}(r_{k})}{\partial \sigma_{ij}(p)}
    \; \frac{\partial \sigma_{ij}(p)}{\partial \sigma_{ij}'(p)}
    \; \frac{\partial \sigma_{ij}'(p)}{\partial U_{lm}(i)} \right ]
  \label{eq_d/du1}
\end{equation}

\noindent With $\partial T_{ij} / \partial \sigma_{ij}$ already
calculated in (\ref{eq_d/da1}) and $\partial \sigma_{ij} /
\partial \sigma_{ij}' = \phi^{2}(r_{ij})/2\sigma_{ij}$ we only need

\begin{equation}
  \frac{\partial \sigma_{ij}'}{\partial U_{lm}(i)} =
    \frac{1}{2 r_{ij}^{2} \sigma_{ij}'} r_{ij}^{l}r_{ij}^{m}
  \label{eq_d/du2}
\end{equation}

%------------------------------------------------------------------------

\subsection*{Atomic positions}

The atomic positions ${\bf v} = (v_{1},v_{2},v_{3})$ of one specific
atom $i$ in phase $p$ appear only in $T_{ij}(r_{k})$ in form of
$\sigma_{ij}$ and $r_{ij}$. We can simply substitute $a_{n}$ in equations
(\ref{eq_d/da1}) and (\ref{eq_d/da2}) with the fractional coordinate and get

\begin{eqnarray}
  \frac{\partial T_{ij}(r_{k})}{\partial v_{n}(i)} & = &
     \frac{\partial T_{ij}(r_{k})}{\partial \sigma_{ij}}
     \frac{\partial \sigma_{ij}} {\partial v_{n}(i)} +
     \frac{\partial T_{ij}(r_{k})}{\partial r_{ij}}
     \frac{\partial r_{ij}}      {\partial v_{n}(i)} \nonumber \\
  & = & \frac{T_{ij}(r_{k})}{\sigma_{ij}} \left \{ \left [
     \frac{(r_{k} - r_{ij})^{2}}{\sigma_{ij}^{2}} - 1 \right ]
     \frac{\partial \sigma_{ij}} {\partial v_{n}(i)} +
     \left [ \frac{r_{k} - r_{ij}}{\sigma_{ij}}
           - \frac{\sigma_{ij} r_{k}}{r^{2}_{ij}} \right ]
     \frac{\partial r_{ij}}{\partial v_{n}(i)} \right \}
  \label{eq_d/df1}
\end{eqnarray}

\begin{equation}
    \frac{\partial \sigma_{ij}}{\partial v_{n}(i)} =
    \frac{\phi^{2}(r_{ij})}{2 \sigma_{ij}} \left [
    \frac{\partial \sigma'_{ij}}{\partial v_{n}(i)} \cdot 2
    \sigma'_{ij} +
    \frac{\partial r_{ij}}{\partial v_{n}(i)} \left [
    2 \alpha^{2} r_{ij} + \frac{2 \delta}{r^{3}_{ij}} +
    \frac{\gamma}{r^{2}_{ij}} \right ] \right ]
  \label{eq_d/df2}
\end{equation}

\noindent
with

\begin{equation}
  \frac{\partial \sigma_{ij}'} {\partial v_{n}(i)} =
    \frac{1}{r_{ij}^{2}\sigma_{ij}'} \left [ \sum_{m=1}^{3}
    a_{n}a_{m}d_{ij}^{m} ( U_{nm}(i) + U_{nm}(j)) \right ] -
    \frac{\sigma_{ij}'}{r_{ij}} \;
    \frac{\partial r_{ij}} {\partial v_{n}(i)}
  \label{eq_d/df3}
\end{equation}

\noindent
and

\begin{eqnarray}
  \frac{\partial r_{ij}} {\partial v_{1}(i)} =
  \frac{\partial r_{ij}}     {\partial d_{ij_{1}}} \;
  \frac{\partial d_{ij_{1}}} {\partial v_{1}(i)} & = &
   -\frac{1}{r_{ij}}(a^{2}d_{ij_{1}}+ab\cos\gamma d_{ij_{2}}+
                                     ac\cos\beta  d_{ij_{3}})\\
  \frac{\partial r_{ij}} {\partial v_{2}(i)} & = &
   -\frac{1}{r_{ij}}(b^{2}d_{ij_{2}}+ab\cos\gamma d_{ij_{1}}+
                                     bc\cos\alpha d_{ij_{3}})\\
  \frac{\partial r_{ij}} {\partial v_{3}(i)} & = &
   -\frac{1}{r_{ij}}(c^{2}d_{ij_{3}}+ac\cos\beta  d_{ij_{1}}+
                                     bc\cos\alpha d_{ij_{2}})
  \label{eq_d/df4}
\end{eqnarray}

\noindent
Note that $\partial T_{ij}(r_{k}) / \partial v_{n}(i) = - \partial
T_{ij}(r_{k}) / \partial v_{n}(j)$ because $d_{ij} = - d_{ji}$.

%------------------------------------------------------------------------

\subsection*{Site occupancies}

The site occupancies $c_{n}$ for a given phase $p$ appear in the
terms $A_{ij}$, $N_{p}$ and $\rho_{0}$. Thus the derivative is

\begin{eqnarray}
  \frac{\partial G_{p}}{\partial c_{n}} & = &
      \frac{\partial G_{p}} {\partial N_{p}}
      \frac{\partial N_{p}} {\partial c_{n}} +
      \frac{\partial G_{p}} {\partial A_{ij}}
      \frac{\partial A_{ij}}{\partial c_{n}} +
      \frac{\partial G_{p}} {\partial \rho_{0}}
      \frac{\partial \rho_{0}}{\partial c_{n}}\\
  & = & f_{p}f_{s} S_{p}(r_{k},s_{i}) B_{k}(s) \left \{
      \frac{1}{N_{p}r_{k}} \left [
      \frac{1}{c_{n}}\sum_{j}A_{nj}T_{nj}(r_{k}) -
      \frac{1}{N_{p}}\sum_{i}\sum_{j}A_{ij}T_{ij}(r_{k}) \right ] -
      \frac{4 \pi r_{k}}{V_{p}} \right \} \nonumber
  \label{eq_d/dc}
\end{eqnarray}

%------------------------------------------------------------------------

\subsection*{Particle envelopes}

In this section we give the derivatives with respect to the
different particle shape envelope functions $S_{p}(r_{k},s_{i})$ for
the implemented shapes.

\subsubsection*{Sphere}

Since only the shape function $S_{p}(r_{k},R)$ depends on the radius
$R$, we can use equation \ref{eq_shape_sphere} and write the
derivative as:

\begin{eqnarray}
  \frac{\partial G(r_{k},s)}{\partial R} & = &
     f_{s} B_{k}(s) f_{p} G_{p}(r_{k},s) \cdot
     \frac{\partial S_{p}(r_{k},R)}{\partial R} \nonumber \\
  & = &
     f_{s} B_{k}(s) f_{p} G_{p}(r_{k},s)
     \cdot 3 r_{k}
     \left [ \frac{1}{4 R^{2}} - \frac{r_{k}^{2}}{16 R^{4}}\right ]
     \Theta (2 R - r_{k})
\end{eqnarray}

%------------------------------------------------------------------------

\section{Calculating standard deviations}

The derivatives presented in the last section were with respect to the
experimental and structural parameters, which we will refer to as
$x_{i}$ in this section. {\it PDFFIT} allows the user
to freely define their relationship to the refinement parameters
$p_{i}$. However, for the least square refinement the derivatives with
respect to those refinement parameters are needed. They can be obtained
using the following relationship

\begin{equation}
  \frac{\partial G(r_{k},s)}{\partial p_{i}} =
    \sum_{j=1}^{N} \frac{\partial G(r_{k},s)}{\partial x_{j}} \;
                   \frac{\partial x_{j}}     {\partial p_{i}}
\end{equation}

The sum goes over the number of parameters $x$. The derivatives
$\partial x_{j} / \partial p_{i}$ must be specified analytically by
the user using the command {\tt par}. These derivatives are also
used to calculate the resulting standard deviations, $\Delta x_{i}$,
of the parameters $x_{i}$ from the errors, $\Delta p_{i}$ of the
refinement parameters using

\begin{equation}
  \Delta x_{i} = \sqrt {\sum_{n} \left (
    \frac{\partial x_{i}}{\partial p_{n}} \Delta p_{n} \right ) ^{2}}
\end{equation}

The sum goes over all refinement parameters $p_{i}$, however in practice
each structural parameter $x_{i}$ is only allowed to depend on a small
number of refinement parameters, so only a few derivatives contribute
to the sum. \par

Finally we will give the definitions of the R-values calculated in
each cycle of the refinement. The expected R-value is defined as

\begin{equation}
  R_{exp} = \sqrt {\frac{ N - p}
                  {\sum\limits_{i=1}^{N} w(r_{i}) G_{obs}^{2}(r_{i})}}
  \label{eq_rexp}
\end{equation}

\noindent
with $G_{obs}$ being the experimental PDF, $N$ the number of data
points and $p$ the number of (refined) parameters. The weight for
each data point is given by $w(r_{i})$. The weighted R-value is calculated
as follows

\begin{equation}
  R_{w} = \sqrt {\frac{\sum\limits_{i=1}^{N} w(r_{i})
                       [G_{obs}(r_{i}) - G_{calc}(r_{i})]^{2}}
                {\sum\limits_{i=1}^{N} w(r_{i}) G_{obs}^{2}(r_{i})}}
  \label{eq_rw}
\end{equation}

\noindent
Here $G_{calc}$ is the calculated PDF. The unweighted R-value is
simply defined as in (\ref{eq_rw}) just with the weights $w(r_{i})$
set to unity.

%------------------------------------------------------------------------
