\chapter{PDFFIT commands}
\section{Summary}
Here is a short summary of the PDFFIT specific commands currently 
available: 
\par
\begin{MacVerbatim}
alloc   : Allocates space for calculating PDFs without reading data
bang    : Calculates bond angles with standard deviations
blen    : Calculates bond lengths with standard deviations
calc    : Calculates PDF from current structure
cycle   : Sets the maximum number of refinement cycles
i/jdese : Selects atoms used for calculating the PDF
i/jsele : Deselects atoms used for calculating the PDF
occ     : Sets occupancy for a given site
par     : Sets a parameter definition
phase   : Switches active structural phase
psel    : Associates structural phases with data sets
range   : Sets r-range used for refinement
read    : Reads structure and PDF data files
reset   : Resets PDFFIT
run     : Starts a refinement
save    : Saves resulting structure, PDF data and fit results
scat    : Overwrites internal scattering factor for given element
shape   : Sets particle shape parameters
show    : Shows various information
temp    : Sets anisotropic temperature factor
urf     : Sets URF (magic refinement number)
xray    : Specifies Q-value for calculating X-ray form factors
\end{MacVerbatim}
\section{alloc}
{\bf alloc $ \{$"x"$| $"n"$\} $,$ <$qmax$> $,$ <$qsig$> $,$ <$rmin$> $,$ <$rmax$> $,$ <$np$> $ \par }
\par
\vspace{3pt}
This command allows the user to allocate space when just calculating 
a PDF. The first three parameters are similar to 'read data' and 
the next three parameters specify the range in r to be from $ <$rmin$> $ 
to $ <$rmax$> $ and the number of points $ <$np$> $. Note that this command 
allocates the space for one data set without actually reading any 
data ! Use the command 'reset' before reading "real" data for a 
refinement. 
\section{bang}
{\bf bang $ <$a1$> $,$ <$a2$> $,$ <$a3$> $ \par }
\par
\vspace{3pt}
This command calculates the bond angle and its standard deviation 
for the atoms numbered $ <$a1$> $, $ <$a2$> $ and $ <$a3$> $. The angle is measured 
at $ <$a2$> $. Note that no periodic boundaries are applied and it might 
be necessary to add +-1 to the coordinates of some of the atoms to 
get the desired angle (-$> $ variables). 
\section{blen}
{\bf blen $ <$a1$> $,$ <$a2$> $ \par }
{\bf blen $ \{$"all"$| $$ <$name$> $$| $$ <$number$> $$\} $,$ \{$"all"$| $$ <$name$> $$| $$ <$number$> $$\} $,$ <$b1$> $,$ <$b2$> $ \par }
\par
\vspace{3pt}
The simple form of this command calculates the bond length and its 
standard deviation between the atoms number $ <$a1$> $ and $ <$a2$> $. Note 
that the 'real' distance within the crystal is given and *no* 
periodic boundary conditions are applied. 
\par
The second command calculates the bond lengths between the atoms selected 
by the first two parameters that fall in the range specified by 
$ <$b1$> $ and $ <$b2$> $. As usual atoms can be selected by name or type number 
or simply use "all" to select all atoms. Here periodic boundaries 
are applied like for the PDF calculation itself. Thus the results 
of the two versions of 'blen' might be different for the same atom 
number pair ! 
\section{calc}
{\bf calc \par }
\par
\vspace{3pt}
This command starts the calculation of the PDF for the current 
settings. Note that the current refinement parameter settings are 
applied before the PDFs corresponding to all currently loaded data 
sets are calculated. Note that the command 'save' must be used 
to save the result(s). 
\section{cycle}
{\bf cycle $ <$n$> $ \par }
\par
\vspace{3pt}
This command sets the maximum number of refinement cycles to $ <$n$> $. 
If a minimum is found in less cycles, the fit will stop before $ <$n$> $ 
cycles are finished. 
\section{functions}
\par
The following PDFFIT specific functions exist. For a listing 
of general intrinsic functions see help entry 'functions' in 
the 'command language' section of the online help. 
\par
\begin{MacVerbatim}
bang(u1,u2,u3,v1,v2,v3[,w1,w2,w3]) :
    Returns the bond angle at the site v. If only vectors u and v
    are given, the angle between u and v is returned.

blen(u1,u2,u3[,v1,v2,v3]) :
    Returns the length of vector v-u. Vector v defaults to zero.

dstar(h1,h2,h2[,k1,k2,k3]) :
    Returns the length of reciprocal vector k-h. Vector k defaults
    to zero.

rang(h1,h2,h3,k1,k2,k3[,l1,l2,l3]) :
    Returns the angle between vectors k-h and k-l at reciprocal site k.
    If l is omitted, the angle between the reciprocal vectors h and k
    is returned.

gran(val,typ) :
    Returns Gaussian distributed pseudo random number with mean zero
    an a width given by parameter <val>. If <typ> is "s" <val> is taken
    as sigma, if <typ> is "f", <val> is taken as FWHM.

gbox(r1,r2,r3) :
    Returns pseudo random number with distribution given by a box
    centered at 0 with a width of <r2> and two half Gaussian
    distributions with individual sigmas of <r1> and <r3> to the
    left and right, respectively.
rval(s [,rmin,rmax])
    Returns the weighted R-value corresponding to data set <s>.
    The optional parameters allow the user to calculate the R-value
    for a specific region in r.
\end{MacVerbatim}
\section{i/jdese}
{\bf idese $ <$is$> $, $ \{$ "all" $| $ $ <$name$> $ $| $ $ <$number$> $ $\} $, [ ... ] \par }
{\bf jdese $ <$is$> $, $ \{$ "all" $| $ $ <$name$> $ $| $ $ <$number$> $ $\} $, [ ... ] \par }
\par
\vspace{3pt}
This command deselects atom types given either by $ <$name$> $ or $ <$number$> $ 
for the PDF calculation. The two commands allow one to deselect 
atom types for each atom in a pair 'ij' contributing to the PDF 
calculation. The first parameters $ <$is$> $ specifies the number of the 
corresponding data set. 
\section{i/jsele}
{\bf isele $ <$is$> $, $ \{$ "all" $| $ $ <$name$> $ $| $ $ <$number$> $ $\} $, [ ... ] \par }
{\bf jsele $ <$is$> $, $ \{$ "all" $| $ $ <$name$> $ $| $ $ <$number$> $ $\} $, [ ... ] \par }
\par
\vspace{3pt}
This command selects atom types given either by $ <$name$> $ or $ <$number$> $ 
for the PDF calculation. All other atoms are ignored. This allows 
the calculation of differential or partial PDFs. The two commands 
allow one to select atom types for each atom in a pair 'ij' 
contributing to the PDF calculation. The first parameters $ <$is$> $ 
specifies the number of the corresponding data set. 
\section{occ}
{\bf occ $ \{$"all" $| $ $ <$name$> $ $| $ $ <$number$> $$\} $,$ <$o$> $ \par }
\par
\vspace{3pt}
This command allows the user to set the occupancy of a given 
site. The parameter "all" will result in the assignment of 
the occupancy $ <$o$> $ to all atoms of the currently active phase 
(-$> $ phase). Alternatively a specific atom name $ <$name$> $ or even 
more specific the number $ <$number$> $ of a single atom type can be 
used. Note that for DISCUS compatibility the occupancies are 
not part of the structure file and must be set separately. 
The default value is 1.0. 
\section{par}
{\bf par $ <$x=f(p(i))$> $,$ <$dx/dp1$> $,$ <$dx/dp2$> $,... \par }
{\bf par "reset" \par }
\par
\vspace{3pt}
This command defines the relation between fit parameters (p[i]) 
and experimental and structural parameters, e.g. x[i]. There is 
no predefined sequence or order. For each parameter p[i] that is 
used in the definition, the corresponding partial derivative must 
be given. The user has also to take care of proper starting 
values for the parameters which can be assigned with p[i]=$ <$expr$> $. 
The parameter settings can be deleted using the argument "reset". 
Note that all expressions within arguments ([]) are evaluated 
after the command is entered, thus parameter definitions can be 
made using loops. However, other variables are saves as variables 
and may change during the fit. For a detailed discussion refer to 
the PDFFIT users guide. Here are some examples. 
\par
\begin{MacVerbatim}
Examples
par x[1]=p[1],1.0        : x of atom 1 is refined using p[1]
par o[1]=r[1]*p[5],r[1]  : occupancy of atom 1 is r[1]*p[5] and
                           the derivative is r[1] (simple isn't it)
\end{MacVerbatim}
\section{phase}
{\bf phase [$ <$ip$> $] \par }
\par
\vspace{3pt}
PDFFIT allows the usage of multiple phases. All structural 
commands and setting will affect the currently active phase. 
The command 'phase' allows the user to set the active phase 
to $ <$ip$> $. Alternatively the command can be used without a 
parameter in which case the current setting will be displayed. 
\section{psel}
{\bf psel $ <$is$> $,"all" \par }
{\bf psel $ <$is$> $,$ <$ip1$> $ [,$ <$ip2$> $, ...] \par }
\par
\vspace{3pt}
This command allows the user to associate certain phases $ <$ip1$> $ 
to $ <$ipn$> $ with a given data set $ <$is$> $. One extreme would be to have 
e.g. two data sets and one phase each completely separate which is 
not really different from running the refinements separately. 
However, by using identical refinement parameters one can constrain 
any structural parameters between the phases. An example is given 
in the Users Guide. The default is that all structural phases are 
associated with all loaded data sets. 
\section{range}
{\bf range $ \{$"all" $| $ $ <$is$> $$\} $,$ <$rmin$> $,$ <$rmax$> $ \par }
\par
\vspace{3pt}
This command sets the range in R that is used for fitting 
either for "all" read data sets or for just the specified 
data set $ <$is$> $. The limits $ <$rmin$> $ and $ <$rmax$> $ are given in 
A. 
\section{read}
This command reads various information from a specified file. 
The following formats are currently supported: 
\par
\subsection*{stru}
{\bf read "stru",$ <$file$> $ \par }
\par
\vspace{3pt}
The command 'read stru' allows the user to read a DISCUS 
type structure from the file $ <$file$> $. Note that some DISCUS 
keywords and the use of molecules are not supported by 
PDFFIT. To read structural information for multiple phases 
simply repeat the 'read' command. In order to discard 
the current structure use the command -$> $ 'reset'. Note that 
DISCUS only supports an isotropic temperature factor B. 
When reading a DISCUS structure the temperature factor is 
converted according to: $ <$u(ii)$> $ = Bglm-1/(8pi**2al*am*). 
PDFFIT automatically recognizes the structure file format 
of the input file ! 
\subsection*{data}
{\bf read "data",$ \{$"n" $| $ "x"$\} $,$ <$qmax$> $,$ <$qsig$> $,$ <$file$> $ \par }
\par
\vspace{3pt}
The command 'read data' reads the observed PDF. The file 
format is ASCII and contains 'r G(r) dummy dG(r)' in each line. 
The value of 'r' is in A, G(r) is the reduced PDF. The 
third column is ignored (needed for KUPLOT) and the last 
value 'dG' is the error of the PDF used to calculate the 
weight (w=1/dg**2) for this point to be used for the 
refinement. Alternative formats are 'r G(r) dG(r)' in each 
line or simple 'r G(r)'. In the later case, the weights are 
set to unity. This is also done in case the value of dG(r) 
is found as zero. Additional to the filename $ <$file$> $ the commands 
needs the following parameters: First the type of radiation 
is specified, "n" stands for neutrons and "x" for X-rays. 
Next the maximum value of Q $ <$qmax$> $ is given followed by the 
value of $ <$qsig$> $ defining the Q resolution. To read multiple 
data sets just repeat the 'read' command. 
\par
If the data file contains a history part created by PDFgetN, 
some of the information is returned in the res[n] variables. 
Currently the following information is available after the 
'read data' command: 
\par
\begin{MacVerbatim}
res[1]   : Temperature where the data were collected (in K)
res[2]   : Qmax (only AFTER the data were read !)
\end{MacVerbatim}
\subsection*{diff}
{\bf read "diff",$ \{$"n" $| $ "x"$\} $,$ <$qmax$> $,$ <$qsig$> $,$ <$file$> $,$ <$rfile$> $ \par }
\par
\vspace{3pt}
This command is similar to 'read data' but rather than reading 
a PDF, a difference PDF is read from file $ <$file$> $. The format 
is identical to the PDF file format. The additional parameter 
$ <$rfile$> $ specifies a reference PDF, so that the model PDF fitted 
to the read data becomes CALC - REFERENCE. See Users Guide for 
more details. 
\section{reset}
{\bf reset \par }
\par
\vspace{3pt}
This command resets PDFFIT, e.g. all current structural and 
PDF information is lost. 
\section{run}
{\bf run \par }
\par
\vspace{3pt}
This command starts the least square refinement ... 
\section{save}
{\bf save $ <$subcommand$> $ \par }
\par
\vspace{3pt}
This command allows to write different information to a file. 
The valid subcommands are: 
\par
\subsection*{atoms}
{\bf save "atoms",$ <$ip$> $,$ <$file$> $ \par }
\par
\vspace{3pt}
This command saves the structure of phase $ <$ip$> $ in a format suitable 
for import into ATOMS using the 'File - Import - Free Form' function. 
In case of multiple unit cells, a supercell containing the complete 
crystal is created. Anisotropic thermal parameters are exported and 
thermal ellipsoids can be viewed using ATOMS. 
\subsection*{const}
{\bf save "const",$ <$file$> $ \par }
\par
\vspace{3pt}
This command saves the parameter constraints to a text file. 
The output is similar to the screen output created by 'show const'. 
\subsection*{dif}
{\bf save "dif",$ <$is$> $,$ <$file$> $ \par }
\par
\vspace{3pt}
This command saves the difference between observed and calculated 
PDF of data set $ <$is$> $ to a file named $ <$file$> $. 
\subsection*{discus}
{\bf save "discus",$ <$ip$> $,$ <$file$> $ \par }
\par
\vspace{3pt}
This command saves the structural phase $ <$ip$> $ to the file named 
$ <$file$> $ in DISCUS format. Note that DISCUS uses only an isotropic 
B value which is computed according to B=8pi**2/3(u11+u22+u33). 
The information about anisotropic temperature factors, occupancies 
and standard deviations is NOT saved. 
\subsection*{mark}
{\bf save "mark", $ <$ip$> $,$ <$is$> $,$ <$file$> $ \par }
\par
\vspace{3pt}
This commands saves the positions and frequency of atom pairs 
for phase $ <$ip$> $ in the r-range given by data set $ <$is$> $ to file $ <$file$> $. 
The file is a simple two column ASCII file with columns "r" and 
"Number", giving the distance "r" and the amoutn of pairs "Number". 
This can be used to plot distance markers. 
\subsection*{pdf}
{\bf save "pdf" ,$ <$is$> $,$ <$file$> $ \par }
\par
\vspace{3pt}
The command 'save pdf' saves the calculated PDF corresponding 
to the given data set $ <$is$> $ to the file $ <$file$> $. The format 
is identical to the input format (-$> $ read). 
\subsection*{result}
{\bf save "result",$ <$file$> $ \par }
\par
\vspace{3pt}
This command saves all information about the refinement to 
a text file named $ <$file$> $. The output is similar to the screen 
output of the 'show' commands. 
\subsection*{stru}
{\bf save "stru",$ <$ip$> $,$ <$file$> $ \par }
\par
\vspace{3pt}
The command 'save stru' saves the structure of the phase 
specified by $ <$ip$> $ to the file $ <$file$> $. The file format is 
related to the DISCUS format. However it contains additional 
information like anisotropic temperature factors, occupancies 
and standard deviations for all values. These files can be 
read again by PDFFIT. To save a structural phase in DISCUS 
format use -$> $ 'save discus,...'. 
\section{scat}
{\bf scat $ <$s$> $,$ \{$$ <$name$> $$| $$ <$number$> $$\} $,$ <$a1$> $,$ <$b1$> $,$ <$a2$> $,$ <$b2$> $,$ <$a3$> $,$ <$b3$> $,$ <$a4$> $,$ <$b4$> $,$ <$c$> $ \par }
{\bf scat $ <$s$> $,$ \{$$ <$name$> $$| $$ <$number$> $$\} $,$ <$c$> $ \par }
{\bf scat $ <$s$> $,$ \{$$ <$name$> $$| $$ <$number$> $$\} $,"internal" \par }
\par
\vspace{3pt}
This command allows the user to specify a new scattering factor 
for data set $ <$s$> $ and atom $ <$name$> $ (or $ <$number$> $) of the current 
phase. In case data set $ <$s$> $ are neutron scattering data just a 
single scattering length $ <$c$> $ is needed, in case of X-rays, the 
scattering factor is given in exponential form. Finally if the 
last parameter is "internal", the internal scattering factor 
are used again. 
\section{shape}
{\bf shape $ \{$'none','sphere'$\} $,[$ <$s1$> $, $ <$s2$> $, ..] \par }
\par
\vspace{3pt}
 This commands allows one to set parameters when calculating the 
 PDF of a finite particle. The fist parameter is describing the 
 particle shape, following parameters are related to the selected 
 shape. 
\par
\begin{MacVerbatim}
 none     This deselects any particle shape.
 sphere   This selects a shperical particle. The only parameter <s1>
\end{MacVerbatim}
          is the radius of the particle. 
\section{show}
{\bf show $ <$subcommand$> $,.. \par }
\par
\vspace{3pt}
This command allows to obtain various information about the 
model crystal, PDFFIT settings etc. 
\par
Valid subcommands are : 
\par
\subsection*{atom}
{\bf show "atom",$ <$number$> $ [,$ <$end$> $] \par }
{\bf show "atom","all" \par }
{\bf show "atom","last" \par }
\par
\vspace{3pt}
Information about the atom(s) such as position, thermal factors, 
atom type and occupancy are listed for the current phase (-$> $ phase). 
If the optional second parameter $ <$end$> $ is given, all atoms in the 
range $ <$number$> $ to $ <$end$> $ are shown.  If the second parameter is "all", 
all atoms in the crystal are shown.  WARNING, this could last a while 
:-) If the second parameter is "last", only the last atom of the 
crystal is shown. This is identical to setting the second parameter 
to "n[1]", which contains the number of atoms in the crystal. 
\subsection*{config}
{\bf show "config" \par }
\par
\vspace{3pt}
This commands lists the limits of various arrays like the maximum 
number of phases, atoms, parameters etc. In case one of the limits 
is not sufficient, PDFFIT needs to be recompiled with adjusted 
limits. 
\subsection*{const}
{\bf show "const" [,$ <$ip$> $] \par }
\par
\vspace{3pt}
This command displays all constraint equations and derivatives 
that contain parameter $ <$ip$> $. If $ <$ip$> $ is omitted, all definitions 
are show, which can be quite a long list ! 
\subsection*{elem}
{\bf show "elem" [,$ <$ip$> $] \par }
\par
\vspace{3pt}
This command lists information about the elements within the 
crystal. If the parameters $ <$ip$> $ is omitted, the corresponding 
information for all phases is shown, otherwise just for the 
specified phase $ <$ip$> $. 
\subsection*{fit}
{\bf show "fit" \par }
\par
\vspace{3pt}
This command outputs refinement information such as R-values 
and the correlation matrix. 
\subsection*{metric}
{\bf show "metric" \par }
\par
\vspace{3pt}
This command show the current lattice parameters, metric tensor 
as well as reciprocal lattice parameters and the reciprocal 
metric tensor for the currently active structural phase 
(-$> $ phase). 
\subsection*{par}
{\bf show "par" \par }
\par
\vspace{3pt}
This command lists the current fit parameters on the screen. 
\subsection*{pdf}
{\bf show "pdf" [,$ <$is$> $] \par }
\par
\vspace{3pt}
This command lists information about PDF related settings. 
If the parameters $ <$is$> $ is omitted, the corresponding 
information for all data sets is shown, otherwise just for the 
specified set $ <$is$> $. 
\subsection*{phase}
{\bf show "phase" \par }
\par
\vspace{3pt}
This command displays all information about the currently 
active phase on the screen. 
\subsection*{scat}
{\bf show "scat",$ \{$ "all" $| $ $ <$name$> $ $| $ $ <$number$> $ $\} $ \par }
\par
\vspace{3pt}
This command allows the user to display the scattering lengths 
for all or just a specific atom within the currently selected 
phase (-$> $ phase). 
\section{temp}
{\bf temp $ \{$"all"$| $$ <$name$> $$| $$ <$number$> $$\} $,$ <$u11$> $ \par }
{\bf temp $ \{$"all"$| $$ <$name$> $$| $$ <$number$> $$\} $,$ <$u11$> $,$ <$u22$> $,$ <$u33$> $ \par }
{\bf temp $ \{$"all"$| $$ <$name$> $$| $$ <$number$> $$\} $,$ <$u11$> $,$ <$u22$> $,$ <$u33$> $,$ <$u12$> $,$ <$u13$> $,$ <$u23$> $ \par }
\par
\vspace{3pt}
This command sets the anisotropic temperature factor U(ij) for 
"all" or the atoms specified either by $ <$name$> $ or atom type 
$ <$number$> $. If just a single further parameter $ <$u11$> $ is given, 
an isotropic temperature factor is assumed. Otherwise just 
the diagonal or all elements of the tensor can be specified. 
Note that U(i,j)=$ <$u(i)u(j)$> $ and specifically for the diagonal 
elements U(ii)=$ <$u(i)**2$> $ ! 
\section{urf}
{\bf urf $ <$u$> $ \par }
\par
\vspace{3pt}
This command sets the URF (some German: Unterer Relaxations Faktor) 
for the fit. This value determines who 'fast' the fit will move to 
its minimum or how much the parameter values are changed in each 
cycle depending on the deviations. A small value (e.g. 0.1) might 
lead to a fast convergence but might also miss the minimum. A larger 
value (e.g. 100.0) will give a slow convergence which more certain 
finds the minimum, but might be caught in local minima rather than in 
the global one. 
\par
Understood ? Well just try different values until your fit converges 
nicely to the global minimum. 
\section{variables}
{\bf Variables \par }
\par
\vspace{3pt}
PDFFIT allows the usage of variables. The general variables 
i[n], r[n] and res[n] are explained in the 'command language' 
section of the online help (-$> $ command language). The variables 
marked with RO below are READONLY. 
\begin{MacVerbatim}

x[n]      : x-position of atom n of current phase (-> phase)
y[n]      : y-position of atom n of current phase (-> phase)
z[n]      : z-position of atom n of current phase (-> phase)
m[n]      : Number of atom type on site n (current phase)
u[6,n]    : Aniso. thermal factor U(ij) for site n (curr. phase)
o[n]      : Occupancy of site n (current phase)

lat[6]    : Lattice parameters (a,b,c,alpha,beta,gamma) f. phase
delt[n]   : Value of DELTA (sharpening !) for phase 'n'
gamm[n]   : Value of GAMMA (sharpening !) for phase 'n'
rcut[n]   : Cutoff r-value for additional sharpening for phase 'n'
srat[n]   : Ratio of peak width below and above 'rcut' for phase 'n'
csca[n]   : Value of scaling factor for phase 'n'
rhoz[n]   : Value of number density RHO0 phase 'n' (RO)
cmax[n]   : Maximum correlation 'r' for phase 'n'
shap[n,i] : Particle shape parameter 'i' for phase 'n'

qsig[s]   : Value of QSIGMA (resolution dampening) for data set 's'
qalp[s]   : Value of QALPHA (resolution broadening) for data set 's'
dsca[s]   : Value of scaling factor for data set 's'
bave[s]   : Average scatt. length <b> data set 's' (current phase)

dx[n]     : Standard deviation of x-position
dy[n]     : Standard deviation of y-position
dz[n]     : Standard deviation of z-position
du[6,n]   : Standard deviation of u[6,n]  (RO)
do[n]     : Standard deviation of o[n]    (RO)
dlat[6]   : Standard deviation of lat[6]
ddelt[n]  : Standard deviation of delt[n] (RO)
dgamm[n]  : Standard deviation of gamm[n] (RO)
dsrat[n]  : Standard deviation of srat[n] (RO)
dcsca[n]  : Standard deviation of csca[n] (RO)
drhoz[n]  : Standard deviation of rho0[n] (RO)
ddsca[s]  : Standard deviation of dsca[s] (RO)
dqsig[s]  : Standard deviation of qsig[s] (RO)
dqalp[s]  : Standard deviation of qalp[s] (RO)
dshap[n,i]: Standard deviation of shap[p,i]  (RO)

n[1]      : Total number of atoms of current phase (RO)
n[2]      : Number of different atoms of current phase (RO)
n[3]      : Number of atoms per unit cell (current phase) (RO)
n[4]      : Number of phases (RO)
n[5]      : Number of current phase (change with -> phase) (RO)
n[6]      : Number of loaded experimental PDFs (RO)
n[7]      : Number of possible parameters (RO)
n[8]      : Number of used refinement parameters (RO)

p[n]      : Value of fit parameter n
dp[n]     : Value of standard deviation for parameter n (RO)
pf[n]     : Refinement flag for parameter n, 1=refine 0=fixed
cl[n,m]   : Correlation matrix element n,m (RO)

rw[1]     : expected R-value (after refinement) (RO)
rw[2]     : achieved R-value (after refinement) (RO)
rw[3]     : achieved weighted R-value (after refinement) (RO)

np[s]     : Number of data points of data set 's' (RO)
pc[n,s]   : Value of calculated PDF point 'n' of data set 's'
po[n,s]   : Value of observed PDF point 'n' of data set 's'
pw[n,s]   : Weight for PDF point 'n' of data set 's'
delr[s]   : Value of DELTAR of data set 's'
qmax[s]   : Value of QMAX for data set 's'
rang[2,s] : Value of minimum (1) and maximum (2) R for refinement
rmin[s]   : Value of maximum R of data set 's'
rmax[s]   : Value of maximum R of data set 's'

fa[i,s,p] : Value of matrix A(i,s,p) (see Users Guide)
fb[i,s,p] : Value of matrix B(i,s,p)
fc[p,dp]  : Value of matrix C(p,dp)
\end{MacVerbatim}
\section{xray}
\begin{MacVerbatim}
xray [<xq>]
\end{MacVerbatim}
This command sets the Q-value used to calculate the scattering 
length used in the PDF calculation. The default value is xq=0 
which results in a weight corresponding to the number of electrons 
of the contributing atoms. Other settings could be the Q value 
of the first Bragg peak or the average Q value of the data set. 
Calling the command without parameters prints the current setting 
on the screen. 
\section{errors}
\par
The program has been written such that it should handle almost 
any typing error when giving commands and hopefully all errors 
that result from calculation with erroneous data. When an error 
is found an error message is displayed that should get you back 
on track. See the manual for a complete list of error messages. 
\par
Error messages concerning the command language are explained in 
the -$> $ command language section of the online help. Application 
specific commands are described here and are grouped as follows: 
\par
\begin{MacVerbatim}
APPL   Errors specific to PDFFIT structure functions
PDF    Errors specific to PDFFIT PDF calculation and fit func.
\end{MacVerbatim}
Each error message is displayed together with the corresponding 
category $ <$cccc$> $ and the error number $ <$numb$> $ in the form: 
\par
{\bf ****CCCC****message                    **** numb **** \par }
\par
\vspace{3pt}
Type help error $ <$cccc$> $ $ <$numb$> $ to get an explanation for the error 
message and hints for possible steps to correct the situation. 
\par
In the default mode PDFFIT returns the standard prompt and you can 
continue the execution from this point. You can set the error status 
to "exit" by the ==$> $'set' command. In this case PDFFIT terminates 
if an error is detected. This option is useful to terminate a faulty 
sequence of commands when running PDFFIT in the batch mode of your 
operating system. 
\par
\subsection*{appl}
\par
This category lists error messages that are specific to 
structure related functions of PDFFIT. 
\par
\subsubsection{Error -2: Improper limits for atom number}
\par
Either of the upper or lower limits used on the 'append' command is 
outside the range of atoms present in the crystal. Check whether 
the limits are both positive, the upper limit is higher or equal to 
the lower limit and whether both limits are less or equal to the 
number of atoms present in the crystal. The number of atoms in the 
crystal can be checked with the command: "eval n[1]". 
\subsubsection{Error -3: No atoms selected yet}
\par
The plot and waves can only be run for selected atoms. Use the 'select' 
command to select individual atom types or to select all atoms present 
in the crystal. 
\subsubsection{Error -7: Unknown space group symbol}
\par
The crystal file contains an unknown space group. Check the spelling of 
the space group symbol. The allowed space groups are all 230 space 
groups in the Int. Tables for Crystallography plus the space groups 
given for alternative settings and cell choices of the monoclinic 
space groups. 
\subsubsection{Error -10: Too many Atoms in crystal}
\par
The maximum number of atoms that can be stored in the structure 
is determined by the parameter NMAX in the file "crystal.inc". 
By inserting new atoms or by defining too large a crystal on the 
'read' command, this number was exceeded. If necessary, change the 
value of the parameter NMAX and recompile the program. 
\subsubsection{Error -12: Number of points must be $> $ zero}
\par
The value of the parameter given on the 'abs' or 'ord' command is 
less than one. This value represents the number of data points 
calculated along the respective direction. 
The value must be at least one or higher. 
\subsubsection{Error -14: Invalid space group \& lattice constants}
\par
While reading a new cell the program checks the space group and the 
lattice constants for consistency. Either a space group was given 
that is not included in the program, or the lattice constants do 
not fulfill the constraints imposed by the space group. Check the 
space group symbol and the lattice constants given in the input file. 
\subsubsection{Error -19: Atom number outside limits}
\par
The number of the atom is either less than one, larger than the current 
number of atoms in the crystal or even larger than the maximum 
number of atoms allowed in your implementation. 
Check the value of the parameter(s) on the 'remove' and 'switch' 
commands or check the index of the variables "m", "x", "y" or "z". 
Check the number of atoms present in the crystal by the command: 
'eval n[1]'. 
\subsubsection{Error -20: Unknown element, no Fourier calculated}
\par
An element was detected in the list of atoms for which there is no 
scattering curve available. The Fourier transform is not calculated 
at all. Check the name of all atoms present in the crystal using 
the 'asym' and 'chem' commands. If a charged ion was given, this 
valence might not be present in the list of scattering curves. 
Refer to Appendix b of the manual for a list of internally stored 
names. 
If the 'scat' and/or 'delf' commands were used, any name may be used. 
Check whether the commands were used properly. 
\subsubsection{Error -21: No element present, no Fourier calculated}
\par
There are no elements present at all in the crystal. The Fourier 
transform is not calculated at all. Most likely, the Fourier 
was called before a structure or unit cell was read, or an error 
occurred during the reading of the structure or unit cell. 
\subsubsection{Error -26: Too many different atoms in crystal}
\par
The maximum number of different atoms allowed in your implementation 
was exceeded. No more new types of atoms can be inserted into the 
structure. Check the chemistry of your crystal by the 'asym' and 
'chem' commands. All atoms are considered different types that are 
chemically different, have different charge or a different temperature 
coefficient. If all types are needed, modify the parameter "maxscat" 
in the file "param.inc" and recompile the program. See chapter 9.1 
of the manual for further information. 
\subsubsection{Error -27: No atom of this type present in crystal}
\par
An atom was selected for displacement by a wave or for plotting that 
does not exist within the crystal. Check the spelling of the atom 
name, and the chemistry of the crystal by the 'asym' and 'chem' 
commands. 
\subsubsection{Error -28: Input parameters must be $> $ zero}
\par
This function/command requires non-negative parameters. 
Check the values of the parameters and the explanation for the function 
or command used for valid ranges of numerical input. 
\subsubsection{Error -32: Length of vector is zero}
\par
An attempt was made to calculate the angle between two vectors 
while one of them is of length zero. Check the parameters given 
on the 'bang' or 'rang' command for proper numbers. 
\subsubsection{Error -35: Volume of unit cell $ <$= zero}
\par
The volume of the unit cell was calculated as zero or a negative 
value. 
Check the lattice parameters given in the input file. Are there 
any accidental "-" signs ? Do the angles form an impossible shape ? 
\subsubsection{Error -37: No filename defined yet}
\par
An attempt was made to write output to or read from a file without 
defining a file name. PDFFIT does not provide default names for the 
output of the 'plot' command or the input filenames. Check the 
'outfile' command at sublevels 'output' and 'plot' or the 'content' 
command at sublevel 'microdomains'. 
\subsubsection{Error -43: Not enough parameter for filename format}
\par
An attempt was made to generate a file name from a string like 
"text\%dtext" without supplying enough numerical parameters. 
Check that the spelling of the sting within " " is correct. Are 
there any unwanted \%d combinations?. Check the number and type 
of parameters following the file name. 
\subsubsection{Error -44: Right quotation mark missing in format}
\par
An attempt was made to generate a file name from a string like 
"text\%dtext" without supplying the right quotation mark. 
Check the line and provide the missing ". 
\subsubsection{Error -45: Too many atoms in environment}
\par
The indices of all atoms found are stored in the internal variable 
"res". More atoms were found that fit into the dimensions of "res". 
Restrict the search for the environment to a smaller shell or 
change the dimension of "res\_para" in file "param.inc". 
\subsubsection{Error -46: Error reading title of structure}
\par
An error occurred while reading the title line of a structure or 
unit cell file. Check the file for any garbage. 
\subsubsection{Error -47: Error reading space group symbol}
\par
An error occured while reading the space group  line of a structure or 
unit cell file. Check the file for any garbage. 
\subsubsection{Error -48: Error reading lattice constants}
\par
An error occured while reading the lattice constants of a structure 
or unit cell file. Check the file for any garbage or accidental letters. 
\subsubsection{Error -49: Error reading atom coordinates}
\par
An error occured while reading the atom coordinates of the atom 
listed.  Check the file for any garbage. Is the line of the type 
Name x y z B 
Are there letters among the numerical values? 
\subsubsection{Error -54: Index outside limits}
\par
The value given is outside the proper limits allowed by this command. 
This usually means that an array element is outside the current 
dimension of an array, for example a correlation matrix or you are 
trying to include too many atoms in the crystal. Check the section 
on dimensions in the manual. 
\subsection*{pdf}
\par
This category lists error messages that are specific PDF 
and refinement functions of PDFFIT. 
\par
\subsubsection{Error -1: Invalid structure phase selected}
\par
The number of an invalid phase was given. One can only enter 
phase specific commands when the corresponding structure was 
actually read via 'read stru'. 
\subsubsection{Error -2: Number of phases outside limits}
\par
When trying to read another structural phase, the maximum number 
of phases was exceeded. You might need to adjust the parameter 
MAXPHA in 'config.inc' and recompile PDFFIT. 
\subsubsection{Error -3: Invalid occupancy specified}
\par
A site occupancy not in the range 0 to 1 was specified for an 
atomic site. Check your input. 
\subsubsection{Error -4: Maximum number of data sets exceeded}
\par
When reading a PDF data set, the maximum number of data sets was 
exceeded. You might need to adjust the parameter MAXDSET in the 
file 'config.inc' and recompile PDFFIT. 
\subsubsection{Error -5: Invalid radiation type selected}
\par
The only radiation types allowed as parameter for the 'read data' 
command are 'n' for neutrons and 'x' for X-rays. Check your input. 
\subsubsection{Error -6: Invalid data set specified}
\par
The number of an invalid data set was given. One can only enter 
data set specific commands when the corresponding data file was 
actually read via 'read data'. 
\subsubsection{Error -7: Invalid parameter constraint specified}
\par
The parameter constraint specified using 'par' is invalid. Either 
it contains no "=" or tries to assign a non refinable variable to 
a refinement parameter. 
\subsubsection{Error -8: Too many different parameters in constr.}
\par
You have entered a definition, that depends on too many different 
refinement parameters. You might need to adjust the variable 
MAXDPP in the file 'config.inc' and recompile PDFFIT. 
\subsubsection{Error -9: Inconsistency between NCELL and \# atoms}
\par
When reading a structure file, the number of atoms actually found 
does not match the number of atoms expected from the parameters 
of the 'ncell' keyword. For 'ncell nx,ny,nz,n' you expect to read 
nx.ny.nz.n atoms. 
\subsubsection{Error -10: Invalid parameter selected}
\par
An invalid refinement parameter was selected. Use the command 
'show config' to see what the maximum number of refinement 
parameters is you can use. 
\subsubsection{Error -11: Too many data points in experimental PDF}
\par
The PDF data file to be read contains too many data points. You 
might need to adjust the parameter MAXDAT in the file 'config.inc' 
and recompile PDFFIT. 
\subsubsection{Error -12: Cannot extend r-range for convolution}
\par
In order to calculate the PDF, the calculated function is convoluted 
with a SINC function with a width determined by Qmax. In oder to 
carry out the convolution a finite number of data points outside 
the refinement range are needed. You have to use a lower RMAX for 
the refinement or reprocess the PDF up to a higher R value. 
\subsubsection{Error -13: R-value out of range}
\par
You have specified a value of R that is outside the range for the 
specified data set. Check your input. 
\subsubsection{Error -14: Parameter index in p[i] out of range}
\par
You have used a parameter index 'i' that is out of range. Note that 
the maximum allowed value of 'i' is determined by the maximum number 
of parameters that can be used for a given set of data and structural 
phases. 
