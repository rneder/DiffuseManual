%------------------------------------------------------------------------
% Example
%------------------------------------------------------------------------

\chapter{Example: CeF$_3$\label{s:exa}}

In this chapter we will illustrate the use of \Mixscat using the example 
of CeF$_3$. The neutron data were collected on the NPDF neutron powder
diffractometer NPDF at the Lujan Neutron Scattering Center. The x-ray data
were collected at beamline ID11-B at the Advanced Photon Source. The 
data are shown in Fig. \ref{f:data}. All input and macro files used 
in this example are included in the program distribution and can be found 
in the directory {\tt examples/mixscat} in the installation directory.

%------------------------------------------------------------------------
\section{Preparing the data\label{s:exa:dat}}

As we discussed in the introduction, \Mixscat used the real space 
pair distribution function, $G(r)$ and calculates the desired differential
PDF, $G_{\overline{AB}}(r)$. For details on the related notation and 
equations, refer to Appendix \ref{app}. \Mixscat can read files produced by 
\textsc{PDFgetN}\,\citep{pdfgetn} and \textsc{PDFgetX2}\,\citep{pdfgetx2} 
directly and extract some information such as $Q_{\mbox{\footnotesize{max}}}
$ from the header. Alternatively data files are simple multi column ASCII
files containing $r$, $G(r)$ and an optional $\sigma(G(r))$.

Since we are taking a difference of $G(r)$ data, it is important to process 
the PDFs as similar as possible. In particular the truncation value $Q_
{\mbox{\footnotesize{max}}}$ should be the same for both data sets.

\begin{figure}[!bt]
  \centering
  \includegraphics[angle=270.0,width=0.9\linewidth]{cef-data.eps}
  \caption{The top panel shows the experimental neutron PDF of CeF$_3$
  collected on the instrument NPDF at the Lujan Center. The bottom
  panel shows the x-ray data of the same sample collected at beamline
  11ID-B at the Advanced Photon Source.}
  \label{f:data}
\end{figure}

%------------------------------------------------------------------------
\section{Extracting differential PDFs\label{s:exa:calc}}

\begin{figure}[!bt]
  \centering
  \includegraphics[angle=270.0,width=0.9\linewidth]{cef-result.eps}
  \caption{The plot shows the three extracted differential 
  $G_{\overline{mn}}$ for CeF$_3$. The line with error bars is the
  extracted differential PDF form the neutron and x-ray data. The
  solid thick line is calculated from a model using DISCUS (see
  next section).}
  \label{f:res}
\end{figure}

As the other programs of the \textsc{Discus} package, \Mixscat is controlled by a command language which is described in more detail in the \textsc{Discus} reference guide included in the package. Let us go through the CeF$_3$ example in detail. The resulting differential PDFs as well as model calculations are shown in Fig. \ref{f:res}.

\begin{MacVerbatim}
 reset
 read dat,x,CeF3_bulk_binned.gr
 read dat,n,CeF3_Bulk_npdf_03902.gr
\end{MacVerbatim}

First we use the command {\tt reset} to put \Mixscat in its initial state. 
Next the x-ray and neutron data files are imported using the command {\tt 
read}. Note the first parameter specifying the type of data.

\begin{MacVerbatim}  
 match 0.5,2.0,24./320.864
 elem Ce,1,F,3
 remove Ce,Ce
 show
 calc
\end{MacVerbatim}

The section above contains the setup. The command {\tt match} allows to 
automatically scale the data, so the slopes $-4 \pi r \rho_0$ match. The 
first two parameters specify the range, here $0.5 < r < 2.0$\AA. The last 
value is the number density $\rho_0$ calculated as the number of atoms in 
the unit cell divided by the unit cell volume. As an alternative individual 
scale factors for each data set can be specified using the command {\tt scal}. 
In a perfect world, of course, non of this is needed. The command {\tt elem} 
specifies the elements and their abundance (CeF$_3$). Next we specify the 
contribution {\tt Ce-Ce} to be removed. Finally {\tt show} will list the 
current settings and {\tt calc} execute the calculation.

\begin{MacVerbatim}  
 save pdf,CeCe_new_diff.gr
 save wei,w_CeCe.mac
 save res,CeCe.res
\end{MacVerbatim}

This section saves different data and results. The command {\tt save pdf} saves 
the extracted differential PDF, $G_{\overline{CeCe}}(r)$. Next {\tt save wei} 
saves a macro file for the program \textsc{Discus} containing the weights for 
the remaining pairs. This way \textsc{Discus} can calculate the corresponding 
differential PDF from a structural model. The output in our example is shown 
below.

\begin{MacVerbatim}  
 set partial,CE  ,CE  ,       0.0000000    
 set partial,CE  ,F   ,       16.266106    
 set partial,F   ,F   ,       21.585993    
 set rden,       18.241911 
\end{MacVerbatim}

Finally the command {\tt save res} will save a text file with the setup similar 
to the information shown on the screen by the command {\tt show}. The output for 
our example is shown below.

\begin{MacVerbatim}
 ------------------------------------------------------------------------------
 Program MIXSCAT - Version 1.0.0     
 ------------------------------------------------------------------------------
 Build date : Tue Jan  5 21:02:52 JST 2010       
 Save date  : Tue Jan  5 21:06:04 2010
 Homepage   : http://discus.sourceforge.net
 ------------------------------------------------------------------------------

 ------------------------------------------------------------------------------
 DATA SETS
 ------------------------------------------------------------------------------
 Data set    :                         1                           2
 Filename    :       CeF3_bulk_binned.gr     CeF3_Bulk_npdf_03902.gr
 Radiation   :       x-rays (  0.0 A^-1)                    Neutrons
 Rmin (A)    :                     0.010                       0.010
 Rmax (A)    :                    99.990                      20.000
 Dr (A)      :                     0.010                       0.010
 Qmax (A^-1) :                     0.000                      35.000
 Scale       :                1.42648721                  0.05844205
 ------------------------------------------------------------------------------
 SETUP - Removing contribution : CE - CE
 ------------------------------------------------------------------------------
 Data set    :                         1                           2
 Weight w_mn :                0.46553183                  0.04904367
 <b>^2       :              451.33862305                  0.29729757
 b(CE  )     :               57.98102951                  0.48300001    c=0.250
 b(F   )     :                8.99930000                  0.56599998    c=0.750
 ------------------------------------------------------------------------------
 DIFFERENTIAL WEIGHTS
 ------------------------------------------------------------------------------
 w(CE  -CE  ):                0.00000000
 w(CE  -F   ):                3.04989457
 w(F   -F   ):               12.14212132
 w(rho_0)    :               18.24191093
 ------------------------------------------------------------------------------
\end{MacVerbatim}

The output contains information about the data sets and the respective weights 
for each atom and atom pair. The bottom of the output file shows the result from 
a simple error estimate of the differential PDF (see below).

\begin{MacVerbatim}
 ------------------------------------------------------------------------------
 ERROR ANALYSIS - Removed contribution :   CE -   CE        Difference:  2 -  1
 ------------------------------------------------------------------------------
 Data                       File      Average data      Average sigma  Relative
    1        CeF3_bulk_binned.gr        2.75641990         0.00012269     0.00%
    2    CeF3_Bulk_npdf_03902.gr       29.27724457         5.72750998    19.56%
 ------------------------------------------------------------------------------
          Extracted differential       30.33419609         6.82508755    22.50%
 ------------------------------------------------------------------------------
\end{MacVerbatim}

The other differential PDFs, $G_{\overline{CeF}}(r)$ and $G_{\overline{FF}}(r)$ 
can be calculated in a similar way.

%------------------------------------------------------------------------
\section{Calculating differential PDFs from a model\label{s:exa:model}}

As shown in Fig. \ref{f:res} one can calculate the corresponding differential PDF  
from a structural model. The program \textsc{Discus} allows to read custom 
weights to be used in the PDF calculation. These weights are set using the macro 
file {\tt w\_CeCe.mac} we saved earlier. Simply add the line below in the {\tt 
pdf} module of \textsc{Discus}.

\begin{MacVerbatim}  
  @w_CeCe.mac 
\end{MacVerbatim}

Details about the underlying equations as well as information how to calculate 
differential PDFs with other programs such as \textsc{PDFgui} are given in 
Appendix \ref{app-mod}.

