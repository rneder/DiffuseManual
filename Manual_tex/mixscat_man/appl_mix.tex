\chapter{MIXSCAT commands}
\section{Summary}
Here is a short summary of the MIXSCAT specific commands currently 
available: 
\par
\begin{MacVerbatim}
calc    : Starts the calculation of the differential function
elem    : Sets sample composition
match   : Scales data by fitting low r slope
read    : Reads data and model files
remo    : Sets the partial to be removed
save    : Save commands
scal    : Sets scale factor for dataset
scat    : Overwrite internal scattering powers
show    : Display various settings
xray    : Specifies Q-value for calculating X-ray form factors
var     : Shows the available variables
\end{MacVerbatim}
\section{calc}
{\bf calc \par }
\par
\vspace{3pt}
This command starts the actual calculation of the differential 
function. 
\section{elem}
{\bf elem $ <$a1$> $,$ <$c1$> $,$ <$a2$> $,$ <$c2$> $,.. \par }
\par
\vspace{3pt}
This command sets the composition of the sample. For each element, 
the name $ <$a1$> $ and concentration $ <$c1$> $ is given. For example, CeF3 
would be given as elem Ce,1.0,F,3.0. 
\section{match}
{\bf match $ <$rmin$> $,$ <$rmax$> $,$ <$rho0$> $ \par }
\par
\vspace{3pt}
This command fits the low r region of each data set between $ <$rmin$> $ 
and $ <$rmax$> $. The respective scale factors for the data sets are 
then determined by scaling the refined slope to the specified 
number density $ <$rho0$> $. It is important to use a good guess of the 
value of $ <$rho0$> $, e.g. derived from the average structure. 
\par
The resulting slope and corresponding sigma are stored in the 
variables res[i]. As usual res[0] contains the number of parameters 
available. The slope of data set 1 is in res[1], the error in res[2]. 
The next set is in res[3] and res[4] and so on. 
\section{read}
This command reads various information from a specified file. 
The following formats are currently supported: 
\subsection*{data}
{\bf read "data",$ \{$"n"$| $"x"$\} $,$ <$file$> $ \par }
\par
\vspace{3pt}
The command 'read data' reads the observed PDF. The file 
format is ASCII and contains 'r G(r) dummy dG(r)' in each line. 
The value of 'r' is in A, G(r) is the reduced PDF. The 
third column is ignored (needed for KUPLOT) and the last 
value 'dG' is the error of the PDF used to calculate the 
weight (w=1/dg**2) for this point to be used for the 
refinement. Alternative formats are 'r G(r) dG(r)' in each 
line or simple 'r G(r)'. In the later case, the weights are 
set to unity. This is also done in case the value of dG(r) 
is found as zero. Additional to the filename $ <$file$> $ the commands 
needs the following parameters: First the type of radiation 
is specified, "n" stands for neutrons and "x" for X-rays. 
To read multiple data sets just repeat the 'read' command. 
\par
If the data file contains a history part created by PDFgetN, 
some of the information is returned in the res[n] variables. 
Currently the following information is available after the 
'read data' command: 
\par
\begin{MacVerbatim}
res[1]   : Temperature where the data were collected (in K)
res[2]   : Qmax (only AFTER the data were read !)
\end{MacVerbatim}
\section{remove}
{\bf remove $ <$a1$> $,$ <$a2$> $ \par }
\par
\vspace{3pt}
This commands sets the partial to be removed to $ <$a1$> $-$ <$a2$> $. For 
example in the case of CeF3, the command remove Ce,Ce would 
generate a differential function containing only Ce-F and F-F 
contributions. 
\section{save}
This command allows one to save various data or settings. 
The following formats are currently supported: 
\subsection*{pdf}
{\bf save "pdf",$ <$name$> $ \par }
\par
\vspace{3pt}
This commands allows one to save the differential PDF, G(r), to 
the file called $ <$name$> $. 
\subsection*{results}
{\bf save "results",$ <$name$> $ \par }
\par
\vspace{3pt}
This command save information about the setup and error analysis 
of the last calculation to the file named $ <$name$> $. 
\subsection*{weights}
{\bf save "weights",$ <$name$> $ \par }
\par
\vspace{3pt}
In order to calculate the corresponding differential PDF from a 
structural mode, the modified weights for each remaining partial 
g\_ij(r) is needed. This command saves the weights to a DISCUS 
macro file called $ <$name$> $. 
\section{scat}
{\bf scat $ \{$$ <$name$> $$| $$ <$number$> $$\} $,$ <$a1$> $,$ <$b1$> $,$ <$a2$> $,$ <$b2$> $,$ <$a3$> $,$ <$b3$> $,$ <$a4$> $,$ <$b4$> $,$ <$c$> $ \par }
{\bf scat $ \{$$ <$name$> $$| $$ <$number$> $$| $"all"$\} $, "internal" \par }
\par
\vspace{3pt}
The first command form defines for the element $ <$name$> $ or the scattering 
curve number $ <$number$> $ a new scattering factor in the exponential form. 
For neutron scattering lengths, set a(i) and b(i) to zero. 
\section{scal}
{\bf scal $ <$is$> $,$ <$factor$> $ \par }
\par
\vspace{3pt}
This command sets the scale factor $ <$factor$> $ for data set $ <$is$> $. 
This is used in cases where the data need scaling due to systematic 
errors. 
\section{show}
{\bf show \par }
{\bf show "config" \par }
{\bf show "error" \par }
{\bf show "scat",$ <$is$> $,$ \{$$ <$a$> $$| $"all"$> $ \par }
\par
\vspace{3pt}
This command displays all current settings. The command show "config" 
will show current limits such as maximum number of data points. The 
command show "error" will display results of an error analysis on 
the screen. The subcommand "scat" shows the current scattering lengths 
to be used for data set $ <$is$> $ for atom $ <$a$> $ or "all" atoms. 
\section{variables}
\par
The program MIXSCAT recognizes various variables. The contents of a 
variable can be displayed using the 'eval' command. Some variables 
are READONLY (RO) and can not be changed. 
\par
\begin{MacVerbatim}
i[<n>]      : Integer variables
r[<n>]      : Real variables
res[<n>]    : Results of MIXSCAT commands (RO)
n[1]        : Number of loaded data sets (RO)
\end{MacVerbatim}
\section{xray}
{\bf xray [$ <$xq$> $] \par }
\par
\vspace{3pt}
This command sets the Q-value used to calculate the scattering 
length used in the PDF calculation. The default value is xq=0 
which results in a weight corresponding to the number of electrons 
of the contributing atoms. Other settings could be the Q value 
of the first Bragg peak or the average Q value of the data set. 
Calling the command without parameters prints the current setting 
on the screen. 
