%------------------------------------------------------------------------
% Introduction
%------------------------------------------------------------------------

\chapter{Introduction\label{s:intro}}

So what is \Mixscat all about ? Essentially one can combine total 
total scattering data with different weights to extract partial or in our 
case differential PDFs. A partial PDF contains only information from pairs of 
a particular atom type. For example $G_{AB}$ would contain only contributions 
from AB pairs. For a $N$ component system, one would need $N(N+1)/2$ 
independent measurements. A differential PDF, $G_{\overline{AB}}$ on the 
other hand contains all atom-atom contributions \textit{except} the ones from 
AB pairs. Here only two data sets are needed.

The different data sets can either be neutron data from samples containing 
different isotopes of the same element or x-ray data taken close to one 
elements absorption edge or, as in our case, a combination of x-ray and 
neutron data. This has been done, and is by no means new. What makes \Mixscat 
new, is our attempt to create a simple to use tool to combine neutron and x-
ray data. More discussion about other methods and plenty of references can be 
found in our \Mixscat paper\,\citep{mixscat}. A more detailed discussion and 
summary of the equations behind this approach are given in Appendix \ref
{app}.

In principle \Mixscat can be used to process real space data, $G(r)$, as well 
as reciprocal space data, $S(Q)$. However, we find working with real space 
data is more straight forward, as for example differences in instrument 
resolution manifest themselves at higher distances $r$ whereas they effect $S
(Q)$ over the complete range. Also in many cases, a differential PDF is 
extracted to aid model building and focussed on the low $r$ region where PDF 
peaks are well separated. Once a useable structural model might is 
determined, it can be refined against all available data sets yielding 
partials extracted from the model.
