%------------------------------------------------------------------------
% Chapter:  Introduction
%------------------------------------------------------------------------

\chapter{Introduction \label{intro}}
\section{What is DIFFEV ?}

\diffev is a generic evolutionary refinement program that implements the
differential evolutionary algorithm \cite{prstla2005}. Evolutionary or
genetic refinement algorithms allow the refinement of models, functions,
or more generally speaking the parameters of a cost function to obtain
a good solution. 

A least squares based refinement of a function 
$y = F(p_{0}, p_{1}, ..., p_{n})$ requires the calculation of all
partial derivatives $\partial y/ \partial p_{i}$, either from an 
analytical or a numeric solution. If these derivatives cannot be 
calculated, either because they cannot be derived analytically
or because the numeric computation is too time consuming, evolutionary
algorithms offer a possibility to refine these parameters. All
algorithms are population based, i.e. several different parameter sets
$P_I ~:~ [p_{0}, p_{1}, ..., p_{n}]$ are created simultaneously. For 
each of these parameter sets, the value of the cost function is 
computed. In the next step, a new group of parameter sets is
generated and the cost function calculated anew. The respective
values of the cost function are compared and those parameter sets
that yield the better cost function will in turn be taken to generate
the next generation of parameter sets. By carefully designed modification
of the parameter values from generation to generation and by weeding 
out those parameter sets that lead to a bad fit, the algorithm will
eventually find parameter sets that provide a good fit to the 
experimentally determined function. 

\Diffev provides the refinement part of such an evolutionary 
algorithm. It creates the group of parameter sets, compares the
cost function values between two successive generations and creates
the next generation based on a comparison between the old and new
cost function values. It does, however, not calculate the cost
function itself. This task is handed over to a slave program.
Since this slave program could calculate any cost function, 
\Diffev is not limited to the refinement of a particular physical
problem. 

%------------------------------------------------------------------------

\section{What is new ? \label{intro-new}}

\Diffev is available as a stand alone programmed and may also be 
used within the \suite. The \Suite is optimized with respect to
the performance on a large scale computing facility. Several new
features are available within the \suite. These are explained in
the \Suite manual.

%------------------------------------------------------------------------

\section{Getting started \label{intro-get}}

After the program {\it DIFFEV} is installed properly and the
environment variables are set, the program can be started by typing
'diffev' at the operating systems prompt.

\begin{table}[!tbh]
\centering
\begin{tabularx}{\textwidth}{|p{30mm}|X|}
  \hline
  {\bf Symbol} & {\bf Description} \\
  \hline\hline
  "text"     &  Text given in double quotes is to be understood as typed. \\
  \hline
  $<$text$>$ &  Text given in angled brackets is to be replaced by an
                appropriate value, if the corresponding line is used
                in \diffev. It could, for example be the actual name
                of a file, or a numerical value. \\
  \hline
  {\tt text} &  Text in single quotes exclusively refers to \Diffev
                commands. \\
  \hline
  $[$text$]$ &  Text in square brackets describes an optional parameter or
                command. If omitted, a default value is used, else
                the complete text given in the square brackets is to
                be typed. \\
  \hline
  \{text $|$ text\} &  Text given in curly brackets is a list of alternative
                parameters. A vertical line separates two alternative,
                mutually exclusive parameters. \\
  \hline
\end{tabularx}
\caption{\label{sym-tab}Used symbols}
\end{table}

The program uses a command language to interact with the user.  The
command {\tt exit} terminates the program and returns control to the
shell.  All commands of \Diffev consist of a command verb,
optionally followed by one or more parameters.  All parameters must
be separated from one another by a comma ",".  There is no
predefined need for any specific sequence of commands.  \Diffev     
is case sensitive, all commands and alphabetic parameters MUST be
typed in lower case letters.  If \Diffev has been compiled
using the {\tt -DREADLINE} option (see installation files) basic
line editing and recall of commands is possible.  For more
information refer to the reference manual or check the online help
using ({\tt help command input}).  Names of input or output files
are to be typed as they will be expected by the shell.  If necessary
include a path to the file.  All commands may be abbreviated to the
shortest unique possibility. At least a single space is needed
between the command verb and the first parameter.  No comma is to
precede the first parameter. A line can be marked as comment by
inserting a "{\tt \#}" as first character in the line.\par

The symbols used throughout this manual to describe commands,
command parameters, or explicit text used by the program \Diffev     
are listed in Table \ref{sym-tab}. There are several sources
of information, first \Diffev  has a build in online help, which
can be accessed by entering the command {\tt help} or if help for a
particular command $<$cmd$>$ is wanted by {\tt help $<$cmd$>$}. This
manual describes background and principle functions of \Diffev
and should give some insight in the ways to use this program. \par

\Diffev is distributed as part of the diffuse scattering
simulation software \discus.  However, \Diffev can be used
as general refinement program separate from the \Discus program
package. 

%------------------------------------------------------------------------

\section{Command language}

\begin{table}[!tb]
\centering
\begin{tabularx}{\textwidth}{|p{30mm}|X|}
  \hline
  {\bf Variable} & {\bf Description} \\
  \hline \hline
  pop\_gen[1]    &  Current population number \\
  pop\_n[1]      &  Number of members in the population\\
  pop\_c[1]      &  Number of children in the population\\
  pop\_dimx[1]   &  Number of parameters to be refined  \\
  \hline 
  diff\_cr[1]     &  Cross over probability\\
  diff\_f[1]      &  Scale factor for the difference vectors\\
  diff\_k[1]      &  point along line between parent and donor base\\
  diff\_lo[1]     &  Local search probability\\
  diff\_sel[1] *  &  Selection mode: 0 compare to parent;
                     1 use best of (members and children);
                     2 use best of (children)\\
  \hline \hline
  pop\_xmin[i]   &  Minimum allowed value for parameter no. i  \\
  pop\_xmax[i]   &  Maximum allowed value for parameter no. i  \\
  pop\_smin[i]   &  Minimum allowed starting value for parameter no. i  \\
  pop\_smax[i]   &  Maximum allowed starting value for parameter no. i  \\
  pop\_sig[i]    &  Global sigma for parameter no. i  \\
  pop\_lsig[i]   &  Local search sigma for parameter no. i  \\
  \hline \hline
  pop\_v[i,j] *  &  Value of parameter no. i for member no. j  \\
  pop\_t[i,j] *  &  Value of current trial parameter no. i for child no. j  \\
  rvalue[i] *    &  R-value for member no. i\\
  child\_val[i]  &  R-value for child no. i\\
  bestm[1] *     &  Number of member with best R-value\\
  bestr[1] *     &  Best R-value\\
  worstm[1] *    &  Number of member with worst R-value\\
  worstr[1] *    &  Worst R-value\\
  \hline
\end{tabularx}
\caption[\Diffev structural variables]
        {\label{v1-tab}\Diffev variables. Variables marked
         with $^{*}$ are read-only and cannot be altered.}
\end{table}

The program includes a FORTRAN style interpreter that allows the
user to program complex modifications. A detailed discussion about the 
command language which is common to all \Discus package programs can be 
found in the separate \Discus package reference guide which is included with 
the package. Table \ref{v1-tab} shows a summary of \Diffev specific 
variables. Some of these variables can not be modified, others like can be 
altered, thus allowing to modify the refinement strategy.
