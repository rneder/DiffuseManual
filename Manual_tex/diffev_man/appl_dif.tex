\chapter{DIFFEV commands}
\section{News}
\subsection*{2019\_Jan}
\par
Improved the writing of the diffev\_best.mac macro. 
The state of "compute:" is evaluated and multiple "run\_mpi" lines 
are taken into account. 
\subsection*{2018\_Nov}
\par
Added a new command 'restart $ <$user\_generation$> $' This command compares 
the user value to the current value in GENERATION. If the user 
value is smaller, the Logfiles, Summary and Last files are 
shortened and the GENERATION file is adapted. This lets you step 
back to an earlier refinement status. Keep in mind that the later 
cycles are irrevocably lost. If in doubt make a backup first. 
\par
reinstated the ==$> $ 'read' command 
\par
Added further options to the ==$> $ 'release' command 
\subsection*{2018\_Oct}
\par
Added a ==$> $ 'release' command that is intended to act 
complementary to a previopus ==$> $ 'fix' command. 
\subsection*{2018\_July}
\par
A small modification to the "diffev\_best" macro. 
The value of REF\_NINDIV is written and at the end a 
"set error, continue' instruction has been added. 
\subsection*{2018\_June}
\par
Revised the reaction to a CTRL-C 
\par
Added a ==$> $ 'set error, ... , "save" option 
\par
Revised the internal workings of the distribution within 
run\_mpi. This has no effects on the user. It should anable you, 
however, to distribute the children much better on a system with 
many nodes and many CPU's per node. See the manual for furtrher info. 
\subsection*{2018\_Feb}
\par
Added an optional parameter "partial:$ <$value$> $" to the 
==$> $ 'restrial' command. This parameter tells DIFFEV how many 
partial R-values the slave program will calculate and return. 
If set or if the value is larger than 1, DIFFEV will create 
logfiles "Summary.Rvalue.0001", "Parameter.Rvalue.0001" and 
"Current.Rvalue.0001" etc. for each partial R-value. 
Use the KUPLOT commands 'rval' and 'cost' to set the 
individual R-values and a (weighted) average. 
You can display the development of the partial R-values within 
KUPLOT" with the kuplot command 'kpara'. 
\par
removed 'read' command 
\subsection*{2018\_Jan}
\par
The logical comparisons may now take the operators: 
$ <$, $ <$=, ==, /=, $> $=, $> $/ 
The classical fortran77 operators are still valid 
\par
New logical functions "isvar" and "isexp" can be used within an 
"if" construction. See help entry ==$> $'function' in the 
general "Command\_lang" section 
\par
The ==$> $ function par\_number($ <$char\_variable$> $) returns the 
number for the refinement parameter in the character variable 
$ <$char\_variable$> $. 
\par
The ==$> $ function par\_name($ <$number$> $) returns the name of the 
refinement parameter number $ <$i$> $. 
\par
Finished the transformation from parameter numbers to 
parameter names. All Log files now have an extension of the 
parameter name instead the parameter number. 
\par
The new command ==$> $'reset' can be used to reset DIFFEV to the 
conditions at program start. 
\subsection*{2018\_Dec}
\par
Major revision of the refinement parameter handling. 
The new command 'newparam' effectively replaces for most of the 
common refinements the need of the individual commands to set the 
parameter values. Throughout the command language much more 
epmphasis is placed on the parameter names instead of the 
parameter numbers. See ==$> $'refine' and 'init' commands. 
The parameter names are alos placed into the user variable 
environment and handed down to the slave program (discus or kuplot) 
where they will have the appropriate values for the members of 
the population. 
\par
Added a 'read' command that reads GENERATION and the ==$> $ 'logfile' 
This allows for easier changes of the population size, see the 
entry 'Description / Population\_size' 
\subsection*{2017\_Sep}
\par
Throughout the program the internal calculation of random numbers 
was changed to the FORTRAN 90 intrinsic function. 
\subsection*{2017\_July}
DIFFEV has been modified to log the status of the random number generator 
at the beginning of each slave calculation. This status is documented 
internally for the current best member of the population. Once an 'exit' 
command is executed, DIFFEV will write a macro called "diffev\_best.mac" 
that can be used to recreate the current best solution. 
\subsection*{2017\_Jan}
An unfortunate typing error in News/2016\_Oct regaring the new 
refinement variable 
ref\_para[1...]   ( was misspelled as ref\_param[1...] ) 
is corrected in the  on-line help. 
\par
Another typing error caused a an error in the macro parameters 
transferred with a NON MPI command: run\_mpi, making these 
not backward compatible. This has been fixed. 
\subsection*{2016\_Dec}
\par
At a few select points colors are introduced into the output. 
Currently these are just the error messages. 
\par
\subsection*{2016\_Oct}
\par
Global variables have been introduced that use the same syntax as 
user defined variables. This include just "pi" and variables related 
to the refinement. 
DIFFEV sets the value to these variables: 
REF\_GENERATION  Current generation 
REF\_MEMBER      Current population size 
REF\_CHILDREN    Current children size 
REF\_DIMENSION   Number of parameters 
REF\_KID         Current child Updated for DISCUS and KUPLOT only 
REF\_INDIV       Current individuum Updated for DISCUS and KUPLOT only 
REF\_NINDIV      Number of individual repetitions 
ref\_para[1..]   Current trial parameters for current child 
\subsection*{2016\_june}
\par
DIFFEV may now be interruted gracefully with a CTRL-c. 
This will cause DIFFEV to shut down MPI if active. 
\par
The 'run\_mpi' command can now be used as a generic interface to identical 
slave macros, regardless of the MPI status. 
\subsection*{2016\_april}
\par
The initialise command was augmented by a second form to  initialise 
just the logfiles, see ==$> $ 'logfile', 'summary'. 
This might be helpful if you want to reset the refinent cycle 
to a smaller generation number. For very lengthy refinements with 
a few thousand refinement cycles and many parameters, a continuation 
will take appreciable time to read all previous cycles. If the 
actual development accross the cycles is not relevant to you you 
can reduces the file sizes drastically be a sequence like: 
pop\_gen[1] = 1 
initialise logfile 
\par
New command 'lastfile' 
\par
This new command creates a short copy of the ==$> $ 'logfiles'. 
This short form contains the parameters just for the last 
refinement generation. 
\subsection*{2013\_May}
\par
The run\_mpi command has been augmented by a "socket" qualifier, which 
will cause DIFFEV to run the application program controlled via a 
socket. This should speed up the process on a multi-core/multi-cpu 
system. 
\subsection*{2011\_June}
\par
The program is now a fully functional fortran2003 program. 
There are no changes that must be followed by the user, but important 
differences exist in the memory allocation. The internal arrays that hold 
the population are now all allocated automatically, when you define the 
number of parameters to be refined, the size of the population and the 
constraint equations. Most of the time you will not have to bother with 
these computational details. If you whish, you can allocate appropriate 
array sizes, see ==$> $ 'allocate', 'deallocate'. 
\par
With this version, the program allows the user to change the population 
size and the number of parameters to be refined during a refinement cylce. 
\par
If the number of parameters to be refined in increased during a cycle, 
the program will automatically write new child parameter sets and patch 
the logfile and summary file. 
\par
See the entry ==$> $ "Description" ==$> $ "Increase\_Dimension" for an example. 
\par
Related to this new feature, the ==$> $ "initialize" command now allows to 
(re-)initialise an individual parameter. 
\par
\section{Synopsis}
\par
\begin{MacVerbatim}
Description ! A description of the program
News        ! Information on recent changes
allocate    ! Allocate array sizes
adapt       ! Adaption of global/local search width
backup      ! Backup current best solutions
compare     ! Compares the results of the current population
constraint  ! Defines a constraint condition
deallocate  ! Deallocate array sizes
dismiss     ! Dismiss the worst parents; replace by new children
donor       ! Defines which donor to use
fix         ! Fixes a parameter
initialize  ! Initializes the generation zero
lastfile    ! Defines a file name for the parameters for the last generation
logfile     ! Defines the file name for the parameters for each generation
newparam    ! Defines Ra new parameter or new values for a parameter
pop_name    ! Defines names for the individual parameters
refine      ! Defines which parameters are refined/fixed
release     ! Initialize a parameter that had been fixed
restrial    ! Defines the file name for the current R-values
run_mpi     ! Run the cost function program in parallel through MPI
selection   ! Defines how children/parents survive into next generation
summary     ! Defines the file name for a summary of the parameter changes
trialfile   ! Defines the file name for the current parameters
type        ! Sets the numerical type for a parameter (integer or real)
write       ! Writes new Children or new GENERATION file
\end{MacVerbatim}
\section{Description}
\par
The program DIFFEV uses the differential evolution algorithm to refine 
a set of parameters to a set of observations. 
\par
DIFFEV provides the handling of the parameters and their evolution. 
An external program must be used to calculate the cost function or 
R-value that corresponds to a given set of parameters. 
\par
The differential evolution algorithm compares simultaneously the resulting 
cost function for several sets of parameters. The number of parameter sets 
is called the population. For each member of the population a set of 
parameters is used. This is called a parameter vector. Its dimension 
depends on the model that you need to describe. In order to describe 
a parabola you would need three parameters. The whole set of parameters 
is called a generation. 
\par
For each generation, the cost function is evaluated for each member of 
the population. The next generation is determined from the current 
parent generation through the following procedure: 
\par
Loop over all members 
  - Choose a member (at random or at will), this will be the 
    donor base. 
  - Set a point along the line between current member and donor base 
    to create the effective donor base ==$> $ diff\_k[1] 
  - Select two other members by random choice 
  - Loop over all parameters: 
     - Take the difference between the corresponding parameters of the 
       two other randomly chosen members. 
     - Multiply this difference by a user provided value ==$> $ diff\_f[1]. 
     - Add the difference vector to the effective donor base to 
       create a parameter set called the donor. 
       - One parameter of the donor is always chosen for the child. 
         All other parameters are then randomly chosen from: 
         Either:  the parent 
         Or    :  the donor 
       - The probability for this choice is weighted by a user provided 
         probability ==$> $ diff\_cr[1]. 
\par
Once a new generation has been determined the corresponding cost function 
is calculated. 
\par
Next the selection process determines, which current children survive 
into the next generation. This is done either by: 
- Direct comparison between a parent member and its immediate child. 
  Only those new members survive, whose cost function is 
  less than that of the parent. Otherwise the parent is retained. 
- All parents and all children are pooled into one set. From this set 
  the best members survive, irrespective whether they were a parent 
  or a child. 
\par
Such an algorithm is able to search for parameters if a standard refinement 
algorithm fails or is difficult to adapt. This might be the case for: 
- Undefined parameter values within the possible range. The parameter 
  P may for example NOT be equal to 1 for a function: 1/(1-P) but values 
  larger AND smaller are allowed. 
- The calculation of the cost function involves existing extensive 
  algorithms or several different programs. 
- The calculation involves the averaging of several calculations that 
  rely on data created by (Gaussian-) randomly distributed parameters. 
\par
Files 
\par
DIFFEV uses several files to store the parameter values, the current 
status etc these are: 
\par
"GENERATION"   Fixed file name in the current directory! 
\par
The file "GENERATION" contains twelve fixed lines with the current 
generation number, and all relevant file names. 
Further lines contain the backup file names if defined, the random 
number seeds and the finally the parameter names and their allowed 
numerical ranges. 
\par
\begin{MacVerbatim}
 # generation members children parameters
      158        45        90         4
 # trial file
 DIFFEV/Trials
 # result file
 DIFFEV/Results
 # log file
 DIFFEV/Parameter
 # summary file
 DIFFEV/Summary
 # last    file
 DIFFEV/Current
\end{MacVerbatim}
Trial files (obsolete) 
\par
DIFFEV writes a short file for each member that contains the current 
parameters for one member. 
The user defined filename is appended by a four digit member number. 
The first lines contain the information 
on current generation number, the number of members in the population, 
the number of children and the number of parameters 
\begin{MacVerbatim}
# generation members children parameters
     181        45        90         4
# current member
    1
# parameter list
    0.8284027688E-02
    0.5815573883E+02
    0.2050311089E+01
    0.3000000000E+01
\end{MacVerbatim}
Result files (obsolete) 
\par
DIFFEV expects to read the R-values for each member from a short file. 
The user defined filename is appended by a four digit member number. 
\par
The file must contain the member number and the R-value in free format 
within the first line. 
\par
Parameter files 
\par
For the R-value and for each parameter a SPEC type file is written that 
contains all old R-values and parameters, respectively. 
Each generation makes up a scan. The first column is the member number, 
the second column is the R-value and the third column is the respective 
parameter value. 
The base name of the parameter files can be set by the user via command 
==$> $ 'logfile'. As of Version 5.16.1 the base name is appended by an 
extension that is identical to the parameter name. Parameter file 
$ <$name$> $.Rvalue has the same structure, except that both column two and 
three are the R-values. 
\par
Prior to version 5.16.1 the base naem was appended by a four digit 
number with leading zeros. Parameter file no 0000 has the same 
structure, except that both column two and three are the R-values. 
\par
Last file 
\par
A short copy of the parameter file that contains the parameters 
just for the last generation. 
\par
Summary files 
\par
For the R-value and for each parameter a SPEC type summary files 
contains a single scan. Each generation creates one line within the 
scan. Five values arew written to each line. The first column is the 
generation number 
For the R-value and each of the parameters four further columns are 
written. The first of these is the average value, the second the 
minium value, the third the maximum value and the fourth the sigma of 
the parameter distribution. 
The base name of the summary files can be set by the user via command 
==$> $ 'summary'. This base name is appended by a four digit number with 
leading zeros. Summary file no 0000 has the same structure, except that 
both column two and three are the R-values. 
\par
Further help topics are: 
\par
\subsection*{Basic\_Example}
\par
This example illustrates the commands to refine the three parameters 
that describe a parabola: y = P1*x**2 + P2*x + P3 
\par
You will find the data and the macros in the diffev/Example directory 
within the program source directories. 
\begin{MacVerbatim}
======================diffev.mac======================================
#
#
pop_gen[1]  = 0         # initialize the current generation to zero
#
pop_n[1]    = 15        # The population shall have 15 members
pop_c[1]    = 15        # The population shall have 15 children
pop_dimx[1] = 3         # We need three parameters
#
pop_name      1,square  # The parameter is called "square"
#
type          1,real    # Parameter 1 is a floating number
#
pop_xmin[1] = -1.0      # The first parameter is restricted to the
pop_xmax[1] =  1.0      # range -1 to +1
#
pop_smin[1] = -0.8      # The starting parameters of the first parameter
pop_smax[1] = +0.8      # are restricted to the range -0.8 to +0.8
#
pop_sig[1]  =  0.02     # The sigma of minimum parameter "noise".
pop_lsig[1] =  0.002    # The sigma for local searches
#
adapt  sigma, 1, 0.025  # After generation 0 the value of pop_sig[1]
#                       # is adjusted to 0.025*(largest parameter -
#                       #                       smallest parameter )
#
adapt lsigma, 1, 0.0025 # After generation 0 the value of pop_lsig[1]
#                       # is adjusted to 0.0025*(largest parameter -
#                       #                        smallest parameter )
#
pop_name      2,linear
pop_xmin[2] = -2.0
pop_xmax[2] =  2.0
pop_smin[2] = -1.8
pop_smax[2] = +1.8
#
pop_name      3,constant
pop_xmin[3] = -5.0
pop_xmax[3] =  5.0
pop_smin[3] = -4.8
pop_smax[3] = +4.8
#
constraint  p[1].lt.1.3 # Several constraint may be imposed on the
constraint  p[2]+p[3].gt.0.0
#                       # parameters. Here the first parameter must be
#                       # less than 1.3. In the second constraint condition,
#                       # The sum of the second and third parameter must be
#                       # greater than zero.
#
diff_cr[1]  = 0.9       # The cross over probability is 90%
diff_f[1]   = 0.81      # The difference vectors are multiplied by 0.81
diff_k[1]   = 1.0       # For diff_k = 1, the difference vector is added
#                       # to the donor, for diff_k = 0 to the parent.
diff_lo[1]  = 0.1       # In 10% of all cases, a member creates its child
#                       # not by diffev algorithm, but from a local Gaussian
#                       # distributed search.
#
trialfile  silent       # Instead of writing to a file DIFFEV will pass the
#                       # values silently down to DISCUS/KUPLOT
restrial   silent       # Instead of reading a resultfiel, DIFFEV wil
#                       # obtain the cost function results directly from
#                       # the slave program.
logfile    Parameter    # The log file for the parameters
summary    Summary      # A shorter log file
#
init       silent       # Initializes diffev, Generation zero is written.
#
do i[1]=1,15            # A loop over 15 generations
  run_mpi kuplot, kcompare.mac, 0,  /dev/null
#                       # diffev starts the kuplot program with input
#                       # from file "kcompare.mac". This macro instructs
#                       # kuplot to read the 'trialfiles', to calculate the
#                       # cost function for each member and to write the
#                       # results into the 'restrial' files.
#
  compare  silent       # diffev reads the 'restrial' files, compares the
#                       # cost functions of children and parents and
#                       # creates the next generation.
#
enddo                   # End of the loop
\end{MacVerbatim}
\subsection*{Increase\_Dimension}
\par
Here is an example for a macro that should be used to increase the 
number of parameters to be refined: 
\par
\begin{MacVerbatim}
newparam cube, -5.0, 5.0, -4.8, 4.8, keep:initialize
#                                # Defines a new parameter called "cube"
#                                # that is allowed to be in
#                                # the range [-5:+5]
#                                # It is initialized in the
#                                # range [-4.8:+4.8]
dismiss          pop_n[1]/2      # Set R-value of half the population
                                 # to a very high value, thus they will
                                 # be replaced in the next generation
\end{MacVerbatim}
Prior to version 5.19 the macro would have taken the following command: 
\par
\begin{MacVerbatim}
variable integer, ipar           # just a nice variable name
pop_dimx[1]    = pop_dimx[1] + 1 # increase dimension
ipar           = pop_dimx[1]     # copy into variable

pop_name         ipar,cube       # Define name etc for new parameter
pop_xmin[ipar] = -5.0
pop_xmax[ipar] =  5.0
pop_smin[ipar] = -4.8
pop_smax[ipar] = +4.8
type real,       ipar
refine           ipar            # Set refinement flag
init             ipar            # Initialize just this new parameter
dismiss          pop_n[1]/2      # Set R-value of half the population
                                 # to a very high value, thus they will
                                 # be replaced in the next generation
\end{MacVerbatim}
\subsection*{Hints}
\par
These are some hints regarding useful parameters. They are derived 
from the authors experience and are to be carefully adopted. 
\par
The population size MUST be at least 4! 
\par
The population size should be at least about ten times as large as the 
number of parameters, twenty times will give you a good sampling of the 
parameter space. A smaller population runs the risk of converging into 
a local minimum. 
\par
The cross over probability ==$> $diff\_cr[1] should be about 0.8 
 A smaller value (lets say around 0.3), seems to prevent convergence, 
   While a small value prevents the special properties of the 
   differential algorithm to be applied at all. At diff\_cr[1]=0, 
   all children would always be identical to their parents! 
\par
The multiplier for the difference vector ==$> $ diff\_f[1] should be 
around 0.8 
   A small value prevents the children from being very different 
   from their parents, while a large value ~$> $ 1.5(?) seems to 
   prevent convergence, since all children always jump too far away 
   from their parents. 
\par
selection mode 
   If the dependency of the cost function/R-value on the parameters 
   is (or seems to be) fairly straightforward, the convergence is 
   much faster if you choose the selection best,all scheme. By increasing 
   the number of children in comparison to the parents, the selection 
   pressure also increases and the algorithm will move faster into 
   the minimum. This will, however, also happen if you happen to be close 
   to a local minimum, instead of the global minimum! 
\par
I cannot give a hint regarding the number of generations required. 
This depends too much on the problem at hand, parameter correlation, 
the initial choice of parameters etc. 
\subsection*{Population\_size}
\par
Usually the population size will remain fixed during the course of 
a refinement. The population size is defined by setting appropriate 
values to the variable 'pop\_n' and 'pop\_c' for the parent and 
children respectively. 
\par
If you want to change the population size during a refinement, use 
a sequence of commands as follows: 
\par
read         ! This forces DIFFEV to update the parameters from 
             ! GENERATION and the logfiles 
pop\_n[1] = $ <$new\_value$> $ 
pop\_c[1] = $ <$new\_value$> $ 
\par
You may increase or decrease the population size as needed. 
\section{allocate}
{\bf allocate \par }
{\bf allocate "default" \par }
{\bf allocate "constraint", $ <$max\_constr$> $ \par }
{\bf allocate "population", $ <$max\_members$> $, $ <$max\_parameters$> $ \par }
{\bf allocate "show" \par }
\par
\vspace{3pt}
DIFFEV allows to to allocate memory for the arrays needed to store the 
population and the constraints. DIFFEV will dynamically allocate 
the population size, the number of refinement parameters and the 
number of constraints. 
Thus one often may not have to use this command. 
\par
If previously allocated arrays are reallocated, DIFFEV tries to save 
the old values and you can continue to use these. If the new array 
sizes are smaller than the previous ones, this can obviously not be 
done. DIFFEV will perform the new allocation, but all old data are lost. 
A short warning will be printed. 
\par
{\bf allocate \par }
{\bf allocate "show" \par }
\par
\vspace{3pt}
Without parameter or with the parameter "show", the allocate command 
shows the current memory allocations. 
\par
{\bf allocate "default" \par }
\vspace{3pt}
Allocates all array sizes to default values. 
\par
{\bf allocate "constraint", $ <$max\_constr$> $ \par }
\par
\vspace{3pt}
Allocates the maximum number of constraints that DIFFEV shall handle. 
\par
{\bf allocate "population", $ <$max\_members$> $, $ <$max\_parameters$> $ \par }
\vspace{3pt}
Allocate the maximum population size and the maximum number of 
parameters that can be refined. 
\section{adapt}
{\bf adapt "sigma" ,$ <$parameter\_number$> $,[,$ \{$"yes"$| $"no"$| $$ <$value$> $$\} $] \par }
{\bf adapt "lsigma",$ <$parameter\_number$> $,[,$ \{$"yes"$| $"no"$| $$ <$value$> $$\} $] \par }
\par
\vspace{3pt}
DIFFEV uses to "sigmas" to handle special situations. 
\par
The global sigma pop\_sig[$ <$i$> $] is used in the following two situations: 
- a new parameter falls outside the range allowed by pop\_xmin[$ <$i$> $] and 
  pop\_xmax[$ <$i$> $]. In this case the new parameter is chosen by adding a 
  Gaussian random distributed value with sigma pop\_sig[$ <$i$> $] next to the 
  respective boundary. 
- the difference between two parameters is zero. This will usually 
  occur for integer parameters only. In this case the new parameter is 
  chosen by adding a Gaussian random distributed value with sigma 
  pop\_sig[$ <$i$> $] to the value of the effective donor. 
\par
If sigma is allowed to adapt during the fit, its value is set to 
(maximum parameter value - minimum parameter value) * $ <$value$> $ 
Thus, as the population converges to a smaller parameter spread, sigma 
dynamically becomes smaller as well. 
\par
The adaptation can be set for each parameter $ <$parameter\_number$> $ 
separately. 
\par
The local sigma is used to modify a member not by differential evolution, 
but by a adding a local shift to the member. The local shift is a 
Gaussian distributed random value with sigma = $ <$lsigma$> $. 
Whether DIFFEV uses this mode, is determined  by the ==$> $ variable 
diff\_lo[1]. This gives the probability, that a given member will be 
modified by the local change instead of differential evolution. 
\section{backup}
{\bf backup "NONE" \par }
{\bf backup $ <$input$> $ [,$ <$extension] , $ <$output$> $ \par }
\par
\vspace{3pt}
This command allows you to back up the current best solutions. 
Diffev expects that the current trial solutions are called 
inputfile.???? 
or 
inputfile.????.extension 
where "????" is a four digit integer number with leading zeros. 
Note that the '.' before the number "????" and before the extension 
are mandatory parts of the filename. You need to ensure that your 
version of "kup.diffev.mac" adheres to this standard. 
If the output file name includes a path, make sure that the output 
directory does exist. DIFFEV does not create the output directory. 
\par
\par
Examples 
backup CALC/calc, FINAL/final 
This form will back up the files 
      "CALC/calc.????" as "FINAL/final.????" 
backup TEMP/calc, tth, FINAL/final 
This form will back up the files 
      "TEMP/calc.????.tth" as "FINAL/final.????.tth" 
\par
If multiple files need to be backed up, use the 'backup' command 
repeatedly: 
\par
backup TEMP/calc, tth,    FINAL/final 
backup TEMP/calc, grcalc, FINAL/final 
These two lines will back up the files 
      "TEMP/calc.????.tth"    as "FINAL/final.????.tth" 
      "TEMP/calc.????.grcalc" as "FINAL/final.????.grcalc" 
\par
backup NONE 
  Turns off the back up option 
\par
The backup option is useful, if the calculation of the solutions 
takes a long time and involves random configurations. In these cases, 
the extra time required to copy the files may outway the time to 
calculates these again after the end of the refinement. 
If the calculation involves random configurations, a repeated 
calculation of the solutions on which the cost function depends 
may not yield exactly the same result. With the backup option you 
ensure that you always have those backed up solutions correspond 
to the actual cost function values that DIFFEV used. 
\par
If the calculation of the solutions is quick or if it does not 
involve random configurations it is faster not to run the backup 
during the refinements. 
\section{branch}
{\bf branch kuplot [, "-macro" $ <$macro\_name$> $ [ $ <$par1$> $ [ , $ <$par2$> $ ...]]] \par }
{\bf branch discus [, "-macro" $ <$macro\_name$> $ [ $ <$par1$> $ [ , $ <$par2$> $ ...]]] \par }
\par
\vspace{3pt}
Active within the discus suite only! 
\par
Branches to the "kuplot" or "discus" section. 
\par
Within this section any standard KUPLOT command can be 
given. The behaviour of "kuplot" is essentially the same 
as in the stand alone version. Likewise for DISCUS. 
\par
The main use will branch to KUPLOT while the discus section 
is run via run\_mpi from a DIFFEV slave. 
\par
Optionally the "-macro" qualifier instructs the suite to run the 
macro $ <$macro\_name$> $ (with its optional parameters) before the 
interactive session is started. 
\section{compare}
{\bf compare ["silent"] \par }
\par
\vspace{3pt}
This is the main part of the differential evolution section. 
This command reads the current results that the slave program stored 
in files ==$> $ 'restrial' and compares these to the results of the 
parent generation. A new set of children is calculated according to 
the differential evolution algorithm. The successful parents are 
written to the file ==$> $ 'logfile', which stores their respective cost 
function, and the full parameter set. At the same time the short summary 
file ==$> $ 'summary' is appended with abbreviated information about the 
last generation.  The new children parameters are 
written to the temporary files ==$> $ 'trialfile'. The current 
generation is increased in file 'GENERATION'. 
\par
If the optional "silent" qualifier is specified, DIFFEV will not read 
the result files. Instead, DIFFEV must have received the current results 
by explicitly setting the values of child\_val[*] for all children. 
This option will work only within the discus\_suite, which is a 
collection of DIFFEV, DISCUS, and KUPLOT into a common program. 
\section{constraint}
{\bf constraint $ <$logical expression$> $ \par }
\par
\vspace{3pt}
The parameter range may be restricted by defining one or several constraint 
conditions. Each condition must be a valid logical expression. For 
details of the syntax see the manual entry under ==$> $ Command language. 
The parameters within the condition are referred to by p[$ <$i$> $], where 
$ <$i$> $ is parameter number, 1 up to the dimension of the problem at hand. 
The dimension is fixed through parameter ==$> $ 'pop\_dimx', see the 
variable entry. 
\par
Example 
\par
\begin{MacVerbatim}
constraint  p[1].gt.2   # The first parameter must be larger than 2
constraint  4.le. p[2]**2 + p[3]**2
                        # The sum of parameters 2 and 3 squared must be
                        #  equal or greater than 4.
\end{MacVerbatim}
If a constraint equation is not met, DIFFEV will create a new parameter 
set. This process is repeated until a valid parameter set is found, or 
until DIFFEV has tried so for MAX\_CONSTR\_TRIAL times. In this latter 
case the program stops. 
\section{deallocate}
{\bf deallocate $ \{$"all" $| $ "constraint" $| $ "population"$\} $ \par }
\par
\vspace{3pt}
This command allows to deallocate memory for the specified program 
segments. This helps to conserve memory, if program sections are no 
longer needed. 
\par
This deallocation applies to memory that you allocated yourself 
and also to memory that DIFFEV has allocated automatically during 
runtime. As DIFFEV does not necessarily know, when you do not need 
the results of certain calculations any longer, it does not 
deallocate the automatically allocated memory sections unless you 
tell DIFFEV to do so. 
\par
\begin{MacVerbatim}
"constraint"
\end{MacVerbatim}
   Free memory associated to the maximum number of constraints. 
\par
\begin{MacVerbatim}
"population"
\end{MacVerbatim}
   Free memory associated to the population size and maximum number 
   of refineable parameters. 
\section{dismiss}
{\bf dismiss $ \{$$ <$n$> $ $| $ "all" $\} $ \par }
\par
\vspace{3pt}
Set the R-value of the worst $ <$n$> $ parents to a very high value. 
Thus $ <$n$> $ of the next children will definetly have a better R-value 
and replace these parents. 
\par
This command should be used after you changed the parameter dimension 
and initialised ==$> $ 'init' some or all of the trial values. 
Otherwise, many or eval all of the new children might not have 
a better R-value than their parents, and as a consequence the 
(re-)initilization of the trial values may get lost. 
\section{donor}
{\bf donor $ \{$"best" $| $ "random"$\} $ \par }
\par
\vspace{3pt}
The donor vector may be chosen in two different ways. 
"best" chooses the parent member that has the best parameter set. 
"random" chooses at random one of the parent vectors as donor vector. 
Two other parent vectors are always chosen at random to form the 
difference vector that is added to the donor vector. 
DIFFEV chooses the effective donor base along the straight line 
from the current parent vector to the donor. The point along this 
line is determined by the value of "diff\_k[1]" (==$> $ variables). 
For diff\_k[1] = 0  the effective donor is the parent vector, 
for diff\_k[1] = 1  the effective donor is the donor vector itself. 
\section{fix}
{\bf fix $ <$parameter\_no$> $, $ \{$$ <$value$> $ $| $ "best"$\} $ \par }
{\bf fix $ <$parameter\_name$> $, $ \{$$ <$value$> $ $| $ "best"$\} $ \par }
\par
\vspace{3pt}
This command fixes the value of parameter $ <$parameter\_no$> $ or 
$ <$parameter\_name$> $ to 
$ <$value$> $ for all members of the population. The ==$> $ refine 
flag is turned of for this parameter. 
If the second parameter is "best", the parameter is set to the 
corresponding value of the current member with the best R-value. 
\par
As a side effect, the values of pop\_xmin, pop\_xmax, 
pop\_smin, pop\_smax are fixed to this new value as well. 
\section{functions}
\par
The following DISCUS specific functions exist. For a listing 
of general intrinsic functions see help entry 'functions' in 
the 'Command language' section of the online help. 
\par
{\bf par\_number($ <$char\_variable$> $) \par }
\par
\vspace{3pt}
Returns the number of the refinement parameter whose name is 
encoded in the character variable. 
Example: 
line = 'P\_length' 
eval par\_number(line) 
\par
{\bf par\_name($ <$number$> $) \par }
\par
\vspace{3pt}
Returns the name of the refinement variable number $ <$number$> $ 
\section{initialize}
{\bf initialize [ $ <$par\_number1$> $ [, $ <$par\_number2$> $]] [, "silent"] \par }
{\bf initialise "logfile" \par }
\par
\vspace{3pt}
This command initializes the differential evolution sequence. 
\par
Before using this command, you must have defined: 
\begin{MacVerbatim}
Number of parameters to be defined ==> 'pop_dimx'
Size of the parent population      ==> 'pop_n'
Size of the children population    ==> 'pop_c'
Boundaries for each parameter      ==> 'pop_xmin'
                                   ==> 'pop_xmax'
Starting intervals for parameters  ==> 'pop_smin'
                                   ==> 'pop_smax'
Sigmas for parameter adjustment    ==> 'pop_sig'
Local sigmas                       ==> 'pop_lsig'
Cross over probability             ==> 'diff_cr'
Fraction of the difference vector  ==> 'diff_f'
Point between parent an donor base ==> 'diff_k'
Local search probability           ==> 'diff_lo'
\end{MacVerbatim}
Without the optional parameters, 'initialize' is used to start 
the generation zero. 
\par
'Initialize' will use this information to generate the zero's generation. 
The file 'GENERATION' is set to generation zero, the population size 
and the number of parameters is written. 
The files ==$> $ 'logfile' and 'summary' are initialized. Old versions 
with the same name are overwritten! 
The starting parameter values are written to files ==$> $ 'trialfile' 
After the header, each line contains a one parameter, 
'pop\_n' (i.e. the size of the population) lines are written. 
\par
If you want to reinitialize one or several parameters, the 
'initialize' command may be used with the optional parameter(s). 
In this case 'initialize' will simply set the corresponding 
parameter, or parameter range from $ <$par\_number1$> $ to $ <$par\_number2$> $ 
to the range defined by the values of pop\_smin[*] and pop\_smax[*]. 
A new set of children and the GENERATION file is written. 
\par
If the last parameter is the string "silent", the trial files are 
note written to your disk. DIFFEV expects to be part of a suite 
program and will transfer the trial parameters directly to the 
slave program. See also ==$> $ 'trialfile', 'run\_mpi', 'compare' 
This option will work only within the discus\_suite, which is a 
collection of DIFFEV, DISCUS, and KUPLOT into a common program. 
\par
The second form of the command can be used to initialize just 
the logfiles, see ==$> $ 'logfile', 'summary'. 
This might be helpful if you want to reset the refinent cycle 
to a smaller generation number. For very length refinements with 
a few thousand refinement cycles and many parameters, a continuation 
will take appreciable time to read all previous cycles. If the 
actual development accross the cycles is not relevant to you you 
can reduces the file sizes drastically be a sequence like: 
pop\_gen[1] = 1 
initialise logfile 
\section{lastfile}
{\bf lastfile $ <$filename$> $ \par }
\par
\vspace{3pt}
This defines the short log file of the parameter evolution. 
\par
After each generation, the short lastfile $ <$filename$> $ is overwritten 
by the parameters of the current generation. 
\par
It is a SPEC type file that contains all old R-values and parameters. 
Only this last generation makes up a scan. The first column is the 
member number, the second the R-value and all further columns the 
respective parameter values. 
\section{logfile}
{\bf logfile $ <$filename$> $ \par }
\par
\vspace{3pt}
This defines the log file of the parameter evolution. 
\par
After each generation, the logfile $ <$filename$> $ is appended by the 
parameters of the current generation. 
\par
It is a SPEC type file that contains all old R-values and parameters. 
Each generation makes up a scan. The first column is the member number, 
the second the R-value and all further columns the respective parameter 
values. 
\section{newparam}
{\bf newparam $ <$name$> $, $ <$xmin$> $, $ <$xmax$> $,$ <$smin$> $, $ <$smax$> $ \par }
\vspace{3pt}
         [,init:$ \{$"keep"$| $"initialize"$\} $ 
         [,real:$ \{$"real"$| $"integer"$\} $ 
\par
This command defines a new parameter name or changes the setting for 
an existig parameter name. 
$ <$xmin$> $, $ <$xmax$> $ are the absolute lower and upper windows for the 
               parameter. DIFFEV restricts the refinement to this 
               interval. 
$ <$smin$> $, $ <$smax$> $ are the lower and uper limit of the starting 
               window that is used when a parameter is initialized. 
\par
The optional parameter "init" defines if the current parameter 
values for the population are kept at their current values or if 
this parameter shall be initialized within the start window 
$ <$smin$> $, $ <$smax$> $. 
\par
The optional parameter "type" defines if the current parameter 
shall be treated as a real valued number (the default) or as a 
whole, integer number. 
\par
The command summarizes the individual commands 
  par\_name $ <$name$> $ 
  pop\_xmin[$ <$ipar$> $] = $ <$xmin$> $ 
  pop\_xmax[$ <$ipar$> $] = $ <$xmax$> $ 
  pop\_smin[$ <$ipar$> $] = $ <$smin$> $ 
  pop\_smax[$ <$ipar$> $] = $ <$smax$> $ 
  pop\_sig[$ <$ipar$> $]  = 0.001 
  pop\_lsig[$ <$ipar$> $] = 0.0001 
  type $ \{$"real"$ <$"integer"$\} $.$ <$ipar$> $ 
\par
\section{read}
{\bf read \par }
\par
\vspace{3pt}
This command has been removed. 
Activated again in version 5.29.0 
\par
The read command instructs DIFFEV to read the GENERATION file 
and to determine the population from the Parameter files. 
\par
This command is only needed if you want to add a new parameter 
to a refinement that you want to continue and the ==$> $ 'newparam' 
command is the first command in DIFFEV after starting the 
program prior to the ==$> $ 'run\_mpi'command. 
\par
Example: 
read 
newpara  P\_new, 0.0, 10.0,  10.0, 9.0 
run\_mpi discus, dis.diffev.mac, repeat:5, compute:parallel 
compare 
\par
The 'read' command is not necessary if now new parameters 
are defined. In this case, the 'run\_mpi' command will 
automatically determine the refinement state from 'GENERATION' 
\section{refine}
{\bf refine $ \{$"all"$| $"none"$| $$ <$number$> $ [,$ <$number$> $...]$\} $ \par }
{\bf refine $ \{$"all"$| $"none"$| $$ <$par\_name$> $ [,$ <$par\_name$> $...]$\} $ \par }
\par
\vspace{3pt}
This command allows you to set, which of the variables are refined. 
If you give the parameter number as negative number, the corresponding 
parameter is not refined. 
\par
Currently you can only specify up to 20 variable numbers on 
one "refine" command line. If you need to specify the behaviour 
for more than 20 variables, please use several "refine" commands. 
\section{release}
{\bf release $ \{$$ <$par\_name$> $$| $ $ <$number$> $$\} $ , range:$ <$sigma$> $ \par }
{\bf   [value:$ <$setpoint$> $] \par }
{\bf   [min:$ <$pop\_xmin$> $] \par }
{\bf   [max:$ <$pop\_xmax$> $] \par }
{\bf   [dismiss:$ <$number$> $] \par }
{\bf   [dismiss:"all"] \par }
{\bf   [dismiss:"best"] \par }
{\bf   [dismiss:"none"] \par }
\par
\vspace{3pt}
This command will initialize a parameter named $ <$par\_name$> $ 
to a range around the current best value. Its intent is 
to act complementary to a ==$> $ 'fix' command. It works 
in a similar fashion as the ==$> $ 'init' command, but the user 
does not have to ensure that the ranges of 
pop\_xmin to pop\_xmax and pop\_smin to pop\_smax are non-zero. 
\par
The parameter will be refined, i.e. the 'release' command 
implies a ==$> $ refine $ <$par\_name$> $ command. 
\par
If the optional parameter 'value:$ <$setpoint$> $' is present, 
it defines the setpoint around which the parameter values 
will be initialized. If the optional parameter is omitted, 
the setpoint defaults to the value of the current best 
population member. 
The initalization range is set to the setpoint +- range. 
\par
The absolute limits, ==$> $ 'pop\_xmin', 'pop\_xmax' are set 
to a range of +- 3* $ <$range$> $ around the $ <$setpoint$> $. 
The original user provided values of 
pop\_min or pop\_max are repleced by the new values. 
\par
If you need different ranges for pop\_xmin and pop\_xmax, 
or if the new limits might be 
outside a physical limit you need to set the proper 
limits explicitely with the optional parameters 
"min" and "max": 
   min:pop\_xmin 
   max:pop\_xmax 
\par
If all the children will produce R-values that are worse 
than the parents, DIFFEV will discard these children and 
create the next generation based on all parents. As all 
parents reflect the state prior to the release, they all 
correspond to identical values of the parameter that you 
wanted to release. In effect, the release is lost. 
To prevent this, the 'dismiss:' option will set the R-value 
of a user specified number of parents to a very large 
value. This ensures that the corresponding children will 
certainly survive into the next generation. 
The values that the 'dismiss:' parameter may take are: 
$ <$number$> $  : A number from 0 to the population size pop\_n[1] 
            The $ <$number$> $ worst parents are dismissed. 
"all"     : All members are dismissed. 
"best"    : All but the best member are dismissed. 
"none"    : No member is dismissed. Same as $ <$number$> $=0. 
\section{reset}
{\bf reset \par }
\par
\vspace{3pt}
Resets DIIFEV to the conditions at program start. The generation 
number is set to zero, the population size, children size and the 
number of parameters is set to the default value of one. All parameter 
names are removed from the list of user variables. 
\par
Use this command if you want to combine several refinements 
for different experimental data to be executed one after the other. 
Without the reset, the old parameters would persist and may 
interfere with your new parameters. 
\section{restart}
\begin{MacVerbatim}
restart <user_generation>
\end{MacVerbatim}
If you set $ <$user\_generation$> $ to a value that is less than 
the current value in GENERATION, 
the Logfiles, Summary and Last files are 
shortened and the GENERATION file is adapted. 
This lets you step back to an earlier refinement status. 
Keep in mind that the later cycles are irrevocably lost. 
If in doubt make a backup first. 
\section{summary}
{\bf summary $ <$filename$> $ \par }
\par
\vspace{3pt}
This defines the log file of the R-value/cost function evolution. 
\par
The SPEC type summary files contains a single scan. Each generation 
creates one line within the scan. The first columns is the generation 
number. For the R-value and each of the parameters four columns are 
written. The first of these is the average value, the second the 
minium value, the third the maximum value and the fourth the sigma of 
the parameter distribution. 
\section{restrial}
{\bf restrial "silent" $ \{$,partial:$ <$no$> $$\} $ \par }
{\bf restrial $ <$result$> $ $ \{$,partial:$ <$no$> $$\} $ \par }
\par
\vspace{3pt}
Within the discus\_suite always use the "silent" mode! 
\par
These temporary files are used to communicate the R-value from the 
slave program back to DIFFEV. 
DIFFEV expects a separate file for each member of the population. 
The filename $ <$result$> $ is automatically augmented by a four digit 
member number. Thus, if the filename is Results, DIFFEV expects to 
find the files: 
\begin{MacVerbatim}
Results.0001
Results.0002
etc.
\end{MacVerbatim}
The file contains one line with two numbers, the member number and 
the R-value obtained for the corresponding set of parameters in 
file ==$> $ 'trialfile'. 
\par
If the file name is the string "silent", the result files are 
note read from your disk. DIFFEV expects to be part of a suite 
program and will expect that the result values have been transfered 
directly from the slave program. See also ==$> $ 'trialfile', 
'run\_mpi', 'compare' 
\par
The optional parameter "partial:$ <$no$> $" tells DIFFEV how many 
partial R-values to expect from the the slave program. 
Within DIFFEV the parrtial R-values are just archived, the slave 
program has to provide the properly weighted average R-value. 
\par
\section{run\_mpi}
{\bf run\_mpi $ <$program\_name$> $, $ <$macro\_name$> $, $ <$no\_repetitions$> $, $ <$output\_base$> $ [,"socket"] \par }
\par
\vspace{3pt}
This command starts the processing of the slave program $ <$program\_name$> $. 
If the program has been compiled with MPI and started with mpiexec, 
a parallel computation is started. Otherwise the slave program is 
executed in a serial loop over all children and individual repatitions. 
\par
This command starts parallel processing of program $ <$program\_name$> $. 
The program will be executed $ <$pop\_c$> $*$ <$no\_repetitions$> $ times. 
$ <$pop\_c$> $ is the number of children for the current refinement and 
each child corresponds to one ==$> $ trialfile. 
The calculations can be repeated $ <$no\_repetions$> $ times, with the identical 
parameter set. This may be necessary, if you need to average several 
calculations. 
\par
MPI option: 
The standard output of the program will be directed into a file 
$ <$output\_base$> $. The current child number ( and the current repeat 
number, if present) will be appended as four digit wide field. 
On a UNIX system you can redirect the output to /dev/null. 
\par
NON MPI option: 
Regardless of the value of $ <$output\_base$> $, the output is displayed 
at the screen. The usual options for the ==$> $ 'set prompt,' command 
do hold. 
\par
Starting with version 5.7 the communication between the diffev section 
and the discus or kuplot section is done via internal variables. 
These are: 
REF\_GENERATION  : the current generation number 
REF\_MEMBER      : the number of members in the population 
REF\_CHILDREN    : the number of children in the population 
REF\_DIMENSION   : the number of parameters defined in DIFFEV 
REF\_KID         : the current child 
REF\_INDIV       : the current repetition for the current child REF\_KID 
REF\_NINDIV      : the intended total number of repetitions 
\par
The actual parameters are transferred to DISCUS/KUPLOT via variables 
ref\_para[...]   : Indices from 1 to REF\_DIMENSION 
\par
The capitalization of the refinement variables is intended to 
distinguish these from user variables. other than that they should be 
used just as any user defined variable, with no square brackets: 
kid = REF\_KID 
\par
IN a similar fashion, the KUPLOT rvalue command transfers its 
result back to DIFFEV internally. See 'resfile silent' 
\par
Essentially, DIFFEV branches to the section, executes the macro file 
and returns to DIFFEV. As the communication is done via the internal 
variables, the need for the macro parameters included in previous 
version 5.5 and earlier has ceased. You are encouraged not to rely 
on macro parameters but to use the refinement variables. 
\par
\par
If the last optional parameter is "socket", the program will be 
started via a socket connection, else via a single "system" call. 
\par
system == non-socket option: 
\par
Relevant for older verions (5.6 an earleir only): 
\par
The actual line that starts the program has syntax: 
"program\_name" -macro $ <$macro.name$> $ $ <$cwd$> $ $ <$child$> $ $ <$indiv$> $ $> $ $ <$output$> $ 
\par
$ <$macro\_name$> $ name of the macro file to be executed 
$ <$cwd$> $        1st macro parameter string with current directory 
$ <$child$> $      current child number (1 to pop\_c) 
$ <$indiv$> $      current repetion number (1 to $ <$no\_repetions$> $) 
             This parameter is omitted, if $ <$no\_repetions$> $ is zero 
$ <$output$> $     File for standard output written by $ <$program\_name$> $ 
\par
Example 
run\_mpi discus, discus.mac, 5, LOGFILES/d 
        Runs discus with macro "discus.mac". For each child the 
        calculation is repeated 5 times. DISCUS output is written 
        into files "LOGFILES/d.xxxx.yyyy", where "xxxx" is the 
        child number and "yyyy" the repetion number. 
run\_mpi discus, discus.mac, 5, /dev/null 
        Same, except that the output is written to "/dev/null" 
        i.e. it is thrown away. 
run\_mpi discus, discus.mac, 0, LOGFILES/d 
        Same, except that no repetions are requested. Only one 
        calculation is performed per child and the output is 
        written to "LOGFILES/d.xxxx". 
\par
socket option: 
\par
The program is startet via socket at the beginning of a refinement. It 
remains active until diffev is terminated. A second, third program may 
be started as well, such as a discus run followed by kuplot. 
\par
The socket option starts the program via the line: 
\par
"program\_name" -remote -access=127.0.0.0 -port=$ <$port$> $ $> $ $ <$output$> $ 
\par
Thereafter the vatiables generation, member, children, parameters, 
kid, indiv, nindiv are defined and placed at their proper values. 
The corresponding lines are 
variable integer,generation 
generation = $ <$value$> $ 
And so on for the other variables. 
For each calculation i.e. combination of kid, indiv the current 
values are sent for "kid" and "indiv". All corresponding trial 
values are placed into the array "r[*]", into entries 201 to 
201 + $ <$parameters$> $. Then the calculation macro is started, which 
does of course not have to read the GENERATION and the trial file. 
\section{selection}
{\bf selection $ \{$"compare" $| $ "best","all"$\} $ \par }
\par
\vspace{3pt}
This command governs the selection criterium that determines which 
children and parents survive into the next generation. 
\par
\begin{MacVerbatim}
"compare"    Each child is compared to its immediate parent. The better
             of these two will survive into the next generation. It will
             serve as one of the parents from which the next children
             are derived using the Differential Evolution Algorithm.
             The number of children ==> 'variables' pop_c, should be
             identical to the number of members i.e. parents
             ==> 'variables' pop_n.
"best","all" All parents and all children are put into a single list.
             Of this list the best <pop_n[1]> members survive into
             the next generation. No restriction applies to keep any
             parent or any children.
             You are free to use any number of children. If the number of
             children is increased compared to the number of parents, the
             selection becomes "tougher" since a smaller percentage of the
             whole population survives into the next generation. This will
             speed up convergence, yet run a higher risk of getting stuck
             in a local minimum.
\end{MacVerbatim}
\section{trialfile}
{\bf trialfile $ <$filename$> $$| $"silent" \par }
\par
\vspace{3pt}
These temporary files are used to communicate the current set of parameters 
between DIFFEV and the slave program. 
\par
DIFFEV writes a separate file for each member of the population. 
The filename $ <$filename$> $ is automatically augmented by a four digit 
member number. The if the filename is Trials, DIFFEV expects to 
find the files: 
\begin{MacVerbatim}
Trials.0001
Trials.0002
etc.
\end{MacVerbatim}
The file has format: 
\par
\begin{MacVerbatim}
# generation members children parameters
     181        45        90         4
# current member
    1
# parameter list
    0.8284027688E-02
    0.5815573883E+02
    0.2050311089E+01
    0.3000000000E+01
\end{MacVerbatim}
The first states the current generation, the number of members in 
the population, the number of children in each population, and the 
number of parameters. 
The next two lines states the number of the current member. 
This is followed by a list of all parameter values, each in a separate 
line. 
\par
If the file name is the string "silent", the trial files are 
note written to your disk. DIFFEV expects to be part of a suite 
program and will transfer the trial parameters directly to the 
slave program. See also ==$> $ 'trialfile', 'run\_mpi', 'compare' 
This option will work only within the discus\_suite, which is a 
collection of DIFFEV, DISCUS, and KUPLOT into a common program. 
\section{type}
{\bf type $ \{$"integer" $| $ "real"$\} $,$ <$number$> $ \par }
\par
\vspace{3pt}
Defines the number that the parameter $ <$number$> $ assumes. Valid options 
are: 
\begin{MacVerbatim}
"integer" The parameter is restricted to integer numbers.
"real"    The parameter may take on any real,floating number.
\end{MacVerbatim}
See the entry on 'variables' regarding options to limit the allowed range, 
and the entry ==$> $ 'constraint' on possible constraints. 
\section{variables}
\par
Like all programs of the Diffuse suite, diffev offers integer and 
real variable for standard calculations == Command Language/variables 
\par
Unique diffev variables are: 
\par
\begin{MacVerbatim}
pop_gen[1]    The number of the current generation
pop_n[1]      The size of the parent population
              The size of the population may be changed during a
              refinement. In this case the GENERATION file is updated
              automatically. If the population is increased, the R-values
              for the new members are set to ten times the maximum
              current R-value. Parameter values are copied from old
              member no. one.
pop_c[1]      The size of the children population
              The size of the children population may be changed during a
              refinement. In this case, a new set of trial files is
              automatically generated and the GENERATION file is updated
              as well.
pop_dimx[1]   The dimension, i.e. the number of parameters
              The dimension may be increased during a refinement. To do
              so you need to increase pop_dimx[1], set all parameter
              related values for tis new parameter and run the
              ==> 'init <par_number>' command. This will initialize the
              parameter to the starting range and update the trial files
              and the generation file.

pop_xmin[<i>] Minimum allowed value for parameter <i>
pop_xmax[<i>] Maximum allowed value for parameter <i>

pop_smin[<i>] Minimum starting value for parameter <i>
pop_smax[<i>] Maximum starting value for parameter <i>

pop_sig[<i>]  Sigma of Gaussian distribution for parameter <i>. If the
              difference between two parent parameters is zero, or if a
              child parameter is outside the limits defined by pop_xmin
              and pop_xmax, the corresponding parameter is modified by
              a Gaussian distributed random number.
              Set pop_sig to zero to switch off this option.

pop_lsig[<i>] Sigma of local Gaussian distribution for parameter <i>.
              The probability diff_lo[1] determines if a given member
              is changed locally only or takes part in the usual
              differential evolution algorithm. If it is changed only
              locally, the parent parameters are modified by adding
              a Gaussian distributed random number, with mean zero
              and sigma pop_lsig.

pop_v[<i>,<j>] Value of parameter <i> for member <j>
              This parameter is read only.

pop_t[<i>,<j>] Current trial value of parameter <i> for child <j>

rvalue[<i>]   R-value for member <j>. This is the R-value for the
              current parent generation.

bestm[1]      Parent member that currently has the lowest R-value

bestr[1]      Currently lowest R-value.

worstm[1]     Parent member that currently has the highest R-value

worstr[1]     Currently highest R-value.

p[<i>]        Parameter symbol used in the constraint conditions.

diff_cr[1]    Cross over probability
diff_f[1]     Multiplier for difference vector
diff_k[1]     Multiplier for vector between parent vector and donor
diff_lo[1]    Probability for local refinement of a population member
diff_sel[1]   Selection mode for compare command, READ_ONLY
\end{MacVerbatim}
\section{write}
{\bf write $ \{$"children" $| $ "generation"$\} $ \par }
\par
{\bf write "children" \par }
\vspace{3pt}
The write command will generate a new set of child values and update 
the corresponding trial files and the GENERATION file. The generation 
number is not changed. 
\par
{\bf write "generation" \par }
\vspace{3pt}
Writes the GENERATION file. No changes are done to the file. 
