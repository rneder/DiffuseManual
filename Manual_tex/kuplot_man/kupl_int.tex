%------------------------------------------------------------------------
% Chapter:  Introduction
%------------------------------------------------------------------------

\chapter{Introduction}
\section{What is KUPLOT ?}

\Kuplot is an interactive plotting program controlled by a
command language.  \Kuplot is part of the diffuse scattering
simulation package {\sc Discus}, however it can be used totally
independent of the {\sc Discus} software.  \par

\Kuplot is now using the {\it PGPLOT} library for the actual
plotting and supports all graphic devices supported by {\it
PGPLOT} such as X-window terminal or POSTSCRIPT output. As
mentioned before, the program is controlled by a command language
which includes a FORTRAN style interpreter which allows the use of
variables, loops and conditional i statements. This results in a
high degree of flexibility and allows the creation of quite
complex graphics. It also allows \Kuplot to be used to
process large numbers of data files and produce the desired plots
automatically. \par

\Kuplot can process simple 1d data files.  The program
supports normal line graphs, marker, error bars as well as spline
interpolation between the data points.  Line color, marker color,
line type, line width and various other parameters can be
adjusted.  \Kuplot allows one to plot 3D data sets using
contour lines, colored bitmaps or both.  The color map for the
bitmap can be freely changed using the FORTRAN interpreter (see
section \ref{var}).  The program also allows one to define
different contour line sets for one data set, e.g.  finer spaced
lines for diffuse scattering and larger space lines in a different
color for the Bragg peaks. A page can be divided into different
plot areas (see chapter \ref{frame}). Each frame can contain
graphs or the contents of a text file.  Frames can have different
background colors.  Each frame has its own parameter set like e.g.
title, axis labels, fonts.  \par

Loaded data sets can be manipulated using the FORTRAN style
interpreter or a variety of build-in functions.  An integrated FIT
sub level, which allows to fit the following functions to a given
1D data set: polynomial, n Gaussians and n Lorentzians.
Furthermore 2D data sets can be fitted using a set of 2D
Gaussians. Additionally \Kuplot allows fitting a user defined
function to a loaded data set.

%------------------------------------------------------------------------

\section{Getting started \label{get}}

After the program \Kuplot is installed properly and the
environment variables are set, the program can be started by
typing: 'kuplot' at the operating systems prompt.  Information
about how to compile and install the program is described in 
detail in the {\sc Discus} package reference guide.
All array sizes that might need to be adjusted to your needs are
defined in the file `{\it kuplot.inc}'. Check the comments within
this file for an explanation of the various variables.\par
%
\begin{table}[!tb]
\centering
\begin{tabularx}{\textwidth}{|p{30mm}|X|}
  \hline
  {\bf Symbol} & {\bf Description} \\
  \hline\hline
  "text"     &  Text given in double quotes
   is to be understood as typed. \\
  \hline
  $<$text$>$ &  Text given in angled brackets,
   is to be replaced by an appropriate value,
   if the corresponding line is used in \Kuplot.
   It could, for example be the
   actual name of a file, or a numerical
   value. \\
  \hline
  {\tt text} &  Text in type writer font
   exclusively refers to \Kuplot commands. \\
  \hline
  $[$text$]$ &  Text in square brackets
   describes an optional parameter or command.
   If omitted, a default value is used, else
   the complete text given in the square
   brackets is to be typed. \\
  \{text $|$ text\} & Text given in curly
   brackets is a list of alternative parameters.
   A vertical line separates two alternative,
   mutually exclusive parameters. \\
  \hline
\end{tabularx}
\caption{\label{sym-tab}Used symbols}
\end{table}
%
The program uses a command language to interact with the user. The
command 'exit' terminates the program and returns control to the
shell.  All commands of \Kuplot consist of a command verb,
optionally followed by one or more parameters.  All parameters must
be separated from one another by a comma `,'. There is no predefined
need for any specific sequence of commands. \Kuplot is case
sensitive, all commands and alphabetic parameters MUST be typed in
lower case letters.  If \Kuplot has been compiled using the
'{\it -DREADLINE}' option (see installation files) basic line
editing and recall of commands is possible.  For more information
about the command language refer to the {\sc Discus} reference manual 
or check the online help. A list of variables specific to \Kuplot
is given in Tab. \ref{v1-tab}. 

\begin{table}[!tb]
\centering
\begin{tabularx}{\textwidth}{|p{40mm}|X|}
  \hline
  {\bf Variable} & {\bf Description} \\
  \hline \hline
  n[1]               & Number of loaded data sets ({\bf now writable})\\
  n[2]$^{*}$         & Maximum allowed number of data sets \\
  n[3]$^{*}$         & Maximum allowed number of frames \\
  n[4]$^{*}$         & Maximum allowed number of annotations \\
  n[5]$^{*}$         & Maximum allowed number of bonds \\
  \hline
  nx[$<$ik$>$]$^{*}$ & Number of points in x-direction for data set $<$ik$>$ \\
  ny[$<$ik$>$]$^{*}$ & Number of points in y-direction for data set $<$ik$>$ \\
  ni[$<$ik$>$]$^{*}$ & Type of data set $<$ik$>$. (0 for 1D
                       and 1 for 2D data) \\
  np[$<$ik$>$]$^{*}$ & Number of points of 1D data set $<$ik$>$ \\
  \hline
  xmin[$<$ik$>$]$^{*}$ & Minimum x-value of data set $<$ik$>$ \\
  xmax[$<$ik$>$]$^{*}$ & Maximum x-value of data set $<$ik$>$ \\
  ymin[$<$ik$>$]$^{*}$ & Minimum y-value of data set $<$ik$>$ \\
  ymax[$<$ik$>$]$^{*}$ & Maximum y-value of data set $<$ik$>$ \\
  zmin[$<$ik$>$]$^{*}$ & Minimum z-value of data set $<$ik$>$ \\
  zmax[$<$ik$>$]$^{*}$ & Maximum z-value of data set $<$ik$>$ \\
  pwin[i]$^{*}$        & Current plotting dimensions
                         (1:xmin,2:xmax,3:ymin,4:ymax) \\
  \hline
   x[$<$ik$>$,$<$ip$>$] & X-value of data point $<$ip$>$ of data set $<$ik$>$\\
   y[$<$ik$>$,$<$ip$>$] & Y-value of data point $<$ip$>$ of data set $<$ik$>$\\
  dx[$<$ik$>$,$<$ip$>$] & Value of $\sigma_{x}$ for data point $<$ip$>$ of
                          data set $<$ik$>$\\
  dy[$<$ik$>$,$<$ip$>$] & Value of $\sigma_{y}$ for data point $<$ip$>$ of
                          data set $<$ik$>$\\
   z[$<$ik$>$,$<$ix$>$,$<$iy$>$] &
                          Z-value of point $<$ix$>$,$<$iy$>$ of
                          data set $<$ik$>$\\
  \hline
  axis[1,i]       & Angle of labels for axis i (1=x,2=y) \\
  axis[2,i]       & Length of major ticks for axis i (1=x,2=y)\\
  axis[3,i]       & Length of minor ticks for axis i (1=x,2=y)\\
  axis[4,i]       & Subdivisions between ticks, axis i (1=x,2=y)\\
  axis[5,i]       & Distance numbers - axis i (1=x,2=y)\\
  axis[6,i]       & Distance label - axis i (1=x,2=y)\\
  \hline
  cmap[$<$ic$>$,3]    & Color map entry in R,G,B (0..1) for color $<$ic$>$ \\
  cmax[1]$^{*}$       & Number of bitmap colors available \\
  \hline
  p[$<$i$>$]          & Fit variable number $<$i$>$ (see section \ref{fit}) \\
  s[$<$i$>$]          & Error of p$<$i$>$ \\
  \hline
\end{tabularx}
\caption{\label{v1-tab}\Kuplot specific variables}
\end{table}

Names of input or output files are to be typed, as
they will be expected by the shell.  If necessary include a path to
the file.  Only the first four letters of any command verb are
significant, all commands may be abbreviated to the shortest unique
possibility.  At least a single space is needed between the command
verb and the first parameter.  No comma is to precede the first
parameter. A line can be marked as comment by inserting a `\#' as
first character in the line. The symbols used throughout this manual
to describe commands, command parameters, or explicit text used by
the program \Kuplot are listed in Table \ref{sym-tab}. \par

There are several sources of information, first \Kuplot has a
build in online help, which can be accessed by entering the command
{\tt help} or if help for a particular command $<$cmd$>$ is wanted
by {\tt help $<$cmd$>$}.  This manual describes background and
principle functions of \Kuplot and should give some insight in
the ways to use this program. \par

%------------------------------------------------------------------------
