\chapter{KUPLOT commands}
\section{Summary}
\par
Here is a list and brief description of valid KUPLOT commands. Further 
help can be obtained by typing the corresponding command name at the 
help prompt. 
\par
\begin{MacVerbatim}
News          : Information on program updates
achx/y/z      : Set labels for x, y and z-axis
alloc         : Allocate space for data set without loading one
angl          : Set angle between x- and y-axis
aver          : Set ratio between units on y and x-axis
branch        : Switch to discus, SUITE version only
buff          : Set reserved space outside of plotting frame
ccal          : Perform calculations with single data set
color         : Set pen colors
conv          : Calculate convolution of two data sets
costvalue     ! Set a value to be returned to DIFFEV
cmap          : Select color map
deriv         : Calculate the derivative
ecol          : Set color of error bars
etyp          : Set type of error bars
eval          : Evaluate expression
excl          : Exclude regions from 1D files
fill          : Set filling style
fit           : Enter least square fit sub level
fnam          : Toggle plot of filenames in upper left corner
font          : Set font size, type and color
fset          : Set type of plot frame (axis, tick marks, ..)
frames        : More information about the use of frames
func          : Create data set from given function
glat          : Smooth data set (sliding average)
grid          : Toggle plotting of a grid at main marker points
hart          : Set type for 2D plots (bitmap, contours, ..)
hcol          : Set color of contour lines
hlab          : Labeling of contour lines
hlin          : Set contour line parameters (also for bitmap)
hpak          : Set number of contour line packages
htyp          : Set line type for contour lines
iden          : Toggle plotting of user, data and time information
inte          : Integrate data set
ksav          : Save data set in various formats
kpara         : Plots parameters fitted with DIFFEV
kcal          : Perform calculations with two data sets (add, ..)
lart          : Set type for 1D plot (line, steps, spline)
lcol          : Set line color
load          : Read data set
ltyp          : Set line type
lwid          : Set line width
mark          : Set tick mark interval
match         : Calculate scale and offset to get best match bw. data
mcol          : Set marker color
mean          : Calculate average and stand. deviation for data set
merge         : Merge data sets (2D only)
mouse         : Activate the mouse
mtyp          : Set marker type
msiz          : Set marker size
nexus         : NeXus file support routines
orient        : Set paper orientation (landscape, portrait)
plot          : Display plot
prin          : Print plot
ptyp          : Set marker type for maxima
rdef          : Read KUPLOT settings from file
rebin         : Allows rebinning data to user defined grid
rese          : Reset
sann          : Set text annotations for plot
save          : Save plot as hardcopy
sdef          : Store KUPLOT settings on disk
show          : Show various settings
sleg          : Sets legend of plot
smax          : Determines maxima of data set
smooth        : Smooth data set (Savitzky-Golay algorithm)
sort          : Sort data set
spline        : Interpolate data set on grid given by different set
skal          : Set plot window
tit1/2        : Set plot titles
window        : Select graphics window
variables     : List of variables
errors        : List of error messages
\end{MacVerbatim}
\section{News}
\par
Here you can find updated info on the latest KUPLOT releases. 
\subsection*{2019\_April}
\par
Added "latex" option to ==$> $ 'save' command 
\par
Extended 'merge' command to 2D data. 
\par
Corrected fit weighting scheme "DAT" to 1/sig$**$2 
\subsection*{2019\_February}
\par
Added a 'blank' separator to 'load csv, separator:blank' 
\par
Added a new menu "2dm" that allows to concatenate 
several 2D data sets into a single 3D data set 
\subsection*{2018\_Oct}
\par
Added the trend of the best value to the 'kpara' plot 
\subsection*{2018\_June}
\par
Revised the reaction to a CTRL-C 
\par
Added a ==$> $ 'set error, ... , "save" option 
\subsection*{2018\_May}
\par
Added the possibility to use a macro as fit 'function' 
\subsection*{2018\_Feb}
\par
Added a command 'kpara' to create a plot for DIFFEV 
\par
Added an optional parameter "partial" to the ==$> $ 'rvalue' command 
\par
Added optional parameters to the "load csv" command 
\subsection*{2018\_Jan}
\par
The new command 'costvalue' allows to set a global value 
for the cost funciton that is returned to DIFFEV. 
\par
The logical comparisons may now take the operators: 
$ <$, $ <$=, ==, /=, $> $=, $> $/ 
The classical fortran77 operators are still valid 
\par
New logical functions "isvar" and "isexp" can be used within an 
"if" construction. See help entry ==$> $'function' in the 
general "Command\_lang" section. 
\subsection*{2017\_Oct}
\par
The 'ni' format may now have a leading header that consists of 
lines that start with a '\#' in the first column 
\subsection*{2017\_Sep}
\par
Throughout the program the internal calculation of random numbers 
was changed to the FORTRAN 90 intrinsic function. 
\par
Added error messages for variable nx, ny, modified variable np to 
yield the total number of data points for 2D data sets as well. 
\subsection*{2017\_May}
\par
The 'save' command can now save graphics as a PDF file 
\subsection*{2017\_Jan}
\par
Added the option to read CSV files to 'load' command. 
\par
An unfortunate typing error in News/2016\_oct regarding the new 
refinement variable 
ref\_para[1...]   ( was misspelled as ref\_param[1...] ) 
is corrected in the  on-line help. 
\subsection*{2016\_Dec}
\par
At a few select points colors are introduced into the output. 
Currently these are just the error messages. 
\par
\subsection*{2016\_Oct}
\par
Global variables have been introduced that use the same syntax as 
user defined variables. This include just "pi" and variables related 
to the refinement. 
DIFFEV sets the value to these variables: 
ref\_generation  Current generation 
ref\_member      Current population size 
ref\_children    Current children size 
ref\_dimension   Number of parameters 
ref\_kid         Current child Updated for DISCUS and KUPLOT only 
ref\_indiv       Current individuum Updated for DISCUS and KUPLOT only 
ref\_para[1..]   Current trial parameters for current child 
\par
All colors can be referenced by their respective names as well as by 
their traditional numbers. See 'lcol', 'hcol', 'mcol' 'ecol' 
\par
All line types can be referenced by descriptive names as well as by 
their traditional numbers. See 'ltyp', 'htyp' 
\par
All marker types can be referenced by descriptive names as well as by 
their traditional numbers. See 'etyp', 'mtyp', 'ptyp' 
\subsection*{2016\_Feb}
\par
Minor upgrade, added a threshold function to ==$> $ 'ccal' 
\subsection*{2015\_Dec}
\par
Minor upgrade, increases the size of bitmaps KUPLOT can plot 
\subsection*{2015\_June}
\par
Starting with Version 5.1, we have migrated to a X-Window 
environment for WINDOWS as well. As a small side effect, 
the technique to jump to the desired folder has changed slightly. 
See the help entry on "cd" in the general "Command\_lang" section 
for further information. The process is described in the 
package manual as well. 
\section{2dm}
{\bf 2dm \par }
\par
\vspace{3pt}
Branches into the '2dm' menu. 
This menu allos you to concatenate several 2D i.e. xy data 
sets into a single 3D data set. 
\par
Valid commands at this level are: 
\subsection*{reset}
{\bf reset \par }
\par
\vspace{3pt}
Reset all numerical values back to system start. 
\subsection*{run}
{\bf run \par }
\par
\vspace{3pt}
Performs the actual calculation. The map will be 
stored in the next free data set number. 
\subsection*{back}
{\bf back "off" \par }
{\bf back "csv",$ <$filename$> $ \par }
{\bf         ,skip:$ <$nskip$> $, \par }
{\bf         colx:$ <$icolx$> $, coly:$ <$icoly$> $, coldx:$ <$icoldx$> $, coldy:$ <$icoldy$> $, \par }
{\bf         separator:$ <$name$> $ \par }
{\bf         scale:$ <$factor$> $ \par }
\par
\vspace{3pt}
Defines the name and type of a background file. 
The y-values will be subtracted from the corresponding 
y-values of the files read with 'load'. 
If the command is used with the parameter "off", no 
background file will be read. 
\par
Optionally a scale factor for the background can be 
defined with 
scale:$ <$factor$> $ 
\par
The remaining optional parameters are identical to those 
of the == 'load csv' command at the main menu. 
\subsection*{load}
{\bf load "csv","filename\_format", LOOP \par }
{\bf         ,skip:$ <$nskip$> $, \par }
{\bf         colx:$ <$icolx$> $, coly:$ <$icoly$> $, coldx:$ <$icoldx$> $, coldy:$ <$icoldy$> $, \par }
{\bf         separator:$ <$name$> $ \par }
\par
\vspace{3pt}
Defines the name and type of the 2D input files. 
The filename must be specified within double quotation marks 
and contain a format descriptor that will be replaced by the 
formal "LOOP" loop index. The loop index must be called "LOOP". 
\par
A second, outer loop can be defined as well, which must use 
the counter variable "SLOW". 
\par
The remaining optional parameters are identical to those 
of the == 'load csv' command at the main menu. 
\subsection*{loop}
{\bf loop "counter:"$ \{$"LOOP"$| $"SLOW"$\} $, \par }
{\bf      "start:"$ <$start\_number$> $, "end:"$ <$end\_number$> $ \par }
{\bf      ,"step:"$ <$step\_number$> $ \par }
{\bf      ,"miss:"$ \{$"error"$| $"ignore"$| $"blank"$\} $ \par }
\par
\vspace{3pt}
Defines the range of input file numbers that will be used 
for the "LOOP"  or "SLOW" value on the ==$> $ 'load' command. 
\par
The optional parameters "start:", "end:" and "step:" 
default to a value of 1. 
Allowed values for the numerical values are valid expressions. 
\par
The optional parameter "miss:" with default "error" 
sets the rules for missing or corrupt input files. 
With the default value "error", any missing or corrupt 
input file will interrupt the ==$> $ 'run' command and 
no map will be produced. 
With "ignore" the erroneous file will simply be ignored 
and the map will be shorter by one (ore more) lines. 
With "blank" the corresponding line in the map will be 
set to a value of zero. 
The status of "miss" is a single status. If both loop 
counters "LOOP" and "SLOW" are used, use a common and 
identical status to describe the handling of missing 
files. 
\subsection*{xrange}
{\bf xrange "xmin:"$ \{$"xmin"$| $$ <$x\_min$> $$\} $ , "xmax:"$ \{$"xmax"$| $$ <$x\_max$> $$\} $ \par }
\par
\vspace{3pt}
Defines the x-range of the input files that is to be 
written into the 3D map. 
Both parameters default to "xmin" respectively "xmax" 
the actual minimum and maximum values of the input files. 
\subsection*{yfunction}
{\bf yfunction loop:LOOP \par }
{\bf yfunction loop:inc \par }
{\bf yfunction loop:$ <$expression$> $ \par }
\par
\vspace{3pt}
Defines the function that will be used to calculate the 
y-axis values of the map. 
\par
Default is "loop:LOOP". The y-axis will be the loop numbers 
in the range as defined on the ==$> $ 'loop' command. 
\par
As these numbers may be unnecessarily large, the option 
"loop:inc" will increment the y-axis values from 1 in steps 
of 1. 
\par
Finally you might want to calculate the y-values from 
a user defined function, or use values in a previous 
data set: 
loop:1+LOOP-start\_number 
   If start\_number is a variable that corresponds to the 
   start value on the ==$> $ 'loop' command, this function 
   would be identical to the "inc" option. 
loop:y[1,LOOP] 
   Would take the y-value of data set 1 point LOOP as 
   y-value for the map. This might be appropriate if 
   y[1,*] contains temperature or pressure or time or ... 
   values for the 2D data sets. 
\subsection*{example}
\par
Loop with a single counter 
\begin{MacVerbatim}
reset
back off
back csv, empty_measurement.xy, scale:1.0, skip:0
load csv, "DATA/my_data.%4D", LOOP, colx:1, coly:1, skip:0
loop counter:LOOP, start:1, end:201, step:2, miss:blank
yfunc loop:LOOP
xrange xmin:xmin, xmax:5.00
run
\end{MacVerbatim}
Loop with a two counters, SLOW and LOOP 
\begin{MacVerbatim}
reset
back off
back csv, empty_measurement.xy, scale:1.0, skip:0
load csv, "DATA/my_data_%5D.%4D", SLOW, LOOP, colx:1, coly:1, skip:0
loop counter:SLOW, start:100, end:101, step:1, miss:ignore
loop counter:LOOP, start:1, end:201, step:2, miss:ignore
yfunc loop:LOOP
xrange xmin:xmin, xmax:5.00
run
\end{MacVerbatim}
\section{achx/y/z}
{\bf achx $ \{$$ <$string$> $ $| $"OFF"$\} $ [, $ \{$"lin" $| $ "log"$\} $] \par }
{\bf achy $ \{$$ <$string$> $ $| $"OFF"$\} $ [, $ \{$"lin" $| $ "log"$\} $] \par }
{\bf achz $ \{$$ <$string$> $ $| $"OFF"$\} $ \par }
\par
\vspace{3pt}
This command sets the labels for the x- or y-axis. The label for 
the z-axis is only active when bitmaps are plotted and is used 
to label a wedge on the left side of the plot. As for the other 
axes the parameter "OFF" will turn off the plotting of the 
wedge which is the default. The optional parameter "lin" or 
"log" allows the user to toggle between a linear and logarithmic 
axis. 
\par
Because KUPLOT treats the ',' as a parameter separator it can not 
be used within the label $ <$string$> $. The command allows to build a 
label from a format text and variables/numbers. It is the same 
mechanism as described in -$> $ filenames. If the given parameter is 
"OFF", no label AND no numbers will be plotted at the corresponding 
axis. 
\par
Examples: 
\par
\begin{MacVerbatim}
achx 2 Theta       : will label x-axis with '2 Theta'
achy "PSD #%d",43  : will label y-axis with 'PSD #43'
\end{MacVerbatim}
\section{alloc}
{\bf alloc $ <$name$> $,$ <$len$> $ \par }
{\bf alloc $ <$name$> $,$ <$x$> $,$ <$y$> $ \par }
\par
\vspace{3pt}
This command allocates space for a new data set. This is e.g. required 
if a data set should be created using the FORTRAN interpreter of 
KUPLOT. The command 'alloc $ <$name,$ <$len$> $' allocated space for $ <$len$> $ 
points for a 2d data set. To allocate a 3d data sets, the size in x- 
and y-direction is given after the name of the data set. The parameter 
$ <$name$> $ just specifies a name for the data set (also used as default 
filename if the data set is saved). 
\section{angl}
{\bf angl [$ <$angle$> $] \par }
\par
\vspace{3pt}
This command allows to set the angle between the x- and y-axis for 
the currently active frame. Suitable angles are between 30 and 120 
degrees. If the command is entered without parameters, the value 
is set to 90.0 degrees. 
\section{aver}
{\bf aver [$ <$yx$> $] \par }
\par
\vspace{3pt}
This command sets the aspect ratio between the y- and the x-axis. 
A value of $ <$yx$> $=1.0 will force one unit on the x-axis to be as 
long as one unit on the y-axis. The value $ <$yx$> $ is the ratio 
y/x of the length of a unit of both axis. Is the parameter $ <$yx$> $ 
omitted, the plot will fill the complete drawing area. 
\section{bond}
{\bf bond \par }
{\bf bond $ <$ib$> $,$ <$dist$> $ [,$ <$sig$> $ [,$ <$it$> $,$ <$ic$> $,$ <$iw$> $]] \par }
\par
\vspace{3pt}
This command enables the plotting of lines between points that 
have a given distance, e.g. to include bonds in a crystal file 
plot. The first parameter $ <$ib$> $ is the number of the definition 
you are going to alter. The second parameter $ <$dist$> $ is the 
distance of points that shall be connected by a line. The distance 
is given relative to the current x-axis. To calculate the distances, 
the aspect ratio and the angle between the axes is used. Thus you 
will get the desired connection only for the correct aspect ratio 
(-$> $ aver) and angle between the axes (-$> $ angl). The other parameters 
are optional. The value $ <$sig$> $ sets the allowed relative deviation 
from the given distance, e.g. a value of 0.1 will allow distances 
of +- 10\% of the given value. The parameters $ <$ic$> $, $ <$it$> $ and $ <$iw$> $ 
set the line color, line type and line width of the connecting 
line, respectively (-$> $ lcol, ltyp, lwid). 
\par
In order to disable the plotting of a defined bond, just specify 
a zero distance, i.e. 'bond $ <$ic$> $,0.0'. 
\section{branch}
{\bf branch discus [, "-macro" $ <$macro\_name$> $ [ $ <$par1$> $ [ , $ <$par2$> $ ...]]] \par }
{\bf branch diffev [, "-macro" $ <$macro\_name$> $ [ $ <$par1$> $ [ , $ <$par2$> $ ...]]] \par }
\par
\vspace{3pt}
Active within the discus suite only! 
\par
Branches to the "discus" or "diffev" section. 
\par
Within this section any standard DISCUS command can be 
given. The behavior of "discus" is essentially the same 
as in the stand alone version. Likewise for DIFFEV. 
\par
The main use will branch to DISCUS while the discus section 
is run via run\_mpi from a DIFFEV slave. 
\par
Optionally the "-macro" qualifier instructs the suite to run the 
macro $ <$macro\_name$> $ (with its optional parameters) before the 
interactive session is started. 
\section{buff}
{\bf buff $ <$all$> $ \par }
{\bf buff $ <$left$> $,$ <$right$> $,$ <$bottom$> $,$ <$top$> $ \par }
\par
\vspace{3pt}
This command allows to alter the amount of drawing space KUPLOT 
reserves outside the plotting area for labels, titles etc.. The 
parameters given are relative to the size of the plot, e.g. a 
value of 0.3 will save 30\% of the width/height of the plot itself 
for axis numbers, labels etc. - note that if the reserved space 
is too small text might be wrapped around or cut off. 
\section{branch}
{\bf branch discus \par }
\par
\vspace{3pt}
Active within the discus suite only! 
\par
Branches to the "discus" section. 
\par
Within this section any standard DISCUS command can be 
given. The behavior of "discus" is essentially the same 
as in the stand alone version. 
\section{ccal}
{\bf ccal $ <$oper$> $,$ <$field$> $,$ <$ik$> $ [,$ <$value$> $] \par }
\par
\vspace{3pt}
This command allows data manipulation. All functions could also 
be performed using the FORTRAN type interpreter, but especially 
for large data sets this can be very slow. 
\par
The general syntax is the operation $ <$oper$> $ (see list below) that 
should be performed as first parameter followed by the field 
identifier $ <$field$> $ (see below), followed by the number of the 
data set $ <$ik$> $. Some operations require an additional parameter 
$ <$value$> $. 
\par
Here is a list of allowed operation parameters $ <$oper$> $: 
\par
\begin{MacVerbatim}
  "inv"  :  x -> 1/x         "squ"  :  x -> x**2
  "log"  :  x -> ln(x)       "add"  :  x -> x + <value>
  "exp"  :  x -> exp(x)      "mul"  :  x -> x * <value>
  "sqr"  :  x -> sqrt(x)     "abs"  :  x -> |a|
  "thr"  :  x < value -> -9999; value is thus a threshold,
\end{MacVerbatim}
 data below are considered invalid 
\par
The value "x" in the table above represents the coordinate selected 
by the $ <$field$> $ parameter. Allowed values are: 
\par
\begin{MacVerbatim}
  "wx" : x value of data set
  "dx" : standard deviation of x
  "wy" : y value of data set
  "dy" : standard deviation of y
  "wz" : z value of data set (for 3d data sets only)
\end{MacVerbatim}
\section{color}
{\bf color $ <$ic$> $,$ <$r$> $,$ <$g$> $,$ <$b$> $ \par }
\par
\vspace{3pt}
This command allows the user to change the color settings for 
the different pens. The parameter $ <$ic$> $ gives the pen number 
(0 for background) and $ <$r$> $,$ <$g$> $,$ <$b$> $ are the red, green and blue 
values to be uses ranging from 0.0 to 1.0. If no parameters are 
given, the current settings are displayed. 
\section{conv}
{\bf conv $ <$ik1$> $,$ <$ik2$> $ \par }
\par
\vspace{3pt}
This command calculates the convolution of data sets $ <$ik1$> $ and 
$ <$ik2$> $ and creates a new data set with the result. Both data sets 
must be on the same equidistant grid ! 
\section{cost}
{\bf costvalue  $ <$rvalue$> $ [, $ <$partial\_rvalue1$> $ ...] \par }
\par
\vspace{3pt}
Set a value that is to be retured to DIFFEV as the overall 
R-value. It will overwrite the current weighted R-value 
that was calculated with the ==$> $ 'rval' command. 
The value of $ <$rvalue$> $ can be any expression or real valued 
variable. This allows to calculate a combined, weighted 
R-value from several different calculations like single 
crystal, powder diffraction, PDF, etc. 
\par
If the partial R-values are provided as well, they will 
overwrite the corresponding partial R-values. 
\section{cmap}
{\bf cmap $ \{$ "gray" $| $ "fire" $| $ "kupl"  $\} $ \par }
{\bf cmap $ \{$ "read" $| $ "write"$\} $, $ <$fname$> $ \par }
{\bf cmap "invert" \par }
\par
\vspace{3pt}
This command allows to choose the color map to be used for printing 
bitmaps (see 'hart' command). The three build in color map are 
"gray", "fire" and "kupl" (the previous default color map). Color maps 
can also be read from a file using the command 'cmap read,$ <$fname$> $'. 
The format is of the form '\#RRGGBB' for each color map entry on a 
new line and 'RR', 'GG' and 'BB' are the 'red', 'green' and 'blue' 
values given in hexadecimal ranging from 00 to FF. The current 
color map can be saved to a file using the command 'cmap write,$ <$fname$> $'. 
Finally 'cmap invert' will simply invert the current color map. 
\par
The RGB values of the current color map can be shown using the 
command 'show color'. The color map entries can also be altered 
using the variable 'cmap [$ <$ic$> $,3]'. 
\section{deriv}
{\bf deriv $ <$ik$> $ [,$ <$order$> $] \par }
\par
\vspace{3pt}
This command calculates the $ <$order$> $th derivative of data set $ <$ik$> $. 
The result is stored as new data set. If $ <$order$> $ is omitted, the 
first derivative is calculated. 
\section{dsav}
{\bf dsav "gsas",$ <$file$> $ [,$ <$iname$> $] \par }
{\bf dsav "xml",$ <$file$> $ [,$ <$dtd-path$> $] \par }
{\bf dsav "merge",$ <$file$> $ \par }
\par
\vspace{3pt}
The parameter "gsas"  allows one to save all loaded data sets 
in GSAS file format. NOTE THAT THE CURRENT IMPLEMENTATION ASSUMES 
A LUJAN STYLE GSAS FILE. 
\par
The parameter "xml" will save the plot on PlotML (xml) format 
for viewing with 'plplot'. Currently only 1D data are supported. 
The optional second parameter allows one to specify the location 
of the DTD file needed to render the XML file. 
\par
The parameter "merge" will save a xyz-type file from all the 
loaded data sets with the following condition x=x, y=data set 
number and z=y. This can then be loaded using load zz, .. 
\section{ecol}
{\bf ecol $ <$ik$> $,$ <$c$> $ \par }
\par
\vspace{3pt}
This command sets the error bar color for the data set $ <$ik$> $. 
The colors are numbered like the pens on a HP7475 plotter. Allowed 
values for $ <$c$> $ are listed in the table, Alternatively you can use 
the color name (without blanks i.e. darkblue etc.) 
\par
The allowed default color values $ <$icol$> $ are listed below. Note 
that they can be changed by the user via the command 'color'. 
\par
\begin{MacVerbatim}
  1 : red       5 : yellow        9 : dark blue      13 : cyan
  2 : green     6 : black        10 : dark magenta   14 : dark cyan
  3 : blue      7 : dark red     11 : dark yellow    15 : white (!)
  4 : magenta   8 : dark green   12 : gray
\end{MacVerbatim}
\section{etyp}
{\bf etyp $ <$ik$> $,$ <$p$> $ \par }
\par
\vspace{3pt}
This command sets the type of error bars to be use for the data 
set number $ <$ik$> $. Allowed values for $ <$p$> $ are: 
\par
\begin{MacVerbatim}
  0 : {"noerror"|"none"} : no error bars are drawn
  1 :  "xbar"            : only error bars in x-direction
  2 :  "ybar"            : only error bars in y-direction
  3 :  "cross"           : error bars in both directions
\end{MacVerbatim}
\section{excl}
{\bf excl $ <$ik$> $,$ <$left$> $,$ <$right$> $ \par }
\par
\vspace{3pt}
This command allows the user to exclude the area between $ <$left$> $ 
and $ <$right$> $ from data set $ <$ik$> $. The values within the range 
are replaced by the linear interpolation between $ <$left$> $ and 
$ <$right$> $, i.e. the closest data points to the specified values. 
If $ <$left$> $ is outside the data range, all points up to $ <$right$> $ 
are removed. If $ <$right$> $ is outside the data range, all points 
from $ <$left$> $ onwards are removed. 
\section{fill}
{\bf fill $ <$ik$> $,$ <$fc$> $,$ <$ft$> $ [,$ <$xmin$> $,$ <$xmax$> $] [,$ <$ymin$> $,$ <$ymax$> $] \par }
\par
\vspace{3pt}
This command sets the filling style for data set $ <$ik$> $. The data 
set is treated as a polygon and is filled between the data and 
the baseline. The fill color is defined by $ <$fc$> $ (see lcol for 
color definitions). The following fill types $ <$ft$> $ are currently 
supported: 
\par
\begin{MacVerbatim}
  1: "solid"     : solid fill    5: "bsolid" : type 1 plus border
  2: "slash"     : hashed (/)    6: "bslash" : type 2 plus border
  3: "backslash" : hashed (\)    7: "bback"  : type 3 plus border
  4: "cross"     : cross hashed  8: "bcross" : type 4 plus border
\end{MacVerbatim}
The filled area is defined by the data points and a baseline, 
by default the connection between the end points of the current 
plotting area. However, the area to be filled can be limited 
using the optional parameters $ <$xmin$> $, $ <$xmax$> $ and $ <$ymin$> $, $ <$ymax$> $. 
\section{fit}
{\bf fit $ <$ik$> $ \par }
\par
\vspace{3pt}
This command enters the fit sub level and data set $ <$ik$> $ is taken 
as target data set. For an easy start, select a fit function 
with 'func' and start the fit using the 'run' command. The result 
can be displayed with 'plot' and saved with 'save'. For more 
detailed information see the list of commands below: 
\par
\subsection*{commands}
{\bf Valid commands in this sub level are: \par }
\par
\begin{MacVerbatim}
@      : Executes a macro (see general help)
=      : Algebra (see general help)
cycle  : Sets maximum number of fit cycles
echo   : Echoes a string on the screen (see general help)
eval   : Evaluates an expression  (see general help)
exit   : Exits fit sub level
func   : Sets fit function
help   : Gives on-line help for 'fit' sub level (see general help)
macro  : Writes current fit parameters to a macro file
mark   : Sets interval for tick marks
mfen   : Sets parameter for maxima determination (fit start values)
output : Toggles fit screen output on/off
para   : Sets/modifies fit parameters
plot   : Plot results
range  : Sets data range used for refinement
run    : Start the fit
save   : Save fit results
show   : Show settings and fit results
skal   : Sets region for plotting window (see general help)
system : Executes operating system command (see general help)
urf    : Sets urf value
wait   : Waits for user input (see general help)
wic    : Sets weighting scheme
\end{MacVerbatim}
\subsection*{cycle}
{\bf cycle $ <$n$> $ \par }
\par
\vspace{3pt}
This command sets the maximum number of cycles to $ <$n$> $. If a minimum 
is found in less cycles, the fit will stop before $ <$n$> $ cycles are 
finished. 
\subsection*{func}
\par
This command allows to set the fit function. After this command is 
called, KUPLOT will calculate starting values and destroy possible 
previous fit results, which should be saved ('save') before this 
command is called. The starting values for some functions use the 
'smax' function. You can set the window size using the 'mfen' command. 
See also -$> $ 'smax'. The following table shows the currently allowed 
functions: 
\par
\begin{MacVerbatim}
function   2d   3d   what            <par>
-----------------------------------------------------------
 "poly"    yes  no   polynomial         order of the polynomial
 "cheb"    yes  no   Chebyshev poly. order of the polynomial
 "gaus"    yes  yes  Gaussian        number of Gaussians
 "lore"    yes  no   Lorentzian      number of Lorentzians
 "pseudo"  yes  no   Pseudo-Voigt    number of Pseudo-Voigts
 "fx"      yes  yes  User defined    # parameters and function
 "ma"      yes  yes  User macro      # parameters and macro name
 "back"    yes  no   Background      # of background parameters
 "gsas"    yes  no   TOF profile     see text
\end{MacVerbatim}
Select a fit function from the list below for more information: 
\subsubsection{poly}
{\bf func "poly" [,$ <$par$> $] \par }
\par
\vspace{3pt}
A polynomial function p[1] + p[2]*r[0] + p[3]*r[0]**2 + ... 
is used. The parameter is the order of the polynomial. 
\subsubsection{gaus}
{\bf func "gaus" [,$ <$par$> $] \par }
\par
\vspace{3pt}
The fit uses $ <$par$> $ Gaussian functions. Two parameters (no. 1 
and 2) are refined for the background and each Gaussian peak used 
four parameters: 
\par
\begin{MacVerbatim}
p[3] Peak height
p[4] Position
p[5] FWHM
p[6] Asymmetry
\end{MacVerbatim}
A Gaussian function with linear background is used as: 
p[1] + p[2]*r[0] + p[3]*exp(-4*ln(2)*(r[0]-p[4])**2/p[5]**2) 
Note that the peak height rather than the integral intensity 
is refined. 
\subsubsection{lore}
{\bf func "lore" [,$ <$par$> $] \par }
\par
\vspace{3pt}
The fit uses $ <$par$> $ Lorentzian functions. Two parameters (no. 1 and 2) 
are refined for the background and each Lorentzian peak used four 
parameters: 
\par
\begin{MacVerbatim}
p[3] Peak height
p[4] Position
p[5] FWHM
p[6] Asymmetry
\end{MacVerbatim}
A Lorentzian function with linear background is used: 
p[1] + p[2]*r[0] + p[3]*p[5]**2/(p[5]**2 + 4*(r[0]-p[4])**2) 
Note that the peak height rather than the integral intensity 
is refined. 
\subsubsection{pseudo}
{\bf func "pseudo" [,$ <$par$> $ [,$ <$origin$> $ [,$ <$n\_backgrd$> $ ]] \par }
\par
\vspace{3pt}
The fit uses $ <$par$> $ Pseudo-Voigt functions. $ <$n\_backgrd$> $ parameters 
(no. 1, .. to $ <$n\_backgrd$> $) are refined for the background and each 
Pseudo-Voigt peak used six parameters: 
\par
\begin{MacVerbatim}
p[3] eta
p[4] integral intensity
p[5] Position
p[6] FWHM
p[7] Asymmetry parameter 1
p[8] Asymmetry parameter 2
\end{MacVerbatim}
The Pseudo-Voigt function is used in the form: 
\par
\begin{MacVerbatim}
background + Intensity*(eta*L + (1-eta)G)*Asymmetry
\end{MacVerbatim}
with Lorentzian part as: 
\begin{MacVerbatim}
L = 2/pi*FWHM / ( FWHM**2 + 4*(x-POSITION)**2 )
\end{MacVerbatim}
and Gaussian part as: 
\begin{MacVerbatim}
G = 2*sqrt(ln(2)/pi)/FWHM * exp ( -4ln(2) * (x-POSITION)**2/FWHM**2 )
\end{MacVerbatim}
The asymmetry function is defined as: 
\par
\begin{MacVerbatim}
ASYMMETRY = 1 + (p[7]*FA(z) + p[8]*FB(z))/ tan(POSTION)
FA(z) = 2*z*exp(-z**2)
FB(z) = 2*(2*z**2-3)*FA(z)
z = (x-POSITION)/FWHM
\end{MacVerbatim}
The background is calculated as a polynomial of order 
(n\_backgrd-1): 
\par
\begin{MacVerbatim}
p[1] + p[2]*(x-origin) + p[3]*(x-origin)**2 + ...
or more exactly:
SUM p[i]*(x-origin)**(i-1)
\end{MacVerbatim}
where origin is the third optional parameter from the command 
line. If this parameter is omitted, the origin is set to zero. 
To describe a locally bent background, one should set the 
origin of the background to a position within the refined 
x-range instead of zero. The value of the origin is not refined. 
\par
The last optional parameter defines the number of background 
parameters to be used. If it is omitted, the number of 
background parameters is set to 2. 
\subsubsection{fx}
{\bf func "fx",$ <$ip$> $,$ <$function$> $ \par }
\vspace{3pt}
KUPLOT allows the user to fit a user defined function to a given 
1D as well as 2D data set. The first parameter after "fx" specifies 
the number of parameters. Refer to the parameters in the expression 
for the function as p[1],p[2],.... The x-variable is r[0] and the 
y-variable (for 2D data sets) is r[1]. Note, that possible values 
stored in these variables are lost. It sounds complicated, but the 
following examples should help: 
\par
\begin{MacVerbatim}
func fx,3,p[1]+p[2]*exp(p[3]*r[0])   : Fit a+b*exp(c*x)
func fx,2,p[1]*r[0]+p[2]*r[1]        : Fit a*x + b*y (2D data set)
\end{MacVerbatim}
For a syntax check, the expression is calculated once after entering 
this command. If the parameter values p[i] are not set to proper 
start values, a error might occur, i.e. division by zero. 
\subsubsection{ma}
{\bf func "ma",$ <$ip$> $,$ <$macro\_name$> $ \par }
\vspace{3pt}
KUPLOT allows the user to fit a user defined function inside a 
macro to a given 1D as well as 2D data set. The first parameter 
after "fx" specifies the number of parameters. Refer to the 
parameters in the expression for the function as p[1],p[2],.... 
The x-variable is r[0] and the y-variable (for 2D data sets) 
is r[1]. Note, that possible values stored in these variables 
are lost. 
\par
The macro can use expressions, loops and if constructions, 
nested macros are allowed as well. 
For a full use of the discus\_suite, use a combination of a 
'top' and 'quit' command. With the 'top' command the macro 
steps to the top level at the discus\_suite. From there on 
all commands can be used. 
\par
The final function value must be given on a 'value' command, 
the derivatives on a 'deriv' command. For these command you 
must have left the 'top' section via a 'quit'. 
As simple macro would be: 
value p[1]*r[0]                  ! Fit a*x 
deriv 1, r[0]                    ! derivative with respect to p[1] 
finished 
\par
or absolutely eqivalently: 
r[100] = p[1]*r[0] 
value r[100]                     ! Fit a*x 
deriv 1, r[0]                    ! derivative with respect to p[1] 
finished 
\par
More involved, create a structure, calculate bond length as the 
function value: 
\par
r[101] = p[1]                    ! Copy KUPLOT variables to global 
r[102] = p[2] 
top                              ! Go to top level menu 
  discus 
    read 
      free 
    insert C, r[101], 0,0, 0.1 
    insert C, r[102], 0,0, 0.1 
    r[200] = x[1]-x[2]           ! the function value 
    r[201] =  1.0                ! the derivative with respect to p[1] 
    r[202] = -1.0                ! the derivative with respect to p[2] 
  exit 
quit                             ! Quit the top level menu back to fit 
value r[200] 
deriv 1, r[201] 
deriv 2, r[202] 
finished 
\par
It sounds complicated, but the following examples should help: 
\par
\begin{MacVerbatim}
func ma,3, one.mac                   : Fit a+b*exp(c*x)
\end{MacVerbatim}
Inside macro "one.mac" use the lines: 
value p[1]+p[2]*exp(p[3]*r[0])   ! Fit a+b*exp(c*x) 
deriv 1, 0.0                     ! derivative with respect to p[1] 
deriv 2, exp(p[3]*r[0])          ! derivative with respect to p[2] 
deriv 3, p[2]*exp(p[3]*r[0])*r[0]! derivative with respect to p[3] 
finished 
\par
The last command 'finished' is required inside these macros to 
return to the fit routine. 
\par
\begin{MacVerbatim}
func ma,2, two.mac                   : Fit a*x + b*y (2D data set)
\end{MacVerbatim}
Inside macro "two.mac" use the lines: 
value p[1]*r[0]+p[2]*r[1]        ! Fit a*x + b*y 
deriv 1, r[0]                    ! derivative with respect to p[1] 
deriv 2, r[1]                    ! derivative with respect to p[2] 
finished 
\par
\begin{MacVerbatim}
func ma, 2
\end{MacVerbatim}
For a syntax check, the expression is calculated once after entering 
this command. If the parameter values p[i] are not set to proper 
start values, a error might occur, i.e. division by zero. 
\subsubsection{gsas}
{\bf func "gsas",$ <$ptype$> $,$ <$bank$> $,$ <$iparm-file$> $ \par }
\par
\vspace{3pt}
KUPLOT allows the refinement of a GSAS profile function for TOF 
data. For details about the parameters and definitions of these 
functions, please refer to the GSAS manual. The parameters are 
$ <$ptype$> $ to select the profile function type, $ <$bank$> $ to select 
the bank number and $ <$iparm-file specifies the GSAS style 
instrument parameter file. The starting values are read from this 
file. Please also note, that the x-values are expected to be TOF. 
\subsubsection{back}
{\bf func "back",$ <$icalc$> $ [,$ <$par$> $] \par }
\vspace{3pt}
The fitting of a background is calculated as: 
\par
\begin{MacVerbatim}
y(dataset) = scale*y(dataset icalc) + SUM xx**parameter
\end{MacVerbatim}
Thus KUPLOT takes the data set $ <$icalc$> $ as input values, which are 
scaled and to which a polynomial function is added. This sum is 
fitted to the current dataset which was selected by the 'fit' command. 
\par
The scale parameter is p[1], all higher parameters act as exponents 
to the background polynomial. Thus p[2] is a constant background, 
p[3] a linear background etc. The xx argument of the polynomial 
is calculated as (x - xmin), where x is the x-value of each given 
data point and xmin the minimum x-value. Thus, the values of the 
background polynomial change if you fit to a changed x-range! 
\par
\begin{MacVerbatim}
<par>  effect
1      scale factor only
2      scale factor + constant background
3      scale factor + constant + linear background
N      scale factor + background polynomial of oder (N-2)
\end{MacVerbatim}
\subsection*{help}
{\bf help $ <$command$> $ \par }
\par
\vspace{3pt}
Displays this help text, for help for a certain command type help 
command. See help entry on main level for more details. 
\subsection*{macro}
{\bf macro $ <$filename$> $ \par }
\par
\vspace{3pt}
This command saves the current parameters in a KUPLOT macro file. 
This allows to set the parameter values in a later session using 
@$ <$filename$> $ from the FIT sub level. 
\subsection*{mfen}
{\bf mfen $ <$if$> $ \par }
\par
\vspace{3pt}
This command sets the 'window size' for maxima determination used 
to compute starting values for functions GAUS and LORE. For details 
about the $ <$if$> $ value see command -$> $ smax. 
\subsection*{output}
{\bf output $ \{$ "on" $| $ "off" $\} $ \par }
\par
\vspace{3pt}
This command allows to specify if the progress of the fit should be 
displayed on the screen ("on") or not ("off"). 
\subsection*{para}
{\bf para \par }
{\bf para $ <$ip$> $,$ <$pinc$> $ [,$ <$value$> $] \par }
{\bf para $ \{$ "save" $| $ "load" $\} $ \par }
\par
\vspace{3pt}
This command allows to modify the fit parameters. If the command 
is called alone, the current parameters are displayed. A single 
parameter $ <$ip$> $ can be set to a value $ <$value$> $. The value $ <$pinc$> $ 
specifies if the parameter should be refined ($ <$pinc$> $ = 1) of should 
be kept fixed ($ <$pinc$> $ = 0). If $ <$value$> $ is omitted the current value 
will be used. 
\par
A set of parameters can be internally stored ('para save') and loaded 
later back in the parameter array ('para load'). Each save command 
will overwrite previously stored values. 
\subsection*{plot}
{\bf plot \par }
\par
\vspace{3pt}
This command displays the current fit result with the current plot 
settings (like the plot command in the main level). 
\subsection*{range}
{\bf range $ \{$"all" $| $ "plot"$\} $ \par }
\par
\vspace{3pt}
This command allows one to select which part of the data are used 
for the refinement. If set to "all", all data are used. If set to 
"plot" only the data shown on the current plot window are used. 
This is useful if one wants to refine e.g. just a single Gaussian 
in a data set without having to manipulate the data set itself. 
\subsection*{run}
{\bf run \par }
\par
\vspace{3pt}
This command starts the fit. 
\subsection*{save}
{\bf save \par }
\par
\vspace{3pt}
This command saves the current fit results. The output filename 
is the filename of the input data set with the following extensions: 
\par
\begin{MacVerbatim}
  .erg  : text file with fit results
  .fit  : calculated data set
  .dif  : difference between observed and calculated data set
\end{MacVerbatim}
\subsection*{show}
{\bf show [ $ \{$ "general" $| $ "fit" $| $ "para" $\} $ ] \par }
\par
\vspace{3pt}
This command shows fit settings and results. If the command is called 
without parameters, all information is printed on the screen. The 
command 'show general' will display only general fit settings, 'show 
fit' the fit results like R values and 'show para' will list the 
resulting parameters. 
\subsection*{urf}
{\bf urf $ <$u$> $ \par }
\par
\vspace{3pt}
This command sets the URF (some German: Unterer Relaxations Faktor) 
for the fit. This value determines who 'fast' the fit will move to 
its minimum or how much the parameter values are changed in each 
cycle depending on the deviations. A small value (e.g. 0.1) might 
lead to a fast convergence but might also miss the minimum. A larger 
value (e.g. 100.0) will give a slow convergence which more certain 
finds the minimum, but might be caught in local minima rather than in 
the global one. 
\par
Understood ? Well just try different values until your fit converges 
nicely to the global minimum. 
\subsection*{wic}
{\bf wic $ <$w$> $ [,$ <$K$> $] \par }
\par
\vspace{3pt}
This command defines the weighting scheme to be used for the fit. The 
valid values for $ <$w$> $ are: 
\par
\begin{MacVerbatim}
  "one"  : w(y) = 1.0              "sqa"  : w(y) = y**2
  "sqrt" : w(y) = SQRT(y)          "inv"  : w(y) = 1/y
  "log"  : w(y) = LOG(y)           "isq"  : w(y) = 1/SQRT(y)
  "lin"  : w(y) = y                "dat"  : w(y) = 1/DY**2 (error data set)
  "bck"  : w(y) = exp(-K*(y-ycalc))
\end{MacVerbatim}
The second parameter is only needed for the weighting scheme "bck". 
\section{fnam}
{\bf fnam [ $ \{$"on" $| $ "off"$\} $ ] \par }
\par
\vspace{3pt}
This command determines weather the filename should be plotted for 
every data set in the upper right corner. Called without parameters, 
filename plotting is switched off. 
\section{font}
{\bf font $ <$prop$> $,$ <$value$> $,.. \par }
\par
\vspace{3pt}
This command allows to set font attributes for the active frame. 
The settings can be displayed by entering the command 'font' 
without parameters or using the command 'show font'. For a list 
of escape sequences and special characters see entry PGPLOT below. 
The available font properties $ <$prop$> $ are listed below: 
\par
\subsection*{PGPLOT}
{\bf Special characters \par }
\par
\vspace{3pt}
The use of the PGPLOT library allows KUPLOT users to use a variety 
of escape sequences and special characters in all text related 
functions of KUPLOT. 
\par
\begin{MacVerbatim}
\u  : start superscript, or end subscript
\d  : start sub, or end superscript (e.g. cm\u-2\d)
\b  : backspace (draw next character on top of current one)
\fn : switch to normal font (1)
\fr : switch to roman font  (2)
\fi : switch to italic font (3)
\fs : switch to script font (4)

\\  : backslash              \x  : Multiplication sign
\.  : centered dot           \A  : Angstroem

\gx      : Greek character 'x' (e.g. \ga gives alpha, \gb beta, ..)
\(nnnn)  : character (nnnn) from the Hershey character set
\end{MacVerbatim}
\subsection*{color}
{\bf font "color",$ <$id$> $,$ <$icol$> $ \par }
\par
\vspace{3pt}
This command allows to alter the color of the font used to label the 
graph. The default is black for all fonts. The different types of 
text given by their $ <$id$> $ parameter are listed below: 
\par
\begin{MacVerbatim}
  1 : main title line               2: subtitle line
  3 : axis labels                   4: numbers at axis
  5 : text in text frame            6: filename & caption
\end{MacVerbatim}
The allowed default color values $ <$icol$> $ are listed below. Note 
that they can be changed by the user via the command 'color'. 
Alternatively you can use 
the color name (without blanks i.e. darkblue etc.) 
\par
\begin{MacVerbatim}
  1 : red       5 : yellow        9 : dark blue      13 : cyan
  2 : green     6 : black        10 : dark magenta   14 : dark cyan
  3 : blue      7 : dark red     11 : dark yellow    15 : white (!)
  4 : magenta   8 : dark green   12 : gray
\end{MacVerbatim}
\subsection*{just}
font "just", $ \{$ "left" $| $ "center" $\} $ 
\par
This command sets the text justification for the two title lines and 
the text within a text frame. 
\subsection*{size}
{\bf font "size",$ <$id$> $,$ <$size$> $ \par }
{\bf font "size",$ <$scale$> $ \par }
\par
\vspace{3pt}
This command sets the font size for text type $ <$id$> $ to $ <$size$> $ points. 
If a specified font size is not available, a fixed font will be used 
instead. The allowed $ <$id$> $ parameters are: 
\par
\begin{MacVerbatim}
  1 : main title line               2: subtitle line
  3 : axis labels                   4: numbers at axis
  5 : text in text frame            6: filename & caption
\end{MacVerbatim}
Alternatively if only one parameter is given after "size", all fonts 
will be scaled by the given factor $ <$scale$> $, i.e. a value of 0.5 will 
set all font sizes to half their specified original size. 
\subsection*{typ}
{\bf font "typ",$ <$id$> $,$ <$typ$> $ \par }
\par
\vspace{3pt}
This command sets the font of the text typ $ <$id$> $ to $ <$typ$> $. Allowed 
values for the text typ parameter $ <$id$> $ are: 
\par
\begin{MacVerbatim}
  1 : main title line               2: subtitle line
  3 : axis labels                   4: numbers at axis
  5 : text in text frame            6: filename & caption
\end{MacVerbatim}
Currently implemented fonts are: 
\par
\begin{MacVerbatim}
  1 : Normal single stroke font (default)
  2 : Roman font
  3 : Italics font
  4 : Script font
\end{MacVerbatim}
\section{fset}
{\bf fset $ <$typ$> $ \par }
\par
\vspace{3pt}
This commands allows to alter the frame and axis plotted around 
the view graph for the active frame. The allowed values for $ <$typ$> $ 
are: 
\par
\begin{MacVerbatim}
  0 : no frame, axis and labels
  1 : only a box around the view graph
  2 : setting 1 & tick marks, numbers and labels
  3 : setting 2 & axis at x,y=0.0
\end{MacVerbatim}
If you specify a negative number for $ <$typ$> $ rather than the positive 
number listed above, the y-axis label and numbers will be plotted 
on the right hand side of the view graph rather than the left hand 
side. This allows to plot data sets with two different y-axis in 
one plot by plotting two frames (-$> $ frames) on top of each other. 
\par
Note, that the numbers and label at the axis can be switched off 
using the command 'achx/y' (-$> $ achx, achy). A dotted line parallel 
to x and y at the major tick marks can be plotted using the command 
'grid' (-$> $ grid). The variable axis[i,j] allows one to change 
various details of the axes and frame (-$> $ variables). Note that the 
values are given in character height units. 
\section{four}
{\bf four $ <$ik$> $ [,$ <$Qmin$> $, $ <$Qmax$> $, $ <$deltaQ$> $] [,$ <$code$> $] \par }
\par
\vspace{3pt}
This function allows the user to calculate the discrete Fourier 
Transform of data set $ <$ik$> $. The range of the Fourier transform is 
given by $ <$Qmin$> $ and $ <$Qmax$> $ and the step size is $ <$deltaQ$> $. If the 
parameters are omitted, the range in Fourier space is fixed 
from -1/2D .. 1/2D with D being the grid size in real space. Note, 
that the range can only be explicitly specified for 1D data sets. The 
optional parameter $ <$code$> $ determines how much output will be 
created. The parameter $ <$code$> $ is made up of any combination of the 
following four letters: 
\par
\begin{MacVerbatim}
  r   : save real (cos) part of Fourier Transform
  i   : save imaginary (sin) part of Fourier Transform
  a   : save amplitude of Fourier Transform
  p   : save phase of Fourier Transform
\end{MacVerbatim}
So $ <$code$> $ 'riap' would save all 4 parts in separate new data sets. 
The default is to store the real part. If the parameter $ <$code$> $ 
contains the letter 'H', the factor 2pi is used in the exponential. 
\section{frames}
{\bf Using frames in KUPLOT \par }
\par
\vspace{3pt}
The program KUPLOT allows to divide the plotting space into smaller 
areas called FRAMES. Each frame can be individually positioned on the 
plotting space. The user can define which loaded dataset should be 
displayed in which frame, alternatively, the contents of a text file 
can be printed in a frame. Each frame has its own set of parameters 
like title, axis labels, tick marks, ... . All commands related to 
frames are listed below: 
\par
\subsection*{commands}
{\bf List of commands: \par }
\par
\begin{MacVerbatim}
afra : sets the active frame for user input
bfra : sets background color for specified frame
cfra : copy frame parameters
fram : defines if a border is plotted around each frame
kfra : define contents of frames (data sets or text)
nfra : set number of frames (default = 1)
sfra : define position and size of frame
\end{MacVerbatim}
\subsection*{afra}
{\bf afra $ <$if$> $ \par }
\par
\vspace{3pt}
The active frame is set to $ <$if$> $. All parameters like title, axis labels, 
plotting window etc. are entered for the active frame. Is 'afra' entered 
without parameters, the actual active frame is displayed. The setting 
of one frame can be copied to another frame with the 'cfra' command. 
\subsection*{bfra}
{\bf bfra $ <$if$> $,$ <$r$> $,$ <$g$> $,$ <$b$> $ \par }
\par
\vspace{3pt}
This command sets the background color for frame $ <$if$> $. The color is 
given as RGB (red,green,blue) value ranging from 0.0 to 1.0. The 
default is a white background (1.0 1.0 1.0). 
\par
If the background color is changed frequently, the X-window server 
might be unable to allocate additional colors if bitmaps are used. 
To free the used colors reenter the command 'cmap' which will create 
a new color map. 
\subsection*{cfra}
{\bf cfra $ <$ifr$> $,$ <$i1$> $ [,$ <$i2$> $,$ <$i3$> $,...] \par }
\par
\vspace{3pt}
This command copies the settings of frame $ <$ifr$> $ (like title, contour 
line settings, plot window, ..) to frames $ <$i1$> $, $ <$i2$> $, ... 
\subsection*{fram}
{\bf fram [ $ \{$"on" $| $ "off"$\} $ ] \par }
\par
\vspace{3pt}
This command determines weather a black border is plotted around the 
graph / each individual frame. Called without parameters, no border 
will be plotted. 
\subsection*{kfra}
{\bf kfra $ <$if$> $,$ <$ik1$> $ [,$ <$ik2$> $,...] \par }
{\bf kfra $ <$if$> $,$ <$fname$> $ \par }
{\bf kfra $ <$if$> $,$ \{$"para" $| $ "fit"$\} $ \par }
\par
\vspace{3pt}
This command sets the contents for frame $ <$if$> $. A frame can either 
contain the graphs of dataset(s) $ <$ik1$> $,.. of the contents of a text 
file, specified by $ <$fname$> $. Special text frames are "para" which 
prints data information and "fit" giving the fit results. 
\subsection*{nfra}
{\bf nfra $ <$n$> $ \par }
\par
\vspace{3pt}
This command sets the total number of frames to be used. For most 
values of $ <$n$> $ KUPLOT defines a default layout. If no default layout 
is found, a warning is given. The layout can be changes using the 
command 'sfra'. 
\subsection*{sfra}
{\bf sfra $ <$if$> $,$ <$xmin$> $,$ <$ymin$> $,$ <$xmax$> $,$ <$ymax$> $ \par }
\par
\vspace{3pt}
This command allows to define the layout of the frames on the plot space. 
The lower left corner of the plot space is 0,0, the upper right is 1,1. 
The lower left corner $ <$xmin$> $,$ <$ymin$> $ and the upper right corner $ <$xmax$> $, 
$ <$ymax$> $ for each frame $ <$if$> $ can be specified with the 'sfra' command. 
\section{func}
{\bf func $ <$f(r[0])$> $,     $ <$iref$> $ \par }
{\bf func $ <$f(r[0])$> $,     $ <$xmin$> $,$ <$xmax$> $,$ <$dx$> $ \par }
{\bf func $ <$f(r[0],r[1])$> $,$ <$xmin$> $,$ <$xmax$> $,$ <$dx$> $,$ <$ymin$> $,ymax$> $,$ <$dy$> $ \par }
{\bf func "fit",$ <$iref$> $ \par }
{\bf func "fit",$ <$xmin$> $,$ <$xmax$> $,$ <$dx$> $ \par }
{\bf func "fit",$ <$xmin$> $,$ <$xmax$> $,$ <$dx$> $,$ <$ymin$> $,ymax$> $,$ <$dy$> $ \par }
\par
\vspace{3pt}
This command allows to create a new data set from the function and 
range given as command parameters. Every call of 'func' will create 
a new data set similar to 'load'. Use 'rese' to discard previous 
created data sets. 
\par
The first parameter is the function to be calculated. See -$> $ expressions 
for details on the function syntax. Use variable r[0] as x-variable and 
r[1] as y-variable in case of 3D data sets. Note that stored values 
in r[0] and r[1] will be destroyed ! The parameters $ <$xmin$> $ and $ <$xmax$> $ 
give the x-range for the function. The value of $ <$dx$> $ is the step size 
in x-direction. For 3D data sets the y-range is given as additional 
parameters in a similar way. Alternatively the x-values can be taken 
from the dataset $ <$iref$> $ (not implemented for 3D files). 
\par
if the first parameter is "fit", the value of the current fit function 
(see -$> $ fit) is used. 
\par
\begin{MacVerbatim}
Examples
func sin(r[0]),0.00,6.30,0.10             : sin(x) from 0->6.3, 64 pts
func sin(r[0])*cos(r[1]),0,6,0.1,0,6,0.1  : sin(x)*cos(y) 0->6  61x61 pts.
\end{MacVerbatim}
\section{glat}
{\bf glat $ <$ik$> $,$ <$ip$> $ [, $ \{$"x" $| $ "y"$\} $] \par }
\par
\vspace{3pt}
This command smoothes the data set number $ <$ik$> $ over $ <$ip$> $ points. 
The number $ <$ip$> $ should be a positive odd number. A normal sliding 
averaging without weighting is used (see also command -$> $ smooth). 
Three dimensional files are per default smoothed in x and y direction. 
The user can restrict the smoothing to only one direction by adding the 
optional parameter "x" or "y". 
\section{grid}
{\bf grid [ $ \{$"on" $| $ "off"$\} $ ] \par }
\par
\vspace{3pt}
This command determines whether a dotted line parallel to the axis 
should be plotted at all major tick mark positions. Called without 
parameters, the grid plotting is switched off. 
\section{hart}
{\bf hart $ <$ik$> $,$ <$p$> $ \par }
\par
\vspace{3pt}
This command controls the appearance of two dimensional plots. The 
data set is again specified by $ <$ik$> $, the allowed values for $ <$p$> $ are: 
\par
\begin{MacVerbatim}
  1 : contour map
  2 : bitmap
  3 : contour map & bitmap
\end{MacVerbatim}
The z-range of the values to be converted to the color range of the 
bitmap is determined by the setting of the first contour line range 
(see command hlin). The color map is defined by the 'cmap' command. 
\section{hcol}
{\bf hcol $ <$ik$> $,$ <$ic$> $,$ <$c$> $ \par }
\par
\vspace{3pt}
This command sets the color for contour line set $ <$ic$> $ for the data 
set $ <$ik$> $. The colors are numbered like the pens on a HP7475 plotter. 
Allowed values for $ <$c$> $ are listed in the table, Alternatively you 
can use the color name (without blanks i.e. darkblue etc.) 
\par
The allowed default color values $ <$icol$> $ are listed below. Note 
that they can be changed by the user via the command 'color'. 
\par
\begin{MacVerbatim}
  1 : red       5 : yellow        9 : dark blue      13 : cyan
  2 : green     6 : black        10 : dark magenta   14 : dark cyan
  3 : blue      7 : dark red     11 : dark yellow    15 : white (!)
  4 : magenta   8 : dark green   12 : gray
\end{MacVerbatim}
\section{hlab}
{\bf hlab $ <$ik$> $,$ <$int$> $ \par }
\par
\vspace{3pt}
This command controls optional labeling of the contour lines of 
data set $ <$ik$> $. If the parameter $ <$int$> $ is set to zero (default), no 
labels are drawn. Otherwise every $ <$int$> $ contour level starting at 
the base level is labeled. The font characteristics of the axes 
labels are used, only the font size is reduced by 75\%. 
\section{hlin}
{\bf hlin $ <$ic$> $,$ <$min$> $,$ <$inc$> $,$ <$n$> $ [, $ \{$ "abs" $| $ "\%" $\} $] \par }
\par
\vspace{3pt}
This command sets the contour line parameters. The first parameter 
$ <$ic$> $ specifies the contour line set, which is defined by the following 
numbers. The contour lines starting level is given by $ <$min$> $, the 
increment by $ <$inc$> $ and the number of contours to be plotted by $ <$n$> $. 
As default the given values are taken as absolute values. However, 
is a fifth parameter set to "\%" the values of $ <$min$> $ and $ <$inc$> $ are 
taken relative to the maximum/minimum value of loaded data set 1. 
\par
Is the command given without parameters, the current settings of the 
contour parameters are displayed on the screen. 
\section{hpak}
{\bf hpak $ <$ip$> $ \par }
\par
\vspace{3pt}
KUPLOT allows to define more than one set of contour lines (see 'hlin' 
command). The command 'hpak' allows to set the number of contour line 
sets $ <$ip$> $ to be used. 
\section{htyp}
{\bf htyp $ <$ik$> $,$ <$ic$> $,$ <$p$> $ \par }
\par
\vspace{3pt}
This command sets the line typ for contour line set $ <$ic$> $ for the data 
set $ <$ik$> $. The colors are numbered like the pens on a HP7475 plotter. 
Allowed values for $ <$p$> $ are (number or name): 
\par
\begin{MacVerbatim}
  0 : {"noline"|"none"} : no line
  1 :  "solid"          : solid line
  2 :  "broken"         : ----   ----   ----   ----
  3 :  "dashed"         : - ---- - ---- - ---- - --
  4 :  "shortbroken"    : - - - - - - - - - - - - -
  5 :  "shortdashed"    : - - - ---- - - - ---- - -
\end{MacVerbatim}
\section{ident}
{\bf ident [ $ \{$"on" $| $ "off"$\} $ ] \par }
\par
\vspace{3pt}
This command determines whether user name, date and time of the plot 
are written on the bottom of the view graph. 
\section{inte}
{\bf inte $ <$ik$> $ [,$ <$xmin$> $,$ <$xmax$> $] \par }
{\bf inte $ <$ik$> $ [,$ <$xmin$> $,$ <$xmax$> $,$ <$ymin$> $,$ <$ymax$> $] \par }
\par
\vspace{3pt}
This command calculates the integral of data set $ <$ik$> $. If the additional 
parameters are omitted, the complete actual plotting window is integrated. 
Alternatively, the integration limits $ <$xmin$> $,$ <$xmax$> $ (2d) or $ <$xmin$> $,$ <$xmax$> $, 
$ <$ymin$> $,$ <$ymax$> $ (3d) can be given after the data set number $ <$ik$> $. 
\par
The results of this command are stored in the res[i] variables in 
the following order: 
\par
\begin{MacVerbatim}
  res[0]   : Number of parameters given
  res[1]   : Integral
  res[2]   : Sigma of the integral
  res[3]   : # of points integrated
  res[4]   : Integration limit xmin
  res[5]   : Integration limit xmax
  res[6]   : Integration limit ymin (only for 3d data)
  res[7]   : Integration limit ymax (only for 3d data)
\end{MacVerbatim}
\section{ksav}
{\bf ksav $ <$ik$> $ \par }
\par
\vspace{3pt}
This command enters the 'ksav' level of KUPLOT which allows to save 
the given data set $ <$ik$> $ in different formats to a file. The allowed 
commands in the 'ksav' level are: 
\par
\subsection*{commands}
Here is a list of valid commands in the KSAV sub level: 
\par
\begin{MacVerbatim}
exit    : Leaves KSAV sub level without saving the file
form    : Sets output file format
help    : Displays on-line help (see general help)
m999    : Treatment of excluded data points
outfile : Specifies the output filename
run     : Saves file and leaves KSAV level
show    : Shows the current settings
thresh  : Sets threshold for PGM file output
wait    : wait for user input (see main help)
\end{MacVerbatim}
\subsection*{form}
{\bf form $ <$ff$> $ [,$ <$p1$> $,$ <$p2$> $,...] \par }
\par
\vspace{3pt}
This command determines the file format to be used when the file is 
saved. Some formats require additional parameters. If a parameter 
is not required, the default is the current plotting window. 
The following list shows the valid file formats: 
\par
\begin{MacVerbatim}
<ff>   <p1>,....           required   What
---------------------------------------------------------------------
 xy    xmin,xmax,[<style>]    no      Saves xy file from xmin,xmax
 ni    xmin,xmax,ymin,ymax    no      Nipl file, given area
 pg    xmin,xmax,ymin,ymax    no      PGM file (ASCII), given area
 gn    xmin,xmax,ymin,ymax    no      XYZ file (gnuplot), given area
 sx    y-value               yes      Crossection || x at y-value
 sy    x-value               yes      Crossection || y at x-value
 mx    i                     yes      Crossection || x trough max #i
 my    i                     yes      Crossection || y trough max #i
 sk    ik                    yes      Crossection along xy of data set ik
 sl    x1,y1,x2,y2,n         yes      Crossection from x1,y1 to x2,y2
                                      with n points
\end{MacVerbatim}
For the xy format, the last optional parameter can be the string 
"four" to explicitly write the default four columns x, y, dx, dy. 
Alternatively the last parameter can be "two" to write a two 
column file.  If the uncertainties for x and / or y are not zero, 
the values are lost! 
For either option, the parameters xmin, xmax are still optional, 
and the command might take the forms: 
\begin{MacVerbatim}
 xy    ,xmin,xmax, "four"
 xy    ,xmin,xmax, "two"
 xy    ,xmin,xmax
 xy    ,"four"
 xy    ,"two"
\end{MacVerbatim}
\subsection*{m999}
{\bf m999 "excl"$| $"data" \par }
\par
\vspace{3pt}
KUPLOT treats data points of value "-9999" as excluded data in a 
two dimensional data array. This command specifies the treatment of 
these data points while writing crossections $| $$| $ x or y. 
With "excl" these data points are omitted from the output file, which 
may therefore have a length that varies from the corresponding 
dimension of the two-dimensional data. 
With "data" these data points are treated as regular data points and 
written to the output file. The output file is of constant length 
corresponding to the respective two dimensional dimension. The 
output file contains the data points with value -9999. 
\subsection*{outfile}
{\bf outfile $ <$fname$> $ \par }
\par
\vspace{3pt}
This command specifies the name of the output file. See entry 'filenames' 
for more information about the building of filenames. 
\subsection*{run}
{\bf run \par }
\par
\vspace{3pt}
This command save the file with the current settings and leaves the 
KSAV level. 
\subsection*{show}
{\bf show \par }
\par
\vspace{3pt}
This command shows the current settings. 
\subsection*{threshold}
{\bf thresh $ \{$ "high"$| $"low"$| $"sigma"$| $"zmin"$| $"zmax"$\} $ $ <$value$> $ \par }
\par
\vspace{3pt}
This command sets the threshold that is used when writing BITMAP output. 
All values less than the minimum threshold are set to zero, 
all values higher than the maximum threshold are set to the 
maximum threshold. The values in between are linearly scaled from zero 
to 255. 
\par
Depending on the first parameter, the second parameter is interpreted in 
five different ways: 
\par
\begin{MacVerbatim}
"high"  : Sets maximum threshold for BITMAP in percent of the maximum z
"low"   : Sets minimum threshold for BITMAP in percent of the maximum z
"sigma" : Sets threshold to average z +- <value>*standard deviation z
"zmax"  : Sets maximum threshold for BITMAP
"zmin"  : Sets minimum threshold for BITMAP
\end{MacVerbatim}
\section{kcal}
{\bf kcal $ \{$"add"$| $"sub"$| $"mul"$| $"div"$\} $,$ <$ik1$> $,$ <$ik2$> $[,$ \{$"excl"$| $"neut"$| $"igno"$\} $][,"over"] \par }
\par
\vspace{3pt}
This command allows to multiply ("mul"), add ("add"), divide ("div") 
or subtract ("sub") the y- or z- values of the data sets $ <$ik1$> $ and $ <$ik2$> $. 
For a 2d data sets, the y- values are used, for a 3d data set the 
z-values.  The optional last parameter triggers where the resulting 
data set is stored. The default is to create a new data set. If the 
parameter "over" is specified, the resulting data set will overwrite 
data set $ <$ik1$> $. 
\par
The optional fourth parameter is effective only for 3d data sets. It 
tells KUPLOT how to treat the special z values of "-9999." The 
following parameters are valid: 
\par
\begin{MacVerbatim}
"excl" : These data points are treated as excluded regions and the
         result will be -9999 if one or both z-values of the data
         sets had been -9999. This is also the default used when no
         fourth parameter is given.
"neut" : The data points are treated as neutral element for the
         operation. If both input values are -9999, the result is -9999,
         if only one input value is -9999, the result is the z-value
         of the other curve.
"igno" : The special properties of these data points are ignored and
         the result is the corresponding operation between the two
         numbers.
\end{MacVerbatim}
\section{kpara}
{\bf kpara $ \{$$ <$par\_name$> $ $| $ $ <$par\_number$> $$\} $ \par }
{\bf kpara $ \{$ $ \{$"last"$| $ "-1"$\} $ $| $ $ <$generation\_no$> $ $\} $ , \par }
{\bf       $ \{$$ <$par\_name$> $ $| $ $ <$par\_number$> $$\} $ , $ \{$$ <$par\_name$> $ $| $ $ <$par\_number$> $$\} $ \par }
\par
\vspace{3pt}
This command serves to create a standardized plot of the parameters 
refined via DIFFEV. 
\par
The first form with a single parameter name or number will plot the 
development of this parameter versus the refinement generations. 
The parameter name may include "Rvalue" or 0 to plot the development 
of the R-value. 
The plot displays the upper and lower most parameter value in red, 
the average parameter in blue with error bars and the best parameter 
in black. 
\par
Examples 
kpara Rvalue    ! Show the R-value versus generations 
kpara P\_one     ! Same for a parameter named "P\_one" 
\section{kpar\_par}
{\bf kpar\_par $ \{$ $ \{$"last"$| $ "-1"$\} $ $| $ $ <$generation\_no$> $ $\} $ , \par }
{\bf       $ \{$$ <$par\_name$> $ $| $ $ <$par\_number$> $$\} $ , $ \{$$ <$par\_name$> $ $| $ $ <$par\_number$> $$\} $ \par }
\vspace{3pt}
This command serves to create a standardized plot of the parameters 
refined via DIFFEV. 
\par
The command serves to plot the distribution of two parameters 
in a given generation. If the first command parameter is either 
"last" or "-1" the last generation is used. This can be helpful to 
repeatedly plot the parameters during a refinement. 
The plot will display the distribution of the two parameters. 
The size of the markers indicates the R-value for each parameter 
pair, with the smallest marker signifying the smallest R-value. 
Either of the two parameter names/numbers may be "Rvalue" or "0" 
to show the correlation between parameter value and the R-value. 
\par
Examples 
kpar\_par last, P\_one, Rvalue  ! Show R-values versus Parameter value 
                           ! in the last generation 
kpar\_par last, P\_one, P\_two   ! Dislay (possible) correlation 
                           ! between these two parameters 
\section{lart}
{\bf lart $ <$ik$> $,$ <$p$> $ \par }
\par
\vspace{3pt}
This command sets the who the points of data set $ <$ik$> $ are to be 
connected. Allowed values for $ <$p$> $ are: 
\par
\begin{MacVerbatim}
   1 : points are connected by a line
   2 : histogram style plot
   3 : connection via cubic splines
\end{MacVerbatim}
\section{lcol}
{\bf lcol $ <$ik$> $,$ <$c$> $ \par }
\par
\vspace{3pt}
This command sets the line color for the data set $ <$ik$> $. If $ <$ik$> $ is 
0 the color set is taken for the grid lines (-$> $ grid "on" $| $ "off"). 
The colors are numbered like the pens on a HP7475 plotter. Allowed 
values for $ <$c$> $ are listed in the table, Alternatively you can use 
the color name (without blanks i.e. darkblue etc.) 
\par
The allowed default color values $ <$icol$> $ are listed below. Note 
that they can be changed by the user via the command 'color'. 
\par
\begin{MacVerbatim}
  1 : red       5 : yellow        9 : dark blue      13 : cyan
  2 : green     6 : black        10 : dark magenta   14 : dark cyan
  3 : blue      7 : dark red     11 : dark yellow    15 : white (!)
  4 : magenta   8 : dark green   12 : gray
\end{MacVerbatim}
\section{load}
{\bf load $ <$form$> $,$ <$fname$> $[,$ <$par$> $,$ <$par$> $,..] \par }
\par
\vspace{3pt}
Reads a data set from file $ <$fname$> $. The format of the input file 
is given by the value of $ <$form$> $. Some formats require additional 
parameters $ <$par$> $. If a history (PDFgetN) is prepended, KUPLOT 
extracts the title and stores some information in the res[n] 
array (see PDFFIT 'read data' for details). The parameter $ <$fname$> $ 
can also be a valid expression such as "*.asc" just as a directory 
command. 
\par
Here is a list of the available formats: 
\par
\subsection*{4d}
{\bf load 4d,$ <$file$> $,$ <$ix$> $,$ <$iy$> $,$ <$iz$> $,$ <$ival$> $ \par }
\par
\vspace{3pt}
This command allows to read a section from a NIPL type 4D files 
(x y z value). The first two parameters after the filename define 
which index is mapped to x and y in KUPLOT. The next parameter 
defined which index is used for cutting and $ <$ival$> $ specified the 
value at which the cut is performed. 
\subsection*{cr}
{\bf load cr,$ <$file$> $ \par }
\par
\vspace{3pt}
Reads a "crystal" file format which contains coordinates for 
the atoms and typ, color and size for the marker for the specific 
atom. Those code numbers for typ, color and size are the same used 
with the 'mtyp', 'mcol' and 'msiz' commands. The points are plotted 
in the xy plane, i.e. projected along z. 
\par
\begin{MacVerbatim}
  0.5 0.5 0.5   5 3 1.00
  1.0 1.0 1.0   5 6 0.75
\end{MacVerbatim}
The first atom at (0.5 0.5 0.5) is represented by a red (col=3) 
filled square (typ =5) of the size 1.00 whereas the second atom 
is a black (col=6) filled smaller square. The size is given relative 
to the size set with -$> $ 'msiz'. The first atom is plotted with the 
full size set with 'msiz' whereas the second atom is only 75\% of 
that size. 
\subsection*{csv}
load csv,$ <$file$> $ 
load csv,$ <$file$> $, skip:$ <$nskip$> $, 
        colx:$ <$icolx$> $, coly:$ <$icoly$> $, coldx:$ <$icoldx$> $, coldy:$ <$icoldy$> $, 
        separator:$ <$name$> $ 
\par
Reads a csv type file that has been extracted from PANAlytical 
powder data. 
\par
The second form is a generic attempt to read any CSV file. 
You can specify as optional parameters: 
skip:  Number of header lines to skip     Default : 25 
colx:  Column from which to take x-data   Default :  1 
coly:  Column from which to take y-data   Default :  2 
coldx: Column from which to take uncertainties for x-data Default : 0 
coldy: Column from which to take uncertainties for y-data Default : 0 
separator: Character that separates the columns, possible values are: 
           ";" or "semicolon"             Default  : "semicolon" 
           ":" or "colon" 
           "comma" 
           "tab" 
           "space" Will take as separator a string of blank(s) 
If the error columns are omitted or set to zero no error data are read 
\subsection*{de}
{\bf load de,$ <$file$> $,$ <$gridx$> $,$ <$gridy$> $ \par }
\par
\vspace{3pt}
This command works similar to the 'load zz,..' command but rather 
that reading a xyz-file, a histogram of a xy-file is calculated, 
i.e. z represents the number of points x,y which fall in a grid 
of the size $ <$gridx$> $ and $ <$gridy$> $. 
\subsection*{gs}
{\bf load gs,$ <$file$> $,$ <$ibank$> $ [,$ <$unit$> $] [,$ <$ifile$> $] [,"norm"] \par }
\par
\vspace{3pt}
This command allows to read GSAS files. AT THIS TIME THIS FEATURE 
IS ONLY TESTED FOR TIME-OF-FLIGHT POWDER DIFFRACTION DATA. The value 
$ <$ibank$> $ determines which bank to read. A range of banks can be 
specified like from$> $to and to real all banks use the keyword "all". 
The optional parameter $ <$ifile$> $ is the file name of the corresponding 
GSAS instrument parameter file, which is ONLY needed if certain unit 
conversion are desired. The desired unit is specified with the optional 
parameter $ <$unit$> $. Valid units are: 
\par
\begin{MacVerbatim}
   "T"    : Time-of-Flight
   "D"    : D spacing
   "L"    : Wavelength (lambda)
   "Q"    : Momentum transfer Q
   "X"    : Used to indicate X-ray data in 2THETA
\end{MacVerbatim}
Finally if the last optional parameter is set to "norm", the intensities 
will be normalized by the incident spectrum. A valid instrument parameter 
file is required in this case. 
\subsection*{ma}
{\bf load ma,$ <$file$> $ \par }
\par
\vspace{3pt}
This commands reads a "marker" file which is a normal xy-file, but 
the program will only use the x coordinate , y is set to zero. So 
tick marks at the x-axis can be drawn. 
\subsection*{mc}
{\bf load mc,$ <$file$> $,$ <$snumber$> $ \par }
\par
\vspace{3pt}
This commands reads spectrum number $ <$snumber$> $ from MCA files 
(e.g. those created at A2/CHESS). 
\subsection*{ni}
{\bf load ni,$ <$file$> $[,$ <$white$> $] \par }
\par
\vspace{3pt}
This file format is also an old "Garching format" used by an even 
older plot program. It contains gridded 3d data in the following 
format: 
\par
\begin{MacVerbatim}
line 1    : nx ny                    ! number of points in x and y
line 2    : xmin xmax ymin ymax      ! range in x and y
line 3-xx : z z z z z z z            ! z-values increasing in x and y
\end{MacVerbatim}
\subsection*{pg}
{\bf load pg,$ <$file$> $ \par }
\par
\vspace{3pt}
Here is finally a universal 3d data file format known by KUPLOT, 
PGM files (from netpbm library). The current KUPLOT version can 
only read ASCII PGM files which have the code P2 in the first 
line of the program. 
\subsection*{sc/st}
{\bf load sc,$ <$file$> $ [,$ <$iscan$> $] [,$ <$cx$> $,$ <$cy$> $] [,$ <$cN$> $] [,$ <$cdx$> $] [,$ <$cdy$> $] \par }
{\bf load st,$ <$file$> $ [,$ <$iscan$> $] [,$ <$cx$> $,$ <$cy$> $] [,$ <$cN$> $] [,$ <$cdx$> $] [,$ <$cdy$> $] \par }
{\bf load sc,$ <$file$> $,$ <$iscan$> $,"info" \par }
{\bf load smca,$ <$file$> $,[$ <$iscan$> $ [,$ <$iscan$> $, $ <$ipoint$> $ [,"kev",par1,par2,par3]]] \par }
\par
\vspace{3pt}
This command allows one to read SPEC scan files. The parameter 
$ <$file$> $ is as usual the filename. If no further parameters are 
given, KUPLOT lists the scans present in the file. The next 
parameter $ <$iscan$> $ is the number of the scan. Rather than an 
individual scan, one can specify a range of scans to be read 
via 'start $> $ end' in the $ <$iscan$> $ field. Alternatively one can use 
the string 'all' for $ <$iscan$> $ to extract all scans from the 
given file. 
\par
If the subcommand is 'sc' or 'st', KUPLOT will read a scan, 
if the subcommand is 'smca', KUPLOT will read a MCA data point 
embedded in the spec file. 
\par
With no further parameters again KUPLOT lists information about the 
given scan an descriptions of the columns in that scan. The next 
two parameters $ <$cx$> $ and $ <$cy$> $ specify the columns to be associated 
with x and y. These parameters can either be numbers (positive numbers 
are counted from the left, negative from the right) or the EXACT 
column title. The last optional parameters defines which column KUPLOT 
will use for normalization of the y-data. If the subcommand 'sc' is 
used, KUPLOT calculates the standard deviation 'dy' as SQRT(y). If 
'st' is entered, no errors are calculated.  Columns $ <$cdx$> $ and $ <$cdy$> $ 
correspond to the errors of $ <$cx$> $ and $ <$cy$> $ allowing the user to read 
sigma values from any column of the file. The fields have to be 
specified in the given order, for used field enter "0". 
The scan info that KUPLOT derives from the file is in part stored 
into the result variable ==$> $ 'res[]'. The entries are: 
0  number of values stored in res 
1  number of scans 
2  number of MCA points 
3...2+NSCANS number of data points in each scan 
2+1*NSCANS+1 ... 2+2*NSCANS  xmin for each scan 
2+2*NSCANS+1 ... 2+3*NSCANS  xmax for each scan 
\par
KUPLOT can also extract scan information for a given scan $ <$iscan$> $ 
from the SPEC file when the parameter "info" is specified. Note that 
no data are read. 
\par
\begin{MacVerbatim}
load smca,<file>,[<iscan> [,<iscan>, <ipoint> [,"kev",par1,par2,par3]]]
\end{MacVerbatim}
KUPLOT will extract an MCA from a SPEC file. 
$ <$iscan$> $ is the scan number. If the SPEC files does not contain any 
        scans but just individual MCA data points, set $ <$iscan$> $ to zero. 
$ <$ipoint$> $ is the number of the data point within scan $ <$iscan$> $. If there 
        are no scans in the file, KUPLOT will read the file until it 
        finds the individual MCA data number $ <$ipoint$> $. 
Without the "kev" key word, KUPLOT will use the channel number for the 
x-axis. With the keyword, the channel number is converted into keV 
according to kev = par1*channel**2 + par2*channel + par3 
\par
Examples: 
\par
\begin{MacVerbatim}
load sc,scan.dat,12              ! Shows info for scan 12
load sc,scan.dat,pmQ,Elastic,14  ! Reads Q vs. elastic counts norm.
                                   by data in column 14 (monitor)
load sc,scan.dat,1>3,1,2         ! Read scans 1 to 3 column 1 vs. 2
load sc,scan.dat,1,info          ! Get information for scan 1
\end{MacVerbatim}
\subsection*{xx}
{\bf load xx,$ <$file$> $ \par }
\par
\vspace{3pt}
This command reads just the first column of a data file and takes 
the point number as x-value. 
\subsection*{xy}
{\bf load xy,$ <$file$> $ [,$ <$cx$> $,$ <$cy$> $] [,$ <$cdx$> $,$ <$cdy$> $] [,$ <$header$> $] \par }
\par
\vspace{3pt}
Reads data from a multi column ASCII file. Blank lines and lines 
starting with \# are ignored. If no parameters after $ <$file$> $ are given, 
column 1 is assigned to x and column 2 to y. If three columns are 
present, column 3 is assigned to dy. In case 4 or more columns are 
present, column 3 is dx and column 4 is dy. Alternatively, the 
desired columns can be specified using the optional parameters 
$ <$cx$> $ and $ <$cy$> $ for the data columns x and y as well as parameters 
$ <$cdx$> $ and $ <$cdy$> $ for assigning the errors to columns x and y. The 
final optional parameter $ <$header$> $ allows one to specify the number 
of header lines to be skipped. 
\subsection*{zz}
{\bf load zz,$ <$file$> $,$ <$gridx$> $,$ <$gridy$> $ [,"mod"] \par }
{\bf load zz,$ <$file$> $,$ <$gridx$> $,$ <$gridy$> $ [,$ <$xmin$> $,$ <$xmax$> $,$ <$ymin$> $,$ <$ymax$> $] \par }
\par
\vspace{3pt}
This command reads a xyz-file, each xyz triplet on a single line. 
The data points are binned to a regular grid of the size $ <$gridx$> $ 
and $ <$gridy$> $. The number of points in each direction is calculated 
by the program. This command can also be used to rebin data on 
a regular grid to a broader one. 
Without the optional parameter "mod" the grid starts at the lowest 
x,y values found and continues in steps of $ <$gridx$> $ and $ <$gridy$> $. 
If the optional parameter "mod" is given, the grid is shifted 
to the next integer multiple of the interval $ <$gridx$> $, $ <$gridy$> $. 
Alternatively, the desired boundaries can be given as additional 
parameters. All points outside these limits will be ignored. 
\subsection*{mp}
{\bf load mp,$ <$file$> $,$ <$gridx$> $,$ <$gridy$> $ [,"mod"] \par }
{\bf load mp,$ <$file$> $,$ <$gridx$> $,$ <$gridy$> $ [,$ <$xmin$> $,$ <$xmax$> $,$ <$ymin$> $,$ <$ymax$> $] \par }
\par
\vspace{3pt}
This command reads a binary xyz-file, each xyz triplet is given on a 
single 16 byte record. The first 4 bytes of the first record are 
an integer number that gives the total number of records in the file 
including the first line. Each other record contains the xyz data 
as 4 byte real numbers. In case of data that form a plane in three 
dimensions, the data are: abscissa, ordinate, 'density', third axis. 
\par
The data points are binned to a regular grid of the size $ <$gridx$> $ 
and $ <$gridy$> $. The number of points in each direction is calculated 
by the program. This command can also be used to rebin data on 
a regular grid to a broader one. 
Without the optional parameter "mod" the grid starts at the lowest 
x,y values found and continues in steps of $ <$gridx$> $ and $ <$gridy$> $. 
If the optional parameter "mod" is given, the grid is shifted 
to the next integer multiple of the interval $ <$gridx$> $, $ <$gridy$> $. 
Alternatively, the desired boundaries can be given as additional 
parameters. All points outside these limits will be ignored. 
\section{sm}
{\bf load sm,$ <$file$> $ \par }
\par
\vspace{3pt}
Loads the SIMREF *.plt file. 
\subsection*{special}
{\bf load $ <$form$> $,$ <$file$> $ \par }
\par
\vspace{3pt}
KUPLOT can read certain special file formats produced by the program 
MAN2 and MAN1 used at the Institut fuer Kristallographie of the 
University Munich. Those formats are summed up in the following 
list: 
\par
\begin{MacVerbatim}
T1/T2   MAN2-Pfausdatei            2-Theta gegen Zaehlrohr 1/2
C1/C2   MAN2-Pfausdatei            Chi     gegen Zaehlrohr 1/2
P1/P2   MAN2-Pfausdatei            Phi     gegen Zaehlrohr 1/2
O1/O2   MAN2-Pfausdatei            Omega   gegen Zaehlrohr 1/2
H1/H2   MAN2-Pfausdatei            H gegen Zaehlrohr 1/2
K1/K2   MAN2-Pfausdatei            K gegen Zaehlrohr 1/2
K1/K2   MAN2-Pfausdatei            K gegen Zaehlrohr 1/2
L1/L2   MAN2-Pfausdatei            L gegen Zaehlrohr 1/2
 TE     MAN2-Pfausdatei            Messpunktnummer gegen Temperatur (822)
 DM     MAN2-Pfausdatei            Messpunktnummer gegen DMM-Wert
 ZE     MAN2-Pfausdatei            Messpunktnummer gegen Messzeit
Z1/Z2   Messdatei von PULx         Zaehlrohr 1 bzw. 2
\end{MacVerbatim}
Two other special formats are the output files of the FAST data 
collection software of the area detector used in Munich. The two 
formats are: 
\par
\begin{MacVerbatim}
 DA     MP-Datei (binaer)          Flaechenzaehlerdatei
 AS     MP-Datei (ascii)           Flaechenzaehlerdatei
\end{MacVerbatim}
\section{ltyp}
{\bf ltyp $ <$ik$> $,$ <$p$> $ \par }
\par
\vspace{3pt}
This command sets the line typ to be used for data set $ <$ik$> $. The 
allowed values for $ <$p$> $ are (number or name): 
\par
\begin{MacVerbatim}
  0 : {"noline"|"none"} : no line
  1 : {"solid"}         : solid line
  2 : {"broken"}        : ----   ----   ----   ----
  3 : {"dashed"}        : - ---- - ---- - ---- - --
  4 : {"shortbroken"}   : - - - - - - - - - - - - -
  5 : {"shortdashed"}   : - - - ---- - - - ---- - -
\end{MacVerbatim}
\section{lwid}
{\bf lwid $ <$ik$> $,$ <$width$> $ \par }
\par
\vspace{3pt}
This command sets the line width $ <$width$> $ in cm for the data set $ <$ik$> $. 
If $ <$ik$> $ is zero, the line width for the frame itself is set. The 
current settings can be displayed using the 'show style' command. 
\section{mark}
{\bf mark [$ <$inx$> $] [,$ <$iny$> $] \par }
\par
\vspace{3pt}
This command sets the interval and starting point for the tick marks 
at the x- and y-axis. If all parameters are omitted, KUPLOT will do 
an automatic setting. The two parameters $ <$inx$> $ and $ <$iny$> $ give 
the interval of the major tick marks on the x- and y-axis. If only 
$ <$inx$> $ is given, $ <$iny$> $ remains unchanged. 
\section{match}
{\bf match $ \{$"scal" $| $ "offset" $| $ "all"$\} $,$ <$io$> $,$ <$ic$> $ \par }
{\bf match "mix",$ <$io$> $,$ <$i1$> $,$ <$i2$> $ \par }
\par
\vspace{3pt}
This command calculates the scale factor and/or offset which gives 
the best match between of data set $ <$ic$> $ to $ <$io$> $. The first command 
determines whether only a scale ("scal") or offset ("offset") or 
both ("all") will be calculated. Data set $ <$ic$> $ is then modified 
using these values. This command is currently only available for 
2D data sets (e.g. xy files). NOTE: This command assumes that a 
given data point i in both data sets $ <$io$> $ and $ <$ic$> $ corresponds to 
the same x-value, i.e. the data sets must be equal with respect to 
their x-values ! 
\par
If the first parameter is set to "mix" then data set $ <$io$> $ is matched 
by f*$ <$i1$> $+(1-f)*$ <$i2$> $. The result is added as new data set. All data 
sets $ <$io$> $, $ <$i1$> $ and $ <$i2$> $ need to have identical x data points. 
\par
The scale factor and the offset are written into the result variable 
as entries 1 and 2. res[0] is set to 2 
\section{mcol}
{\bf mcol $ <$ik$> $,$ <$c$> $ \par }
\par
\vspace{3pt}
This command sets the marker color for the data set $ <$ik$> $. The colors 
are numbered like the pens on a HP7475 plotter. Allowed values for 
$ <$c$> $ are: listed in the table, Alternatively you can use 
the color name (without blanks i.e. darkblue etc.) 
\par
The allowed default color values $ <$icol$> $ are listed below. Note 
that they can be changed by the user via the command 'color'. 
\par
\begin{MacVerbatim}
  1 : red       5 : yellow        9 : dark blue      13 : cyan
  2 : green     6 : black        10 : dark magenta   14 : dark cyan
  3 : blue      7 : dark red     11 : dark yellow    15 : white (!)
  4 : magenta   8 : dark green   12 : gray
\end{MacVerbatim}
\section{mean}
{\bf mean $ <$ik$> $ [,$ <$xmin$> $,$ <$xmax$> $] \par }
{\bf mean $ <$ik$> $ [,$ <$xmin$> $,$ <$xmax$> $,$ <$ymin$> $,$ <$ymax$> $] \par }
\par
\vspace{3pt}
This command allows to calculate the average and standard deviation for data 
set $ <$ik$> $. The area used for the calculation can be given by $ <$xmin$> $,$ <$xmax$> $ for 
2d data or $ <$xmin$> $,$ <$xmax$> $,$ <$ymin$> $,$ <$ymax$> $ for 3d data. If these parameters are 
omitted, the complete actual plotting range is used for the calculation. 
\par
The results are stored in the res[i] variables in the following order: 
\par
\begin{MacVerbatim}
   3d        2d
  --------------------------------------------------------
  res[0]   res[0]   = Number of stored values
  res[1]   res[1]   = average of x-values
  res[2]   res[2]   = average of y-values
  res[3]            = average of z-values
  res[4]   res[3]   = standard deviation of x-values
  res[5]   res[4]   = standard deviation of y-values
  res[6]            = standard deviation of z-values
  res[7]   res[5]   = # points used for calculation
  res[8]   res[6]   = used range : xmin
  res[9]   res[7]   = used range : xmax
  res[10]           = used range : ymin
  res[11]           = used range : ymax
\end{MacVerbatim}
\section{merge}
{\bf merge $ <$i1$> $,$ <$i2$> $,... [,"add"] \par }
{\bf merge $ \{$"-1"$| $"all"$\} $  [,"add"] \par }
\par
\vspace{3pt}
This command allows one to merge different data sets given by 
the parameters $ <$i1$> $, $ <$i2$> $ and so on. 
If the first parameter is equal to "-1" or "all", all currently 
loaded data sets are merged. 
\par
If as last parameter "add" 
is given, the values are simply added, otherwise they are 
normalized by the number of contributing data points. The new 
data set is assigned the next free data set number. 
\par
For 1D data sets, the extend of the new data set is defined by 
the maximum x-values of the data sets to be merged. The grid 
size is taken as the (xmax-xmin)/(np-1), i.e. the average step 
size of the first specified data set. 
\par
For 2D dat sets the grid is equivalently determined by the 
average grid size of the first data set. 
\section{mouse}
{\bf mouse \par }
{\bf mouse $ \{$"point" $| $ "line" $| $ "rect" $| $ "xrange" $| $ "yrange" $\} $ \par }
\par
\vspace{3pt}
When the command is given without an additional parameter, KUPLOT 
redraws the plot with a button area and enters the interactive 
'mouse mode'. Use the 'EXIT' button to return to the normal command 
mode. The command 'mouse point' allows the user to select a point 
within the active frame (-$> $ afra) and returns the coordinates 
in the variable res[]: 
\par
\begin{MacVerbatim}
   3d        2d
  ------------------------------------------------------------
  res[0]     res[0]       = Number of values stored
  res[1]     res[1]       = x-coordinate of point selected
  res[2]     res[2]       = y-coordinate of point selected
  res[3]                  = z-coordinate of nearest grid pt.
  res[4]     res[3]       = Mouse button clicked
\end{MacVerbatim}
                            (1=left, 2=middle, 3=right) 
\par
The commands 'mouse line' allows one to select two points 
that are connected with a line. Select the first point using 
the left mouse button. The command 'mouse rect' selects 
two points as well, however they make up the corners of a 
rectangle. Commands 'mouse xrange' and 'mouse yrange' mark the 
two points by horizontal/vertical lines. The resulting coordinates 
are stored in res[] again: 
\par
\begin{MacVerbatim}
  res[0]     = Number of values stored
  res[1]     = x-value of first point
  res[2]     = y-value of first point
  res[3]     = x-value of second point
  res[4]     = y-value of second point
  res[5]     = Mouse button clicked when selecting 2nd point
\end{MacVerbatim}
               or -1 if action was aborted. 
\section{mtyp}
{\bf mtyp $ <$ik$> $,$ <$p$> $ \par }
\par
\vspace{3pt}
This command sets the marker typ to be used for data set $ <$ik$> $. The 
allowed standard values for $ <$p$> $ are listed below. Additionally one 
can access all PGPLOT markers by adding 100 to the PGPLOT marker 
number (see users guide). A value of 118 for example will give the 
PGPLOT symbol 18, a filled star. 
\par
Allowed values for $ <$p$> $ are listed in the table. Alternatively 
you can use the name (without blanks) 
\par
\begin{MacVerbatim}
  0 : no marker|none  5 : filled square    10 : slash (/)
  1 : dot             6 : empty triangle   11 : backslash (\)
  2 : circle empty    7 : times (x)        12 : minus (-)
  3 : filled circle   8 : plus  (+)        13 : vertical line
  4 : square empty    9 : bar  (|)              from y-axis
 -1 : x_coordinate   -2 : y_coordinate     -3 : "xy": both
\end{MacVerbatim}
\section{msiz}
{\bf msiz $ <$ik$> $,[$ <$size$> $ [, $ \{$"x" $| $ "y" $| $ "0" $| $ $ <$data\_set$> $$\} $]] \par }
\par
\vspace{3pt}
This command sets the marker size for data set $ <$ik$> $ to $ <$size$> $. 
The marker size is given in cm (for our friends in the US: 
1 inch = 2.54 cm). 
\par
If the third parameter is given and not equal to "0", the marker 
are scaled proportionally. 
x  The marker size is proportional to the value along the x-axis 
y  The marker size is proportional to the value along the y-axis 
$ <$data\_set$> $ The marker size is proportional to the y-value of 
data set $ <$data\_set$> $. 
\section{nexus}
\par
Here is a list of the NeXus file related commands. You have to 
enable NeXus support when compiling KUPLOT. The commands are 
\par
\subsection*{nxopen}
{\bf nxopen $ <$file$> $ \par }
\par
\vspace{3pt}
This command opens the NeXus file called $ <$file$> $. The file stays 
open until the command 'nxclose' is issued. 
\subsection*{nxclose}
{\bf nxclose \par }
\par
\vspace{3pt}
This command closes the current NeXus file. 
\subsection*{nxdir}
{\bf nxdir \par }
\par
\vspace{3pt}
This command lists all NXdata entries in the currently open NeXus 
file. Note currently it will only list NXdata entries directly 
below the top NXentry level. 
\subsection*{nxload}
{\bf nxload $ <$sdsname$> $ \par }
{\bf nxload $ <$sdsname$> $,$ <$i$> $,$ <$j$> $ \par }
{\bf nxload $ <$sdsname$> $,$ <$cx$> $,$ <$cy$> $ \par }
{\bf nxload $ <$sdsname$> $,$ <$xmin$> $,$ <$xmax$> $,$ <$ymin$> $,$ <$ymax$> $ \par }
{\bf nxload $ <$sdsname$> $,$ <$i$> $,$ <$j$> $,$ <$k$> $,$ <$cx$> $,$ <$cy$> $ \par }
{\bf nxload $ <$sdsname$> $,$ <$i$> $,$ <$j$> $,$ <$k$> $,$ <$xmin$> $,$ <$xmax$> $,$ <$ymin$> $,$ <$ymax$> $ \par }
\par
\vspace{3pt}
This command reads the content of an SDS (data set) of the NeXus 
file into a KUPLOT data set. The first parameter is always the 
name of the data block $ <$sdsname$> $. To list all data blocks, use the 
command 'nxdir'. The other parameters allow one to extract a 
section or rebin the data. In general, parameters $ <$i$> $, $ <$j$> $ and 
$ <$k$> $ allow he selection of a cut or slice from the data. If one of 
these variables is '*', it represents the direction to be extracted. 
The parameters $ <$cx$> $ and $ <$cy$> $ allow one to compress the data, i.e. 
a value of 2 would combine any neighboring data points while 
reading. Finally, $ <$xmin$> $,$ <$xmax$> $,$ <$ymin$> $ and $ <$ymax$> $ allow one to 
extract only the specified subsection. 
\section{orient}
{\bf orient [$ \{$ "landscape" $| $ "portrait" $\} $] \par }
\par
\vspace{3pt}
This command sets the orientation of the resulting plot. The default 
is landscape orientation, which is set, if the command is given without 
parameters. 
\section{plot}
{\bf plot \par }
\par
\vspace{3pt}
This command is used to plot the graph on the screen. 
\section{prin}
{\bf prin [$ <$dev$> $], [$ <$command$> $] \par }
\par
\vspace{3pt}
This command allows to print the current view graph. The output 
device type $ <$dev$> $ can be given as first parameter. The only 
supported printing device is currently 'ps' for POSTSCRIPT. The 
file is then automatically send to the default printer. As an 
alternative the print command to be used can be specified as second 
parameter. Note the command has to be in ". The print device and 
the default print command can be altered in the file 'blk\_appl.f' 
of the KUPLOT source code. 
\par
\begin{MacVerbatim}
Examples:
prin                  : Print to default device/printer
prin ps,"lpr -drscgf" : Print Postscript using command lpr -drscgf ..
\end{MacVerbatim}
\section{ptyp}
{\bf ptyp $ <$ik$> $,$ <$p$> $ \par }
\par
\vspace{3pt}
This command sets the marker typ to be used for the found maxima of 
data set $ <$ik$> $. The allowed values for $ <$p$> $ are listed below. 
Additionally one can access all PGPLOT markers by adding 100 to the 
PGPLOT marker number (see users guide). A value of 118 for example 
will give the PGPLOT symbol 18, a filled star. 
\par
Allowed values for $ <$p$> $ are listed in the table. Alternatively 
you can use the name (without blanks) 
\par
\begin{MacVerbatim}
  0 : no marker|none  5 : filled square    10 : slash (/)
  1 : dot             6 : empty triangle   11 : backslash (\)
  2 : empty circle    7 : times (x)        12 : minus (-)
  3 : filled circle   8 : pluss (+)        13 : vertical line
  4 : empty square    9 : bar  (|)              from y-axis
 -1 : x-coordinate   -2 : y-coordinate     -3 : both
\end{MacVerbatim}
\section{rdef}
{\bf rdef [$ <$fname$> $] \par }
\par
\vspace{3pt}
This command reads the KUPLOT setting from a defaults file. If 
no filename $ <$fname$> $ is give, the setting are read from 'kuplot.def'. 
\section{rebin}
{\bf rebin $ <$ik$> $,$ <$delta$> $ \par }
\par
\vspace{3pt}
This command allows one to rebin a 2D data set $ <$ik$> $ to a new 
equi-distant grid with a spacing of $ <$delta$> $. The command creates 
a new data set with the rebinned data. Note that this command will 
only work for 2D data sets. 3D data can be rebinned by saving them 
in GN format and reading them in via 'load zz,..' with the desired 
grid size. 
\section{rese}
{\bf rese ["all"] \par }
\par
\vspace{3pt}
This command resets KUPLOT. The next data set will be loaded as set 
number 1. Values for aspect ratio, distance of tick marks etc. are 
also reset to automatic determination. The optional parameter "all" 
will reset KUPLOT back to the initial state including titles, colors, 
labels and so on. 
\section{rval}
{\bf rval $ <$ik$> $,$ <$ij$> $[,$ <$weight$> $]  [, partial:$ <$no$> $] \par }
\par
\vspace{3pt}
Calculates the residual between curves $ <$ik$> $ and $ <$ij$> $. 
The last parameter defines the weighting scheme to be used for 
the calculation. The valid values for $ <$weight$> $ are: 
\par
\begin{MacVerbatim}
  "one"  : w(y) = 1.0              "sqa"  : w(y) = y**2
  "sqrt" : w(y) = SQRT(y)          "inv"  : w(y) = 1/y
  "log"  : w(y) = LOG(y)           "isq"  : w(y) = 1/SQRT(y)
  "lin"  : w(y) = y                "dat"  : w(y) = 1/DY (error data set)
\end{MacVerbatim}
The "partial" parameter tells KUPLOT that this is one out of 
several R-values that will be send back to DIFFEV. If these 
partial R-values are to be averaged evenly, it is sufficient to 
provide all partial R-values via the 'rval' command. 
If you want to weigh these R-values differently, use the 
==$> $ 'costvalue' command. 
\par
The R-values are also copied into the result variable as entries 
one and two (unwieghted and weighted). 
\section{sann}
{\bf sann $ <$ia$> $,"OFF" \par }
{\bf sann $ <$ia$> $,$ <$text$> $,$ <$x$> $,$ <$y$> $ [,$ \{$"l"$| $"c"$| $"r"$\} $] [,$ <$angle$> $] [,$ <$xp$> $,$ <$yp$> $] \par }
\par
\vspace{3pt}
This command allows to add annotations to a plot. The first 
parameter $ <$ia$> $ is the annotation number. The text is given 
as second parameter $ <$text$> $. It is possible to use formats 
(-$> $ filenames) to alter the text using variables. The next 
two parameters $ <$x$> $ and $ <$y$> $ give the position where the text 
should appear in plot coordinates. You can use 'mouse' to determine 
the coordinates of a particular point. The default justification 
of the text is LEFT, i.e. $ <$x$> $ and $ <$y$> $ specify the lower left 
corner of the text. Other justifications can be given as 
optional parameter. Next the angle between the text and the 
x-axis $ <$angle$> $ can be given as optional last parameter. 
The last two optional parameters $ <$xp$> $ and $ <$yp$> $ give the coordinates 
to the head of an arrow drawn from the text position to this 
point. If omitted, no arrow is drawn. 
\par
Font type, size and color is similar to the settings for 
filenames and caption (-$> $ font). Call 'sann' without parameters 
or use 'show annotation' to get a list of the currently set 
annotations. 
\section{save}
{\bf save $ <$dev$> $, [$ <$filename$> $] \par }
\par
\vspace{3pt}
$ <$dev$> $ : "ps", "png", "pic", "pdf", "la" 
This command allows to save the current view graph. The output 
device type $ <$dev$> $ can be given as first parameter. The 
choices for $ <$dev$> $ are: 
"ps" for POSTSCRIPT 
"pic" or "png" for a PNG 
"pdf" for a PDF file format 
"latex" Latex picture environment 
\par
If "pdf" is chosen, a postscript file and a pdf file will be generated. 
\par
The hardcopy and graphics file format 
can be changed in the file 'blk\_appl.f90' of the KUPLOT source code 
to any device that is supported by PGPLOT. 
\section{sdef}
{\bf sdef [$ <$fname$> $] \par }
\par
\vspace{3pt}
This command writes the current setting of KUPLOT in a file. If 
no filename $ <$fname$> $ is give, the setting are written to 'kuplot.def'. 
\section{show}
{\bf show $ <$sub$> $ [,$ <$par$> $] \par }
\par
\vspace{3pt}
This command displays various parameters and settings of KUPLOT. 
The current subcommands are: 
\par
\subsection*{annotation}
{\bf show "annotation" \par }
\par
\vspace{3pt}
Shows the currently set annotation for the active frame. 
\subsection*{color}
{\bf show "color" \par }
\par
\vspace{3pt}
This commands shows the list of current color settings for the 
background and pens. These settings might be altered using the 
command 'col'. 
\subsection*{cmap}
{\bf show "cmap" \par }
\par
\vspace{3pt}
This command shows the RGB values of the current color map used 
to print bitmaps (only POSTSCRIPT output and X11). 
\subsection*{config}
{\bf show config \par }
\par
\vspace{3pt}
This command lists configuration information on the screen such 
as the maximum number of data sets that can be read or the maximum 
allowed number of frames. 
\subsection*{data}
{\bf show "data" [,$ <$ik$> $] \par }
\par
\vspace{3pt}
Shows information about the loaded data set $ <$ik$> $. If the 
parameter $ <$ik$> $ is omitted, information about all loaded 
data sets is printed. 
\subsection*{error}
{\bf show "error" \par }
\par
\vspace{3pt}
The error status of KUPLOT is displayed. 
\subsection*{font}
{\bf show "font" \par }
\par
\vspace{3pt}
This command shows the font settings for the active frame. 
\subsection*{frames}
{\bf show "frames" \par }
\par
\vspace{3pt}
This command shows the actual frame settings. For more information 
who to use frames in KUPLOT see help entry 'frames'. 
\subsection*{hlin}
{\bf show "hlin" \par }
\par
\vspace{3pt}
Shows the current contour line settings for the active frame. 
\subsection*{plot}
{\bf show "plot" \par }
\par
\vspace{3pt}
Shows plot information for the active frame. A value of -9999. means 
automatic setting of the corresponding parameter before the next plot. 
\subsection*{result}
{\bf show "result" \par }
\par
\vspace{3pt}
This command shows the values currently stored in the res[i] 
variable array. 
\subsection*{style}
{\bf show "style" \par }
\par
\vspace{3pt}
This command shows plotting style setting like line color, line 
style etc. are displayed on the screen. 
\section{sleg}
{\bf sleg $ <$ik$> $, ["off", $ <$text$> $] [,$ <$ox$> $, $ <$oy$> $] \par }
\par
\vspace{3pt}
This command sets a possible caption for data set $ <$ik$> $. The text 
with a sample line, marker, .. will appear in the upper left corner 
of the plot. The option can be turned off via 'sleg $ <$ik$> $,off'. The 
caption text can also be a format string followed by the necessary 
parameters (see filenames entry !). The optional parameters $ <$ox$> $ 
and $ <$oy$> $ allow to specify the UPPER RIGHT corner of the box containing 
the captions in data coordinates. If the parameters are omitted, 
the upper right corner of the plot is used by default. 
\section{smax}
{\bf smax $ <$ik$> $,$ <$ifen$> $ [,$ <$fname$> $] \par }
\par
\vspace{3pt}
This command searches for maxima within data set $ <$ik$> $. 
The parameter $ <$ifen$> $ determines how many points (in each 
direction) around a possible maximum must have a lower value 
as the maximum itself. For smooth data use larger values, 
for rough data smaller values. If the parameter $ <$fname$> $ is given, 
the results are saved to the file $ <$fname$> $ rather than printed 
in the screen. 
\par
The results of the search are stored in the variable array res[i]: 
\par
\begin{MacVerbatim}
   res[0]              : Number of stored values.
   res[i]              : Number of x point for i-th maximum (2d data)
   res[2i-1],res[2i]   : Number of x,y point for i-th maximum (3d data)
\end{MacVerbatim}
The actual values can be determined by e.g. x[ik,res[5]]. For more 
information, see variables entry. 
\section{smooth}
{\bf smooth $ <$ik$> $,$ <$ip$> $ [,$ <$o$> $] [, $ \{$"x" $| $ "y"$\} $] \par }
\par
\vspace{3pt}
This command smoothes the data set number $ <$ik$> $ over $ <$ip$> $ points. 
The number $ <$ip$> $ should be a positive odd number. KUPLOT uses the 
Savitzky-Golay algorithm. The order of the smoothing polynomial defaults 
to two. Alternatively the user can give a higher order (4,6) via 
the optional parameter $ <$o$> $. Within limits this algorithm allows 
smoothing without resolution (see section 14.8. in Numerical Recipes). 
Three dimensional files are per default smoothed in x and y direction. 
The user can restrict the smoothing to only one direction by adding the 
optional parameter "x" or "y". An normal gliding average smoothing 
is available via the command -$> $ glat. 
\section{sort}
{\bf sort $ <$ik$> $ \par }
\par
\vspace{3pt}
This command sorts the data set $ <$ik$> $ increasing in its x-value. 
\section{spline}
{\bf spline $ <$ik$> $,$ <$ig$> $ \par }
\par
\vspace{3pt}
This command will interpolate data set $ <$ik$> $ for x-values given 
by data set $ <$ig$> $. The result is stored as a new data set. The 
interpolation is done using a spline. 
\section{skal}
{\bf skal [$ <$xmin$> $, $ <$xmax$> $] [,$ <$ymin$> $, $ <$ymax$> $] \par }
\par
\vspace{3pt}
This command sets the plotting window to $ <$xmin$> $ to $ <$xmax$> $ and $ <$ymin$> $ 
to $ <$ymax$> $. If the data exceed the given window, they will be clipped. 
If no parameters are given, the plotting window will be set to the 
maximum / minimum values of the read data sets. If only the parameter 
pair $ <$xmin$> $ and $ <$xmax$> $ are given, the plotted y-range remains 
at the current setting. 
\section{tit1/2}
{\bf tit1 $ <$string$> $ \par }
{\bf tit2 $ <$string$> $ \par }
\par
\vspace{3pt}
This command sets the text for the two title lines. Because KUPLOT 
treats the ',' as a parameter separator it can not be used within 
the label $ <$string$> $. The command allows to build a label from a 
format text and variables/numbers. It is the same mechanism as 
described in -$> $ filenames. 
\par
\begin{MacVerbatim}
Examples:
tit1 Powder data ZrO2                   : 'Powder data ZrO2'
tit1 "Run #%d / Sample #%d",i[1],i[2]   : 'Run #1 / Sample #5'
                                          (for i[1]=1 & i[2]=5)
\end{MacVerbatim}
\section{window}
{\bf window $ <$id$> $ [,$ <$sf$> $] \par }
\par
\vspace{3pt}
This command allows the user to select a graphics output window. 
The first window has $ <$id$> $ of 1, the second window 2 and so on. 
The window command allows to create and toggle between graphics 
windows (provided you use the default $\backslash$XSERV device). Note that 
print and save command will use the content of the currently 
active window. Optionally the windows can be scaled using the 
additional scale factor parameter $ <$sf$> $ (default = 0.7). 
\section{variables}
\par
The program KUPLOT recognizes various variables. The contents of a 
variable can be displayed using the 'eval' command. Some variables 
are READ ONLY (RO) and can not be changed. 
\par
\begin{MacVerbatim}
Free variables :
\end{MacVerbatim}
---------------- 
\par
\begin{MacVerbatim}
"i[<n>]"           : integer variables (currently 0 <= <n> <= 59)
"r[<n>]"           : real variables    (currently 0 <= <n> <= 59)
KUPLOT specific variables :
---------------------------
"x[<ik>,<ip>]"     : x value of point <ip> in loaded data set <ik>.
"y[<ik>,<ip>]"     : y value of point <ip> in loaded data set <ik>.
"dx[<ik>,<ip>]"    : value of sigma x of <ip> in loaded data set <ik>.
"dy[<ik>,<ip>]"    : value of sigma y of <ip> in loaded data set <ik>.
"z[<ik>,<ix>,<iy>" : z value of point <ix>,<iy> of loaded data set <ik>.
"nx[<ik>]"         : # points in x-direction of data set <ik>. (RO)
"ny[<ik>]"         : # points in y-direction of data set <ik>. (RO)
"ni[<ik>]"         : data set <ik> is 2D (=0) or 3D (=1).      (RO)
"np[<ik>]"         : # points of 1D data set <ik>,             (RO)
\end{MacVerbatim}
                   : \# or nx[$ <$ik$> $*ny[$ <$ik$> $] for 2D set $ <$ik$> $.    (RO) 
\par
\begin{MacVerbatim}
"xmin[<ik>]"       : minimum x-value of data set <ik>          (RO)
"xmax[<ik>]"       : maximum x-value of data set <ik>          (RO)
"ymin[<ik>]"       : minimum y-value of data set <ik>          (RO)
"ymax[<ik>]"       : maximum y-value of data set <ik>          (RO)
"zmin[<ik>]"       : minimum z-value of data set <ik>          (RO)
"zmax[<ik>]"       : maximum z-value of data set <ik>          (RO)
"n[1]"             : number of loaded data sets
"n[2]"             : maximum number of data sets               (RO)
"n[3]"             : maximum number of frames                  (RO)
"n[4]"             : maximum number of annotations             (RO)
"n[5]"             : maximum number of bonds                   (RO)
"p[i]"             : Fit parameters
"s[i]"             : Fit parameter sigmas
"axis[1,i]"        : Angle of labels for axis i (1=x,2=y)
"axis[2,i]"        : Length of major ticks for axis i (1=x,2=y)
"axis[3,i]"        : Length of minor ticks for axis i (1=x,2=y)
"axis[4,i]"        : Subdivisions between ticks, axis i (1=x,2=y)
"axis[5,i]"        : Distance numbers - axis i (1=x,2=y)
"axis[6,i]"        : Distance label - axis i (1=x,2=y)
"pwin[i]"          : Current plotting dimensions
                     (1:xmin,2:xmax,3:ymin,4:ymax)             (RO)
"cmap[<ic>,3]"     : color map entry RGB for color <ic>
"cmax[1]"          : maximum number of bitmap colors          (RO)
"size[i]"          : scale factor for output (1=x11,2=ps,3=pic)
Variables containing command results :
--------------------------------------
"res[<n>]"          : Result of several calculations that return more
                      than one value.  res[0] contains the number of
                      values returned, res[1] to res[res[0]] contain the
                      actual values.  The values of "res" change every time,
                      a command is applied that returns more than one
                      value. The old values are lost !
\end{MacVerbatim}
\section{errors}
{\bf Error messages \par }
\par
\vspace{3pt}
The program has been written such that it should handle almost 
any typing error when giving commands and hopefully all errors 
that result from calculation with erroneous data. When an error 
is found an error message is displayed that should get you back 
on track. Command language specific errors can be found in the 
'command language' section of the on-line help. 
\par
The error messages are grouped in the following categories: 
\par
\begin{MacVerbatim}
KUPL   Errors specific to KUPLOT usage
\end{MacVerbatim}
Each error message is displayed together with the corresponding 
category $ <$cccc$> $ and the error number $ <$numb$> $ in the form: 
\par
{\bf   ****CCCC**** Message                    **** numb **** \par }
\par
\vspace{3pt}
More information to a particular error message can be found by 
typing 'help $ <$cccc$> $ $ <$numb$> $' where $ <$cccc$> $ is the category and 
$ <$numb$> $ the error number. 
\par
\subsection*{appl}
{\bf KUPLOT errors \par }
\par
\vspace{3pt}
These messages describe application specific errors (here KUPLOT) 
in contrast to errors concerning the command language handling. 
\par
\subsubsection{Error -1: Maximum number of data sets exceeded}
\par
There is no space left for a new data set. The maximum number of 
data sets is defined by MAXKURVTOT in 'kuplot.inc'. In order to 
change this setting you have to recompile the program. 
\subsubsection{Error -2: Unknown file format}
\par
The given file format with the 'load' command is unknown. Check 
the on-line help for 'load' for a list of valid file formats. 
\subsubsection{Error -3: Maximum number of data points exceeded}
\par
There is not enough space for the requested data set left. The 
maximum number of data points for ALL data sets is defined by 
MAXARRAY in 'kuplot.inc'. 
\subsubsection{Error -4: Invalid data set selected}
\par
The given data set number is not valid. Try 'show data' for a 
list of loaded data sets. 
\subsubsection{Error -5: Not an ASCII PGM file}
\par
The PGM file to be read is not ASCII (Code P2). Currently KUPLOT 
can only read ASCII PGM files. If you have the 'pbmplus' package 
installed, use the tool 'pnmnoraw' to convert from binary to ASCII 
format. 
\subsubsection{Error -6: Number of data points exceeds limit}
\par
There is not enough space for the requested data set left. The 
maximum number of data points for ALL data sets is defined by 
MAXARRAY in 'kuplot.inc'. 
\subsubsection{Error -7: Invalid parameter value entered}
\par
The given parameter value like color, line typ or font id is invalid. 
Check the on-line help for the corresponding command for valid 
parameters. 
\subsubsection{Error -8: Use only if NO data set is loaded}
\par
The command can only be used, if NO data sets are loaded. Use the 
command 'rese' and reload the data sets after executing the command. 
\subsubsection{Error -9: Too many excluded regions}
\par
The number of excluded regions exceeds the upper limit. The number 
is defined in 'load.f' and the program must be recompiled after 
changing the value. 
\subsubsection{Error -10: Too many major tick marks}
\par
The number of major tick marks is too large to be displayed. Use the 
'mark' command to adjust the number of tick marks. 
\subsubsection{Error -11: Invalid plot device selected}
\par
The output device selected is invalid. Check the on-line help for 
supported output device types. 
\subsubsection{Error -12: No data present to plot}
\par
There are no data present to be plotted. You are doing something 
wrong :-) 
\subsubsection{Error -13: Too much points for spline / steps}
\par
The number of points for the current data set is too large for 
'step' or 'spline' mode. The value is set by MAXSP in the file 
'config.inc'. The program needs recompiling after the value is 
changed. 
\subsubsection{Error -14: Invalid contour line set selected}
\par
An invalid contour line set was selected. The maximum number of 
contour line sets is specified by MAXHL in the file 'config.inc'. 
The program must be recompiled after the variable was altered. 
\subsubsection{Error -15: Invalid frame selected}
\par
An invalid frame was selected. Check 'show frames' for the current 
frame settings. 
\subsubsection{Error -17: Maximum number of frames exceeded}
\par
You want more frames than KUPLOT can handle. If you absolutely need 
more frames, change the setting of MAXFRAME in 'config.inc' and re- 
compile the program. 
\subsubsection{Error -18: Invalid RGB color found}
\par
RGB colors must have values between 0.0 and 1.0. The entered value 
is invalid. 
\subsubsection{Error -19: Invalid peak number found}
\par
There is no maximum with the specified number. Use the 'smax' 
command to see a list of found maxima. 
\subsubsection{Error -20: No data points in selected area}
\par
The area selected for the command contains no points. 
\subsubsection{Error -21: Too many maxima found at search}
\par
The search routine called by 'smax' has found too many maxima. Try 
limiting the number by changing the parameter $ <$ifen$> $ (see help for 
smax). The maximum number is set by MAXMAX in the file 'plot.f' and 
'math.f'. 
\subsubsection{Error -22: No maxima found}
\par
No maxima were found. Try a different setting of the $ <$ifen$> $ parameter 
(see help for command 'smax') or are there no maxima after all ??? 
\subsubsection{Error -23: Incompatible data sets for KCAL}
\par
The command 'kcal' works only for data sets of the same typ (2d or 
3d) and identical size. 
\subsubsection{Error -24: Size of res[] array exceeded}
\par
The size of the result variable array res[i] is exceeded. The size 
is defined in 'param.inc'. 
\subsubsection{Error -25: Invalid fit function selected}
\par
An invalid fit function was selected. Note that not all functions 
are available for 3D data sets. Check the help entry for 'fit func' 
for more information. 
\subsubsection{Error -26: Invalid fit parameter selected}
\par
An invalid fit parameter was selected. 
\subsubsection{Error -27: Invalid weighting scheme selected}
\par
The selected weighting scheme is invalid. Check the help entry for 
'fit wic' for more information. 
\subsubsection{Error -28: Invalid annotation number selected}
\par
The given annotation number (command 'sann') is not valid. If you 
need more annotations, you have to change MAXAN in the file 
'kuplot.inc' and recompile KUPLOT. 
\subsubsection{Error -29: x or y-range of data set is zero}
\par
The x- or y-range of the read 3D file is zero. Check your input 
file. 
\subsubsection{Error -30: Not enough maxima found for start values}
\par
The number of Gaussians or Lorenzians exceeds the number of maxima 
found in your data set. You can set the start parameters manually 
(-$> $ 'par') or adjust the size of the search window (-$> $ 'mfen') 
until all needed maxima are found. 
\subsubsection{Error -31: To many fit parameters}
\par
The number of fit parameters exceeds the maximum limit. Either 
use a theory function with less parameters or adjust the value of 
MAXPARA in 'config.inc' and recompile KUPLOT. 
\subsubsection{Error -33: Invalid bond definition selected}
\par
An invalid bond definition was specified. Check your input. 
\subsubsection{Error -34: Data set to large for bitmap drawing}
\par
In order to use PGPLOT contour and bitmap routines, the data set 
must fit in an array of the size MAXZ x MAXZ regardless of the 
current setting of MAXDAT. Either use a smaller data set or adjust 
the value of MAXZ in 'config.inc' and recompile KUPLOT. 
\subsubsection{Error -35: No valid frame selected}
\par
During the select frame mouse operation and error occurred. Try 
selecting the appropriate frame again. Note that there might be 
areas of the plotting surface that do not belong to any frame. 
\subsubsection{Error -36: Configuration mismatch file \& KUPLOT}
\par
The loading of KUPLOT defaults works only for default files that 
were created with a KUPLOT version using the identical array 
size definitions. 
\subsubsection{Error -37: Scan not found in SPEC file}
\par
An invalid scan number was specified. Check your SPEC file for 
the correct number (\#S xxx). 
\subsubsection{Error -38: Invalid column value specified}
\par
The column name or number was not found for the specified scan. 
Note that names must be given in full and are case sensitive. 
\subsubsection{Error -39: Invalid smoothing size specified}
\par
An invalid smoothing size was given. Valid numbers are odd 
integers. The minimum value for a smoothing polynomial of the 
order of 2 is 5, for 4th order 7. 
\subsubsection{Error -40: Maximum derivative exceeded}
\subsubsection{Error -41: MCA scan not found}
\subsubsection{Error -42: Invalid window ID selected}
\subsubsection{Error -43: Invalid value/range for log. axis}
\subsubsection{Error -44: Data sets must have same x-range}
\subsubsection{Error -45: Step size of data sets is different}
\subsubsection{Error -46: Invalid bank number found in iparm. file}
\subsubsection{Error -47: Error reading instrument parameter file}
\subsubsection{Error -48: Unsupported GSAS binning type found}
\subsubsection{Error -49: No TIME\_MAP entry found in GSAS file}
\subsubsection{Error -50: Requested bank not found in GSAS file}
\subsubsection{Error -51: Unit for GSAS TOF conversion invalid}
\subsubsection{Error -52: Unsupported incident spectr. function}
\subsubsection{Error -53: This KUPLOT has NeXus support disabled}
\subsubsection{Error -54: Close current NeXus file first}
\subsubsection{Error -55: No NeXus file currently open}
\subsubsection{Error -56: NeXus SDS has too many dimensions}
\subsubsection{Error -57: Invalid SDS name specified (try nxdir)}
\subsubsection{Error -58: SDS section value out of range}
\subsubsection{Error -59: Invalid column number specified}
\subsubsection{Error -60: Multiple point with same x, no spline}
\subsubsection{Error -61: Singular matrix in Savitzky calculation}
\par
\subsubsection{Error -62: Data set is not 2D}
\subsubsection{Error -63: Parameter no outside range}
\subsubsection{Error -64: DIFFEV Summary file not found}
\subsubsection{Error -65: DIFFEV LOG  or Current file not found}
\subsubsection{Error -66: DIFFEV Rvalue file not found}
